\documentclass{article}
\usepackage[utf8]{inputenc}
\usepackage[spanish]{babel}
\usepackage{amsmath}
\usepackage{amsfonts}
\usepackage{graphicx}
\usepackage{caption}
\usepackage{subcaption}
\graphicspath{ {./figuras/} }
\usepackage{dsfont}
\usepackage{physics}
\usepackage[top=2cm,bottom=2cm,left=2cm,right=2cm]{geometry}
\usepackage{fouriernc}
\usepackage{hyperref}
\usepackage{bm}
\DeclareMathAlphabet{\mathcal}{OMS}{zplm}{m}{n}
\usepackage{titlesec}
\setlength{\parskip}{1em}
\setcounter{secnumdepth}{4}

\titleformat{\paragraph}
{\normalfont\normalsize\bfseries}{\theparagraph}{1em}{}
\titlespacing*{\paragraph}
{0pt}{3.25ex plus 1ex minus .2ex}{1.5ex plus .2ex}

\title{Coarse graining}
\author{Adán Castillo Guerrero}
\date{\today}

\newcommand{\Cc}{\mathcal{C}} %Para escribir la C caligráfica que denota al mapeo CG
\newcommand{\Hh}{\mathcal{H}} %Para escribir la H caligráfica que denota esp de Hilbert
\newcommand{\ident}{\mathbb{1}}% Para escribir el op. indentidad con notación chida
\newcommand{\CG}[1]{\Cc\left[#1\right]}
\newcommand*{\B}[1]{\ifmmode\bm{#1}\else\textbf{#1}\fi}
\begin{document}
\section{Idea General}
Sea $\Cc$ el mapeo de Coarse graining definido por:
\begin{equation}
    \Cc\left[\rho\right]=\Tr_{B}\left(p\rho+(1-p)S\rho S^{\dag}\right)
\end{equation}
donde $\rho\in \Hh^{2}_{A}\otimes \Hh^{2}_{B}$ es el operador de densidad de un sistema de cuatro niveles, y $\Tr_{B}$ denota la traza parcial respecto a $\Hh^{2}_{B}$. Al estado $\rho$ se le llama estado ``microscópico'' y a $\CG{\rho}$, estado ``macroscópico''. En este documento, a los estados macroscópicos se les escribirá en negritas. De esta forma:
\begin{equation}
    \B{\rho}=\CG{\rho}
\end{equation}
La tesis tendría como tema el estudio de la dinámica bajo el mapeo $\Cc$. Esto es, dada una operación $M$:
\begin{equation}
    M:\Hh^{4}\rightarrow\Hh^{4}
\end{equation}
aplicada sobre el sistema descrito por $\rho$, nos interesa saber el mapeo efectivo resultante sobre $\B{\rho}$. Tomando en cuenta que si:
\begin{align}
    \rho'=&M\rho M^{\dag}\\
    \Rightarrow \B{\rho}'=&\CG{M\rho M^{\dag}}
\end{align}
Para un observador externo, que solo tiene acceso a la descripción \textit{gruesa}, la dinámica se ve como:
\begin{equation}\label{eq:obs}
    \B{\rho}\rightarrow\mathcal{A}\left[\B{\rho}\right]=\B{\rho}'
\end{equation}
con $\mathcal{A}$ un mapeo:
\begin{equation}
    \mathcal{A}:\Hh^{2}\rightarrow\Hh^{2}
\end{equation}
Una idea sería determinar la forma de $\mathcal{A}$ mediante tomografía de procesos cuántica. Ahora, dado un estado puro arbitrario $\psi=(\psi_{00},\psi_{01},\psi_{10},\psi_{11})$, el mapeo Coarse Graining es:
\begin{equation}
    \B{\rho} =
    \begin{pmatrix}
        \abs{\psi_{00}}^{2}+p\abs{\psi_{01}}^{2}+(1-p)\abs{\psi_{10}}^{2} & p(\psi_{00}\psi_{10}^{*}+\psi_{01}\psi_{11}^{*})+(1-p)(\psi_{00}\psi_{01}^{*}+\psi_{10}\psi_{11}^{*})\\
        p(\psi_{00}^{*}\psi_{10}+\psi_{01}^{*}\psi_{11})+(1-p)(\psi_{00}^{*}\psi_{01}+\psi_{10}^{*}\psi_{11}) & \abs{\psi_{11}}^{2}+p\abs{\psi_{10}}^{2}+(1-p)\abs{\psi_{01}}^{2}
    \end{pmatrix},
\end{equation}

Claro, un ejercicio muy interesante sería el de estudiar el problema \textit{inverso}, esto es, que a través de la observación dada por la ecuación (\ref{eq:obs}), se obtenga un mapeo $\mathfrak{A}$ tal que:
\begin{equation}
    \rho'=\mathfrak{A}\left[\rho\right]
\end{equation}
La cuestión es que me parece que esto requiere haber resuelto primero el problema del mapeo de asignación inverso para el coarse graining.



Otro punto es repetir resultados considerando $\rho$ que no sean necesariamente de la forma $\ket{\psi}\bra{\psi}$. Una primera pregunta sería la de cómo generar dichos estados de forma uniforme, que recuerdo se había comentado que la idea era generar estados de 8 niveles y sacar trazas parciales.



\end{document}