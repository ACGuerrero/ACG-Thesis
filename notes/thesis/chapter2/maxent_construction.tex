\section{Construcción del estado de máxima entropía}

\acnote{Párrafo iterado: solo notas.}

Sea $\rho_{\ef}\in\mcS(\hilbert_{2})$ el estado grueso accesible al observador, y sea $\{A_{j}\}_{j}$ con $A_{j}\in\obspace{2}$ un conjunto de observables tomográficamente completo.
Si $\rho_{\ef}$ es un estado grueso correspondiente a un estado fino $\varrho\in\mcS(\hilbert_{2}^{\otimes n})$, de acuerdo con lo discutido en el capítulo anterior, podemos asignar a $\rho_{\ef}$ un estado microscópico que maximice la entropía de Von Neumann que satisfaga las restricciones $\expval{A_{j}}=\Tr(\rho_{\ef} A_{j})$.
Escójase ${A_{i}}={\pauli{i}}$, las matrices de Pauli, y la aplicación de grano grueso desarrollada en la sección anterior, aquella del grano grueso y borroso dada por (\ref{eq:CG}). Los valores esperados de los operadores se traducen como las componentes del vector de Bloch del operador $\rho_{\ef}$. Las restricciones a las que se ve sujeto el operador $\varrho_{\max}$ son
\begin{equation}\label{eq:Restrictions}
    r_{j}=\Tr[\pauli{j}\rho_{\ef}]
\end{equation}
\acnote{Párrafo iterado: notación}

Aquí hay un problema: el estado que maximiza la entropía pertenece al espacio $\densityspace{2^{n}}$, mientras que las restricciones están definidas para un operador de densidad en $\densityspace{2}$. Entonces, ¿cómo se traducen dichas restricciones en el nivel microscópico? Naturalmente, esto dependerá de la relación entre el estado efectivo $\rho_{\ef}$ y el estado subyacente, $\rho_{\ef}=\CG{\varrho}$. Esto es, el estado de máxima entropía depende de la aplicación de grano grueso. Sustituyendo la relación entre algún estado microscópico compatible $\varrho$ y el estado efectivo $\rho_{\ef}$ en la ecuación \label{eq:Restrictions}, y manipulando un poco se halla que
\begin{align}
    r_{j}&=\Tr[\sigma_{j}\CG{\varrho}]\nonumber\\
    &=\Tr[\sigma_{j}\Tr_{\overline{1}}\qty(p_{1}\varrho+\sum_{k=2}^{n}p_{k}S_{1,k}\varrho S_{1,k}^{\dagger})]\nonumber\\
    &=\Tr[\sigma_{j}\otimes\Id_{2^{n-1}}\qty(p_{1}\varrho+\sum_{k=2}^{n}p_{k}S_{1,k}\varrho S_{1,k}^{\dagger})]\nonumber\\
    &=\Tr[\qty(p_{1}(\sigma_{j}\otimes\Id_{2^{n-1}})+\sum_{k=2}^{n}p_{k}S_{1,k}^{\dagger}(\sigma_{j}\otimes\Id_{2^{n-1}})S_{1,k})\varrho]\nonumber\\
    &=\Tr[\qty(p_{1}(\sigma_{j}\otimes\Id_{2^{n-1}})+\sum_{k=2}^{n}p_{k}(\Id_{2^{k-1}}\otimes\sigma_{j}\otimes\Id_{2^{n-k}}))\varrho]\nonumber\\
    &=\Tr[\qty(\sum_{k=1}^{n}p_{k}(\Id_{2^{k-1}}\otimes\sigma_{j}\otimes\Id_{2^{n-k}}))\varrho]\nonumber.
\end{align}
Definiendo
\begin{equation}\label{eq:GhatNM}
    \hat{G}_{j}=\sum_{k=1}^{n}p_{k}(\Id_{2^{k-1}}\otimes\sigma_{j}\otimes\Id_{2^{n-k}}),
\end{equation}
las restricciones se pueden esribir como
\begin{equation}\label{eq:MaxEntRestrictions}
    r_{j}=\Tr[\hat{G}_{j}\varrho].
\end{equation}
Estas restricciones ya se hallan en términos de observables y un operador de densidad que actúan sobre $\hilbert_{2^{n}}$. Para obtener el estado que maximiza la entropía se realiza el proceso desarrollado en el capítulo anterior. De acuerdo con la ecuación (\ref{eq:GeneralMaxEnt}), el estado de máxima entropía compatible con (\ref{eq:MaxEntRestrictions}) es
\begin{equation}\label{eq:MaxEntLagMult}
    \varrho_{\max}=\frac{1}{Z}e^{\sum_{j}\lambda_{j}\hat{G}_{j}}.
\end{equation}
Si se sustituye a $\varrho_{\max}$ en las ecuaciones (\ref{eq:MaxEntRestrictions}) (cosa nada recomendable), se obtienen las relaciones entre los multiplicadores de Lagrange y los valores esperados de los operadores $\pauli{j}$ Si se escribe $\lambda=\sqrt{\lambda_{1}^{2}+\lambda_{2}^{2}+\lambda_{3}^{2}}$, los resultados son
\begin{align}\label{eq:MaxEntExpVals}
    \begin{split}
    \expval{\pauli{1}}&=\lambda_{1}\rfroml(\lambda),\\
    \expval{\pauli{2}}&=\lambda_{2}\rfroml(\lambda),\\
    \expval{\pauli{3}}&=\lambda_{3}\rfroml(\lambda),
    \end{split}
\end{align}
donde $\rfroml(\lambda)$ es una función biyectiva de $\lambda$, y cuya expresión será derivada en la siguiente sección de una forma que requiere muchas menos cuentas. Idealmente, la ecuación (\ref{eq:MaxEntLagMult}) está en términos de los valores de expectación $r_{i}=\Tr(\sigma_{i}\rho_{\ef})$, y no de los multiplicadores de Lagrange. Aunque no es posible despejar a los multiplicadores de Lagrange de las ecuaciones de manera algebráica, la naturaleza de $\rfroml(\lambda)$ nos permite asegurar que las relaciones son uno a uno y que tienen inversa.

A partir de este momento, cada vez que se hable del \textit{estado de máxima entropía}, se entiende que se hace referencia al estado dado por la ecuación (\ref{eq:MaxEntLagMult}). Esto es, al estado de máxima entropía que es compatible con un estado efectivo inducido por nuestro modelo de grano grueso particular.