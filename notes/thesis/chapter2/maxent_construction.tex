\section{Construcción de la asignación de máxima entropía}

Sea $\rho\in\mcS(\hilbert_{2})$ el estado grueso accesible al observador, y sea $\{A_{i}\}$ con $A_{i}\in\mcL(\hilbert_{2})$ un conjunto de observables tomográficamente completo. Si $\rho$ es un estado grueso correspondiente a un estado fino $\varrho\in\mcS(\hilbert_{2}\otimes\hilbert_{2})$, de acuerdo con lo discutido en el capítulo anterior, podemos asignar a $\varrho$ un estado que maximice la entropía de Von Neumann sin agregar información externa, y que satisfaga las restricciones $\expval{A_{i}}=\Tr(\rho A_{i})$. ¿Pero cómo se traducen dichas restricciones en el nivel microscópico? Naturalmente, esto dependerá de la relación entre $\varrho$ y $\rho$. Esto es, el estado de máxima entropía depende de la aplicación de grano grueso.

Escójase ${A_{i}}={\pauli{i}}$, las matrices de Pauli, y la aplicación de grano grueso como (\ref{eq:CG}). Los valores esperados de los operadores se traducen como las componentes del vector de Bloch del operador $\rho$. Las restricciones a las que se ve sujeto el operador $\varrho_{max}$ son
\begin{align*}
    r_{i}&=\Tr[\pauli{i}\rho],\\
    &=\Tr[\pauli{i}\Lambda(\varrho)],\\
    &=\Tr[\pauli{i}\Tr_{2}(p\varrho+(1-p)S\varrho S^{\dag})],\\
    &=\Tr[\pauli{i}\otimes\Id(p\varrho+(1-p)S\varrho S^{\dag})],\\
    &=\Tr[(p\pauli{i}\otimes\Id+(1-p)\Id\otimes\pauli{i})\varrho].
\end{align*}
Definiendo
\begin{equation}\label{eq:Ghat}
    \hat{G}_{i}=p\pauli{i}\otimes\Id+(1-p)\Id\otimes\pauli{i},
\end{equation}
las restricciones se pueden esribir como
\begin{equation}\label{eq:MaxEntRestrictions}
    r_{i}=\Tr[\hat{G}_{i}\varrho].
\end{equation}
Una vez obtenidas las restricciones, se utilizan multiplicadores de Lagrange para obtener el estado de maximiza la entropía. De acuerdo con la ecuación (\ref{eq:GeneralMaxEnt}), el estado de máxima entropía compatible con (\ref{eq:MaxEntRestrictions}) es
\begin{equation}\label{eq:MaxEntLagMult}
    \varrho_{\max}(\rho)=\frac{1}{Z}e^{\sum_{i}\lambda_{i}\hat{G}_{i}}.
\end{equation}
Si se sustituye a $\varrho_{\max}$ en las ecuaciones (\ref{eq:MaxEntRestrictions}) (cosa nada recomendable), se obtienen las relaciones entre los multiplicadores de Lagrange y los valores esperados de los observables utilizados para la tomografía. Si se escribe $\lambda=\sqrt{\lambda_{1}^{2}+\lambda_{2}^{2}+\lambda_{3}^{2}}$, los resultados son
\begin{align}\label{eq:MaxEntExpVals}
    \begin{split}
    \expval{\pauli{1}}&=\lambda_{1}f(\lambda),\\
    \expval{\pauli{2}}&=\lambda_{2}f(\lambda),\\
    \expval{\pauli{3}}&=\lambda_{3}f(\lambda),
    \end{split}
\end{align}
donde $f(\lambda)$ es una función biyectiva de $\lambda$, y cuya forma será derivada en la siguiente sección. Idealmente, la ecuación (\ref{eq:MaxEntLagMult}) está en términos de los valores de expectación $r_{i}=Tr(\sigma_{i}\rho_{c})$, y no de los multiplicadores de Lagrange. Aunque no es posible despejar a los multiplicadores de Lagrange de las ecuaciones de manera algebráica, la naturaleza de $f(\lambda)$ nos permite asegurar que las relaciones son uno a uno y que tienen inversa.