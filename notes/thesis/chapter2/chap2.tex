\chapter{Estado de máxima entropía \textit{borroso}}

Con las bases asentadas, es posible pasar al estudio del problema propuesto por este trabajo: el estudio de la dinámica efectiva asociada a un modelo de grano grueso y a una aplicación se asignación.

Supongamos que estudiamos un sistema cuántico de varias particulas, que hemos logrado aislarlo de forma que evolucione como un sistema cuántico cerrado, y que conocemos a ciencia cierta la forma de la evolución seguida por el sistema.

En principio, parecería que no tendríamos ningún problema para conocer el estado del sistema a un tiempo arbitrario, dado que conozcamos el estado inicial del sistema. Sin embargo, nos damos cuenta de que, al comenzar a realizar mediciones sobre el sistema para poder hallar al estado inicial, nos damos cuenta de que nuestra instrumentación no sólo no puede resolver todos los grados de libertad del sistema, sino que tiene una probabilidad no nula de fallar.

Al no poder acceder a todas las dimensiones del sistema, nuetra descripción es, efectivamente, un modelo de grano grueso. ¿Cómo podemos describir la evolución del sistema que estamos estudiando? ¿Cómo evoluciona el estado efectivo accesible?

En este capítulo describiremos matemáticamente la aplicación de grano grueso con la que se trabajará, luego se utilizará el Principio de Máxima Entropía para hallar el mejor estimado posible dada nuestra descripción efectiva, y finalmente se propondrá el modelo de la dinámica efectiva. La dinámica efectiva siendo la evolución que el experimentalista observaría.


\section{Un modelo de grano grueso y borroso}\label{sec:CH2CG}

Como se discutió previamente, es natural suponer que no siempre es posible disponer de toda la información sobre el estado\ddnote{las itálicas están de más, me parece. Cual es tu política del uso de las itálicas?}\acnote{\checkmark} del sistema de interés. Esto ya sea por insuficiencia en la resolución de los aparatos de medición o por el inevitable error inherente a las herramientas de medición. Un prototipo sencillo de error consiste en el inducido por un aparato que no distingue diferentes conjuntos de partículas entre sí. El caso más simple corresponde a la permutación de dos partículas. Este intercambio accidental a la hora de la medición es una \textit{aplicación borrosa} \cite{FuzzyMeasurements}.

Para ilustrar lo anterior, considérese dicha aplicación borrosa sobre un sistema de dos partículas. Para simplificarlo un poco más, supongamos que cada partícula es un sistema de dos niveles, esto es, el sistema está compuesto por los qubits $A$ y $B$. El estado del sistema está caracterizado por un operador de densidad $\varrho_{AB} \in \mcS(\hilbert_2 \otimes \hilbert_2)$. La acción de la aplicación borrosa se escribe como sigue:
\begin{align}
\mcF:&\mcS(\hilbert_2 \otimes \hilbert_2)\to \mcS(\hilbert_2 \otimes \hilbert_2)\nonumber\\
&\varrho \mapsto p\varrho+(1-p)S\varrho S,\nonumber
\end{align}
donde $0<p<1$ es la probabilidad con la que el aparato de medición identifica correctamente \ddnote*{correctamente}{erróneamente}\acnote{\checkmark} a los dos subsistemas y $S$ es el operador de transposición de dos partículas (llamado operador SWAP), definido como 
\begin{equation}
    S\ket{\psi}\otimes \ket{\phi}=\ket{\phi}\otimes \ket{\psi} \ \ \forall \ket{\psi},\ket{\phi}\in \hilbert_2.\nonumber
\end{equation}
El estado resultante, $\Fuzzy{\varrho_{AB}}=p\varrho_{AB}+(1-p)\varrho_{BA}$, es una mezcla estadística del estado accesible con un detector perfecto, $\varrho_{AB}$, y el estado donde los qubits tienen las etiquetas equivocadas, $\varrho_{BA}:=S\varrho_{AB} S$. Así, si quisieramos hallar el valor esperado del observable $\sigma_{3}\otimes\Id$ (el valor esperado de $\sigma_{z}$ en la primer partícula), encontraríamos:
\begin{equation}
    \expval{\sigma_{3}\otimes\Id}_{\mcF}=p\expval{\sigma_{3}\otimes\Id}+(1-p)\expval{\Id\otimes\sigma_{3}}\nonumber
\end{equation}\ddnote{esta ecuación tiene las probabilidades volteadas}\acnote{\checkmark}
donde por $\expval{A}_{\mcF}$ nos referimos al valor esperado con respecto al estado del sistema descrito a través de $\mcF$.

Es importante notar que, auque la aplicación borrosa modela el error asociado al aparato de medición, no constituye por si misma un modelo de grano grueso, pues conserva la dimensión del sistema: el aparato resuelve todos los grados de libertad.

Al error se le añade la falta de resolución: solo se resuelve una partícula. \ddnote*{esto está redundante y escrito raro, no se por que dices que el estado sería el de la primera partícula, eso confunde, parece que estás cambiando repentinamente el contexto. Mejor explica al grano la situación y sin hablar de composición de errores. Eso mas bien se deriva, y justamente lo haces mas adelantito.}{Agreguemos al error por permutación la falta de resolución. Una posibilidad es que el instrumento detecte una partícula en donde en realidad haya dos, y aún más, que en un sistema de dos partículas, este sea capaz de medir únicamente observables asociados al primer subsistema. El estado observado sería, de acuerdo con la sección \ref{sec:Ch1PartialTrace}, la matriz de densidad reducida de la primera partícula.}\acnote{\checkmark} Matemáticamente, la composición del error y de la falta de resolución puede escribirse como
\begin{gather}
    \mcC:\mcS(\hilbert_2 \otimes \hilbert_2)\to \mcS(\hilbert_2)\nonumber\\
    \varrho_{AB} \mapsto p\rho_A+(1-p)\rho_B\rlap{,}\nonumber
\end{gather}
donde $\rho_A=\tr_B \rho_{AB}$ y $\rho_B=\tr_A \rho_{AB}$, es decir, las matrices de densidad reducidas de la partículas $A$ y $B$, respectivamente.


A diferencia de la aplicación borrosa, el modelo de grano grueso disminuye la dimensión del estado resultante. Además se puede mostrar que la ecuación anterior puede reescribirse en términos de la aplicación borrosa \cite{FuzzyMeasurements},
\begin{equation}
\CG{\varrho}=(\Tr_{B}\circ\mcF)[\varrho].\nonumber
\end{equation}
En este contexto, diferenciamos al estado ``microscópico'' o ``fino'', denotado por $\varrho\in \mcS(\hilbert_2\otimes\hilbert_2)$, y al estado ``macroscópico'', ``grueso'', o  ``efectivo'' denotado por $\rho_{\ef}\in \mcS(\hilbert_2)$, a través de la relación
\begin{equation}
    \rho_{\ef}=\CG{\varrho}.\nonumber
\end{equation}
Es extremadamente importante notar que la expresión anterior no es invertible. Pueden existir una infinidad de estados $\varrho$ tales que su descripción gruesa coincida con $\rho_{\ef}$. Como ejemplo, supóngase que el estado efectivo está descrito por $\rho_{\ef}=\frac{1}{2}\Id$, el estado máximamente mezclado. Entonces cualquier sistema fino que se halle en un estado máximamente entrelazado será compatible con la descripción efectiva $\frac{1}{2}\Id$.

Pues bien, como se ha asumido que conocemos la evolución unitaria subyacente, requerimos asignar a $\rho$ un estado microscópico que cumpla con todas las restricciones impuestas por nuestras mediciones. Asumiremos que dicho estado asignado es el que experimenta la evolución. 


La discusión anterior giró alrededor del caso en que el modelo de grano grueso reduce un espacio de dos qubits a uno de un solo qubit. Esto puede generalizarse al caso en que el aparato de medición sólo detecta una partícula cuando el sistema microscópico \ddnote*{conforma?}{comporta} $n$ partículas. En particular, consideramos, nuevamente, subsistemas de dos niveles (qubits), de tal forma que $\varrho\in\mcS\qty( \hilbert_{2}^{\otimes n})$, donde $\hilbert_{2}^{\otimes n}$ representa el producto tensorial del espacio $\hilbert_{2}$ consigo mismo $k$ veces. Sean $p_{i}$ las probabilidades de medir cada una de las partículas. La aplicación borrosa pasa de ser un intercambio de dos partículas, a una serie de permutaciones entre la partícula de interés y el resto (sin pérdida de generalidad, asumiremos que la partícula de interés es la primera). Considerando un sistema de $n$ subsistemas de dos niveles \ddnote*{si ya definiste qubits mas arriba como sistemas de dos niveles, no es necesario el paréntesis}{(qubits)}, la aplicación borrosa \ddnote*{se define entonces como}{se convierte en}
\begin{gather}
    \mcF:\mcS\qty( \hilbert_{2}^{\otimes n})\to \mcS\qty( \hilbert_{2}^{\otimes n})\nonumber\\
    \varrho \mapsto p_{1}\varrho+\sum_{j=2}^{n}p_{j}(S_{1,j})\varrho(S_{1,j})^{\dagger},\nonumber
\end{gather}
donde $S_{1,j}$ es el operador que permuta la primera y la $j$-ésima partícula. De esta manera, donde $\Tr_{\overline{i}}$ denota la traza parcial sobre todos menos el $i$-ésimo qubit, la aplicación de grano grueso que resuelve un qubit donde hay $n$ qubits es
\begin{gather}\label{eq:CG}
    \mcC:\mcS( \hilbert_{2}^{\otimes n})\to \mcS(\hilbert_{2})\nonumber\\
    \varrho\mapsto\Tr_{\overline{1}}(\Fuzzy{\varrho}).
\end{gather}
Reconocemos, según los valores de las probabilidades $p_{j}$ dos tipos de regímenes. El primero corresponde a nuestra discusión: si $p_{1}>p_{j}\forall j\neq 1$ se dice que el modelo tiene una partícula preferencial. Si $p_{j}=\frac{1}{n}\forall j$ entonces no existe ninguna partícula preferencial, y se puede ver al sistema como una caja de gas en la que medir a cualquier partícula es igual de probable.
\section{Construcción del estado de máxima entropía}

Sea $\rho\in\mcS(\hilbert_{2})$ el estado grueso accesible al observador, y sea \ddnote*{$\{A_{i}\}_{i}$}{$\{A_{i}\}$} con $A_{i}\in\obspace{2}$ un conjunto de observables tomográficamente completo.
Si $\rho$ es un estado grueso correspondiente a un estado fino $\varrho\in\mcS(\hilbert_{2}^{\otimes n})$, de acuerdo con lo discutido en el capítulo anterior, podemos asignar a $\varrho$ un estado que maximice la entropía de Von Neumann sin agregar \ddnote*{que es la información externa?, no es mejor, asignar un estado de máxima entropía compatible con las restricciones tal y tal} {información externa, y que satisfaga las restricciones $\expval{A_{i}}=\Tr(\rho A_{i})$}.
Escójase ${A_{i}}={\pauli{i}}$, las matrices de Pauli, y la aplicación de grano grueso desarrollada en la sección anterior, aquella del grano grueso y borroso dada por (\ref{eq:CG}). Los valores esperados de los operadores se traducen como las componentes del vector de Bloch del operador $\rho$. Las restricciones a las que se ve sujeto el operador $\varrho_{\max}$ son
\begin{equation*}
    r_{i}=\Tr[\pauli{i}\rho]
\end{equation*}\ddnote{supongo que estás usando $\rho$ y $\varrho$ para las cosas finas y gruesas, pero no se entiende, a la vista los caracteres son parecidos, sugiero usar $\rho_\text{MaxEnt}$ y $\rho_\text{ef}$ o $\rho_\text{grueso}$}
Aquí hay un problema: el estado que maximiza la entropía pertenece al espacio $\densityspace{2^{n}}$, mientras que las restricciones están definidas para un operador de densidad en $\densityspace{2}$. Entonces, ¿cómo se traducen dichas restricciones en el nivel microscópico? Naturalmente, esto dependerá de la relación entre \ddnote*{si, no no, está muy feo esto}{$\varrho$ y $\rho$}. Esto es, el estado de máxima entropía depende de la aplicación de grano grueso. Sustituyendo la relación entre algún estado microscópico compatible $\varrho$ y el estado efectivo $\rho$ en la ecuación anterior, y manipulando un poco se halla que
\begin{align*}
    r_{i}&=\Tr[\sigma_{i}\CG{\varrho}]\\
    &=\Tr[\sigma_{i}\Tr_{\overline{1}}\qty(p_{1}\varrho+\sum_{j=2}^{n}p_{j}(S_{1,j})\varrho(S_{1,j})^{\dagger})]\\
    &=\Tr[\sigma_{i}\otimes\Id_{2^{n-1}}\qty(p_{1}\varrho+\sum_{j=2}^{n}p_{j}(S_{1,j})\varrho(S_{1,j})^{\dagger})]\\
    &=\Tr[\qty(p_{1}(\sigma_{i}\otimes\Id_{2^{n-1}})+\sum_{j=2}^{n}p_{j}(S_{1,j})^{\dagger}(\sigma_{i}\otimes\Id_{2^{n-1}})(S_{1,j}))\varrho]\\
    &=\Tr[\qty(p_{1}(\sigma_{i}\otimes\Id_{2^{n-1}})+\sum_{j=2}^{n}p_{j}(\Id_{2^{j-1}}\otimes\sigma_{i}\otimes\Id_{2^{n-j}}))\varrho]\\
    &=\Tr[\qty(\sum_{j=1}^{n}p_{j}(\Id_{2^{j-1}}\otimes\sigma_{i}\otimes\Id_{2^{n-j}}))\varrho].
\end{align*}
Definiendo
\begin{equation}\label{eq:GhatNM}
    \hat{G}_{i}=\sum_{j=1}^{n}p_{j}(\Id_{2^{j-1}}\otimes\sigma_{i}\otimes\Id_{2^{n-j}}),
\end{equation}
las restricciones se pueden esribir como
\begin{equation}\label{eq:MaxEntRestrictions}
    r_{i}=\Tr[\hat{G}_{i}\varrho].
\end{equation}
Estas restricciones ya se hallan en términos de observables y un operador de densidad que actúan sobre $\hilbert_{2^{n}}$. Entonces se utilizan multiplicadores de Lagrange para obtener el estado de maximiza la entropía. De acuerdo con la ecuación (\ref{eq:GeneralMaxEnt}), el estado de máxima entropía compatible con (\ref{eq:MaxEntRestrictions}) es
\begin{equation}\label{eq:MaxEntLagMult}
    \varrho_{\max}=\frac{1}{Z}e^{\sum_{i}\lambda_{i}\hat{G}_{i}}.
\end{equation}
Si se sustituye a $\varrho_{\max}$ en las ecuaciones (\ref{eq:MaxEntRestrictions}) (cosa nada recomendable), se obtienen las relaciones entre los multiplicadores de Lagrange y los valores esperados de los observables utilizados para la tomografía. Si se escribe $\lambda=\sqrt{\lambda_{1}^{2}+\lambda_{2}^{2}+\lambda_{3}^{2}}$, los resultados son
\begin{align}\label{eq:MaxEntExpVals}
    \begin{split}
    \expval{\pauli{1}}&=\lambda_{1}\rfroml(\lambda),\\
    \expval{\pauli{2}}&=\lambda_{2}\rfroml(\lambda),\\
    \expval{\pauli{3}}&=\lambda_{3}\rfroml(\lambda),
    \end{split}
\end{align}
donde $\rfroml(\lambda)$ es una función biyectiva de $\lambda$, y cuya forma será derivada en la siguiente sección de una forma que requiere muchas menos cuentas. Idealmente, la ecuación (\ref{eq:MaxEntLagMult}) está en términos de los valores de expectación $r_{i}=Tr(\sigma_{i}\rho_{c})$, y no de los multiplicadores de Lagrange. Aunque no es posible despejar a los multiplicadores de Lagrange de las ecuaciones de manera algebráica, la naturaleza de $\rfroml(\lambda)$ nos permite asegurar que las relaciones son uno a uno y que tienen inversa.

A partir de este momento, cada vez que se hable del \textit{estado de máxima entropía}, se entiende que se hace referencia al estado dado por la ecuación (\ref{eq:MaxEntLagMult}). Esto es, al estado de máxima entropía que es compatible con un estado efectivo inducido por nuestro modelo de grano grueso particular.
\section{Propiedades}



\subsection{El estado de máxima entropía general: dos expresiones}
He estado cometiendo un error terrible: creer que todos los estados de máxima entropía, independientemente de las componentes de Pauli del estado grueso con el que son compatibles, evolucionan de la misma manera que el estado de Máxima entropía compatibñe con un estado grueso alineado en $z$. Un ejemplo claro de este error es el de la unitaria generada por el hamiltoniano del modelo de Ising propuesto en la sección \ref{sec:Ising}. Es necesario, entonces, tener en cuenta la unitaria que conecta a un estado de máxima entropía compatible con un estado $\rho$ aribitrario y el estado alineado en $z$. La unitaria Puede constrirse en base a los parámetros del estado $\rho$. A cada estado de dos niveles, $\rho$, lo definen tres parámetros: su pureza, $r$, y dos ángulos $\alpha$, $\beta$. La unitaria que conecta $\rho$ con el estado alineado en $z$ de pureza $z=r$ es 
\begin{equation}
  V=
  \begin{pmatrix}
      \cos{\alpha} & e^{-i\beta}\sin{\alpha}\\
      e^{i\beta}\sin{\alpha}& \cos{\alpha}\\
  \end{pmatrix}
\end{equation}
Construyendo $\mcV=V\otimes V$, podemos expresar al estado de máxima entropía de dos formas equivalentes,
\begin{align}\label{eq:MaxEntTwoExpr}
  \varrho_{max}(\rho)=\frac{1}{Z}\text{exp}(-\lambda_{3}\mcV\hat{G}_{3}\mcV^{\dag}) && \varrho_{max}(\rho)=\frac{1}{Z}\text{exp}(\sum_{i}\lambda_{i}\hat{G}_{i}).
\end{align}

\subsection{El estado máxima entropía es separable}

Sea $\rho_{z}$ un estado alineado en $z$ como en (\ref{eq:rhoz}), entonces por (\ref{eq:MaxEnt}) el estado de máxima entropía es:
\begin{equation}\label{eq:MaxEntUgly}
\varrho_{max}^{z}=\frac{\text{exp}(-\lambda_{3}\hat{G}_{3})}{\Tr[\text{exp}(-\lambda_{3}\hat{G}_{3})]}
\end{equation}
donde $\hat{G}_{3}$ se define según (\ref{eq:Gop}). Como los dos términos que componen al operador comuntan entre sí, la exponencial puede separarse:
\begin{align*}
\varrho_{max}^{z}&=\frac{1}{Z}e^{-\lambda_{3}p\sigma_{z}\otimes\Id}e^{-\lambda_{3}(1-p)\Id\otimes\sigma_{z}}\\
&=\frac{1}{Z}(e^{-\lambda_{3}p\sigma_{z}}\otimes\Id)( \Id\otimes e^{-\lambda_{3}(1-p)\sigma_{z}})\\
&=\frac{1}{Z}(e^{-\lambda_{3}p\sigma_{z}}\otimes e^{-\lambda_{3}(1-p)\sigma_{z}})\\
\end{align*}
Si se separa a la función de partición como un producto de trazas $Z=Z_{1}Z_{2}$, al estado de máxima entropía se le puede escribir como:
\begin{equation}\label{eq:MaxEntZ}
\varrho_{max}^{z}=\frac{e^{-\lambda_{3}p\sigma_{z}}}{Z_{1}} \otimes \frac{e^{-\lambda_{3}(1-p)\sigma_{z}}}{Z_{2}}
\end{equation}
Esto es válido para el estado alineado en $z$, pero retomando el resultado (\ref{eq:MaxEntTwoExpr}), el estado de máxima entropía compatible con un estado grueso arbitrario es
\begin{equation}\label{eq:MaxEntSeparable}
  \boxed{\varrho_{max}=\frac{e^{-\lambda_{3}pV\sigma_{z}V^{\dag}}}{Z_{1}} \otimes \frac{e^{-\lambda_{3}(1-p)V\sigma_{z}V^{\dag}}}{Z_{2}}}
\end{equation}
Por lo que el estado de máxima entropía compatible con un estado $\rho$ arbitrario es separable.

\subsection{El estado de máxima entropía bajo la aplicación de grano grueso}\label{sec:CG(MaxEnt)}

El problema de la ecuación (\ref{eq:MaxEntZ}) es que el estado de máxima entropía está en términos del multiplicador de Lagrange que se usó para maximizar la entropía, en lugar de estar en términos de la cantidad medible $r_{z}$. Si por alguna razón tuviéramos que resignarnos a trabajar con el estado en términos de $\lambda_{3}$, será necesario conocer la expresión del estado efectivo. Para hallarla, basta con pasar (\ref{eq:MaxEntZ}) y (\ref{eq:MaxEntSeparable}) por la aplicación de grano grueso. Si el estado grueso está alineado en $z$, entonces tiene la forma
\begin{equation}\label{eq:CG(MaxEntZ)1}
    \rho_{z}=\frac{1}{Z}\CG{\varrho_{max}^{z}}=p\frac{e^{-\lambda_{3}p\sigma_{z}}}{Z_{1}}+(1-p)\frac{e^{-\lambda_{3}(1-p)\sigma_{z}}}{Z_{2}},
\end{equation}
un estado arbitrario, por otro lado
\begin{equation}\label{eq:CG(MaxEnt)1}
  \rho=\frac{1}{Z}\CG{\varrho_{max}}=p\frac{e^{-\lambda_{3}pV\sigma_{z}V^{\dag}}}{Z_{1}}+(1-p)\frac{e^{-\lambda_{3}(1-p)V\sigma_{z}V^{z}}}{Z_{2}},
\end{equation}
Las exponenciales de la ecuación (\ref{eq:CG(MaxEntZ)1}) pueden verse como $e^{a\hat{n}\cdot \vec{\sigma}}$. Si se desarollan las series se halla
\begin{equation}\label{eq:PauliVectorExp}
    e^{a\hat{n}\cdot \vec{\sigma}}=\Id \cosh{a}+(\hat{n}\cdot \vec{\sigma})\sinh{a}
\end{equation}
así que, sustituyendo la ecuación (\ref{eq:PauliVectorExp}) en (\ref{eq:CG(MaxEntZ)1}) se encuentra la expresión del estado efectivo en términos de la base de Pauli
\begin{align*}
    \rho_{z}&=p\frac{\Id \cosh{\lambda_{3}p}-\sigma_{z}\sinh{\lambda_{3}p}}{Z_{1}}+(1-p)\frac{\Id \cosh{\lambda_{3}(1-p)}-\sigma_{z}\sinh{\lambda_{3}(1-p)}}{Z_{2}}\\
    &=p\frac{1}{2}(\Id \frac{2\cosh{\lambda_{3}p}}{Z_{1}}-\sigma_{z}\frac{2\sinh{\lambda_{3}p}}{Z_{1}})+(1-p)\frac{1}{2}(\Id \frac{2\cosh{\lambda_{3}(1-p)}}{Z_{2}}-\sigma_{z}\frac{2\sinh{\lambda_{3}(1-p)}}{Z_{2}})
\end{align*}
para que esto sea de la forma $\rho=\sum_{i}p_{i}\rho_{i}$ es necesario que $Z_{1}=2\cosh{\lambda p}$ y $Z_{2}=2\cosh{\lambda (1-p)}$ (cosa que se puede comprobar). El estado efectivo en términos de $\lambda_{3}$ es
\begin{equation}\label{eq:CG(MaxEntZ)2}
    \rho_{z}=p\frac{1}{2}(\Id+\sigma_{z}\tanh{(-\lambda p)})+(1-p)\frac{1}{2}(\Id+\sigma_{z}\tanh{(-\lambda (1-p))})
\end{equation}
Naturalmente, el caso general es la ecuación anterior como $V$ aplicada en ella para obtener el estado rotado. Ahora, si se compara este resultado con la definición de la aplicación de grano grueso y el hecho que el estado de máxima entropía es separable, encontramos que
\begin{align*}
  \varrho_{max}&=\frac{1}{2}(\Id+\sigma_{z}\tanh{\lambda p})\otimes\frac{1}{2}(\Id+\sigma_{z}\tanh{\lambda (1-p)})\\
  &  =\frac{1}{4}(\Id_{4}+\sigma_{z}\otimes\Id_{2}\tanh{\lambda p}+\sigma_{z}\otimes \Id_{2}\tanh{\lambda (1-p)}+\sigma_{z}\otimes \sigma_{z}\tanh{\lambda p}\tanh{\lambda (1-p)})
\end{align*}

\subsection{El estado de máxima entropía en términos de $r_{z}$}

La ecuación (\ref{eq:CG(MaxEntZ)2}) permite expresar la coordenada $r_{z}$ en términos del multiplicador de Lagrange complejo es
\begin{equation}\label{eq:RzTanh}
    r_{z}=p\tanh{\lambda p}+(1-p)\tanh{\lambda (1-p)}.
\end{equation}
Esta expresión la obtuve después de ahaber desarollado los párrafos de discusión que siguen. 
La forma matricial de este estado es:
\begin{equation*}
\left(
\begin{array}{cccc}
 \frac{1}{4} e^{-\lambda_{3}} \text{sech}(\lambda_{3} p)
   \text{sech}(\lambda_{3}-\lambda_{3} p) & 0 & 0 & 0 \\
 0 & \frac{e^{2 \lambda_{3}}}{\left(e^{2 \lambda_{3}
   p}+1\right) \left(e^{2 \lambda_{3}}+e^{2 \lambda_{3}
   p}\right)} & 0 & 0 \\
 0 & 0 & \frac{1}{\left(e^{2 \lambda_{3}}+1\right) e^{-2
   \lambda_{3} p}+e^{2 \lambda_{3}-4 \lambda_{3}
   p}+1} & 0 \\
 0 & 0 & 0 & \frac{1}{4} e^{\lambda_{3}}
   \text{sech}(\lambda_{3} p) \text{sech}(\lambda_{3}-\lambda_{3} p) \\
\end{array}
\right)
\end{equation*}
Hallar el valor de $\lambda_{
3}$ en términos del valor $r_{z}$ implica resolver la ecuación:
\begin{equation}\label{eq:RZ}
rz=-\frac{1}{2}\frac{\sinh(\lambda_{3})+(1-2p)\sinh((1-2p)\lambda_{3})}{\cosh(p\lambda_{3})\cosh((1-p)\lambda_{3})}
\end{equation}
No se ve ninguna forma sencilla de despejar al multiplicador de Lagrange \notaAd{la ecuación (\ref{eq:RzTanh}) y la ecuación (\ref{eq:RZ}) son completamente equivalentes, como debería de ser. La segunda siendo más fea que la primera. }. En realidad, esto solo se puede si la función $r_{z}(\lambda_{3})$ tiene inversa, y esto puede depender del parámetro $p$. Graficar la superficie (Figura \ref{fig:rzsurf}) puede aclarar algo el panorama.
\begin{figure}[h!]
\centering
\begin{subfigure}{0.475\textwidth}
  \centering
  \includegraphics[width=0.6\linewidth]{maxent/figures/LagrangeMult_lambda-8to8.png}
  \caption{$-8<\lambda_{3}<8$}
\end{subfigure}%
\begin{subfigure}{0.475\textwidth}
  \centering
  \includegraphics[width=0.6\linewidth]{maxent/figures/LagrangeMult_lambda-4to0.png}
  \caption{$-4<\lambda_{3}<0$}
\end{subfigure}
\caption{Superficie de $r_{z}$ según (\ref{eq:RZ}) para dos intervalos de $\lambda_{3}$. A valores $\lambda_{3}<0$ corresponden valores $r_{z}>0$ y viceversa.}
\label{fig:rzsurf}
\end{figure}

Después de una breve inspección se concluyen las siguientes cosas:
\begin{itemize}
\item la superficie es simétrica respecto al plano $p=0.5$
\item la superficie es antisimétrica  respecto al plano $\lambda_{3}=0$ i.e. $r_{z
}(\lambda_{3},p)=-r_{z
}(-´\lambda_{3},p)$
\item $\text{sgn}(\lambda_{3})=-\text{sgn}(r_{z})$
\end{itemize}

La simetría respecto al plano $p=0.5$ suguiere un cambio de variable $q=\abs{p-0.5}$. La ecuación (\ref{eq:RZ}) se reescribe como:
\begin{equation}\label{eq:RZq}
r_{z}=-\frac{1}{2}\frac{\sinh(\lambda_{3})+2q\sinh(2q\lambda_{3})}{\cosh((q+\frac{1}{2})\lambda_{3})\cosh((q-\frac{1}{2})\lambda_{3})}
\end{equation}
Y nos limitamos al dominio $\lambda_{3}\leq0$ y $0\leq q\leq\frac{1}{2}$. Podemos graficar la función (\ref{eq:RZq}) para diferentes valores de $q$ (Figura \ref{fig:rzinv}).
\begin{figure}[h!]
\centering
\includegraphics[width=0.6\linewidth]{maxent/figures/rz_has_inverse_lambda-4to4.png}
\caption{$r_{z}$ como función de $\lambda_{3}$ para diferentes valores de $q$. La apariencia uno a uno sugiere la existencia de una inversa.}
\label{fig:rzinv}
\end{figure}

\subsubsection{Dos soluciones particulares}

Considerando el caso $q=\frac{1}{2}$, la ecuación (\ref{eq:RZq}) se reduce a 
\begin{equation}
r_z=-\frac{1}{2}\frac{2\sinh(\lambda_{3})}{\cosh(\lambda_{3})}
\end{equation}
de manera que $\lambda_{3}=-\text{arctanh}(rz)$.

Si $q=0$, la ecuación (\ref{eq:RZq}) se reduce a
\begin{equation}
r_z=-\frac{\sinh(\lambda_{3})}{\cosh(\lambda_{3}+1)}
\end{equation}
Mathematica sugiere la solución:
\begin{equation}\label{eq:lambda0.5}
\lambda_{3}=\log\qty(\frac{1-r_{z}}{1+r_{z}}).
\end{equation}
\newpage
\section{Construcción de la dinámica}\label{sec:ch2dycon}

Ahora que hemos establecido que usaremos como modelo de grano grueso uno que incluye tanto problemas de resolución como errores de permutación, y que hemos contruído nuestra aplicación de asignación a través del Principio de Máxima Entropía, podemos preguntarnos sobre la evolución del sistema efectivo, la ``dinámica gruesa'', denotada como $\Gamma_t$. La dinámica efectiva es una aplicación dinámica que corresponde a la evolución observada por un experimentalista. Dado un estado efectivo inicial $\rho_{\ef}(0)$,
\begin{gather}
\Gamma_{t}:\mcS(\hilbert_2)\rightarrow \mcS(\hilbert_2)\nonumber\\
\rho_{\ef}(0) \mapsto \Gamma_{t}(\rho_{\ef}(0))\rlap{.}\nonumber
\end{gather}
Debido que asumimos que el estado que se propaga debido a la evolución subyacente es justamente un estado compatible con $\rho_{\ef}$, seleccionado a través de una aplicación de asignación, a la dinámica gruesa la definimos como la composición
\begin{equation}\label{eq:EffectiveDynamics}
\Gamma_t:=\mcC \circ \mcV_t \circ \mcA_{\mcC}^{\max},
\end{equation}
como se ha hecho en trabajos similares \cite{CGEmergingDynamics}.

\acnote{Párrafo iterado: reescritura}

Donde $\mcV_{t}$ es la evolución seguida por el sistema microscópico. Esta puede ser unitaria, o un canal cuántico. El siguiente diagrama ilustra la ecuación anterior,
\[\begin{tikzcd}[arrows={<-|}]
    \rho_{\ef}(0)  & \rho_{\ef}(t) \arrow{l}{\Gamma_{t}} \arrow{d}{\mcC}\\
\varrho_{\max}(0) \arrow{u}{\mcA_{\mcC}^{\max}} & \varrho_{\max}(t). \arrow{l}{\mcV_{t}}
\end{tikzcd}
\]

\acnote{Párrafo iterado: reescritura}

Debe notarse que, debido a la no invertibilidad de la aplicación de grano grueso, en general
\begin{equation}
    (\mcU_{t}\circ\mcA_{\mcC}^{\max})(\rho_{\ef}) \neq (\mcA_{\mcC}^{\max}\circ\mcC \circ \mcU_t \circ \mcA_{\mcC}^{\max})(\rho_{\ef}),\nonumber
\end{equation}
que también puede escribirse como
\begin{equation}
    \varrho_{\max}(t)\neq\mcA_{\mcC}^{\max}(\rho_{\ef}(t))\nonumber
\end{equation}
Después de todo, la maximización de la entropía se restringe de acuerdo a las observaciones experimentales, así que estados de máxima entropía que cumplan un conjunto particular de restricciones no tienen por qué satisfacer un conjunto diferente de restricciones.



En el siguiente capítulo se analizarán dinámicas efectivas generadas por diferentes dinámicas subyacentes. Si se asume que el sistema conformado por las partículas es cerrado, entonces la evolución $\mcV_{t}$ será unitaria, generada por un hamiltoniano $H$. Algunos ejemplos de dinámicas subyacentes no unitarias son los canales de ruido usuales, como el canal de \textit{despolarización}, el canal de \textit{amortiguamiento de amplitud}, o el canal de \textit{amortiguamiento de fase}. \ddnote{puse italicas en algunas partes}\acnote{enterado}


\acnote{Párrafo iterado: notas}

A diferencia de los mapas dinámicos usualmente estudiados en teoría de sistemas cuánticos abiertos, la dinámica efectiva $\Gamma_{t}$ no tiene por qué ser lineal (sí debe, por supuesto, mandar estados cuánticos a estados cuánticos), debido a que uno de los elementos de la composición que la originan no siempre es lineal: la aplicación de asignación. El estudio de las particularidades de algunas de estas dinámicas efectivas es el foco de este trabajo.



\newpage
