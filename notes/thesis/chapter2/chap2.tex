\chapter{Modelos de la dinámica efectiva}

\section{A través del estado de máxima entropía}

\notaAd{Por la forma en la que se está desarrollando el trabajo, este capítulo puede que deba ser renombrado, y centrarse en los resultados y propiedades generales del estado de máxima entropía}

\subsection{Construcción del estado de máxima entropía}

Sea $\rho\in\mcS(\hilbert_{2})$ el estado grueso accesible al observador, y sea $\{A_{i}\}$ con $A_{i}\mcL(\hilbert_{2})$ un conjunto de observables tomográficamente completo. Si $\rho$ es un estado grueso correspondiente a un estado fino $\varrho\in\mcS(\hilbert_{2}\otimes\hilbert_{2})$, de acuerdo con lo discutido en la sección (\ref{sec:CH1MaxEnt}), podemos asignar a $\varrho$ un estado que maximice la entropía de Von Neumann sin agregar información externa, y que satisfaga las restricciones $\expval{A_{i}}=\Tr(\rho A_{i})$. ¿Pero cómo se traducen dichas restricciones en el nivel microscópico? Naturalmente, esto dependerá de la relación entre $\varrho$ y $\rho$. Esto es, el estado de máxima entropía depende de la aplicación de grano grueso.

Escójase ${A_{i}}={\sigma_{i}}$, las matrices de Pauli, y la aplicación de grano grueso como (\ref{eq_CG}). Los valores esperados de los operadores se traducen como las componentes del vector de Bloch del operador $\rho$. Las restricciones a las que se ve sujeto el operador $\varrho_{max}$ son
\begin{align*}
    r_{i}&=\Tr[\sigma_{i}\rho],\\
    &=\Tr[\sigma_{i}\Lambda(\varrho)],\\
    &=\Tr[\sigma_{i}\Tr_{2}(p\varrho+(1-p)S\varrho S^{\dag})],\\
    &=\Tr[\sigma_{i}\otimes\Id(p\varrho+(1-p)S\varrho S^{\dag})],\\
    &=\Tr[(p\sigma_{i}\otimes\Id+(1-p)\Id\otimes\sigma_{i})\varrho].
\end{align*}
Definiendo
\begin{equation}\label{eq:Ghat}
    \hat{G}_{i}=p\sigma_{i}\otimes\Id+(1-p)\Id\otimes\sigma_{i},
\end{equation}
las restricciones son
\begin{equation}\label{eq:MaxEntRestrictions}
    r_{i}=\Tr[\hat{G}_{i}\varrho]
\end{equation}
Dadas las restricciones, se utilizan multiplicadores de Lagrange para obtener el estado de maximiza la entropía. De acuerdo con la ecuación (\ref{eq:GeneralMaxEnt}), el estado de máxima entropía compatible con (\ref{eq:MaxEntRestrictions}) es
\begin{equation}
    \varrho_{\max}(\rho)=\frac{1}{Z}e^{-\sum_{i}\lambda_{i}\hat{G}_{i}}.
\end{equation}

\subsection{Propiedades del estado de máxima entropía}

\subsubsection{Relación unitaria}

Un operador unitario se traduce como una rotación en la esfera de Bloch. Entonces dos operadores están relacionados por una unitaria si sus vectores de Bloch tienen la misma norma euclidiana. Dicho de otra forma, dados $\rho_{1},\rho_{2}\in\mcS(\hilbert_{2})$, tales que $\text{Pu}(\rho_{1})=\text{Pu}(\rho_{2})$, entonces existe una matriz unitaria $V$ tal que
\begin{equation}
    \rho_{1}=V\rho_{2}V^{\dag}
\end{equation}
Así, si consideramos un estado sobre el eje z,

\begin{equation}\label{eq:rhoz}
\rho_{z}=\frac{1}{2}\qty(\Id+z\sigma_{z})
\end{equation}

entonces este está relacionado mediante una unitaria a cualquier estado $\rho\in\mcS(\hilbert_2)$ con vector de Bloch $\vec{r}$ tal que $\abs{\vec{r}}=z$.

Considérese un estado grueso $\rho\in\mcS(\hilbert_2)$. Si el observador puede realizar mediciones de $\sigma_{i}$ en el estado grueso, entonces es posible reconstruir un estado de máxima entropía $\varrho_{max}\in\mcS(\hilbert_2 \otimes \hilbert_2)$ según
\begin{equation}\label{eq:MaxEnt}
\varrho_{max}=\frac{1}{\Tr(e^{\sum_{i}\lambda_{i}\hat{G}_{i}})}e^{\sum_{i}\lambda_{i}\hat{G}_{i}}
\end{equation}
donde $\lambda_{i}$ son los multiplicadores de Lagrange y $\hat{G}_{i}$ son operadores no tomográficamente completos \cite{MaxEnt}. Se puede demostrar que los operadores $\hat{G}_{i}$ son
\begin{equation}\label{eq:Gop}
\hat{G}_{i}=p\sigma_{i}\otimes\Id+(1-p)\Id\otimes\sigma_{i}
\end{equation}

Idealmente, la ecuación (\ref{eq:MaxEnt}) está en términos de los valores de expectación $r_{i}=Tr(\sigma_{i}\rho_{c})$, y no de los multiplicadores de Lagrange. Para simplificar el problema, se puedeescoger un estado grueso $\rho_{z}$ como en (\ref{eq:rhoz}), de tal forma que el exponente en (\ref{eq:MaxEnt}) tenga únicamente un término.

\vspace{0.2cm}

En las siguientes líneas se demuestra que para todo $\rho_{c}$ con vector de Bloch $\vec{r}$ es posible reconstruir el estado de máxima entropía a través del estado de máxima entropía asociado a un estado grueso alineado en $z$, $\rho_{z}$ y la unitaria $U$ que relaciona $\rho_{c}$ y $\rho_{z}$ (para que esta exista se debe cumplir que $\abs{\vec{r}}=z$).

\vspace{0.2cm}

Sea, pues $\rho$ tal que $\rho=U\rho_{z}U^{\dag}$, con $\varrho_{max}$ el estado de máxima entropía construído para  $\mcU=U\otimes U$. El valor de expectación de las $\sigma_{i}$ puede calcularse como
\begin{align*}
    r_{i}&=\Tr\{\sigma_{i}U\rho_{z}U^{\dag}\}\\
    &=\Tr\{\sigma_{i}U\CG{\varrho_{max}^{z}}U^{\dag}\}\\
    &=\Tr\{\sigma_{i}\CG{\mcU\varrho_{max}^{z}\mcU^{\dag}}\}\\
    &=\Tr[\hat{G}_{i}\mcU\varrho_{max}^{z}\mcU^{\dag}]\\
    &=\Tr[\mcU^{\dag}\hat{G}_{i}\mcU\varrho_{max}^{z}]\\
    \end{align*}
De esto, se sigue que el estado de máxima entropía asociado a $\rho_{z}$ se puede reconstruir como:
\begin{align*}
\varrho_{max}^{z}&=\frac{1}{\Tr(e^{\sum_{i}\lambda_{i}\mcU^{\dag}\hat{G}_{i}\mcU})}e^{\sum_{i}\lambda_{i}\mcU^{\dag}\hat{G}_{i}\mcU}\\
&=\frac{1}{\Tr(e^{\mcU^{\dag}\qty(\sum_{i}\lambda_{i}\hat{G}_{i})\mcU^{\dag}})}e^{\mcU^{\dag}\qty(\sum_{i}\lambda_{i}\hat{G}_{i})\mcU^{\dag}}\\
&=\frac{1}{\Tr(\mcU^{\dag}e^{\sum_{i}\lambda_{i}\hat{G}_{i}}\mcU)}\mcU^{\dag}\qty(e^{\sum_{i}\lambda_{i}\hat{G}_{i}})\mcU\\
&=\frac{1}{\Tr(\mcU^{\dag}\qty(e^{\sum_{i}\lambda_{i}\hat{G}_{i}})\mcU)}\mcU^{\dag}\qty(e^{\sum_{i}\lambda_{i}\hat{G}_{i}})\mcU\\
&=\frac{1}{\Tr(e^{\sum_{i}\lambda_{i}\hat{G}_{i}})}\mcU^{\dag}\qty(e^{\sum_{i}\lambda_{i}\hat{G}_{i}})\mcU\\
&=\mcU^{\dag}\varrho_{max}\mcU
\end{align*}
Esto significa que si somos capaces de hallar el estado de máxima entropía asociado a un estado alineado en $z$, podemos hallar el asociado a cualquier otro estado mediante:
\begin{equation}
\varrho_{max}=\mcU\varrho_{max}^{z}\mcU^{\dag}
\end{equation}


\subsubsection{Dos expresiones del estado de máxima entropía}

A cada estado de dos niveles, $\rho$, lo definen tres parámetros: su pureza, $r$, y dos ángulos $\theta$, $\phi$. La unitaria que conecta $\rho$ con el estado alineado en $z$ de pureza $z=r$ es 
\begin{equation}
  V=
  \begin{pmatrix}
      \cos{\theta} & e^{-i\phi}\sin{\theta}\\
      e^{i\phi}\sin{\theta}& \cos{\theta}\\
  \end{pmatrix}
\end{equation}
Construyendo $\mcV=V\otimes V$, podemos expresar al estado de máxima entropía de dos formas equivalentes,
\begin{align}
  \varrho_{max}(\rho)=\frac{1}{Z}\text{exp}(-\lambda_{3}\mcV\hat{G}_{3}\mcV^{\dag}) && \varrho_{max}(\rho)=\frac{1}{Z}\text{exp}(\sum_{i}\lambda_{i}\hat{G}_{i}).
\end{align}

\subsubsection{Separabilidad del estado de máxima entropía}

Sea $\rho$ como en (\ref{eq:rhoz}), entonces por (\ref{eq:MaxEnt}) el estado de máxima entropía es:
\begin{equation}\label{eq:MaxEntUgly}
\varrho_{max}=\frac{\text{exp}(-\lambda_{3}\hat{G}_{3})}{\Tr[\text{exp}(-\lambda_{3}\hat{G}_{3})]}
\end{equation}
donde $\hat{G}_{3}$ se define según (\ref{eq:Gop}). Como los dos términos que componen al operador comuntan entre sí, la exponencial puede separarse:
\begin{align*}
\varrho_{max}&=\frac{1}{Z}e^{-\lambda_{3}p\sigma_{z}\otimes\Id}e^{-\lambda_{3}(1-p)\Id\otimes\sigma_{z}}\\
&=\frac{1}{Z}(e^{-\lambda_{3}p\sigma_{z}}\otimes\Id)( \Id\otimes e^{-\lambda_{3}(1-p)\sigma_{z}})\\
&=\frac{1}{Z}(e^{-\lambda_{3}p\sigma_{z}}\otimes e^{-\lambda_{3}(1-p)\sigma_{z}})\\
\end{align*}
Si se separa a la función de partición como un producto de trazas $Z=Z_{1}Z_{2}$, al estado de máxima entropía se le puede escribir como:
\begin{equation}\label{eq:MaxEntZ}
\varrho_{max}=\frac{e^{-\lambda_{3}p\sigma_{z}}}{Z_{1}} \otimes \frac{e^{-\lambda_{3}(1-p)\sigma_{z}}}{Z_{2}}
\end{equation}
Esto es válido para el estado alineado en $z$, 
\subsubsection{Generalización: N a M partículas}

Sea $\ket{\psi}\in\hilbert_{2}^{\otimes n}$ y $\varrho=\dyad{\psi}$. La aplicación de grano grueso que resuelve un qubit donde hay $n$ qubits se puede escribir como
\begin{equation*}
    \CG{\varrho}=\Tr^{i}(\Fuzzy{\varrho})=\Tr^{1}(\Fuzzy{\varrho})
\end{equation*}
sin pérdida de generalidad, y donde $\Tr^{i}$ denota la traza parcial sobre todos menos el $i$-ésimo qubit. La aplicación borrosa permuta el primer qubit con el $j$-ésimo qubit con probabilidad $p_{j}$.

\section{A través del mapeo de asignación promedio}

\section{Partiendo del estado microscópico inicial}
