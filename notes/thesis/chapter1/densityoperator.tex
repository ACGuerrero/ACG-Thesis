\section{El operador de densidad}

\subsection{Mezclas estadísticas}

En el contexto de la mecánica cuántica nos enfrentamos a dos tipos de probabilidades. La primera, la probabilidad cuántica, está codificada dentro de los vectores de estado que se utilizan para describir el estado en el que se puede hallar un sistema. Sin embargo, los vectores de estado no incluyen\ddnote{contemplan} el segundo tipo de probabilidad: la asociada a la ignorancia. Esta no es una probabilidad cuántica \ddnote{propongo añadir este paréntesis: (definida como el valor absoluto al cuadrado de una amplitud de probabilidad)}, sino una clásica. Por esto, y porque será particularmente útil para nuestro trabajo, introducimos el concepto del operador de densidad (también llamado, en el caso discreto, que es el que nos incumbe, matriz de densidad).

Supóngase que en lugar de estudiar un sistema que está completamente descrito por $\ket{\varphi}\in\hilbert_{n}$, con $\hilbert_{n}$ el espacio de Hilbert \ddnote{añadamos: de dimensión $n$, \ie{}} $\hilbert_{n}=\Complex^{n}$; se trabaja con uno que está en el estado $\ket{\varphi_{i}}$ con probabilidad $p_{i}$, donde $\{\ket{\varphi_{i}}\}_{i=1}^n$ \ddnote{añadí rango de la $i$} es un conjunto no necesariamente ortogonal de estados de $n$ niveles $\ket{\varphi_{i}}\in\hilbert_{n}$, y $\{p_{i}\}_{i=1}^n$ \ddnote{añadí rango de la $i$} es un conjunto de números reales no negativos tales que $\sum_{i=1}^n p_{i}=1$ \ddnote{añadí rango de la $i$}.

De este sistema se dice que se halla en un estado de \textit{mezcla estadística}, y no debe confundirse con que el sistema se halle en una superposición de estados $\ket{\varphi_{i}}$ con coeficientes $\sqrt{p_{i}}$, ya que una superposición está bien \ddnote*{determinado}{caracterizada}, y está completamente descrita por $\ket{\psi}=\sum_{i} e^{i \varphi_i} \sqrt{p_{i}}\ket{\varphi_{i}}$ \ddnote{añadí fases a la ecuación, por generalidad}, mientras que la mezcla no lo está: \ddnote*{... :parte del elemento probabilístico...}{el elemento probabilístico} está asociado a un grado de ignorancia sobre la preparación del sistema. La mezcla estadística, en este sentido, toma en cuenta no sólo la probabilidad intrínseca a cada estado cuántico, sino una probabilidad clásica, $p_{i}$. Consideremos ahora un observable descrita por un operador hermítico $A$, se sabe que el valor de expectación del observable, con respecto a un estado $\ket{\varphi_{i}}$, está dado por $\expval{A}=\bra{\varphi_{i}}A\ket{\varphi_{i}}$. El valor esperado de dicho observable con respecto a la mezcla estadística será, justamente, \ddnote*{promedio de los valores esperados respecto a los estados cuánticos en la mezcla}{combinación probabilística de los valores esperados respecto a los elementos de la mezcla}:
\begin{equation*}
\expval{A}=\sum_{i}p_{i}\bra{\varphi_{i}}A\ket{\varphi_{i}}.
\end{equation*}
Pues bien, esta expresión puede ser manipulada a través de una base ortogonal $\{\ket{e_{k}}\}$ del espacio $\hilbert_{n}$:
\begin{align*}
\expval{A}&=\sum_{i}p_{i}\bra{\varphi_{i}}A\ket{\varphi_{i}}\\
&=\sum_{i,j,k}p_{i}\bra{e_{k}}\ket{\varphi_{i}}\bra{\varphi_{i}}\ket{e_{j}} \bra{e_{j}}A\ket{e_{k}}\\
\end{align*}
Esta es una suma sobre los elementos de la matriz del observador $A$ y las de las matrices definidas por $\dyad{\varphi_{i}}$. Agrupando la suma sobre $i$ y tomando en cuenta completez de la base $\{\ket{e_{j}}\}$,
\begin{align*}
\expval{A}&=\sum_{j,k}\bra{e_{k}}\qty(\sum_{i}p_{i}\dyad{\varphi_{i}})\ket{e_{j}} \bra{e_{j}}A\ket{e_{k}}\\
&=\sum_{k}\bra{e_{k}}\qty(\sum_{i}p_{i}\dyad{\varphi_{i}})A\ket{e_{k}}\\
&=\Tr[\qty(\sum_{i}p_{i}\dyad{\varphi_{i}})A]\rlap{,}
\end{align*}
Con lo que la mezcla queda descrita por el operador de densidad $\rho$, definido según
\begin{equation}\label{eq:DensOpMix}
\rho=\sum_{i}p_{i}\dyad{\varphi_{i}}.
\end{equation}
Entonces podemos observar que es posible hallar el valor esperado de un observable respecto a un \ddnote*{estado}{sistema}, \ddnote*{utilizando el}{a través del} operador de densidad, \ddnote*{cuya expresión está dada por}{ que lo describe mediante}
\begin{equation}\label{eq:ExpValFromDensOp}
\expval{A}=\Tr(A\rho).
\end{equation}

El nombre ``operador de densidad'' puede resultar más claro comparando la ecuación (\ref{eq:ExpValFromDensOp}) con el valor esperado en estadística. Si $X$ es una variable aleatoria con dominio discreto $X=x_1,x_2,\dots$ y cuya distribución probabilidad es $p(x_k)$, entonces el valor esperado de una función $A$ de los valores de $X$ es
\begin{equation*}
E[A]=\sum_{k} A(x_k) p(x_k).
\end{equation*}
\ddnote{estandarizé un poco la notación en la ecuación y texto anteriores}
\ddnote*{Este texto está raro. Ciertamente hay una analogía, pero no es claro que va por aquí}{Reconociendo que la operación traza no es sino la suma sobre los elementos diagonales de la matriz, es posible ver que la matriz de densidad ocupa un rol similar al de la función de densidad.}
Ahora, es necesario ser capaz de distinguir si un operador cualquiera corresponde a un operador de la forma (\ref{eq:DensOpMix}). De esta definición destilan dos propiedades que permiten reconocer si un operador arbitrario es un operador de densidad válido, o no \cite{Holevo}:
\begin{enumerate}
    \item $\Tr(\rho)=1$
    \item $\rho\geq 0$,
\end{enumerate}
\ddnote{donde $\rho\geq 0$ se define como $\bra{\varphi}\rho\ket{\varphi}\geq 0$ $\forall$ $\ket{\varphi}\in\hilbert_{n}$}.
La primera propiedad se deriva de la normalización de los estados $\ket{\varphi_{i}}$ que definen al sistema. La segunda establece que $\rho$ es una matriz positiva semidefinida y puede interpretarse como la necesidad de que la probabilidad de que el sistema descrito por $\rho$ se halle en el estado $\ket{\varphi}$ sea mayor o igual a $0$. Un operador es un operador de densidad si \ddnote*{en las definiciones basta solo el ``si''}{y solo si} cumple con estas propiedades. Por esto, estos dos puntos funcionan como una definición alternativa al operador de densidad.

\acnote{En el siguiente párrafo hay tres oraciones idénticas que me toca reescribir.}

A partir de este momento se asume que todos los espacios de Hilbert con los que se trabaja son complejos y de dimension finita. Esto es, son todos del tipo $\hilbert_{n}=\Complex^{n}$. Al conjunto de todos los operadores lineales acotados que actúan sobre un espacio $\hilbert_{n}$ se le denotará como $\mcB(\hilbert_{n})$. Luego, al conjunto de operadores lineales Hermitianos se le denotará mediante $\mcL(\hilbert_{n})$. Finalmente, al conjunto de operadores de densidad se le denotará mediante $\mcS(\hilbert_{n})$. Como nos concentramos en espacios de dimension finita, los operadores tienen representación matricial. Los términos \textit{matriz de densidad} y \textit{operador de densidad} se consideran intercambiables.



\subsection{Pureza}

La diferencia entre una mezcla estadística y una superposición puede no ser del todo clara. ¿Cómo son diferentes un sistema que tiene una probabilidad $p_{i}$ de hallarse en el estado $\ket{\varphi_{i}}$ y otro que se halla en una superposición de cada estado $\ket{\varphi_{i}}$ con coeficientes $\sqrt{p_{i}}$? 

Para responder, considérense dos sistemas de dos niveles. El primero puede hallarse en cualquiera de los siguientes estados
\begin{align*}
    \ket{0}=\begin{pmatrix}
        1\\
        0
    \end{pmatrix} && \text{y} && \ket{1}=\begin{pmatrix}
        0\\
        1
    \end{pmatrix}\rlap{,}
\end{align*}
con la misma probabilidad $p=\frac{1}{2}$. Entonces el operador de densidad que describe al sistema es 
\begin{equation*}
    \rho=\frac{1}{2}(\dyad{0}+\dyad{1})=\frac{1}{2}\Id_{2}.
\end{equation*}
Por otro lado, el segundo sistema se halla en una superposición de los mismos estados, con coeficientes $\sqrt{p}$. El operador de densidad que describe al segundo sistema es 
\begin{align*}
    \dyad{\psi} && \text{con} && \ket{\psi}=\frac{1}{\sqrt{2}}(\ket{0}+\ket{1})\rlap{.}
\end{align*}
Es claro que $\ket{\psi}$ y $\rho$ no describen al mismo objeto, pues $\rho\neq\dyad{\psi}$. Si nos propusiéramos calcular la probabilidad de cada uno de hallarse en el estado $\ket{0}$ encontraríamos que
\begin{align*}
    \bra{0}\rho\ket{0}=\frac{1}{2} && \text{y} &&\langle 0 \dyad{\psi} 0\rangle=\frac{1}{2}\rlap{.}
\end{align*}
y el resultado es el mismo si se hiciera con el estado $\ket{1}$. Parecería entonces que los sistemas se hallan en el mismo estado. Esto es falso. Si realizamos un cambio de base, de $\{\ket{1},\ket{2}\}$ a $\{\ket{+},\ket{-}\}$, donde
\begin{align*}
    \ket{+}=\frac{1}{\sqrt{2}}\begin{pmatrix}
        1\\
        1
    \end{pmatrix} && \text{y} && \ket{-}=\frac{1}{\sqrt{2}}\begin{pmatrix}
        1\\
        -1
    \end{pmatrix}\rlap{,}
\end{align*}
y calculamos la probabilidad de que cada sistema se halle en el estado $\ket{+}$ encontraremos
\begin{align*}
    \bra{+}\rho\ket{+}=\frac{1}{2} && \text{pero} &&\langle + \dyad{\psi} +\rangle=1\rlap{.}
\end{align*}
El segundo resultado es de esperarse, pues $\dyad{\psi}$ se halla en el estado $\ket{+}$. Por otro lado, el sistema $\rho$ siempre tendrá una probabilidad $\frac{1}{2}$ de hallarse en cualquiera de los dos elementos de cualquier base ortogonal que escojamos. La diferencia entre ambos sistemas es que el elemento probabilístico asociado a las mediciones sobre $\dyad{\psi}$ es de naturaleza cuántica, y viene de que el sistema se halla en una superposición de estados ortogonales, mientras que en el caso de $\rho$, el elemento probabilístico se debe a nuestra ignorancia sobre la preparación del estado \cite{Chuang}. El hecho de que hallemos que $\rho$ siempre tenga una probabilidad $\frac{1}{2}$ de hallarse en alguno de los dos elementos de cualquier base ortogonal es una propiedad del estado máximamente mezclado, que puede verse como un estado de cuya preparación somos máximamente ignorantes.

\acnote{Observamos entonces que hay una diferencia fundamental entre los sistemas que pueden ser descritos por un vector de estado y aquellos que no. Considérese el caso en que, dada la expresión \ref{eq:DensOpMix}, el estado del sistema es $\dyad{\varphi_{1}}$ con  probabilidad $p_{1}=1$, i.e. $\rho=\dyad{\varphi_{1}}$. En tal caso decimos que $\rho$ es un estado puro. Claramente, $\rho^{2}=\rho$. Esto hace de $\rho$ un proyector, de lo que se sigue que $\Tr(\rho^{1})=1$. }

\acnote{Ini----------------}

\ddnote*{Observamos entonces que}{Vemos, pues, que} hay una diferencia fundamental entre los sistemas que pueden ser descritos por un vector de estado \ddnote*{este paréntesis está demás y puede crear confusión. Otra cosa, tienes corrector de ortografía en tu editor?, se han ido varios typos}{(para los que es posible contruir una matriz de densidad)}, y aquellos que no. \ddnote*{Esto se puede explicar de forma mucho mas simple, como diciendo: considere el caso en el que el sistema se encuentra en el estado $\ket{\varphi}$ con probabilidad igual a uno, \ie{} $\rho=\dyad{\varphi}$...}{Si para $\rho$ un operador de densidad,
\begin{equation*}
    \rho=\sum_{i}p_{i}\dyad{\varphi_{i}},
\end{equation*}
se cumple que $\rho=\dyad{\varphi_{i}}$ $\forall i$,} entonces decimos que $\rho$ es un estado puro, y está completamente caracterizado por el vector de estado $\ket{\varphi}=\ket{\varphi_{i}}$. \ddnote*{Esto está escrito raro, mencionas la palabra proyector sin definirlo, pero despues defines $\rho$ usando la definición de proyector. Por fas dale una iterada}{Así, los estados puros (aquellos que están descritos por un vector de estado, i.e. su operador de densidad es un proyector) son los puntos extremos del conjunto convexo de operadores de densidad. Estos estados cumplen que
\begin{itemize}
    \item $\rho=\dyad{\psi}$ para algún vector de estado $\ket{\psi}$.
    \item $\rho=\rho^{2}$.
    \item $\Tr(\rho^{2})=1$.
\end{itemize}}
\ddnote{Cambiado $n$ por 2}
\acnote{Fin----------------}

Como en general se cumple que $\Tr(\rho^{2})\leq 1$, definimos a la pureza como una medida de que tan puro es un estado como \cite{Jaeger}
\begin{equation*}
    \text{Pu}(\rho)=\Tr(\rho^{2}).
\end{equation*}
De esta definición es posible afirmar que
\begin{itemize}
    \item Un estado es puro si y sólo si $\text{Pu}(\rho)=1$.
    \item Para todo estado, $\frac{1}{n}\leq \text{Pu}(\rho)\leq 1$.
\end{itemize}

\subsection{Sistemas multipartitos}\label{sec:Ch1PartialTrace}
\ddnote{Sección pendiente por revisar}
Hasta ahora hemos hablado de sistemas descritos por operadores de densidad en $\densityspace{n}$, pero, ¿qué sucede si el sistema que estudiamos está conformado por dos subsistemas, cada uno descrito a través de sus respectivos espacios de Hilbert? Sean, pues, $A$ y $B$ dos sistemas con espacios de Hilbert $\hilbert^{A}$ y $\hilbert^{B}$, y sea $C$ un sistema compuesto por $A$ y $B$. Entonces el producto tensorial de los espacios $\hilbert^{A}$ y $\hilbert^{B}$ es otro espacio de Hilbert, uno asociado al sistema $C$:
 \begin{equation*}
     \hilbert^{C}=\hilbert^{A}\otimes\hilbert^{B}.
 \end{equation*}
 La dimensión del espacio de Hilbert del sistema multipartito cumple
\begin{equation*}
    \text{dim}(\hilbert^{C})=\text{dim}(\hilbert^{A})\text{dim}(\hilbert^{B}).
\end{equation*}
Si $A$ y $B$ representaran dos partículas diferentes, entonces $C$ representa a las partículas como conjunto, como sistema de dos partículas. Si cada una de las partículas puede ser descrita mediante un vector de estado, el estado del sistema es simplemente el producto tensorial de dichos vectores de estado:
\begin{equation*}
    \ket{\psi^{A}}\otimes\ket{\psi^{B}}\in\hilbert^{C}\,\; \; \forall\ket{\psi^{A}}\in\hilbert^{A},\ket{\psi^{B}}\in\hilbert^{B}.
\end{equation*}
Si un estado puede escribirse como un producto tensorial de estados pertenecientes a los subsistemas del sistema multipartito, entonces se dice que es un estado \textit{producto} o \textit{separable}. Nótese que, en general, los estados del sistema compuesto no son estados separables. En realidad, dadas $\{\varphi_{i}^{A}\}$ y $\{\varphi_{j}^{B}\}$ bases ortonormales de los espacios $\hilbert^{A}$ y $\hilbert^{B}$ respectivamente, podemos escribir a todo estado puro $\ket{\psi^{C}}$ del sistema multipartito como
\begin{equation*}
    \ket{\psi^{AB}}=\sum_{j,k}\alpha_{j,k}\ket{\varphi_{j}^{A}}\otimes\ket{\varphi_{k}^{B}}.
\end{equation*}
El significado físico de que un sistema se halle en un estado producto es que el sistema se halla en un estado en el que no hay correlaciones entre sus subsistemas (de esto que puedan separarse). Un estado que no puede separarse tiene cierto grado de entrelazamiento, y por esto deja de tener sentido hablar de vectores de estado individuales a cada partícula. Ahora, sean $G^{A}$ y $G^{B}$ dos operadores que actúan en $\hilbert^{A}$ y $\hilbert^{B}$ respectivamente, correspondientes a observables de cada subsistema. Entonces se cumple:
\begin{equation*}
    G^{A}\ket{\psi^{A}}\otimes G^{B}\ket{\psi^{B}}=(G^{A}\otimes G^{B})\ket{\psi^{A}}\otimes\ket{\psi^{B}}.
\end{equation*}

¿Qué sucede si al científico no le interesa sino uno de los subsistemas? Es en este caso en el que surge el concepto de la matriz de densidad reducida. Si $\rho^{C}$ es la matriz de densidad del sistema compuesto por $A$ y $B$, entonces la matriz de densidad reducida del sistema $A$ es
\begin{equation*}
    \rho^{A}=\Tr_{B}(\rho^{C}),
\end{equation*}
donde $\Tr_{B}$ representa la operación de traza parcial con respecto al subsistema $B$. Si la traza de $\rho^{C}$ es 
\begin{equation*}
    \Tr(\rho^{C})=\sum_{j}\bra{\varphi^{C}_{j}}\rho^{C}\ket{\varphi^{C}_{j}},
\end{equation*}
para toda base ortonormal $\{\ket{\varphi^{C}_{j}}\}$ de $\hilbert^{C}$. Entonces, para toda base ortonormal $\{\ket{\varphi^{B}_{j}}\}$ de $\hilbert^{B}$  la traza parcial respecto a $B$ es \cite{Hardy}
\begin{equation*}
    \Tr_{B}(\rho^{C})=\sum_{j}(\Id^{A}\otimes \bra{\varphi^{B}_{j}})\rho^{C}(\Id^{A}\otimes \ket{\varphi^{B}_{j}}).
\end{equation*}
Puede verse que el resultado de la operación es trazar sobre los elementos del sistema que no es de interés. La matriz reducida del sistema $A$, o traza parcial con respecto al sistema $B$, actúa como matriz de densidad de $A$, ya que contiene toda la descripción estadística de dicho subsistema.

\subsection{Evolución y parametrización}


\acnote{----Sección: Evolución de sistemas cerrados}

\acnote{
La evolución de un sistema cuántico cerrado descrito por un vector de estado está dada por la ecuación de Schrödinger,
\begin{equation*}
    i\hbar\frac{d}{dt}\ket{\psi(t)}=H\ket{\psi(t)},
\end{equation*}
cuya solución formal está dada en términos de un operador unitario $U(t,t_{0})$ según
\begin{align*}
    \ket{\psi(t)}=U(t,t_{0})\ket{\psi(t_{0})} && \text{con} && U(t,t_{0})=e^{-iH(t-t_{0})/\hbar}\rlap{.}
\end{align*}
Pues bien, los postulados de la mecánica cuántica pueden adaptarse al formalismo de operadores de densidad. De la ecuación de Schrödinger se sigue que la evolución de un sistema descrito por un operador de densidad $\rho$ está descrita por ecuación de Liouville-von Neumann,
\begin{equation*}
    i\hbar\frac{d}{d t} \rho(t)=[H,\rho(t)].
\end{equation*}
De la misma forma que antes, la solución queda expresada en términos de un operador unitario,
\begin{equation*}
    \rho(t)=U(t,t_{0})\rho(t_{0})U^{\dagger}(t,t_{0}).
\end{equation*}
}

\acnote{----Sección: evolución de sistemas abiertos}

\acnote{Considérese, en cambio, que el sistema de interés $\rho_{S}$ se halla acoplado a un entorno $\rho_{E}$ y que, inicialmente, el conjunto de estos dos conforma un sistema cerrado descrito por el operador de densidad $\rho(0)=\rho_{S}(0)\otimes\rho_{E}$. Esta condición inicial está justificada experimentalmente: una medición proyectiva sobre el sistema de interés proyecta al sistema a un estado factorizable, así que estos estados son siempre preparables. Como el sistema es cerrado, este evoluciona siguiendo la ecuación de Liouville-von Neumann,
\begin{align*}
    i\hbar\frac{d}{d t} \rho(t)=[H,\rho(t)] && \text{con} && \rho(0)=\rho_{S}(0)\otimes\rho_{E}\rlap{,}
\end{align*}
cuya solución formal es 
\begin{equation*}
    \rho_{S}(t)=\mcE_{t}(\rho(0)),
\end{equation*}
donde $\mcE_{t}$ es un canal cuántico para cualquier $t$, y $\mcE_{0}=Id$. El formalismo cuántico queda fuera del alcance de este trabajo, y sin entrar en más detalle, basta con señalar que los canales cuánticos son aplicaciones lineales de $\mcB(\hilbert_{n})$ en $\mcB(\hilbert_{m})$ que cumplen que  [LISTA PROPIEDADES DE CANALES].
Es particularmente interesante señalar que dado un canal cuántico, por el teorema de Stinespring...}

\subsubsection{La evolución del operador de densidad}

Los postulados de la mecánica cuántica pueden adaptarse al formalismo de operadores de densidad. En particular, reconociendo que la evolución de un sistema cuántico cerrado descrito por un vector de estado $\ket{\psi}$\ddnote{, está dada} por la ecuación de Schrodinger  \ddnote{Aquí no es tan necesario citar, ya es como citar a Newton cada vez que escribes una derivada. Si citas a alguien para cosas más trascendentes, mejor cita al autor del paper orginal} \acnote{cita removida},
\begin{equation*}
    i\hbar\frac{d}{dt}\ket{\psi(t)}=H\ket{\psi(t)},
\end{equation*}
\ddnote*{y cuya solución formal está dada en terminos de un operador unitario}{puede ser representada a través de un operador unitario} $U(t,t_{0})$ según
\begin{align*}
    \ket{\psi(t)}=U(t,t_{0})\ket{\psi(t_{0})} && \text{con} && U(t,t_{0})=e^{-iH(t-t_{0})/\hbar}\rlap{,}
\end{align*}
\ddnote*{Evita este tipo de expresiones, en la física matemática las cosas son o no son. Propongo algo como: Es directo probar,  a partir de la ecuación de Schrödinger, que el operador de densidad evoluciona de acuerdo a la siguiente expresión (en la que aparece el Hamiltoniano), cuya solución formal es (luego pones la expresión en términos de la unitaria)}{es posible afirmar que dado un operador de densidad $\rho(t_{0})$, este evoluciona de acuerdo a
\begin{equation*}
    \rho(t)=U(t,t_{0})\rho(t_{0})U^{\dagger}(t,t_{0}).
\end{equation*}
Derivando respecto al tiempo, se obtiene la \textit{ecuación de Liouville-von Neumann}, que corresponde a la ecuación de evolución para operadores de densidad,
\begin{equation}\label{eq:vonNeumann}
    i\hbar\frac{d}{d t} \rho(t)=[H,\rho(t)]
\end{equation}
}
Si, por otro lado, se piensa que el sistema estudiado se halla acoplado a un entorno (de tal forma que el conjunto sea un sistema \ddnote*{bipartito}{multipartito}), y que el conjunto de estos dos conforma un sistema cerrado descrito por el operador de densidad $\rho$, \ddnote*{evita el ``es posible afirmar'', me parece una expresión débil para conceptos bien determinados. Propongo: es decir, $\rho$ evoluciona de forma unitaria [cita la ecuación pertinente]}{entonces es posible afirmar que la evolución conjunta es de naturaleza unitaria.} De acuerdo con nuestra discusión sobre sistemas \ddnote*{biparititos}{multipartitos}, el sistema y el entrono se ven desritos por operadores de densidad
\begin{align*}
    \rho_{S}=\Tr_{E}(\rho) & & \text{y} & & \rho_{E}=\Tr_{S}(\rho)
\end{align*}
respectivamente. Siendo $\rho_{S}$ el estado de interés, podemos trazar ambos lados de la ecuación de Liouville-von Neumann para hallar una ecuación \ddnote*{Esta discusión tiene varios problemas importantes, lo hablamos en la reunión}{de la dinámica del sistema,
\begin{align*}
    i\hbar\frac{d}{d t} \rho_{S}(t)=\Tr_{S}([H,\rho(t)])
\end{align*}
en donde el claro problema es que el lado derecho de la ecuación, que está en términos de el conjunto y no del sistema de interés. La dinámica del sistema $S$ no será, en general, unitaria, y las diferentes aproximaciones a la dinámica de sistemas abiertos, entre las que se hallan las ecuaciones maestras, quedan por fuera del alcance de este trabajo.}


\subsubsection{Parametrización del operador de densidad}
\acnote{Esta parte aún la tengo que iterar}

\acnote{Es común escoger alguna base Hermítica para poder parametrizar a las matrices de densidad. El beneficio de hacer esto es que, por ser $\mcS(\hilbert_{n})$ un subconjunto de $\mcL(\hilbert_{n})$, dicha parametrización será lineal, y aún más: los parámetros serán cantidades medibles en el laboratorio. Esto significa que, realizando mediciones de forma adecuada, es posible reconstruir el estado de un sistema [cita]. Una elección común de base son las matrices generalizadas de Gell-Mann, junto a la matriz identidad.}

\acnote{Remplaza:-------}
\ddnote*{la base es hermitiana, el espacio no, lo hablamos en la reunión}{Cualquier matriz de densidad puede descomponerse en términos de una base del espacio de matrices hermitianas de $n\times n$.} \ddnote*{Tengo entendido que la identidad es también un generador de SU(2), checa bien esto}{Una elección común de base para el espacio es el de los generadores $\{\varsigma_{k}\}$ del grupo $\text{SU}(n)$, junto a la matriz identidad $\Id_{n}$}. Esto es particularmente útil, \ddnote*{ya que}{pues }permite parametrizar a las matrices de densidad de forma vectorial \cite{Bruning}. 
\acnote{---------fin}


\acnote{En efecto, sea $\{\varsigma_{k}\}$ el conjunto de matrices generalizadas de Gell-Mann que generan a $\text{SU}(n)$ y}
%En efecto, sea $\{\varsigma_{k}\}$ un conjunto de generadores de $\text{SU}(n)$ y 
$\rho$ una matriz de densidad $\rho\in\mcS(\hilbert_{n})$. Entonces $\rho$ está completamente descrita por el vector generalizado de Bloch de dimensión $2n^{2}-1$, $\vec{\gamma}$ definido según
\begin{equation}
    \rho=\frac{1}{n}\Id_{n}+\frac{1}{2}\vec{\gamma}\cdot\vec{\varsigma}.
\end{equation}
Si $n=2$, los generadores corresponden a las matrices de Pauli $\sigma_{i}$. En tal caso, el conjunto de vectores de Bloch corresponde a la bola unitaria tridimensional, con los estados puros en la superficie y las mezclas en el interior. Para casos en los que la dimensión es una potencia de $k$, es posible obtener nuevos generadores a través de los productos tensoriales de las matrices de Pauli consigo mismas y con la matriz identidad correspondiente. El caso $n=4$, por ejemplo \cite{Chuang}:
\begin{equation}\label{eq::BlochParametrization4}
    \rho=\frac{1}{4}\sum_{i,j}\gamma_{ij}\sigma_{i}\otimes \sigma_{j} \ \ i,j\in\{0,1,2,3\},
\end{equation}
donde $\sigma_{0}=\Id$ y 

\acnote{
\begin{equation*}
        \gamma_{i.j}=\Tr(\sigma_{i}\otimes \sigma_{j}\rho).
\end{equation*}
Obsérvese que, como se mencionó previamente, debido a que los parámetros $\gamma_{i.j}$ son cantidades medibles, esto junto a la ecuación [2.4] (meter referencia bién) permite reconstruir el estado del sistema.}\ddnote{...son promedios de las observables $\sigma_i \otimes \sigma_j$... obsérvese que esto junto a la ecuación [tal], permite reconstruir el estado cuántico $\rho$, a esto se le llama \textit{tomografía cuántica} etc.}
