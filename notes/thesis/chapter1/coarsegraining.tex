\section{Modelos de grano grueso}\label{sec:Ch1CG}

\subsection{Descripciones efectivas}

En nuestro día a día no podemos procesar toda la información de todos los sistemas con los que interactuamos: no nos preocupa la energía cinética individual de cada una de las $10^{21}$\ddnote{Cita de donde sacaste este dato, por fis} moléculas de agua presente en cada una de las gotas que caen sobre nosotros, sino de qué tan caliente o frío parece el chorro que sale de la llave. Una descripción de \textit{grano grueso} es aquella que no toma en cuenta todas los detalles de un sistema o fenómeno. \ddnote*{esto se lee raro, en realidad, a voluntad, no hay mucho que uno pueda hacer o no hacer, propongo que reformules}{Dichas particularidades microscópicas pueden omitirse por voluntad del observador (puede que no le sea útil toda la información del sistema, o que la cantidad de información sea demasiado grande como para manjearla) o por simple ignorancia de una porción de los grados de libertad del sistema.}

\begin{figure}[ht]
    \centering
    \includegraphics[width=0.6\linewidth]{chapter1/figures/CGT.png}
    \caption{Reducir las energías cinéticas individuales de $N$ partículas a un temperatura es un modelo de grano grueso}
    \label{fig:KtoT}
\end{figure}


\ddnote*{aguas aquí, me preocupa que puedas estar intercalando termodinámica y física estadística, no son lo mismo. En termodinámica se tratan sistemas termodinámicos, el sistema no tiene mayores detalles que su descripción termodinámica. La termodinámica es independiente de cualquier modelo microscópico del que se tenga conocimiento en otra teoría. Mejor di algo como que la física estadística se puede ver como una herramienta para producir sistemas mas simples a través de hacer coarse graining a sistemas complicados, de tal manera que las descripciones exactas son sustituidas por distribuciones de probabilidad, y de que ciertas cantidades como la presión y temperatura se pueden ver entonces como promedios de observables microscópicas}{La termodinámica es un área de la física que trata casi exclusivamente con modelos de grano grueso.} Las cantidades termodinámicas: temperatura, presión, volumen, no son sino el resultado de una descripción gruesa de sistemas extremadamente complejos, pues promedian las interacciones y propiedades de $10^{23}$ partículas. En particular, la temperatura de un gas ideal se relaciona con la energía cinética de las partículas de este, a través de la distribución de Boltzmann explícitamente a través de un promedio según
\begin{equation}
    T=\frac{1}{k_\text{B}}\frac{2}{3}\expval{E_{\text{cin}}}.
\end{equation}

Cuando se habla de modelos de grano grueso, no se suele hacer referencia a las descripciones efectivas inducidas por la ignorancia. \ddnote*{Aguas aquí, un modelo no se impone, un modelo es un modelo, y ya, no se impone a la realidad ni al experimento}{En realidad, cuando se habla de modelos de grano grueso se hace referencia a un modelo impuesto por el observador sobre el sistema.} Los modelos de grano grueso que buscan simplificar un problema desechando información poco útil son comunes en \ddnote*{física y química o física-química?}{física-química}\cite{PhysChemI,PhysChemII,PhysChemIII}. \ddnote*{Esto está raro, es como arriba que mencionas ``a voluntad'', que significa imponer para un científico y que se diferencia eso con capacidad o incapacidad}{El tipo de modelo de grano grueso en el que se centra este trabajo no es el impuesto por el científico, sino el que proviene de su incapacidad de acceder a toda la información del sistema.}

\ddnote*{Las variables termodinámicas}{La descripción termodinámica} de un sistema de $10^{23}$ partículas corresponde justamente a un modelo de grano grueso inducido por una ignorancia sobre los grados de libertad del sistema. Aún así, aunque el observador no cuente con acceso a dicha información, puede deducir que su descripción es meramente efectiva. \ddnote*{Aguas aquí, recuerda que hay un zoológico de entropías, esto no es cierto. Lo que impone esta relación es la ecuación de Boltzmann.}{En efecto, la entropía de un sistema termodinámico es una cantidad que relaciona las coordenadas gruesas con la realidad microscópica.}

\subsection{Grano grueso en mecánica cuántica}

En el contexto de la mecánica cuántica, un modelo de grano grueso puede obtenerse trazando sobre un subsistema del sistema de interés. Es importante notar que el subsistema ignorado no es necesariamente una parte que puede ser separada del sistema, como en el caso de dos partículas, \ddnote*{sino que puede representar otros grados de libertad del sistema que se decide ignorar, por ejemplo, al describir un ión atrapado en una trampa magnetoóptica, una opción es ignorar los grados de libertad espaciales y quedarse solo con el espín total del ión [cita]}{sino que puede representar un conjunto de información intrínseca al sistema, pero que se ha decidido ignorar. Por ejemplo, puede que se tome en cuenta el momento angular orbital de una partícula, pero no su espín.} Ahora, la separación separación sistema - entorno no es siempre posible \cite{Macro-To-Micro}, por lo que nos limitamos a los casos en los que los grados de libertad ignorados pueden trazarse a través de la operación de traza parcial, como se discutió en la sección \ref{sec:Ch1PartialTrace}. \ddnote*{entonces no es de cualquier manera...}{De cualquier manera, al subsistema desechado se le puede llamar \textit{entorno}, y aunque la separación separación sistema - entorno no es siempre posible \cite{Macro-To-Micro}, nos limitamos a los casos en los que los grados de libertad ignorados pueden trazarse a través de la operación de traza parcial, como se discutió en la sección \ref{sec:Ch1PartialTrace}.}

Un ejemplo sencillo de un modelo de grano grueso es el de un sistema de dos partículas, del cual únicamente nos importa una. En dicho caso, el modelo puede consistir en estudiar únicamente al operador de densidad reducido correspondiente a la partícula de nuestro interés. De forma general en el contexto de este trabajo, \ddnote*{De forma general en el contexto de este trabajo,}{De forma fundamental,} el modelo de grano grueso consiste en reducir la dimensión del sistema estudiado. En nuestro contexto, una aplicación de grano grueso $\Lambda$ es tal que manda a un operador de densidad $\varrho\in\densityspace{n}$ a algún otro operador de densidad $\rho\in\densityspace{m}$ con $m<n$. Esto es:
\begin{align}
    \mcF:\densityspace{n}\to \densityspace{m}, n>m\nonumber
\end{align}