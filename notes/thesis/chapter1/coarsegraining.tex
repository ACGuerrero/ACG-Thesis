\section{Modelos de grano grueso}\label{sec:Ch1CG}

\subsection{Descripciones gruesas en física}

Una descripción de \textit{grano grueso} es aquella que no toma en cuenta todas los detalles de un sistema o fenómeno. Nuestra interacción del día a día con el mundo que nos rodea es fundamentalmente gruesa: al bañarnos, no nos preocupa la energía cinética individial de cada una de las $10^{23}$ moléculas de agua presente en cada una de las gotas que caen sobre nosotros, sino de qué tan caliente, o frío parece el chorro que sale de la llave. Una descripción de grano grueso puede omitir dichos detalles microscópicos por voluntad del observador (puede que no le sea útil toda la información del sistema, o que la cantidad de información sea demasiado grande como para manjearla) o por simple ignorancia de la información omitida.

La termodinámica es un área de la física que trata casi exclusivamente con modelos de grano grueso. Las cantidades termodinámicas: temperatura, presión, volumen, no son sino el resultado de una descripción gruesa de sistemas extremadamente complejos, pues promedian las interacciones y propiedades de $10^{23}$ partículas. La descripción de todo el sistema se reduce a un puñado de coordenadas gruesas.

Cuando se habla de modelos de grano grueso, no se suele hacer referencia a las descripciones efectivas inducidas por la ignorancia. En realidad, cuando se habla de modelos de grano grueso se hace referencia a un modelo impuesto por el observador sobre el sistema. Los modelos de grano grueso que buscan simplificar un problema deshechando información poco útil son comunes en física química. [REFERENCIAS?] El tipo de modelo de grano grueso en el que se centra estre trabajo no es el impuesto por el científico, sino el que proviene de su incapacidad de acceder a toda la información del sistema.

La descripción termodinámica de un sistema de $10^{23}$ partículas corresponde justamente a un modelo de grano grueso inducido por una ignorancia sobre los grados de libertad del sistema. Aún así, auque el observador no cuente con acceso a dicha información, puede deducir que su descripción es meramente efectiva. n efecto, la entropía de un sistema termodinámico es una cantidad que relaciona las coordenadas gruesas con la realidad microscópica.

\subsection{Grano grueso en mecánica cuántica}

En el contexto de la mecánica cuántica, un modelo de grano grueso se obtiene trazando sobre un subsistema del sistema de interés. Al subsistema deshechado se le puede llamar \textit{entorno}, y aunque la separación separación sistema - entrono no es siempre posible \cite{Macro-To-Micro}, nos limitamos a los casos en los que los grados de libertad ignorados pueden trazarse a través de la operación de traza parcial usual.

Un ejemplo sencillo de un modelo de grano grueso es el de un sistema de dos partículas, del cual únicamente nos importa una. En dicho caso, el modelo puede consistir en estudiar únicamente al operador de densidad reducido correspondiente a la partícula de nuestro interés. Es importante notar que el subsistema ignorado no es necesariamente una parte que puede ser separada del sistema, como en el caso de las dos partículas, sino que puede representar un conjunto de información intrínseca al sistema, pero que se ha decidido ignorar. Por ejemplo, puede que se tome en cuenta el momento angular orbital de una partícula, pero no su espín.

Matemáticamente, el modelo de grano grueso corresponde a separar un espacio $\hilbert^{C}$ en dos espacios $\hilbert^{A}$ y $\hilbert^{B}$...


\subsection{Canales cuánticos}

