\section{Dinámicas separables}

En la sección \ref{sec:Ch1PartialTrace} se habló de estados separables como aquellos estados que, descritos por un operador de densidad $\rho\in\densityspace{n}$, tienen la forma
\begin{equation*}
    \rho=\rho_{A}\otimes\rho_{B}
\end{equation*}
donde $\rho_{A}\in\densityspace{m}$, $\rho_{B}\in\densityspace{l}$ y $l+m=n$. Siguiendo esta línea de pensamiento, con \textit{dinámicas separables} nos referimos a dinámicas unitarias descritas por operadores $U\in\unitaryspace{n}$ que pueden reescribirse como
\begin{equation*}
    U=U_{A}\otimes U_{B}
\end{equation*}
donde, una vez más, $U_{A}\in\unitaryspace{m}$, $U_{B}\in\unitaryspace{l}$ y $l+m=n$. Los operadores separables están compuestos por operadores que actúan de forma independiente sobre diferentes subsistemas del sistema en custión. En el caso de un sistema compuesto por dos subsistemas de dos niveles, el operador separable está compuesto por dos unitarias que actúan sobre $\hilbert_{2}$. Como el estado de máxima entropía resulta ser separable, las dinámicas separables son una muy buena primera forma de aplicar el formalismo descrito en las secciones anteriores.

\subsection{Caso general}

Consideramos una unitaria $\mcU=U_{1}\otimes U_{2}$ que evoluciona en el tiempo como $\mcU_{t}=(U_{1}\otimes U_{2})^{t}=U_{1}^{t}\otimes U_{2}^{t}$. Retomando la ecuación (\ref{eq:MaxEntSeparable}), la evolución del estado de máxima entropía es simplemente
\begin{equation*}
    \varrho_{\max}(t)=U_{1}^{t}\rho_{A}(U_{1}^{t})^{\dag}\otimes U_{2}^{t}\rho_{B} (U_{2}^{t})^{\dag}.
\end{equation*}
De esto, el estado efectivo evolucionado obtenido del principio de máxima entropía, en términos de los multiplicadores de Lagrange es
\begin{equation*}
    \rho(t)=pU_{1}^{t}\rho_{A}(U_{1}^{t})^{\dag}+(1-p)U_{2}^{t}\rho_{B} (U_{2}^{t})^{\dag}
\end{equation*}

Por supuesto, esta expresión puede expandirse en términos de exponenciales o de funciones hiperbólicas del vector de Bloch del estado efecivo incial, $\vec{r}_{\rho}$. Si se hace esto para hallar la evolución de un observable $\pauli{i}\in\obspace{2}$ se encuentra que
\begin{equation*}
    \expval{\pauli{i}(t)}=\frac{p\tanh(p\lambda)}{2}\Tr[\pauli{i}U_{1}^{t}(\paulivec{r_{\rho}})(U_{1}^{t})^{\dag}]+\frac{p\tanh((1-p)\lambda)}{2}\Tr[\pauli{i}U_{2}^{t}(\paulivec{r_{\rho}})(U_{2}^{t})^{\dag}]
\end{equation*}
En la evolución de los observables (que, depués de todo, son las cantidades que permiten describir al sistema efectivo), se observan, de nueva cuenta, dos términos: el primero está asociado a la evolución de grano grueso sin error. Como el modelo toma en cuenta únicamente a la primera partícula, se espera observar únicamente la acción de la primera parte del operador de evolución separable. El primer término contiene un coeficiente de peso $p\tanh(p\lambda)$ inducido por la aplicación de grano grueso, y el elemento de valor esperado, que depende únicamente de la dirección del vector de Bloch del estado efectivo inicial y de la primera parte del operador de evolución. En contraste, el segundo término contiene la evolución generada por $U_{2}^{t}$, y depende de $(1-p)$, la probabilidad de error. Por esto, este es el término de ruido. Por la naturaleza separable y unitaria de la evolución, se verá que el ruido son oscilaciones periódicas, pero esto es más claro si se toman en cuenta ejemplos particulares.

\subsection{Dinámica simétrica}

Comenzamos con el caso en el que la dinámica separable simétrica, esto es, de una unitaria $mcU\in\text{U}(4)$ de la forma
\begin{equation*}
    \mcU_{t}=(U \otimes U)^{t}
\end{equation*}
donde $U\in\text{U}(2)$. Se realiza el mismo proceso: aplicamos la evolución al estado de máxima entropía compatible con un conjunto de observables tomográficamente completos en $\hilbert_{2}$ y propagamos al estado con la uitaria subytacente, para luego pasarlo por la aplicación de grano grueso y recuperar el estado efectivo evolucionado. El caso de la dinámica separable es quizá el caso más sencillo, pues la simetría de la unitaria permite factorizarla:
\begin{align*}
\CG{(U^{t}\otimes U^{t})\varrho_{max}(U^{t}\otimes U^{t})^{\dag}}&=p\frac{1}{Z_{1}}e^{\lambda p U^{t}\sigma_{z}(U^t)^{\dag}}+(1-p)\frac{1}{Z_{2}}e^{\lambda (1-p)U^{t}\sigma_{z}(U^t)^{\dag}}\\
&=p\frac{1}{Z_{1}}U^{t}e^{\lambda p \sigma_{z}}(U^t)^{\dag}+(1-p)\frac{1}{Z_{2}}U^{t}e^{\lambda (1-p)\sigma_{z}}(U^t)^{\dag}\\
&=U^{t}\qty(p\frac{1}{Z_{1}}e^{\lambda p \sigma_{z}}+(1-p)\frac{1}{Z_{2}}e^{\lambda (1-p)\sigma_{z}})(U^t)^{\dag}\\
\end{align*}
La dinámica efectiva tiene la forma:
\begin{equation}
    \rho\xrightarrow{U\otimes U}U\rho U^{\dagger}
\end{equation}
Así como demostramos previamente que si el estado efectivo es puro, entonces el único estado de máxima entropía compatible es justamente el producto tensorial del estado efectivo consigo mismo, podemos mostrar que si la evolución efectiva es unitaria, la dinámica subyacente es el producto tensorial de dicha dinámica consigo misma:

\subsection{Identidad de un lado}

Retomando a expresión (\ref{eq:SeparableDynamics}), y en virtud de (\ref{eq:PauliVectorExp}), vemos que el estado efectivo inicial $\rho$ puede verse como una combinación de dos operadores con vector de Bloch con dirección $\hat{r}_{\rho}$. El vector de Bloch de $\rho$ se ve modificado al ser una de sus dos componentes (paralelas) rotada. La rotación siendo $U_{1}$ \notaAd{Creo que dependo mucho de las parametrizaciones de Bloch para entender lo que está pasadno, ¿qué sucede en el espacio de operadores de densidad?}. En general:
\begin{equation}\label{eq:SeparableDynamicsUxI}
    \rho\xrightarrow{\mcU=U_{1}\otimes \Id} p\frac{1}{Z_{1}}U_{1}e^{\lambda p\hat{r}_{\rho}\cdot\vec{\sigma}}U_{1}^{\dag}+(1-p)\frac{1}{Z_{2}}e^{\lambda(1-p)\hat{r}_{\rho}\cdot\vec{\sigma}}
\end{equation}
En términos del vector de Bloch, denotando $r_{A}=p\tan(p\lambda)$, $r_{B}=(1-p)\tan((1-p)\lambda)$, y $O$ la rotación generada por $U_{1}$:
\begin{equation}
    r\hat{r}_{\rho}\xrightarrow{\mcU=U_{1}\otimes \Id}r_{A}O\hat{r}_{\rho}+r_{B}\hat{r}_{\rho}=O(r\hat{r}_{\rho}-r_{B}\hat{r}_{\rho})+r_{B}\hat{r}_{\rho}
\end{equation}\label{eq:SeparableDynamicsUxIBloch}
El resultado es una rotación alrededor de una línea que no pasa por el origen. Una rotación de esta naturaleza puede descomponerse en una rotación a través de un eje que pasa por el origen $R$ y una traslación $T$ como $T^{-1}\circ R\circ T$. Notar que una transformación así no tendría por qué mantener a los estados dentro de la esfera de Bloch, por lo que esta debe depender del estado mismo. En efecto traslación tiene una magnitud $r_{B}$ en la dirección opuesta a la del estado (depende del estado tanto en magnitud como en dirección). Así que, aunque esto podría parecer una transformación afín, no lo es, pues depende enteramente del estado.

De (\ref{eq:SeparableDynamicsUxIBloch}) también se ve que si $U=e^{it\hat{n}\cdot\vec{\sigma}}$ entonces se ve que cualquier estado con vector de Bloch $r\hat{n}$ será invariante bajo la transformación subyacente. Lo que es mejor, esto aplica para cualquiera de los casos $U_{1}=\Id$, $U_{2}=\Id$ o $U_{1}=U_{2}$.

\subsubsection{Cambio de fase $H=\sigma_{z}$}
Considérese el hamiltoniano $H=\sigma_{z}$. La rotación en la esfera debida a la unitaria generada por el hamiltoniano es una alrededor del eje $z$. La representación de esto, y de el resultado general (\ref{eq:SeparableDynamicsUxIBloch}) puede verse en la figura \ref{fig:ZRot}.


En el espacio de operadores de densidad, esto equivale a insertar una fase relativa en el primer subsistema fino. En efecto, ignorando fases globales,
\begin{equation}
    e^{it\sigma_{z}}=\begin{pmatrix}
        1&0\\0&e^{-i2t}
    \end{pmatrix}
\end{equation}
El estado grueso siente el cambio de fase relativa en su primera componente \notaAd{¿Qué son las trazas del MaxEnt?}.
\begin{equation}
    \rho\xrightarrow{\mcU=e^{it\sigma_{z}}\otimes \Id} p\frac{1}{Z_{1}}e^{it\sigma_{z}}e^{\lambda_{3}p\hat{r}_{\rho}\cdot\vec{\sigma}}e^{-it\sigma_{z}}+(1-p)\frac{1}{Z_{2}}e^{\lambda_{3}(1-p)\hat{r}_{\rho}\cdot\vec{\sigma}}
\end{equation}

\subsubsection{Tengo que ver cómo interpreto esta $H=a\sigma_{x}+b\sigma_{y}$}
La transformación es una rotación respecto al eje $(a,b,0)$. Al aplicarse sobre el primer subsistema, el resultado es una rotación de la primera componente del estado grueso. De nuevo, la condición de normalización entre dichas componentes asegura que el estado grueso se mantenga dentro de la esfera de Bloch. 

\subsection{Régimen de error pequeño y ejemplos particulares}

El caso $p\rightarrow 1$ puede interpretarse como aquel en el que el aparato de medición tiene una baja probabilidad de fallar (poco ruido). Las evoluciones separables $U=U_{1}\otimes U_{2}$ pueden verse como una evolución gruesa $U_{1}$ más una perturbación. La perturbación, al ser unitaria, es también una rotación, así que lo que se observa es una especie de hélice. El estado precesa alrededor de una traslación de la que sería su trayectoria no perturbada. Siempre que $U_{2}=\Id$ lo que se observa es una traslación en dirección del estado con magnitud $r_{B}$. Si, por el contrario, $U_{1}=\Id$, una vez más el estado grueso se ve desplazado, para luego comenzar a girar según la rotación inducida por la unitaria $U_{2}$. 

Algunos ejemplos pueden verse en las figuras siguientes.
