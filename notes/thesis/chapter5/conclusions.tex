\chapter{Conclusiones}
\begin{itemize}
    \item La aplicación de grano grueso de $n$ a 1 partículas tiene dos interpretaciones dependiendo de la relación entre $p_{1}$ y las demás probabilidades: la de una medición que puede fallar, y la de una caja de partículas idénticas (que es como un caso degenerado porque el MaxEnt es $\rho^{\otimes n}$ y el CG es como una traza y ya)
    \item Como las contracciones en la esfera de Bloch corresponden a procesos irreversibles, las dinámicas efectivas son irreversibles.
    \item En general, las dinámicas efectivas no son necesariamente ni unitarias ni canales cuánticos. Sin embargo, parece que cuando $p=1$, $p=0$, entonces el mapeo de asignación es lineal y, en consecuencia, todo el proceso es lineal. También cuando la dinámica subyacente tiene alguna simetría fuerte, se recuperan dinámicas que sí son canales cuánticos.
    \item Las no linealidades encontradas son todas dependientes del estado inicial. Esto es, no hallamos no linealidades en las componentes de $\rho(t)$ sino dinámicas que varían según la elección del estado efectivo inicial. Mi hipótesis aquí es que esto es una consecuencia de que el único paso no lineal se da antes de la evolución (la asignación).
\end{itemize}