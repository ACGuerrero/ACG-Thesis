\chapter{Demostraciones de relaciones frecuentemente socorridas}

\subsubsection{Cuadrado de vector de pauli}
Se cumple que
\begin{align*}
    (\paulivec{r})(\paulivec{r})=&\sum_{j}r_{j}\pauli{j}\sum_{k}r_{k}\pauli{k}\\
    =&\sum_{j}r_{j}\sum_{k}r_{k}\pauli{j}\pauli{k}\\
    =&\sum_{j}r_{j}\sum_{k}r_{k}(\Id\delta_{jk}+i\epsilon_{jkl}\pauli{l})\\
    =&\sum_{j}r_{j}\sum_{k}r_{k}(\Id\delta_{jk})+i\sum_{j}r_{j}\sum_{k}r_{k}\epsilon_{jkl}\pauli{l})\\
    =&\Id
\end{align*}
Donde en la última línea se ha utilizado la antisimetría del tensor de Lévi-Civita. Se sigue que para todo entero positivo $p$
\begin{equation}\label{ap:PauliSquare}
    (\paulivec{n})^{2p}=\Id
\end{equation}

\subsubsection{Exponencial real de vector de Pauli}
Si se expande la serie de Taylor usando como argumento un vector de pauli $r\paulivec{r}$,
\begin{align*}
    e^{r\paulivec{r}}=&\sum_{k=0}^{\infty}\frac{1}{k!}(r\paulivec{r})^k\\
    =&\sum_{k}\frac{r^{2k}(\paulivec{r})^{2k}}{(2k)!}+\sum_{k}\frac{r^{2k+1}(\paulivec{r})^{2k+1}}{(2k+1)!},
\end{align*}
se puede usar (\ref{ap:PauliSquare}) para ver que
\begin{align*}
    e^{r\paulivec{r}}=&\Id\sum_{k}\frac{r^{2k}}{(2k)!}+\paulivec{r}\sum_{k}\frac{r^{2k+1}}{(2k+1)!},
\end{align*}
que, claro está, corresponde a
\begin{equation}\label{ap:PauliRealExp}
    e^{r\paulivec{r}}=\Id\cosh{r}+\paulivec{r}\sinh{r}
\end{equation}


\subsubsection{Exponencial compleja de un vector de Pauli}
Si se expande la serie de Taylor usando como argumento un vector de pauli $r\paulivec{r}$,
\begin{align*}
    e^{-ir\paulivec{r}}=&\sum_{k=0}^{\infty}\frac{1}{k!}(r\paulivec{r})^k\\
    =&\sum_{k}\frac{r^{2k}(\paulivec{r})^{2k}}{(2k)!}+\sum_{k}\frac{r^{2k+1}(\paulivec{r})^{2k+1}}{(2k+1)!},
\end{align*}
se puede usar (\ref{ap:PauliSquare}) para ver que
\begin{align*}
    e^{r\paulivec{r}}=&\Id\sum_{k}\frac{r^{2k}}{(2k)!}+\paulivec{r}\sum_{k}\frac{r^{2k+1}}{(2k+1)!},
\end{align*}
que, claro está, corresponde a
\begin{equation}\label{ap:PauliCompExp}
    e^{-ir\paulivec{r}}=\Id\cos{r}-i\paulivec{r}\sin{r}
\end{equation}
\subsubsection{Unitaria generada por un operador hermítico}
Toda unitaria de $2\times 2$ puede generarse a través de un operador hermítico $H$ como
\begin{equation*}
    U=e^{-iH}
\end{equation*}
Pues bien, como el conjunto de las matrices de Pauli, junto a la identidad, forman una base del espacio de operadores (respecto al producto interno de Hilbert-Schmidt), $H$ puede expandirse como $H=r_{0}\Id+r_{x}\pauli{x}+r_{y}+\pauli{y}+r_{z}\pauli{z}$. Si se utiliza este para construir una unitaria, desarrollando la serie se encuentra que
\notaAd{NO SÉ POR QUÉ ME SALIÓ UN MENOS JUNTO A LA IDENTIDAD}
\begin{align*}
    e^{-iH}=&e^{-i(r_{0}\Id+r\paulivec{r})}\\
    =&e^{-i r_{0}\Id}e^{-ir\paulivec{r}}\\
    =&e^{-ir\paulivec{r}}\\
    =&\sum_{k=0}^{\infty}\frac{1}{k!}(-ir\paulivec{r})^k\\
    =&\sum_{k}(i)^{2k}(-1)^{2k}\frac{r^{2k}(\paulivec{r})^{2k}}{(2k)!}+\sum_{k}(i)^{2k+1}(-1)^{2k+1}\frac{r^{2k+1}(\paulivec{r})^{2k+1}}{(2k+1)!}\\
    =&-\Id\sum_{k}(-1)^{2k}\frac{r^{2k}}{(2k)!}-i(\paulivec{r})\sum_{k}(-1)^{2k}\frac{r^{2k+1}}{(2k+1)!}\\
    =&-\Id \cos{r}-i(\paulivec{r})\sin{r}
\end{align*}
\subsubsection{Vector de Pauli sobre vector de Pauli}
Si se aplica un vector de Pauli sobre otro se halla lo siguiente

\begin{align*}
    (\hat{n}\cdot\vec{\sigma})(\hat{m}\cdot\vec{\sigma})
\end{align*}

\subsubsection{Evolución de operador de densidad por operador unitario}
\begin{align*}
    e^{-i\omega t \paulivec{r}}\rho e^{i\omega t \paulivec{r}}=&(\Id \cos(\omega t)-i\paulivec{r} \sin(\omega t))\rho(\Id \cos(\omega t)+i\paulivec{r} \sin(\omega t))\\
    =&\rho\cos^{2}(\omega t)+(\paulivec{r})\rho(\paulivec{r})\sin^{2}(\omega t)+i\rho(\paulivec{r})\cos(\omega t)\sin(\omega t)-i(\paulivec{r})\rho \sin(\omega t)\cos(\omega t)\\
    =&\rho\cos^{2}(\omega t)+(\paulivec{r})\rho(\paulivec{r})\sin^{2}(\omega t)+i\sin(\omega t)\cos(\omega t)[\rho,\paulivec{r}]
\end{align*}