\documentclass[12pt,twoside]{book}
\usepackage[spanish]{babel}
\decimalpoint
\usepackage[utf8]{inputenc}
\usepackage{cmap}
\usepackage[T1]{fontenc}
\usepackage{amssymb}
\usepackage[margin=1in]{geometry}
\usepackage{amsfonts}
\usepackage{dsfont}
\usepackage{physics} %indispensable
\usepackage{xcolor} %colores, notas
\usepackage{tikz-cd} %para diagrama conmutativo
\usepackage{multicol} %para la lista de operadores
\usepackage{hyperref} %para referencias clickeables
\usepackage{caption}
\usepackage{subcaption}
\usepackage{pdfpages} % para la portada
\usepackage[mathlines]{lineno}

%==============================
%========== Comandos ==========
%==============================
\newcommand{\mcU}{\mathcal{U}}
\newcommand{\mcV}{\mathcal{V}}
\newcommand{\mcO}{\mathcal{O}}
\newcommand{\mcI}{\mathcal{I}}
\newcommand{\mcL}{\mathcal{L}}
\newcommand{\mcS}{\mathcal{S}}
\newcommand{\hilbert}{{\sf H}}
\newcommand{\mcB}{\mathcal{B}}
\newcommand{\mcH}{\mathcal{H}}
\newcommand{\mcF}{\mathcal{F}}
\newcommand{\mcC}{\mathcal{C}}
\newcommand{\mcT}{\mathcal{T}}
\newcommand{\mcE}{\ensuremath{\mathcal{E}} }
\newcommand{\mcG}{\ensuremath{\mathcal{G}} }
\newcommand{\mcM}{\mathcal{M}}
\newcommand{\mcN}{\mathcal{N}}
\newcommand{\nnn}{\mathcal{N}}
\newcommand{\mmm}{\mathcal{M}}
\newcommand{\sss}{\mathcal{S}}
\newcommand{\mcD}{\mathcal{D}}
\newcommand{\mcA}{\mathcal{A}}
\newcommand{\mcP}{\mathcal{P}}
\newcommand{\rmi}{\text{i}}
\newcommand{\ie}{i.e.}
\newcommand{\avg}{\text{avg}}
\newcommand{\ef}{\text{ef}}
\newcommand{\Complex}{\mathbb{C}} %Para escribir al espacio de hilbert complejo
\newcommand{\Id}{\mathds{1}}% Para escribir el op. indentidad con notación chida
\newcommand{\CG}[1]{\mcC\left[#1\right]}
\newcommand{\Fuzzy}[1]{\mcF\left[#1\right]}
\newcommand{\nota}[1]{{\color{red} [#1]}}
\newcommand{\notaAd}[1]{{\color{gray} [#1]}} %Notas pero mías
%%Fixnote
\newcommand{\pauli}[1]{\sigma_{#1}} %Para las matrices de pauli
\newcommand{\paulivec}[1]{\hat{#1}\cdot\vec{\sigma}} %Para vectores de Pauli unitarios
\newcommand{\cnot}{\text{C}_{\text{X}}} %Para el CNOT
\newcommand{\purity}[1]{\text{Pu}(#1)} %la pureza
\newcommand{\Abs}{\text{abs}} %abs
\newcommand{\rfroml}{f} %la función r(lambda)
\newcommand{\densityspace}[1]{\mcS(\hilbert_{#1})} %el espacio de operadores de densidad
\newcommand{\unitaryspace}[1]{\text{U}(\hilbert_{#1})} %el espacio de operadores unitarios
\newcommand{\obspace}[1]{\mcL(\hilbert_{#1})} %el espacio de observables
\DeclareMathOperator*{\Motimes}{\text{\raisebox{0.25ex}{\scalebox{0.8}{$\bigotimes$}}}} %para los productos tensoriales
\usepackage[draft,inline,nomargin]{fixme} \fxsetup{theme=color}
\newcommand\dsone{\mathds{1}}
\FXRegisterAuthor{ac}{acc}{\color{red}AC}
\FXRegisterAuthor{dd}{ddg}{\color{blue}DD}
\FXRegisterAuthor{cp}{acp}{\color{orange}CP}
%
%
%Comandos David
%===================
\newcommand{\blue}{\color{blue}}
\pagestyle{plain}
\title{Dinámica de un sistema de N qubits bajo un modelo de grano grueso}
\author{Adán Castillo Guerrero}
\date{\today}

\begin{document}

\includepdf[pages=-]{portada.pdf}

%\maketitle
%\cleardoublepage

\section*{Dedicatoria}
A mi familia, a Gabriela, a todos mis seres queridos.

\cleardoublepage

\section*{Agradecimientos}
A las personas de hace ratito, a mi director, David, al grupo de Carlos, al grupo de Fernando de Melo, a Fernando L.

\cleardoublepage
%==============================
%Opciones de numeración de las líneas
%==============================
\linenumbers
\setlength\linenumbersep{3pt}
\makeatletter
\let\LN@align\align
\let\LN@endalign\endalign
\renewcommand{\align}{\linenomath\LN@align}
\renewcommand{\endalign}{\LN@endalign\endlinenomath}
\let\LN@gather\gather
\let\LN@endgather\endgather
\renewcommand{\gather}{\linenomath\LN@gather}
\renewcommand{\endgather}{\LN@endgather\endlinenomath}
\makeatother
%==============================
%==============================

\section*{Resumen}

En el presente trabajo se utiliza el Principio de Máxima Entropía para crear una aplicación de asignación que permita hacer una conjetura sobre el estado microscópico de una descripción gruesa del estado de un sistema conformado por un número arbitrario de qubits. En particular, la descripción gruesa corresponde a la obtenida de un aparato de medición al que se le asocian dos tipos de errores: por un lado es capaz de resolver únicamente una de las partículas, y por otro lado, existe una probabilidad no nula de que mida una partícula diferente a la de interés. A través de la asignación de Máxima Entropía, y asumiendo que se conoce la evolución microscópica, se estudiaron diferentes dinámicas efectivas. Esto es, el cambio observable en la descripción gruesa del sistema. Algunas de las dinámicas estudiadas resultaron ser no lineales y dependientes en el estado efectivo inicial. Otras, como en el caso de las dinámicas factorizables, experimentan la pérdida de la periodicidad de la dinámica subyacente. En algunos casos particulares, como el de el canal de despolarización y el canal de estabilización, la dinámica efectiva resulta ser el mismo tipo de dinámica y por lo mismo, un canal cuántico. Se encontró también que las dinámicas son lineales en los casos extremos de la probabilidad de error (esto es, cuando la probabilidad de error es uno o cero), debido a que el único elemento no lineal de la dinámica, la aplicación de asignación, se hace lineal.  \acnote{no lineales como?} \ddnote{tal cual que el propagador/canal/mapa/dinámica es no lineal, es decir, no cumple que $\mcE[\alpha A+B]=\alpha\mcE[A]+\mcE[B]$} \acnote{Jaja, esa nota era para mi. Voy a especificar que la no linealidad viene de quemás de una dinámica obtenida es dependiente del estado inicial.}. Finalmente, se compararon los resultados analíticos obtenidos a través de Principio de Máxima Entropía con los resultados numéricos obtenidos de otro tipo de asignación: la asignación promedio.

\ddnote{Va bien, solo evita poner expresiones matemáticas, tipo en lugar de $n$ qubits, poner para un número aribitrario de qubits.}
\pagestyle{plain}
\tableofcontents
\newpage
\section{Introducción}

\begin{frame}{Modelos de grano grueso}
    \lipsum[1]
\end{frame}
\chapter{Preliminares}
El desarrollo del presente trabajo descansa completamente sobre los hombros de tres conceptos: el operador de densidad, el principio de máxima entropía, y los modelos de grano grueso. En este capítulo se introducirán cada una de estas ideas. Primero, se discutirá el formalismo del operador de densidad, introducido mediante la necesidad de este a la hora de estudiar mezclas estadísticas de estados cúanticos. Luego se analizará el concepto de entropía en los contextos de las teorías de información tanto clásica como cúantica, a través del cual se derivará la expresión de la inferencia de máxima entropía de un sistema cuántico. Finalmente, se comentará sobre los modelos de grano grueso, como descripciones \textit{efectivas} de sistemas de dimensión alta, donde por \textit{efectivas} nos referimos a descripciones en las que no se tiene acceso a todos los grados de libertad del sistema. 

\ddnote{sería bueno mencionar efectivas en que sentido: cuando no se tiene acceso a todos los grados de libertad}. 

\acnote{¿Es suficente eso? Se profundiza más en la sección correspondiente.}

\ddnote{Aunque se profundice en la sección es bueno ser precisos aquí, pues esto es como una microintroducción al capitulo. Mejor pon algo sobre que no se tiene acceso a todos los grados de libertad, porque la realidad es que si se están tomando en cuenta, vamos, sabemos que existen.}

\acnote{¿Así queda?}

\section{El operador de densidad}
\subsection{Derivación del operador de densidad}

Los vectores de estado no pueden describir a todos los sistemas estudiables en el contexto de la mecánica cuántica. Por esto, y porque será particularmente útil para nuestro trabajo, introducimos el concepto del operador de densidad (también llamado, en el caso discreto, que es el que nos incumbe, matriz de densidad) de forma similar a como lo hizo L.D. Landau en 1927 \cite{Landau}.

Considérese un sistema descrito por el vector de estado $\ket{\varphi}\in\hilbert_{n}$, con $\hilbert_{n}$ el espacio de Hilbert $\hilbert_{n}=\Complex^{n}$; y $A$ un observable (un operador hermítico). Se sabe que el valor de expectación del observable está dado por $\expval{A}=\bra{\varphi}A\ket{\varphi}$. Pues bien, esta expresión puede ser manipulada a través de una base ortogonal $\{\ket{e_{k}}\}$ del espacio $\hilbert_{n}$:
\begin{align*}
\expval{A}&=\qty(\sum_{i}\dyad{e_{i}})\bra{\varphi}A\ket{\varphi}\qty(\sum_{j}\dyad{e_{j}})\\
&=\bra{\varphi}\qty(\sum_{i}\dyad{e_{i}})A\qty(\sum_{j}\dyad{e_{j}})\ket{\varphi}\\
&=\sum_{i,j}\bra{e_{j}}\ket{\varphi}\bra{\varphi}\ket{e_{i}}\bra{e_{i}}A\ket{e_{j}}\rlap{.}\\
\end{align*}
Esta es una suma sobre los elementos de dos matrices: la del observable $A$ y la definida por $\dyad{\varphi}$. Por completez de la base $\{\ket{e_{i}}\}$,
\begin{align*}
\expval{A}&=\sum_{j}\bra{e_{j}}\ket{\varphi}\bra{\varphi}\qty(\sum_{i}\dyad{e_{i}})A\ket{e_{j}}\\
&=\sum_{j}\bra{e_{j}}\ket{\phi}\bra{\varphi}A\ket{e_{j}}\\
&=\Tr(\dyad{\varphi}A)\rlap{,}
\end{align*}
donde definimos al operador de densidad $\rho$ para un sistema descrito por $\ket{\varphi}$ como
\begin{equation}\label{eq:DensOpPure}
\rho=\dyad{\varphi},
\end{equation} 
y vemos que es posible hallar el valor esperado de un observable respecto a un estado a través del operador de densidad de este según
\begin{equation}\label{eq:ExpValFromDensOp}
\expval{A}=\Tr(A\rho).
\end{equation}

El nombre ``operador de densidad'' puede resultar más claro comparando la ecuación (\ref{eq:ExpValFromDensOp}) con el valor esperado en estadística. Si $X$ es una variable aleatoria cuya función de densidad de probabilidad es $\rho(x)$, entonces el valor esperado de una función $A$ de los valores de $X$ es
\begin{equation*}
E[A(x)]=\int A(x) \rho(x) dx.
\end{equation*}

En este sentido, la matriz de densidad ocupa un rol similar al de la función de densidad.

\subsection{Mezclas estadísticas}

Ahora que conocemos el operador de densidad para un sistema descrito por un vector de estado, supóngase que en lugar de estudiar un sistema que está completamente descrito por $\ket{\varphi}$, se trabaja con uno que está en el estado $\ket{\varphi_{i}}$ con probabilidad $p_{i}$, donde $\{\ket{\varphi_{i}}\}$ es un conjunto no necesariamente ortogonal de estados de $n$ niveles $\ket{\varphi_{i}}\in\hilbert_{n}$, y $\{p_{i}\}$ es un conjunto de números reales tales que $\sum_{i}p_{i}=1$. A este sistema se le llama ``mezlca estadística'', y no debe confundirse con una superposición de estados $\ket{\varphi_{i}}$ con coeficientes $\sqrt{p_{i}}$, ya que una superposición está bien caracterizada, y está completamente descrita por $\ket{\psi}=\sum_{i}\sqrt{p_{i}}\ket{\varphi_{i}}$, mientras que la mezcla no lo está: el elemento probabilístico está asociado a un grado de ignorancia sobre la preparación del sistema.

Ahora vuélvase a considerar un observable $A$. El valor esperado de dicho observable con respecto al sistema será, justamente
\begin{equation}
\expval{A}=\sum_{i}p_{i}\bra{\varphi_{i}}A\ket{\varphi_{i}}.
\end{equation}
Sea $\{\ket{e_{k}}\}$ una base ortogonal del espacio $\hilbert_{n}$.La expresión puede manipularse de forma similar a como se hizo antes:
\begin{align*}
\expval{A}&=\sum_{i}p_{i}\bra{\varphi_{i}}A\ket{\varphi_{i}}\\
&=\sum_{i,j,k}p_{i}\bra{e_{k}}\ket{\varphi_{i}}\bra{\varphi_{i}}\ket{e_{j}} \bra{e_{j}}A\ket{e_{k}}\\
&=\sum_{j,k}\bra{e_{k}}\qty(\sum_{i}p_{i}\dyad{\varphi_{i}})\ket{e_{j}} \bra{e_{j}}A\ket{e_{k}}\\
&=\sum_{k}\bra{e_{k}}\qty(\sum_{i}p_{i}\dyad{\varphi_{i}})A\ket{e_{k}}\\
&=\Tr[\qty(\sum_{i}p_{i}\dyad{\varphi_{i}})A]\rlap{,}
\end{align*}
Con lo que la mezcla queda descrita por el operador de densidad $\rho$ definido según
\begin{equation}\label{eq:DensOpMix}
\rho=\sum_{i}p_{i}\dyad{\varphi_{i}}.
\end{equation}
\subsection{Propiedades del operador de densidad}
\subsubsection{Validez}
De la definición del operador de densidad destilan algunas propiedades que permiten reconocer si un operador es un operador de densidad válido, o no \cite{Holevo}:
\begin{enumerate}
    \item $\Tr(\rho)=1$
    \item $\bra{\varphi}\rho\ket{\varphi}\geq 0$ $\forall$ $\ket{\varphi}\in\hilbert_{n}$
\end{enumerate}
Estas dos propiedades funcionan como una definición alternatica del operador de densidad. La primera propiedad se deriva de la normalización de los estados $\ket{\varphi_{i}}$ que definen a la matriz de densidad. La segunda puede interpretarse como la necesidad de que la probabilidad de que $\rho$ se halle en el estado $\dyad{\varphi}$ sea mayor o igual a $0$.
\subsubsection{Pureza}
La diferencia entre una mezcla estadística y una superposición puede no ser del todo clara. ¿Cómo son diferentes un sistema que tiene una probabilidad $p_{i}$ de hallarse en el estado $\ket{\varphi_{i}}$ y otro que se halla en una superposición de cada estado $\ket{\varphi_{i}}$ con coeficientes $\sqrt{p_{i}}$? 

Para responder, considérense dos sistemas de dos niveles. El primero puede hallarse en cualquiera de los siguientes estados
\begin{align*}
    \ket{0}=\begin{pmatrix}
        1\\
        0
    \end{pmatrix} && \text{y} && \ket{1}=\begin{pmatrix}
        0\\
        1
    \end{pmatrix}\rlap{,}
\end{align*}
con la misma probabilidad $p=\frac{1}{2}$. Entonces el operador de densidad que describe al sistema es 
\begin{equation*}
    \rho=\frac{1}{2}(\dyad{0}+\dyad{1})=\frac{1}{2}\Id_{2}.
\end{equation*}
Por otro lado, el segundo sistema se halla en una superposción de los mismos estados, con coeficientes $\sqrt{p}$. El operador de densidad que describe al segundo sistema es 
\begin{align*}
    \dyad{\psi} && \text{con} && \ket{\psi}=\frac{1}{\sqrt{2}}(\ket{0}+\ket{1})\rlap{.}
\end{align*}
Es claro, al menos matemáticamente, que los sistemas no se hallan en el mismo estado. Si nos propusiéramos calcular la probabilidad de cada uno de hallarse en el estado $\ket{0}$ encontraríamos que
\begin{align*}
    \bra{0}\rho\ket{0}=\frac{1}{2} && \text{y} &&\langle 0 \dyad{\psi} 0\rangle=\frac{1}{2}\rlap{.}
\end{align*}
y el resultado es el mismo si se hiciera con el estado $\ket{1}$. Parecería entonces que, experimentalmente, los sistemas se hallan en el mismo estado. Esto es falso. Si realizamos un cambio de base, de $\{\ket{1},\ket{2}\}$ a $\{\ket{+},\ket{-}\}$, donde
\begin{align*}
    \ket{+}=\frac{1}{\sqrt{2}}\begin{pmatrix}
        1\\
        1
    \end{pmatrix} && \text{y} && \ket{-}=\frac{1}{\sqrt{2}}\begin{pmatrix}
        1\\
        -1
    \end{pmatrix}\rlap{,}
\end{align*}
y calculamos la probabilidad de que cada sistema se halle en el estado $\ket{+}$ encontraremos
\begin{align*}
    \bra{+}\rho\ket{+}=\frac{1}{2} && \text{pero} &&\langle + \dyad{\psi} +\rangle=1\rlap{.}
\end{align*}
Este resultado puede interpretarse como que el sistema descrito por $\rho_{2}$ se halla en el estado $\ket{+}$, mientras que el sistema $\rho_{1}$ siempre tendrá una probabilidad $\frac{1}{2}$ de hallarse en cualquiera de los dos elementos de cualquier base ortogonal que escojamos. Esta es una propiedad del estado máximamente mezclado. La diferencia entre ambos sistemas es que ele elemento probabilístico asociado a las mediciones sobre $\dyad{\psi}$ es de naturaleza cuántica, y se debe a que el sistema se halla en una superposición de estados ortogonales, mietras que en el caso de $\rho$, el elemento probabilístico se debe a nuestra ignorancia sobre la preparación del estado \cite{Chuang}.

Vemos, pues, que hay una diferencia fundamental entre los sistemas que pueden ser descritos por un vector de estado (para los que es posible contruir una matriz de densidad), y aquellos que no. Si para $\rho$ un operador de densidad,
\begin{equation*}
    \rho=\sum_{i}p_{i}\dyad{\varphi_{i}},
\end{equation*}
se cumple que $\rho=\dyad{\varphi_{i}}$ $\forall i$, entonces decimos que $\rho$ es un estado puro, y está completamente caracterizado por el vector de estado $\ket{\varphi}$. En este sentido, los estados puros (aquellos que están descritos por un vector de estado, i.e. su operador de densidad es un proyector) son los puntos extremos del conjunto convexo de operadores de densidad. Estos estados cumplen que
\begin{itemize}
    \item $\rho=\dyad{\psi}$
    \item $\rho=\rho^{n}$
    \item $\Tr(\rho^{2})=1$
\end{itemize}
La pureza es una medida de qué tan puro es un estado, y, dado $\rho\in\mcS(\hilbert_{n})$, se define como \cite{Jaeger}
\begin{equation}
    \text{Pu}(\rho)=\Tr(\rho^{2}).
\end{equation}
Destilan de dicha definición las siguientes dos propiedades:
\begin{itemize}
    \item Un estado es puro si y sólo si $\text{Pu}(\rho)=1$.
    \item Para todo estado, $\frac{1}{n}\geq \text{Pu}(\rho)\geq 1$.
\end{itemize}
\subsubsection{Evolución del operador de densidad}
El postulado de la mecánica cuántica asociado al vector de estado puede reformularse para que funcione con operadores de densidad.\cite{Breuer}

\subsubsection{Parametrización del operador de densidad}
Cualquier matriz de densidad puede descomponerse en términos de una base del espacio de matrices hermitianas de $n\times n$. Una elección común de base para el espacio es el de los generadores $\{\varsigma_{k}\}$ del grupo $\text{SU}(n)$, junto a la matriz identidad $\Id_{n}$. Aunque no está dentro del alcance de este trabajo estudiar las propiedades y caracterizaciones de lestos generadores, su utilizacion permite parametrizar a las matrices de densidad de forma vectorial \cite{Bruning}. En efecto, sea $\{\varsigma_{k}\}$ un conjunto de generadores de $\text{SU}(n)$ y $\rho$ una matriz de densidad $\rho\in\mcS(\hilbert_{n})$. Entonces $\rho$ está completamente descrita por el vector generalizado de Bloch de dimensión $2n^{2}-1$, $\vec{\gamma}$ definido según
\begin{equation}
    \rho=\frac{1}{n}\Id_{n}+\frac{1}{2}\vec{\gamma}\cdot\vec{\varsigma}.
\end{equation}
Si $n=2$, los generadores corresponden a las matrices de Pauli $\sigma_{i}$. En tal caso, el conjunto de vectores de Bloch corresponde a la bola unitaria tridimensional, con los estados puros en la superficie y las mezclas en el interior. Para casos en los que la dimensión es una potencia de $k$, es posible obtener nuevos generadores a través de los productos tensoriales de las matrices de Pauli consigo mismas y con la matriz identidad correspondiente. El caso $n=4$, por ejemplo \cite{Chuang}:
\begin{equation}
    \rho=\frac{1}{4}\sum_{i,j}\gamma_{ij}\sigma_{i}\otimes \sigma_{j} \ \ i,j\in\{0,1,2,3\},
\end{equation}
donde $\sigma_{0}=\Id$ y $\gamma_{i.j}=\sigma_{i}\otimes \sigma_{j}\Tr(\rho)$.
\newpage
\section{Entropía}
\label{sec:ch2_entropy}

\subsection{Entropía de Shannon}
A finales de los años cuarenta, Claude Shannon se preguntaba sobre una medida de la \textit{incertidumbre}, o de la \textit{información} \footnote{En teoría de información clásica, los términos \textit{información}, \textit{incertidumbre} y \textit{sorpresa} se utilizan de manera intercambiable.} asociada a un proceso cuyo resultado estuviera descrito por una variable aleatoria $X$ con distribución de probabilidad $p(x_{j})$.

\acnote{Párrafo iterado: notas}

La cantidad de información provista por el resultado de un experimento depende de la probabilidad asociada a dicho suceso. Por ejemplo, al tirar un dado es mucho menos informativo saber que no cayó un $6$ que saber que cayó un $6$, ya que cada número tiene una probabilidad de $\frac{5}{6}$ de no caer, pero sólo $\frac{1}{6}$ de caer. Con la misma línea de razonamiento, conocer el resultado de un evento que ocurre con probabilidad $p=1$ no transmite ninguna información. Si a cada valor de $X$ se le puede asociar una cantidad de información, entonces debe poder calcularse la cantidad de información promedio: esta es la medida que buscaba Shannon. La forma de esta medida, denotada $H(p)$, vino de las propiedades que el matemático estadounidense afirmó que debía cumplir \cite{Shannon,Wilde}
\begin{enumerate}
    \item $S_{\text{S}}(p)$ debe ser continua en $p$.
    \item $S_{\text{S}}(p)$ debe ser una función creciente, monotónica de $n$ cuando $p_{j}=\frac{1}{n}$.
    \item Si $X$ e $Y$ son procesos independientes, $S_{\text{S}}(p_{X}(x_{j})p_{Y}(y_{l}))=S_{\text{S}}(p_{X}(x_{j}))+S_{\text{S}}(p_{Y}(y_{l}))$.
\end{enumerate}
Además, demostró que
\begin{equation}\label{eq:ShannonEntropy}
    S_{\text{S}}=-k\sum_{j}p(x_{j})\log{p(x_{j})},
\end{equation}
donde $k$ es una constante que depende de la naturaleza del sistema estudiado. Fue a través de discusiones con von Neumann que Shannon descubrió que su medida ya era ampliamente utilizada en física, y que llevaba el nombre de \textit{entropía} \cite{McIrvine}. En efecto, la entropía de Gibbs es
\begin{equation}\label{eq:GibbsEntropy}
    S_{\text{G}}=-k_{\text{B}}\sum_{j}p_{j}\log{p_{j}},
\end{equation}
donde $k_{B}$ es la constante de Boltzmann, y $p_{j}$ es la probabilidad de que el sistema se halle en la $j$-ésima configuración microscópica posible.

Como medida de incertidumbre, la entropía de Shannon (\ref{eq:ShannonEntropy}) es máxima para distribuciones equiprobables. Retomando la idea del dado bien balanceado, como no es posible tener ningún tipo de seguridad sobre el resultado de un tiro, la incertidumbre (la entropía) es máxima.

\acnote{Párrafo iterado: notas}

En teoría de información clásica, la entropía de Shannon se suele utilizar como la cantidad promedio de bits requerida para trasmitir un mensaje, tomando el logaritmo en base $2$ y $k=1$ en la ecuación (\ref{eq:ShannonEntropy}). Para ejemplificar la naturaleza de ``medida de información'' de la entropía de Shannon, supóngase que se desea transmitir un mensaje encriptado en el que únicamente se utilizan los caracteres A, B, C y D. Si el método de encriptación es tal que todos los caracteres tienen la misma probabilidad de aparecer, entonces una forma de transmitir el mensaje es asignándoles los valores $00$, $01$, $10$, y $11$ respectivamente\footnote{Esta forma de codificar los caracteres no es única, pero es la más sencilla. Otra sería asignarle a los caracteres A, B, C y D los valores $101$, $1001$, $10001$ y $100001$. Esta codificación, aunque produzca una cadena de bits que se traduzca de forma única a la cadena de caracteres, es altamente ineficiente. }. Calculando la entropía de Shannon se halla que, en promedio, se requieren dos bits para transmitir cada caracter del mensaje. 

Si, en cambio, las probabilidades de que aparezca cada uno de los caracteres son $p(A)=\frac{1}{2}$, $p(B)=\frac{1}{4}$, $p(C)=\frac{1}{8}$, $p(D)=\frac{1}{8}$. En este caso, una codificación posible, tal que no haya ambigüedad en la cadena de bits, es $A \rightarrow 0$, $B\rightarrow 10$, $C\rightarrow 110$, $D\rightarrow 111$. Nótese que ahora solo se requiere un bit para transmitir la letra más común. Pues bien, si se calcula la entropía de Shannon, se encuentra que cada letra requerirá $1.75$ bits para ser transmitida \cite{Cryptography}.

\subsection{Entropía de von Neumann}

La entropía de von Neumann, a pesar de haber sido obtenida veinte años antes, puede verse como la extensión cuántica de la entropía clásica de Shannon. Von Neumann introdujo el concepto del operador de densidad de forma paralela e independendiente a L. Landau, y definió la entropía $S$ asociada a un sistema descrito por un operador de densidad $\rho$ como \cite{vonNeumann}
\begin{equation}\label{eq:VonNeumannEntropy}
    S_{\text{N}}(\rho)=-\Tr(\rho\ln{\rho}).
\end{equation}
\acnote{El siguiente párrafo ha sido iterado, solo notas}

La entropía de von Neumann puede interpretarse de manera similar a la entropía de Shannon. Si se desea transmitir un qubit preparado como $\ket{\psi_{i}}$ con probabilidad $p_{i}$, entonces el operador de densidad que representa al estado enviado es justamente $\rho=\sum p_{i}\dyad{\psi_{i}}$. La cantidad de información recibida, o la incertidumbre sobre el qubit enviado, es justamente $S(\rho)$. Debe hacerse hincapié en el hecho que la entropía de un sistema cuántico es fundamentalmente diferente a la de un sistema clásico. El sistema cuántico presenta dos tipos de incertidumbres: la incertidumbre clásica, relacionada a nuestra falta de conocimiento relativa a un sistema, y la incertidumbre cuántica, una propiedad intrínseca a los sistemas ondulatorios, matemáticamente expresada a través del Principio de Incertidumbre de Heisenberg \cite{Wilde}.

\acnote{Lista iterada: notas y reescritura}

De la entropía de von Neumann de un sistema descrito por un operador de densidad $\rho\in\mcS(\hilbert_{n})$, nos interesan las siguientes propiedades \cite{Chuang}:
\begin{enumerate}
    \item La entropía puede escribirse en términos de los eigenvalores de $\rho$, $\eta_{j}$, como $S_{\text{N}}(\rho)=-\sum_{j}\eta_{j}\ln{\eta_{j}}$. Esto coincide con la entropía de Shannon si se envían los eigenestados de $\rho$ con probabilidades $\eta_{j}$.
    \item La entropía es no negativa, y es nula si y sólo sí $\rho$ es de la forma $\dyad{\psi}$ con $\ket{\psi}\in\hilbert_{n}$.
    \item La entropía es máxima cuando $\rho=\frac{1}{n}\Id_{n}$, y $S_{\text{N}}(\rho)=n$. Esto es de esperarse, de acuerdo con nuestra discusión previa, el estado máximamente mezclado es aquel del que somos máximamente ignorantes, y por lo mismo debe ser el que tiene la máxima entropía (recordando a la entropía como medida de incertidumbre).
    \item La entropía de un estado producto es igual a la suma de las entropías de cada factor, $S_{\text{N}}(\rho_{A}\otimes\rho_{B})=S_{\text{N}}(\rho_{A})+S_{\text{N}}(\rho_{B})$.
\end{enumerate}
Nótese que la última propiedad es análoga al caso clásico en el que se tienen dos variables aleatorias independientes.
\section{El principio de máxima entropía}\label{sec:CH1MaxEnt}

\subsection{El principio de máxima entropía clásico}

Supóngase que los resultados de un proceso corresponden a los valores $x_{i}$ de una variable aleatoria $X$. Sea, además $f$, una función sobre $X$, de la que conocemos el valor esperado
\begin{equation}\label{eq:JaynesRestrictions}
    \expval{f(x)}=\sum_{i}p(x_{i})f(x_{i}).
\end{equation}
Con esta información, nos interesa hallar la distribución $p(x_{i})$. Introducimos así al Principio de Máxima Entropía.

El principio de máxima entropía fue introducido por E. T. Jaynes en 1957. En su artículo, \textit{Information Theory and Statistical Mechanics}, Jaynes afirma que la distribución de probabilidad obtenida a través del principio de máxima entropía es la mejor estimación que se puede hacer a través de la información disponible, independientemente de si las predicciones coinciden, o no, con los resultados experimentales \cite{JaynesI}.

El problema de hallar una distribución de probabilidad adecuada es también un problema de contaminación de la información accesible. Esta contaminación proviene de suposiciones arbitrarias, y sin sustento físico, que pueden hacerse sobre el sistema. El objetivo es, entonces, hallar la estimación de $p$ menos sesgada posible.

Jaynes relaciona la teoría de información clásica con la mecánica estadística no por la simple coincidencia en la forma de las entropías de Shannon y de Gibbs, sino a través de una reinterpretación de la mecánica estadística como una forma de inferencia estadística. En este contexto, viendo la entropía física como una medida de la incertidumbre asociada a una distribución de probabilidad, una distribución $p$ que no maximice la entropía, es una distribución que introduce información arbitraria no incluída en las hipótesis iniciales.

A través del método de multiplicadores de Lagrange, Jaynes demuestra que la distribución de probabilidad $p$ que maximiza la entropía de Shannon (\ref{eq:ShannonEntropy}), sujeta a las restricciones (\ref{eq:JaynesRestrictions}) es 
\begin{equation}
    p(x_{i})=e^{-\lambda-\mu f(x_{i})}
\end{equation}

Como ejemplo, supongamos que se tira un dado de seis caras $40$ veces, y el promedio de los tirajes es $\expval{X}=3.2$. Si tratáramos con un dado perfectamente equilibrado, y dispusiéramos del tiempo para hacer una infinidad de tirajes, hallaríamos que $\expval{X}=3.5$. Podríamos asumir, entonces, que el dado está bien equilibrado (una susposición de ergodicidad), y que el valor de expectación hallado experimentalmente difiere por simple falta de tirajes, pero esto equivale a hacer una suposición sobre algo que no sabemos a ciencia cierta.

\notaAd{Aqui voy a sacar la distribución de máxima entropía. No es complicado.}

\subsection{Extensión a la mecánica cuántica}

En su segundo artículo, Jaynes

\newpage
\section{Modelos de grano grueso}\label{sec:Ch1CG}

\subsection{Descripciones gruesas en física}

Una descripción de \textit{grano grueso} es aquella que no toma en cuenta todas los detalles de un sistema o fenómeno. Nuestra interacción del día a día con el mundo que nos rodea es fundamentalmente gruesa: al bañarnos, no nos preocupa la energía cinética individial de cada una de las $10^{23}$ moléculas de agua presente en cada una de las gotas que caen sobre nosotros, sino de qué tan caliente, o frío parece el chorro que sale de la llave. Una descripción de grano grueso puede omitir dichos detalles microscópicos por voluntad del observador (puede que no le sea útil toda la información del sistema, o que la cantidad de información sea demasiado grande como para manjearla) o por simple ignorancia de la información omitida.

La termodinámica es un área de la física que trata casi exclusivamente con modelos de grano grueso. Las cantidades termodinámicas: temperatura, presión, volumen, no son sino el resultado de una descripción gruesa de sistemas extremadamente complejos, pues promedian las interacciones y propiedades de $10^{23}$ partículas. La descripción de todo el sistema se reduce a un puñado de coordenadas gruesas.

Cuando se habla de modelos de grano grueso, no se suele hacer referencia a las descripciones efectivas inducidas por la ignorancia. En realidad, cuando se habla de modelos de grano grueso se hace referencia a un modelo impuesto por el observador sobre el sistema. Los modelos de grano grueso que buscan simplificar un problema deshechando información poco útil son comunes en física química. [REFERENCIAS?] El tipo de modelo de grano grueso en el que se centra estre trabajo no es el impuesto por el científico, sino el que proviene de su incapacidad de acceder a toda la información del sistema.

La descripción termodinámica de un sistema de $10^{23}$ partículas corresponde justamente a un modelo de grano grueso inducido por una ignorancia sobre los grados de libertad del sistema. Aún así, auque el observador no cuente con acceso a dicha información, puede deducir que su descripción es meramente efectiva. n efecto, la entropía de un sistema termodinámico es una cantidad que relaciona las coordenadas gruesas con la realidad microscópica.

\subsection{Grano grueso en mecánica cuántica}

En el contexto de la mecánica cuántica, un modelo de grano grueso se obtiene trazando sobre un subsistema del sistema de interés. Al subsistema deshechado se le puede llamar \textit{entorno}, y aunque la separación separación sistema - entrono no es siempre posible \cite{Macro-To-Micro}, nos limitamos a los casos en los que los grados de libertad ignorados pueden trazarse a través de la operación de traza parcial usual.

Un ejemplo sencillo de un modelo de grano grueso es el de un sistema de dos partículas, del cual únicamente nos importa una. En dicho caso, el modelo puede consistir en estudiar únicamente al operador de densidad reducido correspondiente a la partícula de nuestro interés. Es importante notar que el subsistema ignorado no es necesariamente una parte que puede ser separada del sistema, como en el caso de las dos partículas, sino que puede representar un conjunto de información intrínseca al sistema, pero que se ha decidido ignorar. Por ejemplo, puede que se tome en cuenta el momento angular orbital de una partícula, pero no su espín.

Matemáticamente, el modelo de grano grueso corresponde a separar un espacio $\hilbert^{C}$ en dos espacios $\hilbert^{A}$ y $\hilbert^{B}$...


\subsection{Canales cuánticos}



\newpage
\chapter{Estado de máxima entropía \textit{borroso}}

Con las bases asentadas, es posible pasar al estudio del problema propuesto por este trabajo: el estudio de la dinámica efectiva asociada a un modelo de grano grueso y a una aplicación se asignación.

Supongamos que estudiamos un sistema cuántico de varias particulas, que hemos logrado aislarlo de forma que evolucione como un sistema cuántico cerrado, y que conocemos a ciencia cierta la forma de la evolución seguida por el sistema.

En principio, parecería que no tendríamos ningún problema para conocer el estado del sistema a un tiempo arbitrario, dado que conozcamos el estado inicial del sistema. Sin embargo, nos damos cuenta de que, al comenzar a realizar mediciones sobre el sistema para poder hallar al estado inicial, nos damos cuenta de que nuestra instrumentación no sólo no puede resolver todos los grados de libertad del sistema, sino que tiene una probabilidad no nula de fallar.

Al no poder acceder a todas las dimensiones del sistema, nuetra descripción es, efectivamente, un modelo de grano grueso. ¿Cómo podemos describir la evolución del sistema que estamos estudiando? ¿Cómo evoluciona el estado efectivo accesible?

En este capítulo describiremos matemáticamente la aplicación de grano grueso con la que se trabajará, luego se utilizará el Principio de Máxima Entropía para hallar el mejor estimado posible dada nuestra descripción efectiva, y finalmente se propondrá el modelo de la dinámica efectiva. La dinámica efectiva siendo la evolución que el experimentalista observaría.


\section{Un modelo de grano grueso y borroso}\label{sec:CH2CG}

Como se discutió previamente, es natural suponer que no siempre es posible disponer de toda la información sobre el estado\ddnote{las itálicas están de más, me parece. Cual es tu política del uso de las itálicas?}\acnote{\checkmark} del sistema de interés. Esto ya sea por insuficiencia en la resolución de los aparatos de medición o por el inevitable error inherente a las herramientas de medición. Un prototipo sencillo de error consiste en el inducido por un aparato que no distingue diferentes conjuntos de partículas entre sí. El caso más simple corresponde a la permutación de dos partículas. Este intercambio accidental a la hora de la medición es una \textit{aplicación borrosa} \cite{FuzzyMeasurements}.

Para ilustrar lo anterior, considérese dicha aplicación borrosa sobre un sistema de dos partículas. Para simplificarlo un poco más, supongamos que cada partícula es un sistema de dos niveles, esto es, el sistema está compuesto por los qubits $A$ y $B$. El estado del sistema está caracterizado por un operador de densidad $\varrho_{AB} \in \mcS(\hilbert_2 \otimes \hilbert_2)$. La acción de la aplicación borrosa se escribe como sigue:
\begin{align}
\mcF:&\mcS(\hilbert_2 \otimes \hilbert_2)\to \mcS(\hilbert_2 \otimes \hilbert_2)\nonumber\\
&\varrho \mapsto p\varrho+(1-p)S\varrho S,\nonumber
\end{align}
donde $0<p<1$ es la probabilidad con la que el aparato de medición identifica correctamente \ddnote*{correctamente}{erróneamente}\acnote{\checkmark} a los dos subsistemas y $S$ es el operador de transposición de dos partículas (llamado operador SWAP), definido como 
\begin{equation}
    S\ket{\psi}\otimes \ket{\phi}=\ket{\phi}\otimes \ket{\psi} \ \ \forall \ket{\psi},\ket{\phi}\in \hilbert_2.\nonumber
\end{equation}
El estado resultante, $\Fuzzy{\varrho_{AB}}=p\varrho_{AB}+(1-p)\varrho_{BA}$, es una mezcla estadística del estado accesible con un detector perfecto, $\varrho_{AB}$, y el estado donde los qubits tienen las etiquetas equivocadas, $\varrho_{BA}:=S\varrho_{AB} S$. Así, si quisieramos hallar el valor esperado del observable $\sigma_{3}\otimes\Id$ (el valor esperado de $\sigma_{z}$ en la primer partícula), encontraríamos:
\begin{equation}
    \expval{\sigma_{3}\otimes\Id}_{\mcF}=p\expval{\sigma_{3}\otimes\Id}+(1-p)\expval{\Id\otimes\sigma_{3}}\nonumber
\end{equation}\ddnote{esta ecuación tiene las probabilidades volteadas}\acnote{\checkmark}
donde por $\expval{A}_{\mcF}$ nos referimos al valor esperado con respecto al estado del sistema descrito a través de $\mcF$.

Es importante notar que, auque la aplicación borrosa modela el error asociado al aparato de medición, no constituye por si misma un modelo de grano grueso, pues conserva la dimensión del sistema: el aparato resuelve todos los grados de libertad.

Al error se le añade la falta de resolución: solo se resuelve una partícula. \ddnote*{esto está redundante y escrito raro, no se por que dices que el estado sería el de la primera partícula, eso confunde, parece que estás cambiando repentinamente el contexto. Mejor explica al grano la situación y sin hablar de composición de errores. Eso mas bien se deriva, y justamente lo haces mas adelantito.}{Agreguemos al error por permutación la falta de resolución. Una posibilidad es que el instrumento detecte una partícula en donde en realidad haya dos, y aún más, que en un sistema de dos partículas, este sea capaz de medir únicamente observables asociados al primer subsistema. El estado observado sería, de acuerdo con la sección \ref{sec:Ch1PartialTrace}, la matriz de densidad reducida de la primera partícula.}\acnote{\checkmark} Matemáticamente, la composición del error y de la falta de resolución puede escribirse como
\begin{gather}
    \mcC:\mcS(\hilbert_2 \otimes \hilbert_2)\to \mcS(\hilbert_2)\nonumber\\
    \varrho_{AB} \mapsto p\rho_A+(1-p)\rho_B\rlap{,}\nonumber
\end{gather}
donde $\rho_A=\tr_B \rho_{AB}$ y $\rho_B=\tr_A \rho_{AB}$, es decir, las matrices de densidad reducidas de la partículas $A$ y $B$, respectivamente.


A diferencia de la aplicación borrosa, el modelo de grano grueso disminuye la dimensión del estado resultante. Además se puede mostrar que la ecuación anterior puede reescribirse en términos de la aplicación borrosa \cite{FuzzyMeasurements},
\begin{equation}
\CG{\varrho}=(\Tr_{B}\circ\mcF)[\varrho].\nonumber
\end{equation}
En este contexto, diferenciamos al estado ``microscópico'' o ``fino'', denotado por $\varrho\in \mcS(\hilbert_2\otimes\hilbert_2)$, y al estado ``macroscópico'', ``grueso'', o  ``efectivo'' denotado por $\rho_{\ef}\in \mcS(\hilbert_2)$, a través de la relación
\begin{equation}
    \rho_{\ef}=\CG{\varrho}.\nonumber
\end{equation}
Es extremadamente importante notar que la expresión anterior no es invertible. Pueden existir una infinidad de estados $\varrho$ tales que su descripción gruesa coincida con $\rho_{\ef}$. Como ejemplo, supóngase que el estado efectivo está descrito por $\rho_{\ef}=\frac{1}{2}\Id$, el estado máximamente mezclado. Entonces cualquier sistema fino que se halle en un estado máximamente entrelazado será compatible con la descripción efectiva $\frac{1}{2}\Id$.

Pues bien, como se ha asumido que conocemos la evolución unitaria subyacente, requerimos asignar a $\rho$ un estado microscópico que cumpla con todas las restricciones impuestas por nuestras mediciones. Asumiremos que dicho estado asignado es el que experimenta la evolución. 


La discusión anterior giró alrededor del caso en que el modelo de grano grueso reduce un espacio de dos qubits a uno de un solo qubit. Esto puede generalizarse al caso en que el aparato de medición sólo detecta una partícula cuando el sistema microscópico \ddnote*{conforma?}{comporta} $n$ partículas. En particular, consideramos, nuevamente, subsistemas de dos niveles (qubits), de tal forma que $\varrho\in\mcS\qty( \hilbert_{2}^{\otimes n})$, donde $\hilbert_{2}^{\otimes n}$ representa el producto tensorial del espacio $\hilbert_{2}$ consigo mismo $k$ veces. Sean $p_{i}$ las probabilidades de medir cada una de las partículas. La aplicación borrosa pasa de ser un intercambio de dos partículas, a una serie de permutaciones entre la partícula de interés y el resto (sin pérdida de generalidad, asumiremos que la partícula de interés es la primera). Considerando un sistema de $n$ subsistemas de dos niveles \ddnote*{si ya definiste qubits mas arriba como sistemas de dos niveles, no es necesario el paréntesis}{(qubits)}, la aplicación borrosa \ddnote*{se define entonces como}{se convierte en}
\begin{gather}
    \mcF:\mcS\qty( \hilbert_{2}^{\otimes n})\to \mcS\qty( \hilbert_{2}^{\otimes n})\nonumber\\
    \varrho \mapsto p_{1}\varrho+\sum_{j=2}^{n}p_{j}(S_{1,j})\varrho(S_{1,j})^{\dagger},\nonumber
\end{gather}
donde $S_{1,j}$ es el operador que permuta la primera y la $j$-ésima partícula. De esta manera, donde $\Tr_{\overline{i}}$ denota la traza parcial sobre todos menos el $i$-ésimo qubit, la aplicación de grano grueso que resuelve un qubit donde hay $n$ qubits es
\begin{gather}\label{eq:CG}
    \mcC:\mcS( \hilbert_{2}^{\otimes n})\to \mcS(\hilbert_{2})\nonumber\\
    \varrho\mapsto\Tr_{\overline{1}}(\Fuzzy{\varrho}).
\end{gather}
Reconocemos, según los valores de las probabilidades $p_{j}$ dos tipos de regímenes. El primero corresponde a nuestra discusión: si $p_{1}>p_{j}\forall j\neq 1$ se dice que el modelo tiene una partícula preferencial. Si $p_{j}=\frac{1}{n}\forall j$ entonces no existe ninguna partícula preferencial, y se puede ver al sistema como una caja de gas en la que medir a cualquier partícula es igual de probable.
\section{Construcción del estado de máxima entropía}

Sea $\rho\in\mcS(\hilbert_{2})$ el estado grueso accesible al observador, y sea \ddnote*{$\{A_{i}\}_{i}$}{$\{A_{i}\}$} con $A_{i}\in\obspace{2}$ un conjunto de observables tomográficamente completo.
Si $\rho$ es un estado grueso correspondiente a un estado fino $\varrho\in\mcS(\hilbert_{2}^{\otimes n})$, de acuerdo con lo discutido en el capítulo anterior, podemos asignar a $\varrho$ un estado que maximice la entropía de Von Neumann sin agregar \ddnote*{que es la información externa?, no es mejor, asignar un estado de máxima entropía compatible con las restricciones tal y tal} {información externa, y que satisfaga las restricciones $\expval{A_{i}}=\Tr(\rho A_{i})$}.
Escójase ${A_{i}}={\pauli{i}}$, las matrices de Pauli, y la aplicación de grano grueso desarrollada en la sección anterior, aquella del grano grueso y borroso dada por (\ref{eq:CG}). Los valores esperados de los operadores se traducen como las componentes del vector de Bloch del operador $\rho$. Las restricciones a las que se ve sujeto el operador $\varrho_{\max}$ son
\begin{equation*}
    r_{i}=\Tr[\pauli{i}\rho]
\end{equation*}\ddnote{supongo que estás usando $\rho$ y $\varrho$ para las cosas finas y gruesas, pero no se entiende, a la vista los caracteres son parecidos, sugiero usar $\rho_\text{MaxEnt}$ y $\rho_\text{ef}$ o $\rho_\text{grueso}$}
Aquí hay un problema: el estado que maximiza la entropía pertenece al espacio $\densityspace{2^{n}}$, mientras que las restricciones están definidas para un operador de densidad en $\densityspace{2}$. Entonces, ¿cómo se traducen dichas restricciones en el nivel microscópico? Naturalmente, esto dependerá de la relación entre \ddnote*{si, no no, está muy feo esto}{$\varrho$ y $\rho$}. Esto es, el estado de máxima entropía depende de la aplicación de grano grueso. Sustituyendo la relación entre algún estado microscópico compatible $\varrho$ y el estado efectivo $\rho$ en la ecuación anterior, y manipulando un poco se halla que
\begin{align*}
    r_{i}&=\Tr[\sigma_{i}\CG{\varrho}]\\
    &=\Tr[\sigma_{i}\Tr_{\overline{1}}\qty(p_{1}\varrho+\sum_{j=2}^{n}p_{j}(S_{1,j})\varrho(S_{1,j})^{\dagger})]\\
    &=\Tr[\sigma_{i}\otimes\Id_{2^{n-1}}\qty(p_{1}\varrho+\sum_{j=2}^{n}p_{j}(S_{1,j})\varrho(S_{1,j})^{\dagger})]\\
    &=\Tr[\qty(p_{1}(\sigma_{i}\otimes\Id_{2^{n-1}})+\sum_{j=2}^{n}p_{j}(S_{1,j})^{\dagger}(\sigma_{i}\otimes\Id_{2^{n-1}})(S_{1,j}))\varrho]\\
    &=\Tr[\qty(p_{1}(\sigma_{i}\otimes\Id_{2^{n-1}})+\sum_{j=2}^{n}p_{j}(\Id_{2^{j-1}}\otimes\sigma_{i}\otimes\Id_{2^{n-j}}))\varrho]\\
    &=\Tr[\qty(\sum_{j=1}^{n}p_{j}(\Id_{2^{j-1}}\otimes\sigma_{i}\otimes\Id_{2^{n-j}}))\varrho].
\end{align*}
Definiendo
\begin{equation}\label{eq:GhatNM}
    \hat{G}_{i}=\sum_{j=1}^{n}p_{j}(\Id_{2^{j-1}}\otimes\sigma_{i}\otimes\Id_{2^{n-j}}),
\end{equation}
las restricciones se pueden esribir como
\begin{equation}\label{eq:MaxEntRestrictions}
    r_{i}=\Tr[\hat{G}_{i}\varrho].
\end{equation}
Estas restricciones ya se hallan en términos de observables y un operador de densidad que actúan sobre $\hilbert_{2^{n}}$. Entonces se utilizan multiplicadores de Lagrange para obtener el estado de maximiza la entropía. De acuerdo con la ecuación (\ref{eq:GeneralMaxEnt}), el estado de máxima entropía compatible con (\ref{eq:MaxEntRestrictions}) es
\begin{equation}\label{eq:MaxEntLagMult}
    \varrho_{\max}=\frac{1}{Z}e^{\sum_{i}\lambda_{i}\hat{G}_{i}}.
\end{equation}
Si se sustituye a $\varrho_{\max}$ en las ecuaciones (\ref{eq:MaxEntRestrictions}) (cosa nada recomendable), se obtienen las relaciones entre los multiplicadores de Lagrange y los valores esperados de los observables utilizados para la tomografía. Si se escribe $\lambda=\sqrt{\lambda_{1}^{2}+\lambda_{2}^{2}+\lambda_{3}^{2}}$, los resultados son
\begin{align}\label{eq:MaxEntExpVals}
    \begin{split}
    \expval{\pauli{1}}&=\lambda_{1}\rfroml(\lambda),\\
    \expval{\pauli{2}}&=\lambda_{2}\rfroml(\lambda),\\
    \expval{\pauli{3}}&=\lambda_{3}\rfroml(\lambda),
    \end{split}
\end{align}
donde $\rfroml(\lambda)$ es una función biyectiva de $\lambda$, y cuya forma será derivada en la siguiente sección de una forma que requiere muchas menos cuentas. Idealmente, la ecuación (\ref{eq:MaxEntLagMult}) está en términos de los valores de expectación $r_{i}=Tr(\sigma_{i}\rho_{c})$, y no de los multiplicadores de Lagrange. Aunque no es posible despejar a los multiplicadores de Lagrange de las ecuaciones de manera algebráica, la naturaleza de $\rfroml(\lambda)$ nos permite asegurar que las relaciones son uno a uno y que tienen inversa.

A partir de este momento, cada vez que se hable del \textit{estado de máxima entropía}, se entiende que se hace referencia al estado dado por la ecuación (\ref{eq:MaxEntLagMult}). Esto es, al estado de máxima entropía que es compatible con un estado efectivo inducido por nuestro modelo de grano grueso particular.
\section{Propiedades}



\subsection{El estado de máxima entropía general: dos expresiones}
He estado cometiendo un error terrible: creer que todos los estados de máxima entropía, independientemente de las componentes de Pauli del estado grueso con el que son compatibles, evolucionan de la misma manera que el estado de Máxima entropía compatibñe con un estado grueso alineado en $z$. Un ejemplo claro de este error es el de la unitaria generada por el hamiltoniano del modelo de Ising propuesto en la sección \ref{sec:Ising}. Es necesario, entonces, tener en cuenta la unitaria que conecta a un estado de máxima entropía compatible con un estado $\rho$ aribitrario y el estado alineado en $z$. La unitaria Puede constrirse en base a los parámetros del estado $\rho$. A cada estado de dos niveles, $\rho$, lo definen tres parámetros: su pureza, $r$, y dos ángulos $\alpha$, $\beta$. La unitaria que conecta $\rho$ con el estado alineado en $z$ de pureza $z=r$ es 
\begin{equation}
  V=
  \begin{pmatrix}
      \cos{\alpha} & e^{-i\beta}\sin{\alpha}\\
      e^{i\beta}\sin{\alpha}& \cos{\alpha}\\
  \end{pmatrix}
\end{equation}
Construyendo $\mcV=V\otimes V$, podemos expresar al estado de máxima entropía de dos formas equivalentes,
\begin{align}\label{eq:MaxEntTwoExpr}
  \varrho_{max}(\rho)=\frac{1}{Z}\text{exp}(-\lambda_{3}\mcV\hat{G}_{3}\mcV^{\dag}) && \varrho_{max}(\rho)=\frac{1}{Z}\text{exp}(\sum_{i}\lambda_{i}\hat{G}_{i}).
\end{align}

\subsection{El estado máxima entropía es separable}

Sea $\rho_{z}$ un estado alineado en $z$ como en (\ref{eq:rhoz}), entonces por (\ref{eq:MaxEnt}) el estado de máxima entropía es:
\begin{equation}\label{eq:MaxEntUgly}
\varrho_{max}^{z}=\frac{\text{exp}(-\lambda_{3}\hat{G}_{3})}{\Tr[\text{exp}(-\lambda_{3}\hat{G}_{3})]}
\end{equation}
donde $\hat{G}_{3}$ se define según (\ref{eq:Gop}). Como los dos términos que componen al operador comuntan entre sí, la exponencial puede separarse:
\begin{align*}
\varrho_{max}^{z}&=\frac{1}{Z}e^{-\lambda_{3}p\sigma_{z}\otimes\Id}e^{-\lambda_{3}(1-p)\Id\otimes\sigma_{z}}\\
&=\frac{1}{Z}(e^{-\lambda_{3}p\sigma_{z}}\otimes\Id)( \Id\otimes e^{-\lambda_{3}(1-p)\sigma_{z}})\\
&=\frac{1}{Z}(e^{-\lambda_{3}p\sigma_{z}}\otimes e^{-\lambda_{3}(1-p)\sigma_{z}})\\
\end{align*}
Si se separa a la función de partición como un producto de trazas $Z=Z_{1}Z_{2}$, al estado de máxima entropía se le puede escribir como:
\begin{equation}\label{eq:MaxEntZ}
\varrho_{max}^{z}=\frac{e^{-\lambda_{3}p\sigma_{z}}}{Z_{1}} \otimes \frac{e^{-\lambda_{3}(1-p)\sigma_{z}}}{Z_{2}}
\end{equation}
Esto es válido para el estado alineado en $z$, pero retomando el resultado (\ref{eq:MaxEntTwoExpr}), el estado de máxima entropía compatible con un estado grueso arbitrario es
\begin{equation}\label{eq:MaxEntSeparable}
  \boxed{\varrho_{max}=\frac{e^{-\lambda_{3}pV\sigma_{z}V^{\dag}}}{Z_{1}} \otimes \frac{e^{-\lambda_{3}(1-p)V\sigma_{z}V^{\dag}}}{Z_{2}}}
\end{equation}
Por lo que el estado de máxima entropía compatible con un estado $\rho$ arbitrario es separable.

\subsection{El estado de máxima entropía bajo la aplicación de grano grueso}\label{sec:CG(MaxEnt)}

El problema de la ecuación (\ref{eq:MaxEntZ}) es que el estado de máxima entropía está en términos del multiplicador de Lagrange que se usó para maximizar la entropía, en lugar de estar en términos de la cantidad medible $r_{z}$. Si por alguna razón tuviéramos que resignarnos a trabajar con el estado en términos de $\lambda_{3}$, será necesario conocer la expresión del estado efectivo. Para hallarla, basta con pasar (\ref{eq:MaxEntZ}) y (\ref{eq:MaxEntSeparable}) por la aplicación de grano grueso. Si el estado grueso está alineado en $z$, entonces tiene la forma
\begin{equation}\label{eq:CG(MaxEntZ)1}
    \rho_{z}=\frac{1}{Z}\CG{\varrho_{max}^{z}}=p\frac{e^{-\lambda_{3}p\sigma_{z}}}{Z_{1}}+(1-p)\frac{e^{-\lambda_{3}(1-p)\sigma_{z}}}{Z_{2}},
\end{equation}
un estado arbitrario, por otro lado
\begin{equation}\label{eq:CG(MaxEnt)1}
  \rho=\frac{1}{Z}\CG{\varrho_{max}}=p\frac{e^{-\lambda_{3}pV\sigma_{z}V^{\dag}}}{Z_{1}}+(1-p)\frac{e^{-\lambda_{3}(1-p)V\sigma_{z}V^{z}}}{Z_{2}},
\end{equation}
Las exponenciales de la ecuación (\ref{eq:CG(MaxEntZ)1}) pueden verse como $e^{a\hat{n}\cdot \vec{\sigma}}$. Si se desarollan las series se halla
\begin{equation}\label{eq:PauliVectorExp}
    e^{a\hat{n}\cdot \vec{\sigma}}=\Id \cosh{a}+(\hat{n}\cdot \vec{\sigma})\sinh{a}
\end{equation}
así que, sustituyendo la ecuación (\ref{eq:PauliVectorExp}) en (\ref{eq:CG(MaxEntZ)1}) se encuentra la expresión del estado efectivo en términos de la base de Pauli
\begin{align*}
    \rho_{z}&=p\frac{\Id \cosh{\lambda_{3}p}-\sigma_{z}\sinh{\lambda_{3}p}}{Z_{1}}+(1-p)\frac{\Id \cosh{\lambda_{3}(1-p)}-\sigma_{z}\sinh{\lambda_{3}(1-p)}}{Z_{2}}\\
    &=p\frac{1}{2}(\Id \frac{2\cosh{\lambda_{3}p}}{Z_{1}}-\sigma_{z}\frac{2\sinh{\lambda_{3}p}}{Z_{1}})+(1-p)\frac{1}{2}(\Id \frac{2\cosh{\lambda_{3}(1-p)}}{Z_{2}}-\sigma_{z}\frac{2\sinh{\lambda_{3}(1-p)}}{Z_{2}})
\end{align*}
para que esto sea de la forma $\rho=\sum_{i}p_{i}\rho_{i}$ es necesario que $Z_{1}=2\cosh{\lambda p}$ y $Z_{2}=2\cosh{\lambda (1-p)}$ (cosa que se puede comprobar). El estado efectivo en términos de $\lambda_{3}$ es
\begin{equation}\label{eq:CG(MaxEntZ)2}
    \rho_{z}=p\frac{1}{2}(\Id+\sigma_{z}\tanh{(-\lambda p)})+(1-p)\frac{1}{2}(\Id+\sigma_{z}\tanh{(-\lambda (1-p))})
\end{equation}
Naturalmente, el caso general es la ecuación anterior como $V$ aplicada en ella para obtener el estado rotado. Ahora, si se compara este resultado con la definición de la aplicación de grano grueso y el hecho que el estado de máxima entropía es separable, encontramos que
\begin{align*}
  \varrho_{max}&=\frac{1}{2}(\Id+\sigma_{z}\tanh{\lambda p})\otimes\frac{1}{2}(\Id+\sigma_{z}\tanh{\lambda (1-p)})\\
  &  =\frac{1}{4}(\Id_{4}+\sigma_{z}\otimes\Id_{2}\tanh{\lambda p}+\sigma_{z}\otimes \Id_{2}\tanh{\lambda (1-p)}+\sigma_{z}\otimes \sigma_{z}\tanh{\lambda p}\tanh{\lambda (1-p)})
\end{align*}

\subsection{El estado de máxima entropía en términos de $r_{z}$}

La ecuación (\ref{eq:CG(MaxEntZ)2}) permite expresar la coordenada $r_{z}$ en términos del multiplicador de Lagrange complejo es
\begin{equation}\label{eq:RzTanh}
    r_{z}=p\tanh{\lambda p}+(1-p)\tanh{\lambda (1-p)}.
\end{equation}
Esta expresión la obtuve después de ahaber desarollado los párrafos de discusión que siguen. 
La forma matricial de este estado es:
\begin{equation*}
\left(
\begin{array}{cccc}
 \frac{1}{4} e^{-\lambda_{3}} \text{sech}(\lambda_{3} p)
   \text{sech}(\lambda_{3}-\lambda_{3} p) & 0 & 0 & 0 \\
 0 & \frac{e^{2 \lambda_{3}}}{\left(e^{2 \lambda_{3}
   p}+1\right) \left(e^{2 \lambda_{3}}+e^{2 \lambda_{3}
   p}\right)} & 0 & 0 \\
 0 & 0 & \frac{1}{\left(e^{2 \lambda_{3}}+1\right) e^{-2
   \lambda_{3} p}+e^{2 \lambda_{3}-4 \lambda_{3}
   p}+1} & 0 \\
 0 & 0 & 0 & \frac{1}{4} e^{\lambda_{3}}
   \text{sech}(\lambda_{3} p) \text{sech}(\lambda_{3}-\lambda_{3} p) \\
\end{array}
\right)
\end{equation*}
Hallar el valor de $\lambda_{
3}$ en términos del valor $r_{z}$ implica resolver la ecuación:
\begin{equation}\label{eq:RZ}
rz=-\frac{1}{2}\frac{\sinh(\lambda_{3})+(1-2p)\sinh((1-2p)\lambda_{3})}{\cosh(p\lambda_{3})\cosh((1-p)\lambda_{3})}
\end{equation}
No se ve ninguna forma sencilla de despejar al multiplicador de Lagrange \notaAd{la ecuación (\ref{eq:RzTanh}) y la ecuación (\ref{eq:RZ}) son completamente equivalentes, como debería de ser. La segunda siendo más fea que la primera. }. En realidad, esto solo se puede si la función $r_{z}(\lambda_{3})$ tiene inversa, y esto puede depender del parámetro $p$. Graficar la superficie (Figura \ref{fig:rzsurf}) puede aclarar algo el panorama.
\begin{figure}[h!]
\centering
\begin{subfigure}{0.475\textwidth}
  \centering
  \includegraphics[width=0.6\linewidth]{maxent/figures/LagrangeMult_lambda-8to8.png}
  \caption{$-8<\lambda_{3}<8$}
\end{subfigure}%
\begin{subfigure}{0.475\textwidth}
  \centering
  \includegraphics[width=0.6\linewidth]{maxent/figures/LagrangeMult_lambda-4to0.png}
  \caption{$-4<\lambda_{3}<0$}
\end{subfigure}
\caption{Superficie de $r_{z}$ según (\ref{eq:RZ}) para dos intervalos de $\lambda_{3}$. A valores $\lambda_{3}<0$ corresponden valores $r_{z}>0$ y viceversa.}
\label{fig:rzsurf}
\end{figure}

Después de una breve inspección se concluyen las siguientes cosas:
\begin{itemize}
\item la superficie es simétrica respecto al plano $p=0.5$
\item la superficie es antisimétrica  respecto al plano $\lambda_{3}=0$ i.e. $r_{z
}(\lambda_{3},p)=-r_{z
}(-´\lambda_{3},p)$
\item $\text{sgn}(\lambda_{3})=-\text{sgn}(r_{z})$
\end{itemize}

La simetría respecto al plano $p=0.5$ suguiere un cambio de variable $q=\abs{p-0.5}$. La ecuación (\ref{eq:RZ}) se reescribe como:
\begin{equation}\label{eq:RZq}
r_{z}=-\frac{1}{2}\frac{\sinh(\lambda_{3})+2q\sinh(2q\lambda_{3})}{\cosh((q+\frac{1}{2})\lambda_{3})\cosh((q-\frac{1}{2})\lambda_{3})}
\end{equation}
Y nos limitamos al dominio $\lambda_{3}\leq0$ y $0\leq q\leq\frac{1}{2}$. Podemos graficar la función (\ref{eq:RZq}) para diferentes valores de $q$ (Figura \ref{fig:rzinv}).
\begin{figure}[h!]
\centering
\includegraphics[width=0.6\linewidth]{maxent/figures/rz_has_inverse_lambda-4to4.png}
\caption{$r_{z}$ como función de $\lambda_{3}$ para diferentes valores de $q$. La apariencia uno a uno sugiere la existencia de una inversa.}
\label{fig:rzinv}
\end{figure}

\subsubsection{Dos soluciones particulares}

Considerando el caso $q=\frac{1}{2}$, la ecuación (\ref{eq:RZq}) se reduce a 
\begin{equation}
r_z=-\frac{1}{2}\frac{2\sinh(\lambda_{3})}{\cosh(\lambda_{3})}
\end{equation}
de manera que $\lambda_{3}=-\text{arctanh}(rz)$.

Si $q=0$, la ecuación (\ref{eq:RZq}) se reduce a
\begin{equation}
r_z=-\frac{\sinh(\lambda_{3})}{\cosh(\lambda_{3}+1)}
\end{equation}
Mathematica sugiere la solución:
\begin{equation}\label{eq:lambda0.5}
\lambda_{3}=\log\qty(\frac{1-r_{z}}{1+r_{z}}).
\end{equation}
\newpage
\section{Construcción de la dinámica}\label{sec:ch2dycon}

Ahora que hemos establecido que usaremos como modelo de grano grueso uno que incluye tanto problemas de resolución como errores de permutación, y que hemos contruído nuestra aplicación de asignación a través del Principio de Máxima Entropía, podemos preguntarnos sobre la evolución del sistema efectivo, la ``dinámica gruesa'', denotada como $\Gamma_t$. La dinámica efectiva es una aplicación dinámica que corresponde a la evolución observada por un experimentalista. Dado un estado efectivo inicial $\rho_{\ef}(0)$,
\begin{gather}
\Gamma_{t}:\mcS(\hilbert_2)\rightarrow \mcS(\hilbert_2)\nonumber\\
\rho_{\ef}(0) \mapsto \Gamma_{t}(\rho_{\ef}(0))\rlap{.}\nonumber
\end{gather}
Debido que asumimos que el estado que se propaga debido a la evolución subyacente es justamente un estado compatible con $\rho_{\ef}$, seleccionado a través de una aplicación de asignación, a la dinámica gruesa la definimos como la composición
\begin{equation}\label{eq:EffectiveDynamics}
\Gamma_t:=\mcC \circ \mcV_t \circ \mcA_{\mcC}^{\max},
\end{equation}
como se ha hecho en trabajos similares \cite{CGEmergingDynamics}.

\acnote{Párrafo iterado: reescritura}

Donde $\mcV_{t}$ es la evolución seguida por el sistema microscópico. Esta puede ser unitaria, o un canal cuántico. El siguiente diagrama ilustra la ecuación anterior,
\[\begin{tikzcd}[arrows={<-|}]
    \rho_{\ef}(0)  & \rho_{\ef}(t) \arrow{l}{\Gamma_{t}} \arrow{d}{\mcC}\\
\varrho_{\max}(0) \arrow{u}{\mcA_{\mcC}^{\max}} & \varrho_{\max}(t). \arrow{l}{\mcV_{t}}
\end{tikzcd}
\]

\acnote{Párrafo iterado: reescritura}

Debe notarse que, debido a la no invertibilidad de la aplicación de grano grueso, en general
\begin{equation}
    (\mcU_{t}\circ\mcA_{\mcC}^{\max})(\rho_{\ef}) \neq (\mcA_{\mcC}^{\max}\circ\mcC \circ \mcU_t \circ \mcA_{\mcC}^{\max})(\rho_{\ef}),\nonumber
\end{equation}
que también puede escribirse como
\begin{equation}
    \varrho_{\max}(t)\neq\mcA_{\mcC}^{\max}(\rho_{\ef}(t))\nonumber
\end{equation}
Después de todo, la maximización de la entropía se restringe de acuerdo a las observaciones experimentales, así que estados de máxima entropía que cumplan un conjunto particular de restricciones no tienen por qué satisfacer un conjunto diferente de restricciones.



En el siguiente capítulo se analizarán dinámicas efectivas generadas por diferentes dinámicas subyacentes. Si se asume que el sistema conformado por las partículas es cerrado, entonces la evolución $\mcV_{t}$ será unitaria, generada por un hamiltoniano $H$. Algunos ejemplos de dinámicas subyacentes no unitarias son los canales de ruido usuales, como el canal de \textit{despolarización}, el canal de \textit{amortiguamiento de amplitud}, o el canal de \textit{amortiguamiento de fase}. \ddnote{puse italicas en algunas partes}\acnote{enterado}


\acnote{Párrafo iterado: notas}

A diferencia de los mapas dinámicos usualmente estudiados en teoría de sistemas cuánticos abiertos, la dinámica efectiva $\Gamma_{t}$ no tiene por qué ser lineal (sí debe, por supuesto, mandar estados cuánticos a estados cuánticos), debido a que uno de los elementos de la composición que la originan no siempre es lineal: la aplicación de asignación. El estudio de las particularidades de algunas de estas dinámicas efectivas es el foco de este trabajo.



\newpage

\chapter{Dinámica efectiva}
\section{Compuertas de cómputo cuántico}

Las compuertas cuánticas son el análogo cuántico de las compuertas lógicas utilizadas en cómputo clásico. En un circuito cuántico, las compuertas permiten manipular la información (qubits). Una compuerta $U$ válida es un operador de evolución unitario que actúa sobre el espacio generado por $n$ qubits (esto es, actúa sobre $\hilbert_{2^{n}}$):
\begin{equation*}
  U:\hilbert_{2^{n}}\rightarrow\hilbert_{2^{n}}.
\end{equation*}
Compuertas de un qubit comunes icluyen a los operadores de Pauli, mientras que para dos qubits existen compuertas como el SWAP, el controlled-not (CNOT), y la compuerta de Hadamard. Como una primera aplicación del formalismo descrito en las secciones anteriores, en este trabajo se analizará la evolución efectiva bajo evoluciones subyacentes descritas por las compuertas SWAP y CNOT. 

\subsection{La compuerta cuántica SWAP}

La compuerta SWAP, $S$, actúa sobre dos qubits permutándolos. Su acción sobre la base de $\hilbert_{4}$ contruida mediante los eigenestados de $\pauli{3}\otimes\pauli{3}$ es
\begin{align*}
    \ket{0}\otimes\ket{0}\mapsto\ket{0}\otimes\ket{0}\\
    \ket{0}\otimes\ket{1}\mapsto\ket{1}\otimes\ket{0}\\
    \ket{1}\otimes\ket{0}\mapsto\ket{0}\otimes\ket{1}\\
    \ket{1}\otimes\ket{1}\mapsto\ket{1}\otimes\ket{0} \rlap{.}
\end{align*}
Si un sistema está descrito por un operador de densidad separable, $\varrho=\rho_{A}\otimes\rho_{B}$, entonces el efecto de la compuerta SWAP es 
\begin{equation*}
    S\varrho S^{\dag}=\rho_{B}\otimes\rho_{A}.
\end{equation*}
La compuerta puede representarse como una matriz de permutación
\begin{equation*}
    S=\begin{pmatrix}
        1&0&0&0\\
        0&0&1&0\\
        0&1&0&0\\
        0&0&0&1
    \end{pmatrix}.
\end{equation*}
Descrito de esta manera, el operador SWAP es un operador de evolución \textit{discreto}. Esto es, lleva un estado $\varrho$ a otro $\varrho'$ sin considerar ninguna dependencia temporal ni ningún estado intermedio. Sin embargo, como esta evolución es unitaria, puede generalizarse a cualquier tiempo de acuerdo con la ecuación (\ref{eq:UnitaryTimeDependence}). Para extenderlo, utilizamos la acción de $S$ sobre la base de $\pauli{3}\otimes\pauli{3}$ y reconocemos que el operador deja invariantes (hasta un factor) a los estados $\ket{00}$, $\ket{11}$, $\ket{+_{2}}=\frac{\ket{01}+\ket{10}}{\sqrt{2}}$ y $\ket{-_{2}}=\frac{\ket{01}-\ket{10}}{\sqrt{2}}$. Dados estos eigenestados (y eigenvalores), la descomposición espectral del operador es
\begin{equation*}
S=P(\dyad{00}+\dyad{11}+\dyad{+_{2}}-\dyad{-_{2}})P^{\dag}.
\end{equation*}
donde $P$ es la matriz formada por los eigenestados del operador, y cumple que $P^{\dag}P=\Id$. Potenciando se halla que
\begin{align*}
S^{t}&=P(\dyad{00}+\dyad{11}+\dyad{+_{2}}+(-)^{t}\dyad{-_{2}})P^{\dag}\\
&=P(\dyad{00}+\dyad{11}+\dyad{+_{2}}+e^{i \pi t}\dyad{-_{2}})P^{\dag}.
\end{align*}
La forma matricial del operador \textsc{SWAP} a un tiempo $t$ es
\begin{equation}\label{eq:SWAP(t)}
S^{t}=\begin{pmatrix}
 1 & 0 & 0 & 0 \\
 0 & \frac{1}{2}(1+e^{i \pi t}) & \frac{1}{2} (1-e^{i \pi t}) & 0 \\
 0 & \frac{1}{2}(1-e^{i \pi t}) & \frac{1}{2}(1+e^{i \pi t}) & 0 \\
 0 & 0 & 0 & 1
\end{pmatrix}=\begin{pmatrix}
  1 & 0 & 0 & 0 \\
  0 & e^{i\frac{t\pi}{2}}\cos{\frac{t\pi}{2}} & -ie^{i\frac{t\pi}{2}}\sin{\frac{t\pi}{2}} & 0 \\
  0 & -ie^{i\frac{t\pi}{2}}\sin{\frac{t\pi}{2}} & e^{i\frac{t\pi}{2}}\cos{\frac{t\pi}{2}}  & 0 \\
  0 & 0 & 0 & 1
 \end{pmatrix}
\end{equation}

\subsubsection{Evolución discreta}

Para estudiar la dinámica efectiva de una evolución subyacente descrita por el operador SWAP, primero analizaremos el caso en que no se ha introducido la dependencia temporal. Sea $\rho\in\densityspace{2}$ el estado efectivo y $\varrho_{\max}\in\densityspace{4}$ el estado de máxima entropía compatible con $\rho$ y la aplicación de grano grueso descrita en la sección \ref{sec:CH2CG} según $\mcA_{\mcC}^{\max}(\rho)=\varrho_{\max}$. Estudiaremos la asignación
\begin{equation}
  \rho\mapsto \mcC(S \varrho_{\max} S^{\dag}).
\end{equation}
Por comodidad, los cálculos se expresarán en términos de los multiplicadores de Lagrange. Como el estado de máxima entropía es separable, la acción del operador SWAP sobre este es
\begin{equation*}
  \frac{e^{p\sum_{i}\lambda_{i}\sigma_{i}}}{Z_{1}} \otimes \frac{e^{(1-p)\sum_{i}\lambda_{i}\sigma_{i}}}{Z_{2}}\mapsto\frac{e^{(1-p)\sum_{i}\lambda_{i}\sigma_{i}}}{Z_{2}}\otimes\frac{e^{p\sum_{i}\lambda_{i}\sigma_{i}}}{Z_{1}}.
\end{equation*}
El estado de la izquierda corresponde a $\varrho_{\max}(t=0)$, mientras que el de la derecha corresponde a $\varrho_{\max}(t=1)$. Con esto, basta con aplicar la aplicación de grano grueso a ambos estados para hallar a los estados efectivos inicial y final en términos de los multiplicadores de Lagrange:
\begin{equation}
\rho(0)=\frac{1}{2}[\Id+(\hat{r}_{\rho}\cdot\vec{\sigma})(p\tanh{-\lambda p}+(1-p)\tanh{-\lambda (1-p)})],
\end{equation}
\begin{equation}
\rho(t=1)=\frac{1}{2}[\Id+(\hat{r}_{\rho}\cdot\vec{\sigma})((1-p)\tanh{-\lambda p}+p\tanh{-\lambda (1-p)})].
\end{equation}
Vemos que ambos estados tienen la misma orientación (orientación significando la dirección del vector de Bloch) pero pureza distinta. Esto significa que el efecto del \textsc{SWAP} subyacente sobre la esfera de Bloch es comprimir al estado efectivo inicial con un coeficiente $\kappa_{1}$ definido según
\begin{equation}\label{eq:SWAPFactor}
  \kappa_{1}=\frac{r_{\rho(1)}}{r_{\rho(0)}}=\frac{(1-p)\tanh{\lambda p}+p\tanh{\lambda (1-p)}}{
    p\tanh{\lambda p}+(1-p)\tanh{\lambda (1-p)}}.
\end{equation}
Claro está, el factor de compresión depende del multiplicador de Lagrange, que a su vez es una función de la pureza del estado inicial. La figura \ref{fig:SWAPFactor2Drl} muestra dicha dependencia. Si la dependencia del factor de compresión en el estado efectivo inicial se denota por un superíndice, la dinámica efectiva puede escribirse como
\begin{equation}\label{eq:EffectiveSWAP1}
  \boxed{\frac{1}{2}(\Id+\vec{r}_{\rho}\cdot\vec{\sigma}) \mapsto \frac{1}{2}(\Id+\kappa_{1}^{\rho}\vec{r}_{\rho}\cdot\vec{\sigma})}
\end{equation}
\begin{figure}[h!]
  \centering
  \includegraphics[width=0.9\linewidth]{chapter3/figures_toy/ContractionFactorSWAP_2D_both.png}
  \caption{Factor de compresión $\kappa_{1}$ como función de $r_{\rho}$ (der.) y como función de $\lambda$ (izq.), para diferentes valores de $p$.}
  \label{fig:SWAPFactor2Drl}
\end{figure}

De las ecuaciones (\ref{eq:SWAPFactor}) y (\ref{eq:EffectiveSWAP1}) distinguimos lo siguiente:
\begin{itemize}
  \item Si $p=\frac{1}{2}$, entonces $\kappa_{1}^{\rho}=1$. Esto se debe a que la aplicación borrosa es invariante bajo el $\textsc{SWAP}$ si $p=0.5$. Así, todos los estado gruesos son puntos fijos bajo una evolución subyacente SWAP con aplicación de grano grueso con parámetro $p=\frac{1}{2}$.
  \item $\kappa_{1}^{\rho}$ no depende de la orientación del vector de Bloch, únicamente depende de la magnitud $r_{\rho(0)}$ y $p$.
  \item En los casos extremos, $p=1$ o $p=0$, la esfera colapsa al origen.
\end{itemize}


Como el factor de compresión depende de $\lambda$, la dinámica no es lineal. Las operaciones cuánticas de un qubit se traducen como aplicaciones afines en la esfera de Bloch. Si quisiéramos ver el proceso asociado al \textsc{SWAP} subyacente como una transformación de la forma
\begin{equation*}
  \vec{r}\rightarrow M\vec{r}+\vec{c}
\end{equation*}
en la que $\vec{c}=0$, y $M=OS$ con $O=\Id$ y $S=\kappa_{1}(\vec{r})\Id$, de tal forma que
\begin{equation*}
  \vec{r}\rightarrow \kappa_{1}(\vec{r})\vec{r}
\end{equation*}
nos daríamos cuenta que la transformación no es afín, y por esto, el proceso no puede ser descrito a través del formalismo de las operaciones cuánticas (no tiene representación en operadores de Kraus) \cite{Chuang}.

\subsubsection{Evolución continua}
Utilizando la forma dependiente del tiempo del operador $S$ dada por la ecuación (\ref{eq:SWAP(t)}), puede seguirse el mismo proceso para hallar una expresión del estado efectivo evolucionado como función del tiempo y en términos de los operadores de Lagrange:
\begin{align}
  \begin{split}
  \rho(t)=\frac{1}{2}\{\Id-(\hat{r_{\rho}}\cdot\vec{\sigma})[&((1-p)\cos^{2}{\frac{\pi t}{2}}+p\sin^{2}{\frac{\pi t}{2}})\tanh{p\lambda}\\
  &+(p\cos^{2}{\frac{\pi t}{2}}+(1-p)\sin^{2}{\frac{\pi t}{2}})\tanh{(1-p)\lambda}]\}.
  \end{split}
\end{align}

El estado efectivo inicial siendo el mismo, el estado final vuelve a tener la misma orientación, y entonces es posible calcular el factor de compresión como la razón entre las normas de los vectores de Bloch de los estados inicial y final:
\begin{equation}\label{eq:SWAPFactort}
  \kappa_{t}^{\rho}=\frac{((1-p)\cos^{2}{\frac{\pi t}{2}}+p\sin^{2}{\frac{\pi t}{2}})\tanh{\lambda p}+(p\cos^{2}{\frac{\pi t}{2}}+(1-p)\sin^{2}{\frac{\pi t}{2}})\tanh{\lambda (1-p)}}{
    p\tanh{\lambda p}+(1-p)\tanh{\lambda (1-p)}}
\end{equation}

Nuevamente, el factor de compresión (y por consiguiente, toda la evolución) depende de la pureza del estado efectivo incial, codificada en los multiplicadores de Lagrange. El efecto gradual de la evolución sobre la esfera de Bloch puede verse en la figura \ref{fig:SWAPFactorSequence}. La dinámica efectiva puede escribirse como
\begin{equation}\label{eq:EffectiveSWAPt}
  \boxed{\frac{1}{2}(\Id+\vec{r}_{\rho}\cdot\vec{\sigma}) \mapsto \frac{1}{2}(\Id+\kappa_{t}^{\rho}\vec{r}_{\rho}\cdot\vec{\sigma})}
\end{equation}

\begin{figure}[h!]
  \centering
  \begin{subfigure}{0.32\textwidth}
    \centering
    \includegraphics[width=0.9\linewidth]{chapter3/figures_toy/sphere_swapcontraction_t=0.0_z=0.9_p=0.9.png}
    \caption{$t=0$}
  \end{subfigure}%
  \begin{subfigure}{0.32\textwidth}
    \centering
    \includegraphics[width=0.9\linewidth]{chapter3/figures_toy/sphere_swapcontraction_t=0.5_z=0.9_p=0.9.png}
    \caption{$t=0.5$}
  \end{subfigure}
  \begin{subfigure}{0.32\textwidth}
    \centering
    \includegraphics[width=0.9\linewidth]{chapter3/figures_toy/sphere_swapcontraction_t=1.0_z=0.9_p=0.9.png}
    \caption{$t=1$}
  \end{subfigure}
  \caption{Efecto de la evolución subyacente si $r_{z}=0.9$, $p=0.9$. La dramática contracción se asocia a una pérdida casi total de información.}
  \label{fig:SWAPFactorSequence}
  \end{figure}

De las ecuaciones (\ref{eq:SWAPFactort}) y (\ref{eq:EffectiveSWAPt}) es posible concluir:
\begin{itemize}
  \item $\kappa_{t}^{\rho}$ es una función periódica del tiempo, y su periodo es de $T=2$ (observable en la figura \ref{fig:SWAPFactor2Dt}). Esto viene de que el operador SWAP, además de ser unitario, es hermitiano, i.e. $SS=\Id$.
  \item $\kappa_{t}^{\rho}$ es una función decreciente en $0\leq t\leq 1$.
  \item Se cumplen las observaciones hechas para el caso discreto: la esfera colapsa al origen si $p=1$ o $p=0$, y los puntos se mantienen fijos si $p=\frac{1}{2}$.
\end{itemize}

\begin{figure}[h!]
  \centering
  \includegraphics[width=0.6\linewidth]{chapter3/figures_toy/ContractionFactorSWAP_z=0.8_t=0_to_t=2.png}
  \caption{Factor de compresión $\kappa_{t}$ como función de $t$, para diferentes valores de $p$ y $r_{\rho(0)}=0.8$.}
  \label{fig:SWAPFactor2Dt}
\end{figure}

En términos del valor esperado del observable $\sigma_{3}$, la evolución del estado se da como
\begin{equation}
  \expval{\sigma_{3}(t)}=\kappa_{t}^{\rho}\expval{\pauli{3}(0)}
\end{equation}
que puede escribirse, también, como las probabilidades de que $\rho(t)$ se halle en el estado $\ket{0}$ o $\ket{1}$
 \begin{align}
  \bra{0}\rho(t)\ket{0}=\frac{1}{2}(1+\kappa_{t}^{\rho}\expval{\pauli{3}(0)}) && \bra{1}\rho(t)\ket{1}=\frac{1}{2}(1-\kappa_{t}^{\rho}\expval{\pauli{3}(0)})
 \end{align}
 donde la dependencia temporal está completamente contenida dentro del factor $\kappa_{t}^{\rho}$. 


\subsection{La compuerta cuántica controlled not}

La compuerta \textit{controlled not}, o CNOT, es el análogo cuántico de la compuerta lógica XOR. La compuerta XOR recibe como entrada dos bits, y arroja uno que puede ser $0$ si los bits de entrada tienen el mismo valor, o $1$ si tienen valores diferentes. Por otro lado, la compuerta cuántica CNOT actúa sobre un sistema de dos qubits, aplicando sobre el segundo qubit la compuerta $\sigma_{1}$ (NOT) si el primer qubit se halla en el estado $\ket{1}$, o dejándolo invariante si el primer qubit se halla en el estado $\ket{0}$. Esto es, cumple que \cite{Chuang}
\begin{align*}
    \ket{0}\otimes\ket{0}\mapsto\ket{0}\otimes\ket{0}\\
    \ket{0}\otimes\ket{1}\mapsto\ket{0}\otimes\ket{1}\\
    \ket{1}\otimes\ket{0}\mapsto\ket{1}\otimes\ket{1}\\
    \ket{1}\otimes\ket{1}\mapsto\ket{1}\otimes\ket{0} \rlap{.}
\end{align*}
En la base de los eigenestados de $\pauli{3}\otimes\pauli{3}$, la compuerta puede representarse como la matriz de permutación
\begin{equation*}
    \cnot=\begin{pmatrix}
        1&0&0&0\\
        0&1&0&0\\
        0&0&0&1\\
        0&0&1&0
    \end{pmatrix}.
\end{equation*}
Claro está, esta matriz corresponde a la evolución discreta. Para hallar la extensión a cualquier tiempo $t$, notamos que los eigenestados de este operador son $\ket{01}$, $\ket{00}$, $\ket{+}_{\cnot}$ y $\ket{-}_{\cnot}$ donde se definen
\begin{align*}
  \ket{+}_{\cnot}=\frac{\ket{10}+\ket{11}}{\sqrt{2}} & & \text{y} & & \ket{-}_{\cnot}=\frac{\ket{10}-\ket{11}}{\sqrt{2}}.
\end{align*}
Con esto es posible escribir la descomposición espectral del operador, y luego potenciarla 
\begin{align*}
\cnot=&P(\dyad{00}+\dyad{01}+\dyad{+}_{\cnot}-\dyad{-}_{\cnot})P^{\dag}\\
\Rightarrow \cnot^{t}=&P(\dyad{00}+\dyad{11}+\dyad{+}_{\cnot}+e^{i \pi t}\dyad{-}_{\cnot})P^{\dag}.
\end{align*}
La forma matricial del operador controlled not a un tiempo $t$ es análoga a la del operador SWAP:
\begin{equation}\label{eq:CNOT(t)}
\cnot^{t}=\begin{pmatrix}
  1 & 0 & 0 & 0 \\
  0 & 1 & 0 & 0 \\
  0 & 0 & e^{i\frac{t\pi}{2}}\cos{\frac{t\pi}{2}} & -ie^{i\frac{t\pi}{2}}\sin{\frac{t\pi}{2}}\\
  0 & 0 & -ie^{i\frac{t\pi}{2}}\sin{\frac{t\pi}{2}} & e^{i\frac{t\pi}{2}}\cos{\frac{t\pi}{2}}
 \end{pmatrix}
\end{equation}

El operador \textsc{CNOT} puede expandirse de la siguiente manera:
\begin{equation*}
        \cnot=\frac{1}{2}(\Id+\pauli{3}\otimes\Id+\Id\otimes\pauli{1}-\pauli{3}\otimes\pauli{1}).
\end{equation*}
Si se calcula el logaritmo de la compuerta es posible hallar el hamiltoniano generador de la unitaria,
\begin{equation*}
    H_{\cnot}=\frac{\pi}{4}\qty(\Id-\pauli{3}\otimes\Id-\Id\otimes\pauli{1}+\pauli{3}\otimes\pauli{1}),
\end{equation*}
que por ser una suma de operadores que conmutan entre sí, nos permite ver al controlled not como una aplicación consecutiva de tres unitarias diferentes
\begin{align*}
    \cnot&=e^{-i\frac{\pi}{4}\Id}e^{i\frac{\pi}{4}\pauli{3}\otimes\Id}e^{i\frac{\pi}{4}\Id\otimes\pauli{1}}e^{-i\frac{\pi}{4}\pauli{3}\otimes\pauli{1}}\\
    &=e^{-i\frac{\pi}{4}} (e^{i\frac{\pi}{4}\pauli{3}}\otimes \Id) (\Id \otimes e^{i\frac{\pi}{4}\pauli{1}}) e^{-i\frac{\pi}{4}\pauli{3}\otimes\pauli{1}}.
\end{align*}
De momento no le vi como sacarle jugo a esto, pero será útil para la extensión a tiempo arbitrario.


\subsubsection{CNOT completo efectivo}

Para estudiar la dinámica efectiva del operador $\cnot$ son particularmente útiles las expresiones (\ref{eq:rhoArhoB}). Como el estado de máxima entropía compatible con la aplicación de grano grueso puede escribirse como $\varrho_{\max}=\rho_{A}\otimes\rho_{B}$, entonces hallar el estado efectivo final es un problema de álgebra, en efecto,
\begin{align*}
    \rho(t=1)=\frac{1}{2}[&p(\rho(0)+\sigma_{3}\rho_{A}\sigma_{3}+\Tr{\sigma_{1}\rho_{B}}[\rho_{A}-\sigma_{3}\rho_{A}\sigma_{3}])\\
    &+(1-p)(\rho(0)+\sigma_{1}\rho_{B}\sigma_{1}+\Tr{\sigma_{3}\rho_{A}}[\rho_{B}-\sigma_{1}\rho_{B}\sigma_{1}])].
\end{align*}
La estructura del estado final es una consecuencia directa de la aplicación borrosa. Para entender el significado de cada uno de los términos, considérese el caso en que el aparato de medición no tiene un error asociado ($p=1$). En dicho caso, a través del principio de máxima entropía, se esperaría que el estado efectivo final fuera
\begin{equation*}
  \rho(t=1)=\rho(0)+\pauli{3}\rho(0)\pauli{3}=\frac{1}{2}(\Id+r_{3}\pauli{3})
\end{equation*}.
Este resultado viene del hecho que, en el caso $p=1$, el estado de máxima entropía es simplemente $\rho\otimes\frac{1}{2}$. Si se conociera la preparación microscópica del estado inicial, el resultado de aplicar la compuerta de controlled not seguido de trazar al segundo sistema sería
\begin{equation*}
  \rho(t=1)=
\end{equation*}
En términos del vector de Bloch, que es una forma rápida de obtener el cambio de los observables y de la esfera, la dinámica se ve como
\begin{equation*}
    \vec{r}_{\rho}=(pr_{A}+(1-p)r_{B})\hat{r}_{\rho}\mapsto\begin{pmatrix}
        r_{B}(pr_{A}(\hat{r}_{\rho,1})^2+(1-p)\hat{r}_{\rho,1})\\
        r_{B}r_{A}(p\hat{r}_{\rho,1}\hat{r}_{\rho,2}+(1-p)\hat{r}_{\rho,2}\hat{r}_{\rho,3})\\
        r_{A}(p\hat{r}_{\rho,3}+(1-p)r_{A}(\hat{r}_{\rho,3})^{2})
    \end{pmatrix}.
  \end{equation*}
  De momento tengo que ver esto en términos de matrices. El controlled not es una aplicación consecutiva de 3 unitarias, así que no debería ser demasiado complicado obtener la transformación de forma más elegante.

\subsubsection{CNOT efectivo a un tiempo arbitrario}

\section{Dinámicas separables}

En la sección \ref{sec:Ch1PartialTrace} se habló de estados separables como aquellos estados que, descritos por un operador de densidad $\rho\in\densityspace{n}$, tienen la forma
\begin{equation*}
    \rho=\rho_{A}\otimes\rho_{B}
\end{equation*}
donde $\rho_{A}\in\densityspace{m}$, $\rho_{B}\in\densityspace{l}$ y $l+m=n$. Siguiendo esta línea de pensamiento, con \textit{dinámicas separables} nos referimos a dinámicas unitarias descritas por operadores $U\in\unitaryspace{n}$ que pueden reescribirse como
\begin{equation*}
    U=U_{A}\otimes U_{B}
\end{equation*}
donde, una vez más, $U_{A}\in\unitaryspace{m}$, $U_{B}\in\unitaryspace{l}$ y $l+m=n$. Los operadores separables están compuestos por operadores que actúan de forma independiente sobre diferentes subsistemas del sistema en custión. En el caso de un sistema compuesto por dos subsistemas de dos niveles, el operador separable está compuesto por dos unitarias que actúan sobre $\hilbert_{2}$. Como el estado de máxima entropía resulta ser separable, las dinámicas separables son una muy buena primera forma de aplicar el formalismo descrito en las secciones anteriores.

\subsection{Caso general}

Consideramos una unitaria $\mcU=U_{1}\otimes U_{2}$ que evoluciona en el tiempo como $\mcU_{t}=(U_{1}\otimes U_{2})^{t}=U_{1}^{t}\otimes U_{2}^{t}$. Retomando la ecuación (\ref{eq:MaxEntSeparable}), la evolución del estado de máxima entropía es simplemente
\begin{equation*}
    \varrho_{\max}(t)=U_{1}^{t}\rho_{A}(U_{1}^{t})^{\dag}\otimes U_{2}^{t}\rho_{B} (U_{2}^{t})^{\dag}.
\end{equation*}
De esto, el estado efectivo evolucionado obtenido del principio de máxima entropía, en términos de los multiplicadores de Lagrange es
\begin{equation*}
    \rho(t)=pU_{1}^{t}\rho_{A}(U_{1}^{t})^{\dag}+(1-p)U_{2}^{t}\rho_{B} (U_{2}^{t})^{\dag}
\end{equation*}

Por supuesto, esta expresión puede expandirse en términos de exponenciales o de funciones hiperbólicas del vector de Bloch del estado efecivo incial, $\vec{r}_{\rho}$. Si se hace esto para hallar la evolución de un observable $\pauli{i}\in\obspace{2}$ se encuentra que
\begin{equation*}
    \expval{\pauli{i}(t)}=\frac{p\tanh(p\lambda)}{2}\Tr[\pauli{i}U_{1}^{t}(\paulivec{r_{\rho}})(U_{1}^{t})^{\dag}]+\frac{p\tanh((1-p)\lambda)}{2}\Tr[\pauli{i}U_{2}^{t}(\paulivec{r_{\rho}})(U_{2}^{t})^{\dag}]
\end{equation*}
En la evolución de los observables (que, depués de todo, son las cantidades que permiten describir al sistema efectivo), se observan, de nueva cuenta, dos términos: el primero está asociado a la evolución de grano grueso sin error. Como el modelo toma en cuenta únicamente a la primera partícula, se espera observar únicamente la acción de la primera parte del operador de evolución separable. El primer término contiene un coeficiente de peso $p\tanh(p\lambda)$ inducido por la aplicación de grano grueso, y el elemento de valor esperado, que depende únicamente de la dirección del vector de Bloch del estado efectivo inicial y de la primera parte del operador de evolución. En contraste, el segundo término contiene la evolución generada por $U_{2}^{t}$, y depende de $(1-p)$, la probabilidad de error. Por esto, este es el término de ruido. Por la naturaleza separable y unitaria de la evolución, se verá que el ruido son oscilaciones periódicas, pero esto es más claro si se toman en cuenta ejemplos particulares.

\subsection{Dinámica simétrica}

Comenzamos con el caso en el que la dinámica separable simétrica, esto es, de una unitaria $mcU\in\text{U}(4)$ de la forma
\begin{equation*}
    \mcU_{t}=(U \otimes U)^{t}
\end{equation*}
donde $U\in\text{U}(2)$. Se realiza el mismo proceso: aplicamos la evolución al estado de máxima entropía compatible con un conjunto de observables tomográficamente completos en $\hilbert_{2}$ y propagamos al estado con la uitaria subytacente, para luego pasarlo por la aplicación de grano grueso y recuperar el estado efectivo evolucionado. El caso de la dinámica separable es quizá el caso más sencillo, pues la simetría de la unitaria permite factorizarla:
\begin{align*}
\CG{(U^{t}\otimes U^{t})\varrho_{max}(U^{t}\otimes U^{t})^{\dag}}&=p\frac{1}{Z_{1}}e^{\lambda p U^{t}\sigma_{z}(U^t)^{\dag}}+(1-p)\frac{1}{Z_{2}}e^{\lambda (1-p)U^{t}\sigma_{z}(U^t)^{\dag}}\\
&=p\frac{1}{Z_{1}}U^{t}e^{\lambda p \sigma_{z}}(U^t)^{\dag}+(1-p)\frac{1}{Z_{2}}U^{t}e^{\lambda (1-p)\sigma_{z}}(U^t)^{\dag}\\
&=U^{t}\qty(p\frac{1}{Z_{1}}e^{\lambda p \sigma_{z}}+(1-p)\frac{1}{Z_{2}}e^{\lambda (1-p)\sigma_{z}})(U^t)^{\dag}\\
\end{align*}
La dinámica efectiva tiene la forma:
\begin{equation}
    \rho\xrightarrow{U\otimes U}U\rho U^{\dagger}
\end{equation}
Así como demostramos previamente que si el estado efectivo es puro, entonces el único estado de máxima entropía compatible es justamente el producto tensorial del estado efectivo consigo mismo, podemos mostrar que si la evolución efectiva es unitaria, la dinámica subyacente es el producto tensorial de dicha dinámica consigo misma:

\subsection{Identidad de un lado}

Retomando a expresión (\ref{eq:SeparableDynamics}), y en virtud de (\ref{eq:PauliVectorExp}), vemos que el estado efectivo inicial $\rho$ puede verse como una combinación de dos operadores con vector de Bloch con dirección $\hat{r}_{\rho}$. El vector de Bloch de $\rho$ se ve modificado al ser una de sus dos componentes (paralelas) rotada. La rotación siendo $U_{1}$ \notaAd{Creo que dependo mucho de las parametrizaciones de Bloch para entender lo que está pasadno, ¿qué sucede en el espacio de operadores de densidad?}. En general:
\begin{equation}\label{eq:SeparableDynamicsUxI}
    \rho\xrightarrow{\mcU=U_{1}\otimes \Id} p\frac{1}{Z_{1}}U_{1}e^{\lambda p\hat{r}_{\rho}\cdot\vec{\sigma}}U_{1}^{\dag}+(1-p)\frac{1}{Z_{2}}e^{\lambda(1-p)\hat{r}_{\rho}\cdot\vec{\sigma}}
\end{equation}
En términos del vector de Bloch, denotando $r_{A}=p\tan(p\lambda)$, $r_{B}=(1-p)\tan((1-p)\lambda)$, y $O$ la rotación generada por $U_{1}$:
\begin{equation}
    r\hat{r}_{\rho}\xrightarrow{\mcU=U_{1}\otimes \Id}r_{A}O\hat{r}_{\rho}+r_{B}\hat{r}_{\rho}=O(r\hat{r}_{\rho}-r_{B}\hat{r}_{\rho})+r_{B}\hat{r}_{\rho}
\end{equation}\label{eq:SeparableDynamicsUxIBloch}
El resultado es una rotación alrededor de una línea que no pasa por el origen. Una rotación de esta naturaleza puede descomponerse en una rotación a través de un eje que pasa por el origen $R$ y una traslación $T$ como $T^{-1}\circ R\circ T$. Notar que una transformación así no tendría por qué mantener a los estados dentro de la esfera de Bloch, por lo que esta debe depender del estado mismo. En efecto traslación tiene una magnitud $r_{B}$ en la dirección opuesta a la del estado (depende del estado tanto en magnitud como en dirección). Así que, aunque esto podría parecer una transformación afín, no lo es, pues depende enteramente del estado.

De (\ref{eq:SeparableDynamicsUxIBloch}) también se ve que si $U=e^{it\hat{n}\cdot\vec{\sigma}}$ entonces se ve que cualquier estado con vector de Bloch $r\hat{n}$ será invariante bajo la transformación subyacente. Lo que es mejor, esto aplica para cualquiera de los casos $U_{1}=\Id$, $U_{2}=\Id$ o $U_{1}=U_{2}$.

\subsubsection{Cambio de fase $H=\sigma_{z}$}
Considérese el hamiltoniano $H=\sigma_{z}$. La rotación en la esfera debida a la unitaria generada por el hamiltoniano es una alrededor del eje $z$. La representación de esto, y de el resultado general (\ref{eq:SeparableDynamicsUxIBloch}) puede verse en la figura \ref{fig:ZRot}.


En el espacio de operadores de densidad, esto equivale a insertar una fase relativa en el primer subsistema fino. En efecto, ignorando fases globales,
\begin{equation}
    e^{it\sigma_{z}}=\begin{pmatrix}
        1&0\\0&e^{-i2t}
    \end{pmatrix}
\end{equation}
El estado grueso siente el cambio de fase relativa en su primera componente \notaAd{¿Qué son las trazas del MaxEnt?}.
\begin{equation}
    \rho\xrightarrow{\mcU=e^{it\sigma_{z}}\otimes \Id} p\frac{1}{Z_{1}}e^{it\sigma_{z}}e^{\lambda_{3}p\hat{r}_{\rho}\cdot\vec{\sigma}}e^{-it\sigma_{z}}+(1-p)\frac{1}{Z_{2}}e^{\lambda_{3}(1-p)\hat{r}_{\rho}\cdot\vec{\sigma}}
\end{equation}

\subsubsection{Tengo que ver cómo interpreto esta $H=a\sigma_{x}+b\sigma_{y}$}
La transformación es una rotación respecto al eje $(a,b,0)$. Al aplicarse sobre el primer subsistema, el resultado es una rotación de la primera componente del estado grueso. De nuevo, la condición de normalización entre dichas componentes asegura que el estado grueso se mantenga dentro de la esfera de Bloch. 

\subsection{Régimen de error pequeño y ejemplos particulares}

El caso $p\rightarrow 1$ puede interpretarse como aquel en el que el aparato de medición tiene una baja probabilidad de fallar (poco ruido). Las evoluciones separables $U=U_{1}\otimes U_{2}$ pueden verse como una evolución gruesa $U_{1}$ más una perturbación. La perturbación, al ser unitaria, es también una rotación, así que lo que se observa es una especie de hélice. El estado precesa alrededor de una traslación de la que sería su trayectoria no perturbada. Siempre que $U_{2}=\Id$ lo que se observa es una traslación en dirección del estado con magnitud $r_{B}$. Si, por el contrario, $U_{1}=\Id$, una vez más el estado grueso se ve desplazado, para luego comenzar a girar según la rotación inducida por la unitaria $U_{2}$. 

Algunos ejemplos pueden verse en las figuras siguientes.

\section{Dinámicas especiales}

\subsection{Modelo de Ising}

\subsection{Canal de despolarización}

\subsection{Amortiguamiento de amplitud}
\newpage
\chapter{La Asignación Promedio}

\section{Definición y acercamiento}

En el contexto de los modelos de grano grueso, otras asignaciones han sido propuestas. Nos interesamos en este capítulo en la asignación promedio \cite{Macro-To-Micro}. Esta aplicación asigna a un estado efectivo $\rho \in \densityspace{n}$ un estado microscópico $\varrho_{\avg} \in \densityspace{m}$ por medio de promediar sobre el conjunto de todos los estados puros microscópicos tales que son compatibles con el estado efectivo bajo una aplicación de grano grueso en particular. Dicho conjunto de estados puros microscópicos queda definido como
\begin{equation}\label{eq:Omega}
    \Omega_{\mcC}(\rho) = \{\ket{\psi}\in\hilbert_{m}:\, \mcC(\dyad{\psi}) = \rho   \}.
\end{equation}
\acnote{Una motivación física de esta asignación es que cada que se prepara un estado microscópico que sea compatible con un estado efectivo, lo que se hace es tomar un elemento del conjunto $\Omega$.}
La aplicación de asignación promedio es el promedio sobre dicho conjunto, \ie 
\begin{equation}\label{eq:AvgMap}
    \mcA_{\avg}(\rho) = \overline{\Omega_{\mcC}(\rho)} = \int d \mu\,\, \delta(\mcC(\dyad{\psi})-\rho)\,\dyad{\psi},
\end{equation}
donde $d\mu$ es la medida de Haar sobre los estados puros de $\hilbert_{m}$. La delta de Dirac asegura que únicamente se tomen en consideración a los estados puros compatibles, y la medida de Haar, que la integración sobre dicho conjunto sea uniforme. El problema de resolver (\ref{eq:AvgMap}) de forma analítica no es para nada sencillo, y es dependiente del modelo de grano grueso en cuestión. En este trabajo se abordará a la aplicación de asignación promedio numéricamente.

Para hallar la asignación de un estado efectivo se generan estados puros en $\hilbert_{m}$ de manera uniforme. Todos aquellos estados cuyas imágenes bajo la aplicación de grano grueso se hallen a una distancia menor a $\epsilon$ del estado efectivo son incluidos en el cálculo del promedio. El mapeo de asignación promedio numérico queda entonces definido según
\begin{equation}\label{eq:AvgMapNum}
    \mcA'_{\avg}(\rho) = \overline{\Omega'_{\mcC}(\rho)} = \frac{1}{N}\sum_{i=1}^{N}\dyad{\psi},
\end{equation}
donde
\begin{equation}\label{eq:OmegaNum}
    \Omega'_{\mcC}(\rho) = \{\ket{\psi}\in\hilbert_{m}:\, \text{d}(\mcC(\dyad{\psi}),\rho)<\epsilon  \}.
\end{equation}

\section{Diferencia entre el MaxEnt y el AssMap}


No tenemos ninguna razón para asegurar que el estado de máxima entropía y el estado asignado por promedio son el mismo. En la asignación promedio se hacen dos suposiciones fuertes: primero, que el sistema microscópico se halla en un estado puro. Segundo, que todos los estados puros son igualmente probables. Aunque estas suposiciones puedan parecer razonables, son el tipo de contaminación de la información de la que habla Jaynes en su artículo. En general no hay razón para hacerlas, y no tienen por qué llevar a mejores resultados.

La figura  muestra que la fidelidad entre ambos estados parece constante siempre que $n>1000$, y que la verdadera dependencia se halla sobre el parámetro $p$. Veamos, pues, la fidelidad entre ambos estados como función de $p$, con $n=1000$.

La figura  es algo burda, y puede que requiera más puntos y observaciones, pero parece revelar que los estados tienden a ser el mismo cuando $p\rightarrow 0,1$. Asumo aquí simetría respecto a $p=0.5$. \acnote{Bastará con generar puntos entre 0.5 y 1}.

\section{Algunas dinámicas}

\subsection{Dinámicas factorizables}

\subsection{La compuerta SWAP}

\subsection{El canal de despolarización}

\subsection{El canal de estabilización}

\section{Comparación de resultados de ambas asignaciones}

\section{Conclusions}

\begin{frame}{Conclusions}
    \begin{itemize}
        \item Faulty instrumentation.
        \item Can only resolve one particle.
        \item May measure the wrong particle.
    \end{itemize}
    \begin{equation*}
        \mcC[\varrho]=\Tr_{2}[(1-p)\varrho+pS\varrho S^{\dag}].
    \end{equation*}
\end{frame}
\appendix
\chapter{Demostraciones de relaciones frecuentemente socorridas}

\subsubsection{Cuadrado de vector de pauli}
Se cumple que
\begin{align*}
    (\paulivec{r})(\paulivec{r})=&\sum_{j}r_{j}\pauli{j}\sum_{k}r_{k}\pauli{k}\\
    =&\sum_{j}r_{j}\sum_{k}r_{k}\pauli{j}\pauli{k}\\
    =&\sum_{j}r_{j}\sum_{k}r_{k}(\Id\delta_{jk}+i\epsilon_{jkl}\pauli{l})\\
    =&\sum_{j}r_{j}\sum_{k}r_{k}(\Id\delta_{jk})+i\sum_{j}r_{j}\sum_{k}r_{k}\epsilon_{jkl}\pauli{l})\\
    =&\Id
\end{align*}
Donde en la última línea se ha utilizado la antisimetría del tensor de Lévi-Civita. Se sigue que para todo entero positivo $p$
\begin{equation}\label{ap:PauliSquare}
    (\paulivec{n})^{2p}=\Id
\end{equation}

\subsubsection{Exponencial real de vector de Pauli}
Si se expande la serie de Taylor usando como argumento un vector de pauli $r\paulivec{r}$,
\begin{align*}
    e^{r\paulivec{r}}=&\sum_{k=0}^{\infty}\frac{1}{k!}(r\paulivec{r})^k\\
    =&\sum_{k}\frac{r^{2k}(\paulivec{r})^{2k}}{(2k)!}+\sum_{k}\frac{r^{2k+1}(\paulivec{r})^{2k+1}}{(2k+1)!},
\end{align*}
se puede usar (\ref{ap:PauliSquare}) para ver que
\begin{align*}
    e^{r\paulivec{r}}=&\Id\sum_{k}\frac{r^{2k}}{(2k)!}+\paulivec{r}\sum_{k}\frac{r^{2k+1}}{(2k+1)!},
\end{align*}
que, claro está, corresponde a
\begin{equation}\label{ap:PauliRealExp}
    e^{r\paulivec{r}}=\Id\cosh{r}+\paulivec{r}\sinh{r}
\end{equation}


\subsubsection{Exponencial compleja de un vector de Pauli}
Si se expande la serie de Taylor usando como argumento un vector de pauli $r\paulivec{r}$,
\begin{align*}
    e^{-ir\paulivec{r}}=&\sum_{k=0}^{\infty}\frac{1}{k!}(r\paulivec{r})^k\\
    =&\sum_{k}\frac{r^{2k}(\paulivec{r})^{2k}}{(2k)!}+\sum_{k}\frac{r^{2k+1}(\paulivec{r})^{2k+1}}{(2k+1)!},
\end{align*}
se puede usar (\ref{ap:PauliSquare}) para ver que
\begin{align*}
    e^{r\paulivec{r}}=&\Id\sum_{k}\frac{r^{2k}}{(2k)!}+\paulivec{r}\sum_{k}\frac{r^{2k+1}}{(2k+1)!},
\end{align*}
que, claro está, corresponde a
\begin{equation}\label{ap:PauliCompExp}
    e^{-ir\paulivec{r}}=\Id\cos{r}-i\paulivec{r}\sin{r}
\end{equation}
\subsubsection{Unitaria generada por un operador hermítico}
Toda unitaria de $2\times 2$ puede generarse a través de un operador hermítico $H$ como
\begin{equation*}
    U=e^{-iH}
\end{equation*}
Pues bien, como el conjunto de las matrices de Pauli, junto a la identidad, forman una base del espacio de operadores (respecto al producto interno de Hilbert-Schmidt), $H$ puede expandirse como $H=r_{0}\Id+r_{x}\pauli{x}+r_{y}+\pauli{y}+r_{z}\pauli{z}$. Si se utiliza este para construir una unitaria, desarrollando la serie se encuentra que
\begin{align*}
    e^{-iH}=&e^{-i(r_{0}\Id+r\paulivec{r})}\\
    =&e^{-i r_{0}\Id}e^{-ir\paulivec{r}}\\
    =&e^{-ir\paulivec{r}}\\
    =&\sum_{k=0}^{\infty}\frac{1}{k!}(-ir\paulivec{r})^k\\
    =&\sum_{k}(i)^{2k}(-1)^{2k}\frac{r^{2k}(\paulivec{r})^{2k}}{(2k)!}+\sum_{k}(i)^{2k+1}(-1)^{2k+1}\frac{r^{2k+1}(\paulivec{r})^{2k+1}}{(2k+1)!}\\
    =&-\Id\sum_{k}(-1)^{2k}\frac{r^{2k}}{(2k)!}-i(\paulivec{r})\sum_{k}(-1)^{2k}\frac{r^{2k+1}}{(2k+1)!}\\
    =&-\Id \cos{r}-i(\paulivec{r})\sin{r}
\end{align*}
\subsubsection{Vector de Pauli sobre vector de Pauli}
Si se aplica un vector de Pauli sobre otro se halla lo siguiente

\begin{align*}
    (\hat{n}\cdot\vec{\sigma})(\hat{m}\cdot\vec{\sigma})
\end{align*}

\subsubsection{Evolución de operador de densidad por operador unitario}
\begin{align*}
    e^{-i\omega t \paulivec{r}}\rho e^{i\omega t \paulivec{r}}=&(\Id \cos(\omega t)-i\paulivec{r} \sin(\omega t))\rho(\Id \cos(\omega t)+i\paulivec{r} \sin(\omega t))\\
    =&\rho\cos^{2}(\omega t)+(\paulivec{r})\rho(\paulivec{r})\sin^{2}(\omega t)+i\rho(\paulivec{r})\cos(\omega t)\sin(\omega t)-i(\paulivec{r})\rho \sin(\omega t)\cos(\omega t)\\
    =&\rho\cos^{2}(\omega t)+(\paulivec{r})\rho(\paulivec{r})\sin^{2}(\omega t)+i\sin(\omega t)\cos(\omega t)[\rho,\paulivec{r}]
\end{align*}
\chapter{Título del apéndice}

Contenido del apéndice

\include{biblio}
\end{document}
