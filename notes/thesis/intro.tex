\chapter{Introducción}


\acnote{Párrafo de introducción al tema}

Un buen número de áreas de la física tratan casi exclusivamente con descripciones efectivas de los sistemas que estudian. La termodinámica y la mecánica clásica tratan con los efectos observables de una realidad microscópica. La interacción entre dos superficies rugosas y la disipación de energía en forma de calor se ve como una fuerza que se opone al movimiento, y la energía cinética de miles de millones de partículas dentro de un recipiente se ve como la temperatura de un solo objeto: el gas. A este tipo de descripciones las llamamos modelos de grano grueso (\textit{coarse-grained models}), mientras que a descripciones detalladas del comportamiento microscópico de un sistema se llaman modelos de grano fino (\textit{fine-grained models}). Aunque \textit{grano grueso} no sea un término que sea comúnmente leído en los libros de texto de física, está al centro del progreso de la física como ciencia. \acnote{algo sobre descripciones cada vez menos gruesas de los modelos atómicos, algo sobre el límite del átomo de Bohr.}


\acnote{Párrafo de motivación}

Algo sobre las diferencias entre el mundo clásico y el cuántico y la transición entre ellos.

\acnote{Enunciación de la problemática}

Con todo esto en mente, nos preguntamos sobre las características de una dinámica que emerja de un modelo de grano grueso motivado no por la voluntad de simplificar un sistema físico, sino por la incapacidad de acceder a toda la información del sistema.

\acnote{Anuncio del plan}

Para abordar dicho problema, en un primer momento se introducen los conceptos que serán fundamentales a lo largo del trabajo. Principalmente, el formalismo de operadores de densidad, el Principio de Máxima Entropía, y los modelos de grano grueso. Una vez establecida una base teórica, se construirán la \textit{aplicación de asignación de máxima entropía} y el tipo de dinámicas que se estudiarán en el trabajo. Con dichas herramientas en mano, se desarrollará el estudio de las dinámicas efectivas generadas por diferentes tipos de dinámicas unitarias microscópicas. Finalmente, se compararán los resultados obtenidos con aquellos que pueden surgir del uso de otra aplicación de asignación: la \textit{aplicación de asignación promedio}.