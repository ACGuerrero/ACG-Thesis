\chapter*{Introducción}


\acnote{Párrafo de introducción al tema}

--------

\acnote{Párrafo de motivación}

-----------

\acnote{Enunciación de la problemática}

-----------

\acnote{Anuncio del plan}

Para abordar dicho problema, en un primer momento se introducen los conceptos que serán fundamentales a lo largo del trabajo. Principalmente, el formalismo de operadores de densidad, el Principio de Máxima Entropía, y los modelos de grano grueso. Una vez establecida una base teórica, se construirán la \textit{aplicación de asignación de máxima entropía} y el tipo de dinámicas que se estudiarán en el trabajo. Con dichas herramientas en mano, se desarrollará el estudio de las dinámicas efectivas generadas por diferentes tipos de dinámicas unitarias microscópicas. Finalmente, se compararán los resultados obtenidos con aquellos que pueden surgir del uso de otra aplicación de asignación: la \textit{aplicación de asignación promedio}.