 \documentclass[onecolumn,11pt]{article}
%*********
%Paquetes
%*********
\usepackage[spanish]{babel}
\usepackage[utf8]{inputenc}
\usepackage[a4paper, total={7in, 9in}]{geometry}
\usepackage{amsfonts}
\usepackage{dsfont}
\usepackage{physics}
\usepackage{xcolor}
\usepackage{tikz-cd} %para diagrama conmutatitvo
\usepackage{multicol} %para la lista de operadores
\usepackage{hyperref}
\usepackage{caption}
\usepackage{subcaption} %para las subfiguras
\title{MaxEnt}
\date{\today}
%*********
%Comandos
%*********
\newcommand{\mcU}{\mathcal{U}}
\newcommand{\mcO}{\mathcal{O}}
\newcommand{\mcI}{\mathcal{I}}
\newcommand{\mcL}{\mathcal{L}}
\newcommand{\mcS}{\mathcal{S}}
\newcommand{\hilbert}{{\sf H}}
\newcommand{\mcB}{\mathcal{B}}
\newcommand{\mcH}{\mathcal{H}}
\newcommand{\mcF}{\mathcal{F}}
\newcommand{\mcC}{\mathcal{C}}
\newcommand{\mcT}{\mathcal{T}}
\newcommand{\mcE}{\ensuremath{\mathcal{E}} }
\newcommand{\mcG}{\ensuremath{\mathcal{G}} }
\newcommand{\mcM}{\mathcal{M}}
\newcommand{\mcN}{\mathcal{N}}
\newcommand{\nnn}{\mathcal{N}}
\newcommand{\choi}{\ensuremath{\mcD} }
\newcommand{\mmm}{\mathcal{M}}
\newcommand{\sss}{\mathcal{S}}
\newcommand{\mcD}{\mathcal{D}}
\newcommand{\mcA}{\mathcal{A}}
\newcommand{\mcP}{\mathcal{P}}
\newcommand{\Complex}{\mathbb{C}} %Para escribir al espacio de hilbert complejo
\newcommand{\Id}{\mathds{1}}% Para escribir el op. indentidad con notación chida
\newcommand{\CG}[1]{\mcC\left[#1\right]}
\newcommand{\Fuzzy}[1]{\mcF\left[#1\right]}
\newcommand{\nota}[1]{{\color{red} [#1]}}
\newcommand{\notaAd}[1]{{\color{blue} [#1]}} %Notas pero mías

\begin{document}
\maketitle
\thispagestyle{empty}
Esta tarea es sobre el estado maxent.
\section{La descripción del sistema}

Pues a ver, lo que entiendo es que un bajo la suposición de máxima entropía es posible reconstruir un estado de la siguiente forma \cite{MaxEnt}:

\begin{equation}\label{eq:reconstruction}
\rho=\frac{1}{Z}\text{exp}\qty(-\sum_{i}\lambda_{i}\hat{f}_{i})
\end{equation}
donde $\lambda_{i}$ son los multiplicadores de Lagrange y $\hat{f}_{i}$ son operadores no tomográficamente completos.

\vspace{0.2cm}

Pues bien, en nuestro caso, podemos hacer mediciones sobre el estado grueso, los valores de expectación de dichos observables son de la forma:

\begin{equation}\label{eq:observable}
\sigma_{i}=\Tr{\hat{\sigma}_{i}\CG{\psi}}
\end{equation}
donde $\Psi$ es el estado fino subyacente que queremos reconstruir mediante (\ref{eq:reconstruction}). Lo que buscamos, entonces, es un observable fino que opere sobre $\psi$ pero que satisfaga (\ref{eq:observable}).

\vspace{0.2cm}


Puede que mis cuentas estén mal, o que no esté entendiendo el problema (por eso me tomé el tiempo de explicar lo que, según yo, estamos haciendo) pero me parece que $\sigma_{i}\otimes\Id$ no es el observable que queremos. O al menos no es el que corresponde a las matrices de pauli en el sistema grueso.

\vspace{0.2cm}


En las siguientes lineas detallo mis cuentas. El primer paso es desarrollar la ecuación (\ref{eq:observable}) (a partir de la segunda línea voy a tirar los sobreros de operador por comodidad):

\begin{align*}
\sigma_{i}=&\Tr{\hat{\sigma}_{i}\CG{\psi}}\\
=&\Tr{\frac{\sigma_{i}}{2}\qty[\Id+\sum_{j}\qty(p\gamma_{j0}+\gamma_{0j})\sigma_{j}]}\\
=&\frac{1}{2}\sum_{j}\Tr{\sigma_{i}\sigma_{j}(p\gamma_{j0}+(1-p)\gamma_{0j})}\\
=&\frac{1}{2}\sum_{j}(p\gamma_{j0}+(1-p)\gamma_{0j})\Tr(\sigma_{i}\sigma_{j})
\end{align*}

Entonces propongamos como operador fino a $\sigma_{i}\otimes\Id$:
\begin{align*}
\Tr{\CG{\sigma_{i}\otimes\Id\psi}}=&\Tr{\Tr_{2}{\qty[p\qty(\frac{1}{4}\sum_{jk}\gamma_{jk}\sigma_{i}\sigma_{j}\otimes\sigma_{k})+(1-p)S\qty(\frac{1}{4}\sum_{jk}\gamma_{jk}\sigma_{i}\sigma_{j}\otimes\sigma_{k})S^{\dag}]}}\\
=&\Tr{\Tr_{2}{\qty[p\qty(\frac{1}{4}\sum_{jk}\gamma_{jk}\sigma_{i}\sigma_{j}\otimes\sigma_{k})+(1-p)\qty(\frac{1}{4}\sum_{jk}\gamma_{jk}\sigma_{k}\otimes\sigma_{i}\sigma_{j})]}}\\
=&\Tr{p\qty(\frac{1}{4}\sum_{j}\gamma_{j0}\sigma_{i}\sigma_{j})+(1-p)\qty(\frac{1}{4}\sum_{jk}\gamma_{jk}\Tr(\sigma_{i}\sigma_{j})\sigma_{k})}\\
=&p\qty(\frac{1}{4}\sum_{j}\gamma_{j0}\Tr(\sigma_{i}\sigma_{j}))+(1-p)\qty(\frac{1}{4}\sum_{jk}\gamma_{jk}\Tr(\sigma_{i}\sigma_{j})\Tr(\sigma_{k}))\\
=&p\qty(\frac{1}{4}\sum_{j}\gamma_{j0}\Tr(\sigma_{i}\sigma_{j}))+(1-p)\qty(\frac{1}{2}\sum_{j}\gamma_{j0}\Tr(\sigma_{i}\sigma_{j}))\\
\end{align*}

Y como le haga, ya no hay términos $\gamma_{0j}$, así que hasta donde veo el operador $\sigma_{i}\otimes\Id$ no es el que buscamos.´
\bibliographystyle{ieeetr}
\bibliography{bibliography}


\end{document}
