 \documentclass[onecolumn,11pt]{article}
%*********
%Paquetes
%*********
\usepackage[spanish]{babel}
\usepackage[utf8]{inputenc}
\usepackage[a4paper, total={7in, 9in}]{geometry}
\usepackage{amsfonts}
\usepackage{dsfont}
\usepackage{physics}
\usepackage{xcolor}
\usepackage{tikz-cd} %para diagrama conmutatitvo
\usepackage{multicol} %para la lista de operadores
\usepackage{hyperref}
\usepackage{caption}
\usepackage{subcaption} %para las subfiguras
\title{MaxEnt}
%*********
%Comandos
%*********
\newcommand{\mcU}{\mathcal{U}}
\newcommand{\mcO}{\mathcal{O}}
\newcommand{\mcI}{\mathcal{I}}
\newcommand{\mcL}{\mathcal{L}}
\newcommand{\mcS}{\mathcal{S}}
\newcommand{\hilbert}{{\sf H}}
\newcommand{\mcB}{\mathcal{B}}
\newcommand{\mcH}{\mathcal{H}}
\newcommand{\mcF}{\mathcal{F}}
\newcommand{\mcC}{\mathcal{C}}
\newcommand{\mcT}{\mathcal{T}}
\newcommand{\mcE}{\ensuremath{\mathcal{E}} }
\newcommand{\mcG}{\ensuremath{\mathcal{G}} }
\newcommand{\mcM}{\mathcal{M}}
\newcommand{\mcN}{\mathcal{N}}
\newcommand{\nnn}{\mathcal{N}}
\newcommand{\choi}{\ensuremath{\mcD} }
\newcommand{\mmm}{\mathcal{M}}
\newcommand{\sss}{\mathcal{S}}
\newcommand{\mcD}{\mathcal{D}}
\newcommand{\mcA}{\mathcal{A}}
\newcommand{\mcP}{\mathcal{P}}
\newcommand{\Complex}{\mathbb{C}} %Para escribir al espacio de hilbert complejo
\newcommand{\Id}{\mathds{1}}% Para escribir el op. indentidad con notación chida
\newcommand{\CG}[1]{\mcC\left[#1\right]}
\newcommand{\Fuzzy}[1]{\mcF\left[#1\right]}
\newcommand{\nota}[1]{{\color{red} [#1]}}
\newcommand{\notaAd}[1]{{\color{blue} [#1]}} %Notas pero mías

\begin{document}
\maketitle
\thispagestyle{empty}
\section{Todos los estados están sobre $z$}
Considérese un estado grueso $\rho_{c}\in \mcL(\mcH_{2})$, y sea $\psi\in\mcL(\mcH_{4})$ su estado subyacente $\CG{\psi}=\rho_{c}$. Si el observador puede realizar mediciones de $\sigma_{i}$ en el estado grueso, entonces es posible reconstruir un estado de máxima entropía $\psi_{max}$ según
\begin{equation}\label{eq:MaxEnt}
\psi_{max}=\frac{1}{\Tr(e^{\sum_{i}\lambda_{i}\hat{G}_{i}})}e^{\sum_{i}\lambda_{i}\hat{G}_{i}}
\end{equation}
donde $\lambda_{i}$ son los multiplicadores de Lagrange y $\hat{G}_{i}$ son operadores no tomográficamente completos \cite{MaxEnt}. Se puede demostrar que los operadores $\hat{G}_{i}$ son
\begin{equation}\label{eq:Gop}
\hat{G}_{i}=p\sigma_{i}\otimes\Id+(1-p)\Id\otimes\sigma_{i}
\end{equation}

Idealmente, la ecuación (\ref{eq:MaxEnt}) está en términos de los valores de expectación $r_{i}=Tr(\sigma_{i}\rho_{c})$, y no de los multiplicadores de Lagrange. Para simplificar el problema, se puede asumir que $\rho_{c}$ está alineado con el eje $z$, de tal forma que el exponente en (\ref{eq:MaxEnt}) tenga únicamente un término.

\vspace{0.2cm}

En las siguientes líneas se demuestra que para todo $\rho_{c}$ con vector de Bloch $\vec{r}$ es posible reconstruir el estado de máxima entropía a través del estado de máxima entropía asociado a un estado grueso alineado en $z$, $\rho_{z}$ y la unitaria $U$ que relaciona $\rho_{c}$ y $\rho_{z}$ (para que esta exista se debe cumplir que $\abs{\vec{r}}=z$).

\vspace{0.2cm}

Sea, pues $\rho_{c}$ tal que $\rho_{c}=U\rho_{z}U^{\dag}$, con $\psi_{max}$ el estado de máxima entropía reconstruído para  $\mcU=U\otimes U$. El valor de expectación de las $\sigma_{i}$ puede calcularse como
\begin{align*}
r_{i}&=\Tr\{\sigma_{i}U\rho_{z}U^{\dag}\}\\
&=\Tr\{\sigma_{i}U\CG{\psi_{z}}U^{\dag}\}\\
&=\Tr\{\sigma_{i}\CG{\mcU\psi_{z}\mcU^{\dag}}\}\\
&=\Tr\{\Tr_{2}[\sigma_{i}\otimes\Id(p\mcU\psi_{z}\mcU^{\dag}+(1-p)S\mcU\psi_{z}\mcU^{\dag}S)]\}\\
&=\Tr[\sigma_{i}\otimes\Id(p\mcU\psi_{z}\mcU^{\dag}+(1-p)S\mcU\psi_{z}\mcU^{\dag}S)]\\
&=\Tr[p\mcU^{\dag}(\sigma_{i}\otimes\Id)\mcU\psi_{z}+(1-p)\mcU^{\dag}S(\sigma_{i}\otimes\Id) S\mcU\psi_{z}]\\
&=\Tr[p\mcU^{\dag}(\sigma_{i}\otimes\Id)\mcU\psi_{z}+(1-p)\mcU(\Id\otimes\sigma_{i})]\\
&=\Tr[\mcU^{\dag}(p\sigma_{i}\otimes\Id+(1-p)\Id\otimes\sigma_{i})\mcU\psi_{z}]\\
&=\Tr[\mcU^{\dag}\hat{G}_{i}\mcU\psi_{z}]\\
\end{align*}
De esto, se sigue que el estado de máxima entropía se recontruye como:
\begin{align*}
\psi_{max}&=\frac{1}{\Tr(e^{\sum_{i}\lambda_{i}\mcU^{\dag}\hat{G}_{i}\mcU})}e^{\sum_{i}\lambda_{i}\mcU^{\dag}\hat{G}_{i}\mcU}\\
&=\frac{1}{\Tr(e^{\mcU^{\dag}\qty(\sum_{i}\lambda_{i}\hat{G}_{i})\mcU^{\dag}})}e^{\mcU^{\dag}\qty(\sum_{i}\lambda_{i}\hat{G}_{i})\mcU^{\dag}}\\
&=\frac{1}{\Tr(\mcU^{\dag}e^{\sum_{i}\lambda_{i}\hat{G}_{i}}\mcU)}\mcU^{\dag}\qty(e^{\sum_{i}\lambda_{i}\hat{G}_{i}})\mcU\\
&=\frac{1}{\Tr(\mcU^{\dag}\qty(e^{\sum_{i}\lambda_{i}\hat{G}_{i}})\mcU)}\mcU^{\dag}\qty(e^{\sum_{i}\lambda_{i}\hat{G}_{i}})\mcU\\
&=\frac{1}{\Tr(e^{\sum_{i}\lambda_{i}\hat{G}_{i}})}\mcU^{\dag}\qty(e^{\sum_{i}\lambda_{i}\hat{G}_{i}})\mcU\\
&=\mcU^{\dag}\psi_{max}\mcU
\end{align*}
\section{El estado de máxima entropía en términos de $r_{z}$}
Si se asume que $\rho_{z}$ es un estado grueso alineado sobre el eje z, por (\ref{eq:MaxEnt}) el estado de máxima entropía es:
\begin{equation}\label{eq:MaxEntZ}
\psi_{max}=\frac{\text{exp}(\lambda_{3}\hat{G}_{3})}{\Tr[\text{exp}(\lambda_{3}\hat{G}_{3})]}
\end{equation}
donde $\hat{G}_{3}$ se define según (\ref{eq:Gop}). La forma matricial de este estado es:

\bibliographystyle{ieeetr}
\bibliography{bibliography}


\end{document}
