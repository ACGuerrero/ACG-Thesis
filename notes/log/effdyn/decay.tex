\section{Decaimientos}

\subsection{Estabilización espontánea ?}
Considérese que un sistema de $n$ partículas evoluciona siguiendo el canal
\begin{equation*}
    \mcE_{\psi,t}[\varrho]=e^{-t\mu}\varrho+(1-e^{-t \mu})\dyad{\psi}
\end{equation*}
donde $\dyad{\psi}\in \densityspace{n}$. Aplicando el modelo de grano grueso se obtiene que:
\begin{equation*}
    \rho(t)=e^{-t\mu}\rho(0)+(1-e^{-t \mu})\dyad{\psi}_{eff}
\end{equation*}
donde $\dyad{\psi}_{eff}=\mcC(\dyad{\psi})$. Obsérvese que el resultado es un canal del mismo tipo. La única diferencia siendo que el estado al que el sistema \textit{decae} (abusando del lenguaje) es la descripción efectiva de $\dyad{\psi}$. Como es natural, el comportamiento de la evolución es completamente dependiente de $\psi$, pero, al fin y al cabo, ¡se obtiene un canal cuántico!

\subsection{Amortiguamiento de amplitud}

