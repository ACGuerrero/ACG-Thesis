\section{Canal de despolarización}


Consideremos una evolución subyacente no unitaria: el canal de despolarización. El canal de despolarización para $n$ partículas está definido como
\begin{equation*}
    D_{\mu}(\varrho)=\mu\varrho+(1-\mu)\Id.
\end{equation*}
Dado un estado efectivo inicial $\rho$, su asignación de máxima entropía es $\varrho_{\max}=\rho_{A}\otimes\rho_{B}$. El resultado de pasar al estado de máxima entropía por el canal de despolarización es simplemente
\begin{equation*}
    D_{\mu}(\varrho_{\max})=\mu\varrho_{\max}+(1-\mu)\Id.
\end{equation*}
Esto significa que la dinámica efectiva es
\begin{equation*}
    \Gamma_{t=1}(\rho)=\mu\rho+(1-\mu)\Id,
\end{equation*}
esto es, ¡el mismo canal de despolarización aplicado a un sistema de menos partículas! Nótese que este resultado es muy similar al obtenido para una evolución unitaria subyacente generada por un Hamiltoniano de la forma $\mcH=H\otimes\Id+\Id\otimes H$. En dicho caso, la dinámica efectiva era, justamente, la unitaria generada por el Hamiltoniano $H$, esto como consecuencia de la simetría de la evolución: la misma para cada partícula, sin interacción. Este caso es el mismo, el canal de despolarización actúa de la misma forma sobre cada partícula, y es completamente isotrópico dentro del subespacio de cada partícula.