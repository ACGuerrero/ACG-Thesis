\section{La dinámica analítica para el caso $\mcU=\textsc{SWAP}$}

Si se escribe $Z=\Tr{\exp(\lambda_{2}\hat{G}_{3})}$, el estado de máxima entropía compatible con $\rho$ es
\begin{equation}
\varrho_{max}=\frac{1}{Z}\begin{pmatrix}
 e^{-\lambda } & 0 & 0 & 0 \\
 0 & e^{-\lambda  (2 p-1)} & 0 & 0 \\
 0 & 0 & e^{\lambda  (2 p-1)} & 0 \\
 0 & 0 & 0 & e^{\lambda } \\
\end{pmatrix}.
\end{equation}
Este operador de densidad es separable. Es decir, tiene la forma $\varrho_{max}=\varrho_{max}^{A}\otimes\varrho_{max}^{B}$.

Si se aplica el mapeo de grano gureso a este estado, el resultado es precisamente, $\rho$. Si se deja esto en términos de $\lambda_{3}$:
\begin{equation}
\rho(0)=\CG{\varrho_{max}}=\frac{1}{Z}\begin{pmatrix}
 e^{-\lambda }+(1-p)e^{\lambda  (2 p-1)}+p e^{-\lambda  (2 p-1)} & 0 \\
 0 & e^{\lambda }+p e^{\lambda  (2 p-1)}+(1-p)e^{-\lambda  (2 p-1)} \\
\end{pmatrix}.
\end{equation}
El resultado de la dinámica efectiva se puede calcular de forma directa, y es
\begin{equation}
\rho(t=1)=\frac{1}{Z}\CG{S\varrho_{z} S}=
\begin{pmatrix}
 e^{-\lambda }+p e^{\lambda  (2 p-1)}+(1-p) e^{-\lambda  (2 p-1)} & 0 \\
 0 & e^{\lambda }+p e^{-\lambda  (2 p-1)}+(1-p)e^{\lambda  (2 p-1)} \\
\end{pmatrix}
\end{equation}

\subsection{Caso $p=\frac{1}{2}$}

Dentro del estado de máxima entropía, los exponentes de los elementos no extremos de la diagonal se anulan. El resultado es
\begin{equation}
\varrho_{max}=\frac{1}{Z}
\begin{pmatrix}
 e^{-\lambda } & 0 & 0 & 0 \\
 0 & 1. & 0 & 0 \\
 0 & 0 & 1. & 0 \\
 0 & 0 & 0 & e^{\lambda } \\
\end{pmatrix}.
\end{equation}
Usando este, se obtienen tanto el estado efectivo inicial como el estado efectivo final, ambos en términos del multiplicador de Lagrange, y son
\begin{align}
\rho(0)=\CG{\varrho_{max}}=\frac{1}{Z}
\begin{pmatrix}
 1.\, +e^{-\lambda } & 0 \\
 0 & 1.\, +e^{\lambda } \\
\end{pmatrix}, && \rho(1)=\CG{S\varrho_{max} S}=\frac{1}{Z}
\begin{pmatrix}
 1.\, +e^{-\lambda } & 0 \\
 0 & 1.\, +e^{\lambda } \\
\end{pmatrix}.
\end{align}
Si se usa (\ref{eq:lambda0.5}) se halla:
\begin{equation}
\varrho_{max}=\frac{1}{Z}
\begin{pmatrix}
 \frac{1}{4} \left(r_z+1\right){}^2 & 0 & 0 & 0 \\
 0 & \frac{1}{4} \left(1-r_z^2\right) & 0 & 0 \\
 0 & 0 & \frac{1}{4} \left(1-r_z^2\right) & 0 \\
 0 & 0 & 0 & \frac{1}{4} \left(r_z-1\right){}^2 \\
\end{pmatrix}.
\end{equation}
Y recuperamos los estados inicial y final esperados:
\begin{align}
\rho(0)=\CG{\varrho_{max}}=\frac{1}{Z}\begin{pmatrix}
 \frac{1}{2} \left(r_z+1\right) & 0 \\
 0 & \frac{1}{2} \left(1-r_z\right) \\
\end{pmatrix}, && \rho(1)=\frac{1}{Z}\CG{S\varrho_{max} S}=
\begin{pmatrix}
 \frac{1}{2} \left(r_z+1\right) & 0 \\
 0 & \frac{1}{2} \left(1-r_z\right) \\
\end{pmatrix}.
\end{align}