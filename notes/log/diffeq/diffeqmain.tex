\chapter{Con algo de suerte, ecuaciones diferenciales}

Para la aplicación de grano grueso de $n$ a una paricula, el estado de máxima entropía compatible con un estado efectivo es
\begin{equation}\label{eq:GeneralMaxEnt}
    \rho_{\max}=\Motimes_{j=1}^{n}\frac{1}{Z_{j}}\text{exp}\qty(-p_{j}\sum_{i}\lambda_{i}\sigma_{i}),
\end{equation}
que puede escribirse como
\begin{equation*}
    \varrho_{\max}=\Motimes_{j=1}^{n}p_{j}\rho_{j}
\end{equation*}

\section{Asignación de Boltzmannn}



\section{Partícula preferencial}

\subsection{¿Boltzmann? ¿Qué haces tú aquí?}

Sea $p_{j}=\frac{1}{n}$ para todas las partículas. Evoluciónese a cada una localmente a través de un Hamiltoniano

\begin{equation*}
    \mcH=\sum_{j=2}^{n}\Id_{2^{j-1}}\otimes H_{j} \otimes \Id_{2^{N-j}}.
\end{equation*}

Entonces la dinámica efectiva es un anal cuántico:

\begin{equation*}
    \Gamma_{t}(\rho)=\frac{1}{N}\sum_{j}U_{j}\rho U_{j}^{\dag}
\end{equation*}

del cual se puede obtener, con algo de álgebra, una ecuación de Lindbland.

Sea la primera partícula la partícula preferencial, con $p=\frac{1}{2}$ y $p_{k}=\frac{1}{2(n-1)}$ para el resto de las partículas. El estado de máxima entropía es
\begin{equation*}
    \varrho_{\max}=\rho_{1}\Motimes_{j=2}^{n}\rho_{2}
\end{equation*}
donde $\rho_{2}=\frac{e^{(\sum \lambda_{i}\sigma_{i})/2(n-1)}}{Z_{2}}$. Si se pasa este estado por la aplicación de grano grueso se recupera
\begin{equation*}
    \CG{\varrho_{\max}}=\frac{1}{2}(\rho_{1}+\rho_{2}),
\end{equation*}
significando que 
\begin{equation*}
    \varrho_{\max}=\rho^{\otimes n}.
\end{equation*}
¡La asignación de Boltzmann! Pues bien, considérese el Hamiltoniano
\begin{equation*}
    \mcH=\sum_{j=2}^{n}\Id_{2^{j-1}}\otimes H \otimes \Id_{2^{N-j}},
\end{equation*}
que deja a la primera partícula invariante y al resto las evoluciona localmente a través de un mismo Hamiltoniano $H$. La dinámica efectiva que se recupera no es otra sino
\begin{equation*}
    \Gamma_{t}(\rho)=\frac{1}{2}(\rho+e^{-itH}\rho e^{itH})
\end{equation*}
que es un canal de desfasamiento en la dirección del Hamiltoniano $H$.
\subsection{Caso simétrico, unitaria factorizable}
Considérese, para empezar, una aplicación de grano grueso con parámetro $p=\frac{1}{2}$. Esta simetría asegura que la asignación de máxima entropía es simplemente $\rho(0)\otimes\rho(0)$ donde $\rho(0)$ es le estado efectivo incial. En este caso particular, el estado de máxima entropía compatible con los observables del estado efectivo coincide con el estado asignado por el mapeo de clonación, que a todo estado $\rho$ asigna el estado $\rho\otimes\rho$. Así pues,
\begin{equation*}
    \varrho(0)=\rho(0)\otimes\rho(0).
\end{equation*}
Además, sea $\mcU_{t}=\Id\otimes e^{-i\pauli{k}t}$ la unitaria microscópica. Bajo estos supuestos, la dinámica efectiva es:
\begin{equation*}
    \Gamma_{t}(\rho(0))=\frac{1}{2}\qty(\rho(0)+e^{-i\pauli{k}t}\rho(0)e^{i\pauli{k}t}).
\end{equation*}
Cuando $k=3$, el resultado es un canal de desfasamiento (\textit{dephasing channel}) gradual. Al tiempo $t=\frac{\pi}{2}$ se completa el desfasamiento y toda la esfera de Bloch se ve contraída al eje $z$. El efecto de esta dinámica sobre el vector de Bloch del estado efectivo es
\begin{align*}
    \vec{r}(t)=T(t)\vec{r}(0) & & \text{donde} & & T(t)=\frac{1}{2}(\Id+R_{k}(2t)).
\end{align*}
Reconociendo que la dinámica es unitaria para el sistema micróscopico, entonces, usando la asignación de máxima entropía, esta se ve descrita por la ecuación de von Neumann
\begin{equation*}
    \frac{\partial \varrho_{\max}(t)}{\partial t}=-\frac{i}{\hbar}\qty[H,\varrho_{\max}(t)].
\end{equation*}
Si se pasa la ecuación por la aplicación de grano grueso, se obtiene la derivada temporal de la matriz del estado efectivo del lado izquierdo, mientras que del lado derecho queda une expresión en términos del estado fino $\varrho_{\max}(t)$. Recordando que $\varrho_{\max}(t)=\mcU_{t}(\varrho_{\max}(0))$, y que $\varrho_{\max}(0)=\mcA_{\max}(\rho(0))$, podemos hallar una ecuación diferencial siempre que la dinámica efectiva $\Gamma_{t}$ sea invertible:
\begin{equation*}
    \frac{\partial \rho(t)}{\partial t}=\mcC\qty([H,\mcU_{t}\circ\mcA_{\max}\circ\Gamma_{t}^{-1}(\rho(t))])
\end{equation*}
En este caso particular la dinámica es invertible, en términos del vector de Bloch, la dinámica inversa es (si $k=3$)
\begin{equation*}
    T^{-1}(t)=\begin{pmatrix}
        1 & \tan(t) & 0 \\
        -\tan(t) & 1 & 0 \\
        0 & 0 & 1
     \end{pmatrix}.
\end{equation*}
Esto es apropiado. La inversa no está definida para tiempos $t=\frac{(2n+1)\pi}{2}$, que son los tiempos en los que la dinámica manda diferentes estados al mismo punto (todo se manda al eje de la dirección de compresión). Con esto, podemos hallar la ecuación diferencial para el vector de Bloch de la matriz de densidad, y es ($para k=3$)
\begin{equation*}
    \frac{\partial \vec{r}(t)}{\partial t}= \begin{pmatrix}
        -\tan(t) & 1 & 0 \\
        -1 & -\tan(t) & 0 \\
        0 & 0 & 0
     \end{pmatrix}\vec{r}.
\end{equation*}
Considerese, por otro lado, las dinámicas:
\begin{align*}
    \frac{\partial \rho(t)}{\partial t}=-\frac{i}{\hbar}[\pauli{k},\rho(t)]& & \text{y} & & \frac{\partial \rho(t)}{\partial t}=\frac{1}{2}\gamma(t)\qty(\pauli{k}\rho(t)\pauli{k}-\rho(t)).
\end{align*}
Resulta que, en términos del vector de Bloch, se traducen en 
\begin{equation*}
    sol
\end{equation*}
y reconociendo $\gamma(t)=\tan(t)$, obtenemos una ecuación diferencial para la dinámica efectiva:
\begin{equation*}
    \frac{\partial \rho(t)}{\partial t}=i[\pauli{k},\rho(t)]-\frac{1}{2}\tan(t)\qty[\pauli{k}\rho(t)\pauli{k}-\rho(t)].
\end{equation*}
Esta es una ecuación de tipo Lindbland, en la que los coeficientes de difusión son dependientes del tiempo, revelando la naturaleza no markoviana del proceso.
\section{Canal de despolarización no lineal}
Nos enfrentamos a una dinámica de la forma 
\begin{equation*}
    \rho(t)=
\end{equation*}