\chapter{Con algo de suerte, ecuaciones diferenciales}
\section{Caso simétrico, unitaria factorizable}
Considérese, para empezar, una aplicación de grano grueso con parámetro $p=\frac{1}{2}$. Esta simetría asegura que la asignación de máxima entropía es simplemente $\rho(0)\otimes\rho(0)$ donde $\rho(0)$ es le estado efectivo incial. En este caso particular, el estado de máxima entropía compatible con los observables del estado efectivo coincide con el estado asignado por el mapeo de clonación, que a todo estado $\rho$ asigna el estado $\rho\otimes\rho$. Así pues,
\begin{equation*}
    \varrho(0)=\rho(0)\otimes\rho(0).
\end{equation*}
Además, sea $\mcU_{t}=\Id\otimes e^{-i\pauli{3}t}$ la unitaria microscópica. Bajo estos supuestos, la dinámica efectiva es:
\begin{equation*}
    \Gamma_{t}(\rho(0))=\frac{1}{2}\qty(\rho(0)+e^{-i\pauli{3}t}\rho(0)e^{i\pauli{3}t})
\end{equation*}