\section{Una unitaria para gobernarlos a todos}

Un operador unitario se traduce como una rotación en la esfera de Bloch. Entonces dos operadores están relacionados por una unitaria si sus vectores de Bloch tienen la misma norma (esto es, si los operadores tienen la misma norma de Frobenius).

Así, si consideramos un estado sobre el eje z,

\begin{equation}\label{eq:rhoz}
\rho_{z}=\frac{1}{2}\qty(\Id+z\sigma_{z})
\end{equation}

entonces este está relacionado mediante una unitaria a cualquier estado $\rho\in\mcS(\hilbert_2)$ con vector de Bloch $\vec{r}$ tal que $\abs{\vec{r}}=z$.

\subsection{El grueso rotado es igual al grueso del rotado}
Sea $\rho_{z}$ como en (\ref{eq:rhoz}), y sea $\{\varrho_{i}\}_{i=1}^{N}$ un conjunto de estados compatibles que satisfacen la medida de Haar. Además, sea $\rho\in\mcS( \hilbert_2)$ un estado grueso tal que $\abs{\vec{r}}=z$.

\begin{equation}
\Rightarrow \exists U \text{ tal que } U\rho_{z}U^{\dag}=\rho
\end{equation}
\ddnote{esto está escrito un poco al revés, más bien sería así:}
\begin{equation}
U\rho_{z}U^{\dag}=\rho \ \  \forall U\in \text{SU}(2)
\end{equation}

Sea $\mcU=U\otimes U$, $\sigma_{i}'=U\sigma_{i}U^{\dag}$. En las siguientes líneas se muestra que si se rota a todo el conjunto $\{\psi_{i}\}$ usando $\mcU$, lo que se halla es el conjunto fino para el mapeo de asignación de $\rho$. Nótese que la rotación no modifica la distribución de los estados siempre que estos satisfagan la medida de Haar.

\begin{align*}
\CG{\mcU \varrho_{i} \mcU^{\dag}}=&\Tr_{2}\qty[p\mcU\varrho_{i}\mcU^{\dag}+(1-p)S\mcU \varrho_{i} \mcU^{\dag}S^{\dag}]\\
=&\Tr_{2}\qty[p\mcU\qty(\frac{1}{4}\sum_{i,j}\gamma_{ij} \sigma_i\otimes\sigma_j)\mcU^{\dag}+(1-p) S\mcU \qty(\frac{1}{4}\sum_{i,j}\gamma_{ij} \sigma_i\otimes\sigma_j) \mcU^{\dag} S^{\dag}]\\
=&\Tr_{2}\qty[p\qty(\frac{1}{4}\sum_{i,j}\gamma_{ij} \sigma_{i}'\otimes\sigma_{j}')+(1-p)S \qty(\frac{1}{4}\sum_{i,j}\gamma_{ij} \sigma_{i}'\otimes\sigma_{j}')S^{\dag}]\\
=&\Tr_{2}\qty[p\qty(\frac{1}{4}\sum_{i,j}\gamma_{ij} \sigma_{i}'\otimes\sigma_{j}')+(1-p)\qty(\frac{1}{4}\sum_{i,j}\gamma_{ij} \sigma_{j}'\otimes\sigma_{i}')]\\
=&\frac{1}{2}\qty[\Id +p\sum_{i=1}\gamma_{i,0}\sigma_{i}'+(1-p)\sum_{j=1}\gamma_{0,j}\sigma_{j}']\\
\end{align*}
\begin{align*}
\Rightarrow\CG{\mcU \varrho_{i} \mcU^{\dag}}
=&\frac{1}{2}\qty[\Id + \sum_{i=1}^3[p\gamma_{i,0}+(1-p)\gamma_{0,i}]U\sigma_{i}U^{\dag}]\\
=&U\frac{1}{2}\qty[\Id + \sum_{i=1}^3[p\gamma_{i,0}+(1-p)\gamma_{0,i}]\sigma_{i}]U^{\dag}\\
=&U\CG{\varrho_{i}}U^{\dag}\\
=&U\rho_{z}U^{\dag}\\
\end{align*}
Como esto es para todo $i$, si se rota cada $\varrho_{i}$, lo que obtenemos es un conjunto que podemos promediar para hallar el estado fino asignado a $\rho$. Aún mejor, como el promedio es lineal:
\begin{align}
\mcA[\rho]=&\frac{1}{N}\sum_{i}\mcU\varrho_{i}\mcU^{\dag}\\
=&\mcU\qty(\frac{1}{N}\sum_{i}\varrho_{i})\mcU^{\dag}\\
=&\mcU\mcA[\rho_{z}]\mcU^{\dag}
\end{align}
Esto significa que podemos hallar el mapeo de asignamiento de un estado $\rho$ a través del mapeo de asignamiento de otro estado $\rho_{z}$, siempre y cuando exista una unitaria entre estos.
\subsubsection{Para el MaxEnt es lo mismo pero no es igual}
Considérese un estado grueso $\rho\in\mcS(\hilbert_2)$. Si el observador puede realizar mediciones de $\sigma_{i}$ en el estado grueso, entonces es posible reconstruir un estado de máxima entropía $\varrho_{max}\in\mcS(\hilbert_2 \otimes \hilbert_2)$ según
\begin{equation}\label{eq:MaxEnt}
\varrho_{max}=\frac{1}{\Tr(e^{\sum_{i}\lambda_{i}\hat{G}_{i}})}e^{\sum_{i}\lambda_{i}\hat{G}_{i}}
\end{equation}
donde $\lambda_{i}$ son los multiplicadores de Lagrange y $\hat{G}_{i}$ son operadores no tomográficamente completos \cite{MaxEnt}. Se puede demostrar que los operadores $\hat{G}_{i}$ son
\begin{equation}\label{eq:Gop}
\hat{G}_{i}=p\sigma_{i}\otimes\Id+(1-p)\Id\otimes\sigma_{i}
\end{equation}

Idealmente, la ecuación (\ref{eq:MaxEnt}) está en términos de los valores de expectación $r_{i}=Tr(\sigma_{i}\rho_{c})$, y no de los multiplicadores de Lagrange. Para simplificar el problema, se puedeescoger un estado grueso $\rho_{z}$ como en (\ref{eq:rhoz}), de tal forma que el exponente en (\ref{eq:MaxEnt}) tenga únicamente un término.

\vspace{0.2cm}

En las siguientes líneas se demuestra que para todo $\rho_{c}$ con vector de Bloch $\vec{r}$ es posible reconstruir el estado de máxima entropía a través del estado de máxima entropía asociado a un estado grueso alineado en $z$, $\rho_{z}$ y la unitaria $U$ que relaciona $\rho_{c}$ y $\rho_{z}$ (para que esta exista se debe cumplir que $\abs{\vec{r}}=z$).

\vspace{0.2cm}

Sea, pues $\rho$ tal que $\rho=U\rho_{z}U^{\dag}$, con $\varrho_{max}$ el estado de máxima entropía construído para  $\mcU=U\otimes U$. El valor de expectación de las $\sigma_{i}$ puede calcularse como
\begin{align*}
r_{i}&=\Tr\{\sigma_{i}U\rho_{z}U^{\dag}\}\\
&=\Tr\{\sigma_{i}U\CG{\varrho_{max}^{z}}U^{\dag}\}\\
&=\Tr\{\sigma_{i}\CG{\mcU\varrho_{max}^{z}\mcU^{\dag}}\}\\
&=\Tr\{\Tr_{2}[\sigma_{i}\otimes\Id(p\mcU\varrho_{max}^{z}\mcU^{\dag}+(1-p)S\mcU\varrho_{max}^{z}\mcU^{\dag}S)]\}\\
&=\Tr[\sigma_{i}\otimes\Id(p\mcU\varrho_{max}^{z}\mcU^{\dag}+(1-p)S\mcU\varrho_{max}^{z}\mcU^{\dag}S)]\\
&=\Tr[p\mcU^{\dag}(\sigma_{i}\otimes\Id)\mcU\varrho_{max}^{z}+(1-p)\mcU^{\dag}S(\sigma_{i}\otimes\Id) S\mcU\varrho_{max}^{z}]\\
&=\Tr[p\mcU^{\dag}(\sigma_{i}\otimes\Id)\mcU\varrho_{max}^{z}+(1-p)\mcU^{\dag}(\Id\otimes\sigma_{i})\mcU\varrho_{max}^{z}]\\
&=\Tr[\mcU^{\dag}(p\sigma_{i}\otimes\Id+(1-p)\Id\otimes\sigma_{i})\mcU\varrho_{max}^{z}]\\
&=\Tr[\mcU^{\dag}\hat{G}_{i}\mcU\varrho_{max}^{z}]\\
\end{align*}
\notaAd{Pues las cuentas de arriba salen mucho más fácil. Las dejo así por si en algún momento las quiero visitar y que tengan todo los pasos de forma explícita. 
\begin{align*}
r_{i}&=\Tr\{\sigma_{i}U\rho_{z}U^{\dag}\}\\
&=\Tr\{\sigma_{i}U\CG{\varrho_{max}^{z}}U^{\dag}\}\\
&=\Tr\{\sigma_{i}\CG{\mcU\varrho_{max}^{z}\mcU^{\dag}}\}\\
&=\Tr[\hat{G}_{i}\mcU\varrho_{max}^{z}\mcU^{\dag}]\\
&=\Tr[\mcU^{\dag}\hat{G}_{i}\mcU\varrho_{max}^{z}]\\
\end{align*}
}

De esto, se sigue que el estado de máxima entropía asociado a $\rho_{z}$ se puede reconstruir como:
\begin{align*}
\varrho_{max}^{z}&=\frac{1}{\Tr(e^{\sum_{i}\lambda_{i}\mcU^{\dag}\hat{G}_{i}\mcU})}e^{\sum_{i}\lambda_{i}\mcU^{\dag}\hat{G}_{i}\mcU}\\
&=\frac{1}{\Tr(e^{\mcU^{\dag}\qty(\sum_{i}\lambda_{i}\hat{G}_{i})\mcU^{\dag}})}e^{\mcU^{\dag}\qty(\sum_{i}\lambda_{i}\hat{G}_{i})\mcU^{\dag}}\\
&=\frac{1}{\Tr(\mcU^{\dag}e^{\sum_{i}\lambda_{i}\hat{G}_{i}}\mcU)}\mcU^{\dag}\qty(e^{\sum_{i}\lambda_{i}\hat{G}_{i}})\mcU\\
&=\frac{1}{\Tr(\mcU^{\dag}\qty(e^{\sum_{i}\lambda_{i}\hat{G}_{i}})\mcU)}\mcU^{\dag}\qty(e^{\sum_{i}\lambda_{i}\hat{G}_{i}})\mcU\\
&=\frac{1}{\Tr(e^{\sum_{i}\lambda_{i}\hat{G}_{i}})}\mcU^{\dag}\qty(e^{\sum_{i}\lambda_{i}\hat{G}_{i}})\mcU\\
&=\mcU^{\dag}\varrho_{max}\mcU
\end{align*}
Esto significa que si somos capaces de hallar el estado de máxima entropía asociado a un estado alineado en $z$, podemos hallar el asociado a cualquier otro estado mediante:
\begin{equation}
\varrho_{max}=\mcU\varrho_{max}^{z}\mcU^{\dag}
\end{equation}

A partir de este momento, no se usará un subíndice $z$ para los estados alineados en $z$.
