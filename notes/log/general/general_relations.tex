\section{Relaciones frecuentemente socorridas}

En esta sección incluiré algunas relaciones que suelo usar, y de las que ya me he cansado de derivar una y otra vez. También son algunas relaciones que dependen de algún tipo de convención (por ejemplo, si la unitaria generada por un hermitiano lleva un signo menos o no dentro del exponente) que no he seguido de forma consistente a lo largo de mi trabajo justamente por no haberme obligado a hacerlo.
\subsubsection{Cuadrado de vector de pauli}
Se cumple que
\begin{align*}
    (\paulivec{r})(\paulivec{r})=&\sum_{j}r_{j}\pauli{j}\sum_{k}r_{k}\pauli{k}\\
    =&\sum_{j}r_{j}\sum_{k}r_{k}\pauli{j}\pauli{k}\\
    =&\sum_{j}r_{j}\sum_{k}r_{k}(\Id\delta_{jk}+i\epsilon_{jkl}\pauli{l})\\
    =&\sum_{j}r_{j}\sum_{k}r_{k}(\Id\delta_{jk})+i\sum_{j}r_{j}\sum_{k}r_{k}\epsilon_{jkl}\pauli{l})\\
    =&\Id
\end{align*}
Donde en la última línea se ha utilizado la antisimetría del tensor de Lévi-Civita. Se sigue que para todo entero positivo $p$
\begin{equation}
    (\paulivec{n})^{2p}=\Id
\end{equation}

\subsection{Unitaria generada por un operador hermítico}
Toda unitaria de $2\times 2$ puede generarse a través de un operador hermítico $H$ como
\begin{equation*}
    U=e^{-iH}
\end{equation*}
Pues bien, como el conjunto de las matrices de Pauli, junto a la identidad, forman una base del espacio de operadores (respecto al producto interno de Hilbert-Schmidt), $H$ puede expandirse como $H=r_{0}\Id+r_{x}\pauli{x}+r_{y}+\pauli{y}+r_{z}\pauli{z}$. Si se utiliza este para construir una unitaria, desarrollando la serie se encuentra que
\notaAd{NO SÉ POR QUÉ ME SALIÓ UN MENOS JUNTO A LA IDENTIDAD}
\begin{align*}
    e^{-iH}=&e^{-i(r_{0}\Id+r\paulivec{r})}\\
    =&e^{-i r_{0}\Id}e^{-ir\paulivec{r}}\\
    =&e^{-ir\paulivec{r}}\\
    =&\sum_{k=0}^{\infty}\frac{1}{k!}(-ir\paulivec{r})^k\\
    =&\sum_{k}(i)^{2k}(-1)^{2k}\frac{r^{2k}(\paulivec{r})^{2k}}{(2k)!}+\sum_{k}(i)^{2k+1}(-1)^{2k+1}\frac{r^{2k+1}(\paulivec{r})^{2k+1}}{(2k+1)!}\\
    =&-\Id\sum_{k}(-1)^{2k}\frac{r^{2k}}{(2k)!}-i(\paulivec{r})\sum_{k}(-1)^{2k}\frac{r^{2k+1}}{(2k+1)!}\\
    =&-\Id \cos{r}-i(\paulivec{r})\sin{r}
\end{align*}
\subsection{Vector de Pauli sobre vector de Pauli}
Si se aplica un vector de Pauli sobre otro se halla lo siguiente

\begin{align*}
    (\hat{n}\cdot\vec{\sigma})(\hat{m}\cdot\vec{\sigma})
\end{align*}