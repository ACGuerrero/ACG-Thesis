\section{Sobre el dominio de la dinámica y la matriz de correlaciones}

\subsection{Dominio}

El objeto de esta primera tarea era caracterizar el \textit{dominio} de una dinámica $\Gamma_{t}$, como se menciona en el artículo \cite{CGEmergingDynamics}.

Para comenzar, consideremos la dinámica  $\Gamma_{t}$ caracterizada por una evolución subyacente $\mcU_{
t}$, un estado fino inicial $\psi_{0}$, nuestra aplicación de grano grueso $\mcC$, y el estado grueso $\rho_{0}$. Si se asume que el estado grueso es puro, entonces lo siguiente es cierto:
\begin{itemize}
\item $\rho_{0}$ está completamente descrito por dos ángulos $\theta$ y $\phi$
\item $\rho_{0}$ tiene vector de Bloch $\vec{\alpha}=(\cos{\phi}\sin{\theta},\sin{\phi}\sin{\theta},\cos{\theta})$
\item $\psi_{0}=\rho_{0}\otimes\rho_{0}$
\end{itemize}
Ahora, si se expande a $\psi_{0}$ en la base de Pauli,
\begin{equation}
    \psi_{0} = \frac{1}{4}\,\sum_{i,j} \tr[\psi_{0}(\sigma_i \otimes \sigma_j)]\, \sigma_i\otimes\sigma_j
\end{equation}
es relativamente sencillo ver que:
\begin{equation}
    \CG{\psi_{0}} = \frac{1}{2}\qty[\Id + \sum_{i=1}^3[p\gamma_{i,0}+(1-p)\gamma_{0,i}]\sigma_i]
\end{equation}
donde $\gamma_{ij} \equiv \tr[\rho(\sigma_i \otimes\sigma_j)]$. 

Los coeficientes $\gamma_{ij}$ corresponden a las componentes del vector de Bloch de $\psi_{0}$, siendo $\gamma_{i,0}$ y $\gamma_{0,i}$ las componentes de los vectores de Bloch de los subsistemas de $\psi_{0}$. Como $\psi_{0}=\rho_{0}\otimes\rho_{0}$, entonces $\gamma_{i,0}=\gamma_{0,i}=\alpha_{i}$, y se recupera $\CG{\psi_{0}}=\rho_{0}$

Retomando el tema de la tarea, como se menciona en el artículo \cite{CGEmergingDynamics}, para fijar la dinámica y modificar únicamente al estado grueso inicial (esto es, deben modificarse $\vec{\alpha}$).\\
Así, en el espacio de vectores de Bloch $\vec{\gamma}$ se definen tres restricciones correspondientes a tres hiperplanos:
\begin{align}
\alpha_{i}=\Tr[\CG{\psi_{0}}\sigma_{i}]=p\gamma^{i}+(1-p)\gamma^{i+3}=r_{i}
\end{align}
Los vectores normales a estos hiperplanos son nulos en todas sus componentes, exceptuando la $i$-ésima y la $i+3$-ésima, en las que valen $p$ y $(1-p)$ respectivamente. En principio, el dominio de una $\Gamma_{t}$ fija tiene la forma:
\begin{align}
\psi=&\frac{1}{4}\left( \Id_{4}+\vec{\gamma}\cdot\vec{\sigma_{4}} \right)\\
\vec{\gamma}=&\vec{\gamma_{0}}+\sum_{i}c_{i}\vec{v_{i}}\label{eq:newgamma}
\end{align}
donde $\vec{v_{i}}$ son los vectores normales y $c_{i}$ son componentes tales que $\vec{\gamma}$ es un vector de Bloch válido. Si quisiéramos escribir la ecuación (\ref{eq:newgamma})  componente por componente el resultado sería:
\begin{equation}
\vec{\gamma}=\begin{pmatrix}
\alpha_{1}\\
\alpha_{2}\\
\alpha_{3}\\
\alpha_{1}\\
\alpha_{2}\\
\alpha_{3}\\
\alpha_{1}^{2}\\
\alpha_{1}\alpha_{2}\\
\alpha_{1}\alpha_{3}\\
\alpha_{2}\alpha_{1}\\
\alpha_{2}^{2}\\
\alpha_{2}\alpha_{3}\\
\alpha_{3}\alpha_{1}\\
\alpha_{3}\alpha_{2}\\
\alpha_{3}^{3}\\
\end{pmatrix}+\begin{pmatrix}
c_{1}p\\
c_{2}p\\
c_{3}p\\
c_{1}(1-p)\\
c_{2}(1-p)\\
c_{3}(1-p)\\
0\\
0\\
0\\
0\\
0\\
0\\
0\\
0\\
0\\
\end{pmatrix}\label{eq:newvec}
\end{equation}
Esto puede escribirse como un operador de densidad $\psi$. Tras aplicar el coarse graining el resultado es:
\begin{equation}
    \CG{\psi} = \frac{1}{2}\qty[\Id + \sum_{i=1}^3(\alpha_{i}+pc_{i}(3p-2))\sigma_i]\label{eq:newcoarse}
\end{equation}
Es importante notar que aunque el dominio (\ref{eq:newvec}) vive en un espacio quince dimensional, el movimiento se ve limitado a un espacio cinco dimensional, y la variedad descrita es de únicamente tres dimensiones (una por coeficiente $c_{i}$).

\subsection{Matriz de correlaciones}

El objeto de esta tarea es determinar los coeficientes de correlación como se ven en la ecuación (6) del artículo \cite{CGEmergingDynamics}.


Para el mapeo de coarse graining,
\begin{equation}
\CG{\psi}=\Tr_{B}[p\psi+(1-p)S\psi S^{\dag}]
\end{equation}
se hallan los siguientes operadores de Kraus:
\begin{align}
    K_{1}&=\begin{pmatrix} \sqrt{p}&0&0&0\\0&0&\sqrt{p}&0\end{pmatrix} & K_{2}&=\begin{pmatrix} \sqrt{p}&0&0&0\\0&0&\sqrt{p}&0\end{pmatrix} \\ K_{3}&=\begin{pmatrix} \sqrt{1-p}&0&0&0\\0&\sqrt{1-p}&0&0\end{pmatrix} & K_{4}&=\begin{pmatrix} 0&0&\sqrt{1-p}&0\\0&0&0&\sqrt{1-p}\end{pmatrix}
\end{align}
Utilizando la forma de bloque para el operador que actúa en el sistema extendido dada en \cite{Chuang}, el unitario resultante es:
\begin{equation}
V=\begin{pmatrix}
    \sqrt{p} & 0 & 0 & 0 & -\sqrt{1-p} & 0 & 0 & 0 \\
    0 & \sqrt{p} & 0 & 0 & 0 & -\sqrt{1-p} & 0 & 0 \\
    \sqrt{1-p} & 0 & 0 & 0 & \sqrt{p} & 0 & 0 & 0 \\
    0 & 0 & \sqrt{1-p} & 0 & 0 & 0 & \sqrt{p} & 0 \\
    0 & 0 & \sqrt{p} & 0 & 0 & 0 & -\sqrt{1-p} & 0 \\
    0 & 0 & 0 & \sqrt{p} & 0 & 0 & 0 & -\sqrt{1-p} \\
    0 & \sqrt{1-p} & 0 & 0 & 0 & \sqrt{p} & 0 & 0 \\
    0 & 0 & 0 & \sqrt{1-p} & 0 & 0 & 0 & \sqrt{p} \\
\end{pmatrix}
\end{equation}
Debí verificar que esta funcionaba cuando lo hice, todo lo demás queda mal porque la unitaria no quedó.
