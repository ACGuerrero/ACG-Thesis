\documentclass[onecolumn,11pt]{article}
%*********
%Paquetes
%*********
\usepackage[spanish]{babel}
\usepackage[utf8]{inputenc}
\usepackage[a4paper, total={7in, 9in}]{geometry}
\usepackage{amsfonts}
\usepackage{dsfont}
\usepackage{physics}
\usepackage{xcolor}
\usepackage{tikz-cd} %para diagrama conmutatitvo
\usepackage{multicol} %para la lista de operadores
\usepackage{hyperref}
\title{Dominio de $\Gamma_{t}$}
\date{\today}
%*********
%Comandos
%*********
\newcommand{\mcU}{\mathcal{U}}
\newcommand{\mcO}{\mathcal{O}}
\newcommand{\mcI}{\mathcal{I}}
\newcommand{\mcL}{\mathcal{L}}
\newcommand{\mcS}{\mathcal{S}}
\newcommand{\hilbert}{{\sf H}}
\newcommand{\mcB}{\mathcal{B}}
\newcommand{\mcH}{\mathcal{H}}
\newcommand{\mcF}{\mathcal{F}}
\newcommand{\mcC}{\mathcal{C}}
\newcommand{\mcT}{\mathcal{T}}
\newcommand{\mcE}{\ensuremath{\mathcal{E}} }
\newcommand{\mcG}{\ensuremath{\mathcal{G}} }
\newcommand{\mcM}{\mathcal{M}}
\newcommand{\mcN}{\mathcal{N}}
\newcommand{\nnn}{\mathcal{N}}
\newcommand{\choi}{\ensuremath{\mcD} }
\newcommand{\mmm}{\mathcal{M}}
\newcommand{\sss}{\mathcal{S}}
\newcommand{\mcD}{\mathcal{D}}
\newcommand{\mcA}{\mathcal{A}}
\newcommand{\mcP}{\mathcal{P}}
\newcommand{\Complex}{\mathbb{C}} %Para escribir al espacio de hilbert complejo
\newcommand{\Id}{\mathds{1}}% Para escribir el op. indentidad con notación chida
\newcommand{\CG}[1]{\mcC\left[#1\right]}
\newcommand{\Fuzzy}[1]{\mcF\left[#1\right]}
\newcommand{\nota}[1]{{\color{red} [#1]}}
\newcommand{\notaAd}[1]{{\color{blue} [#1]}} %Notas pero mías

\begin{document}
\maketitle
\thispagestyle{empty}
El objeto de esta primera tarea era caracterizar el \textit{dominio} de una dinámica $\Gamma_{t}$, como se menciona en el artículo \cite{CGEmergingDynamics}.

\section{Tres planos}
Al fijar $\Gamma_{t}$, lo que puede variarse es al estado inicial $\psi_{0}$. Si se expresa:
\begin{equation}
    \psi_{0} = \frac{1}{4}\,\sum_{i,j} \tr[\psi_{0}(\sigma_i \otimes \sigma_j)]\, \sigma_i\otimes\sigma_j.
\end{equation}
Es relativamente sencillo ver que:
\begin{equation}
    \CG{\psi_{0}} = \frac{1}{2}\qty[\Id + \sum_{i=1}^3[p\gamma_{i,0}+(1-p)\gamma_{0,i}]\sigma_i],
\end{equation}
donde $\gamma_{ij} \equiv \tr[\rho(\sigma_i \otimes\sigma_j)]$. Los coeficientes $\gamma_{ij}$ pueden acomodarse, claro, como el vector de Bloch de $\psi_{0}$. En este espacio de vectores de Bloch se definen tres restricciones correspondientes a tres hiperplanos:
\begin{align}
\alpha_{i}=\Tr[\CG{\psi_{0}}\sigma_{i}]=p\gamma_{0}^{i}+(1-p)\gamma_{0}^{i+3}
\end{align}
donde se ha escogido un ordenamiento particular para el vector $\vec{\gamma_{0}}$ de compontentes $\gamma_{0}^{i}$. Los vectores normales a estos hiperplanos son nulos en todas sus componentes, exceptuando la $i$-ésima y la $i+3$-ésima, en las que valen $p$ y $(1-p)$ respectivamente. En principio, el dominio de una $\gamma_{t}$ fija tiene la forma:
\begin{align}
\psi=&\frac{1}{4}\left( \Id_{4}+\vec{\gamma}\cdot\vec{\sigma_{4}} \right)\\
\vec{\gamma}=&\vec{\gamma_{0}}+\sum_{i}c_{i}\vec{v_{i}}
\end{align}
donde $\vec{v_{i}}$ son los vectores normales y $c_{i}$ son componentes tales que $\vec{\gamma}$ es un vector de Bloch válido. Para asegurar esto nos metemos en las broncas del pructo estrella.

Sinceramente, no entiendo por qué en el artículo se afirma que el dominio es convexo.
\bibliographystyle{ieeetr}
\bibliography{bibliography}

\end{document}
