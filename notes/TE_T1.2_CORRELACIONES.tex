\documentclass[onecolumn,11pt]{article}
%*********
%Paquetes
%*********
\usepackage[spanish]{babel}
\usepackage[utf8]{inputenc}
\usepackage[a4paper, total={7in, 9in}]{geometry}
\usepackage{amsfonts}
\usepackage{dsfont}
\usepackage{physics}
\usepackage{xcolor}
\usepackage{tikz-cd} %para diagrama conmutatitvo
\usepackage{multicol} %para la lista de operadores
\usepackage{hyperref}
\title{Matriz de correlaciones $\Theta$}
\date{\today}
%*********
%Comandos
%*********
\newcommand{\mcU}{\mathcal{U}}
\newcommand{\mcO}{\mathcal{O}}
\newcommand{\mcI}{\mathcal{I}}
\newcommand{\mcL}{\mathcal{L}}
\newcommand{\mcS}{\mathcal{S}}
\newcommand{\hilbert}{{\sf H}}
\newcommand{\mcB}{\mathcal{B}}
\newcommand{\mcH}{\mathcal{H}}
\newcommand{\mcF}{\mathcal{F}}
\newcommand{\mcC}{\mathcal{C}}
\newcommand{\mcT}{\mathcal{T}}
\newcommand{\mcE}{\ensuremath{\mathcal{E}} }
\newcommand{\mcG}{\ensuremath{\mathcal{G}} }
\newcommand{\mcM}{\mathcal{M}}
\newcommand{\mcN}{\mathcal{N}}
\newcommand{\nnn}{\mathcal{N}}
\newcommand{\choi}{\ensuremath{\mcD} }
\newcommand{\mmm}{\mathcal{M}}
\newcommand{\sss}{\mathcal{S}}
\newcommand{\mcD}{\mathcal{D}}
\newcommand{\mcA}{\mathcal{A}}
\newcommand{\mcP}{\mathcal{P}}
\newcommand{\Complex}{\mathbb{C}} %Para escribir al espacio de hilbert complejo
\newcommand{\Id}{\mathds{1}}% Para escribir el op. indentidad con notación chida
\newcommand{\CG}[1]{\mcC\left[#1\right]}
\newcommand{\Fuzzy}[1]{\mcF\left[#1\right]}
\newcommand{\nota}[1]{{\color{red} [#1]}}
\newcommand{\notaAd}[1]{{\color{blue} [#1]}} %Notas pero mías

\begin{document}
\maketitle
\thispagestyle{empty}
El objeto de esta tarea es determinar los coeficientes de correlación como se ven en la ecuación (6) del artículo \cite{CGEmergingDynamics}.

\section{Operadores de Kraus}

Para el mapeo de coarse graining,
\begin{equation}
\CG{\psi}=\Tr_{B}[p\psi+(1-p)S\psi S^{\dag}]
\end{equation}
se hallan los siguientes operadores de Kraus:
\begin{align}
    K_{1}&=\begin{pmatrix} \sqrt{p}&0&0&0\\0&0&\sqrt{p}&0\end{pmatrix} & K_{2}&=\begin{pmatrix} \sqrt{p}&0&0&0\\0&0&\sqrt{p}&0\end{pmatrix} \\ K_{3}&=\begin{pmatrix} \sqrt{1-p}&0&0&0\\0&\sqrt{1-p}&0&0\end{pmatrix} & K_{4}&=\begin{pmatrix} 0&0&\sqrt{1-p}&0\\0&0&0&\sqrt{1-p}\end{pmatrix}
\end{align}
Utilizando la forma de bloque para el operador que actúa en el sistema extendido dada en \cite{Chuang}, el unitario resultante es:
\begin{equation}
V=\begin{pmatrix}
\sqrt{p}&0&0&0&0&0&0&-\sqrt{1 - p}\\
0&0&\sqrt{p}&0&0&0&-\sqrt{1 - p}&0\\
0&\sqrt{p}&0&0&0&-\sqrt{1 - p}&0&0\\
0&0&0&\sqrt{p}&-\sqrt{1 - p}&0&0&0\\
\sqrt{1 - p}&0&0&0&0&0&0&\sqrt{p}\\
0&\sqrt{1 - p}&0&0&0&\sqrt{p}&0&0\\
0&0&\sqrt{1 - p}&0&0&0&\sqrt{p}&0\\
0&0&0&\sqrt{1 - p}&\sqrt{p}&0&0&0 
\end{pmatrix}
\end{equation}
Debí verificar que esta funcionaba cuando lo hice, todo lo demás queda mal porque la unitaria no quedó.
\bibliographystyle{ieeetr}
\bibliography{bibliography}

\end{document}
