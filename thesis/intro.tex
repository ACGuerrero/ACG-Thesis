\chapter{Introducción}


\acnote{Párrafo de introducción al tema}

Un buen número de áreas de la física tratan casi exclusivamente con descripciones efectivas de los sistemas que estudian. Por ejemplo, la mecánica estadística y la mecánica clásica tratan con los efectos observables de una realidad microscópica. La interacción entre dos superficies rugosas y la disipación de energía en forma de calor se ve como una fuerza que se opone al movimiento, y la energía cinética de miles de millones de partículas dentro de un recipiente se ve como la temperatura de un solo objeto: el gas. A este tipo de descripciones las llamamos modelos de grano grueso (\textit{coarse-grained models}), mientras que a descripciones detalladas del comportamiento microscópico de un sistema se llaman modelos de grano fino (\textit{fine-grained models}). Aunque \textit{grano grueso} no sea un término que sea comúnmente leído en los libros de texto de física, está al centro del progreso de la física como ciencia. En efecto, la estructura gruesa atómica (\textit{atomic gross structure}), en la únicamente se estudia la interacción de Coulomb entre electrones puntuales y un núcleo también puntual, puede verse como una descripción efectiva de la estructura fina del átomo, en la que sí se toma en cuenta la interacción espín-órbita, así como una corrección relativista a la energía cinética y el término de Darwin. Esta, a su ves, puede verse como una descripción efectiva de la estructura hiperfina del átomo, que sí considera al espín nuclear y las interacciones que este tiene con el resto del sistema.

\acnote{Párrafo de motivación}

Aunque la mecánica cuántica describe con alta precisión los fenómenos del mundo microscópico, los objetos macroscópicos son mejor descritos por la física clásica. De acuerdo con el principio de correspondencia de Bohr, la mecánica cuántica debería poder realizar las mismas predicciones que la física clásica una vez que los sistemas se vuelven lo suficientemente grandes. La transición cuántica-clásica es objeto de estudio hasta el día de hoy. En este contexto, un modelo de grano grueso puede utilizarse para describir un número arbitrariamente grande de partículas utilizando un número pequeño de variables. De esta manera, los modelos de grano grueso podrían permitir echar una mirada a la transición entre el mundo cuántico y el mundo clásico.

\acnote{Enunciación de la problemática}

Con todo esto en mente, nos preguntamos sobre las características de una dinámica que emerja de un modelo de grano grueso inducido por la incapacidad de un observador en resolver de forma precisa la información de un sistema de un número arbitrario de partículas, asumiendo que el observador conoce tanto el número de partículas como la evolución experimentada por el sistema microscópico.

\acnote{Anuncio del plan}

Para abordar dicho problema, en un primer momento se introducen los conceptos que serán fundamentales a lo largo del trabajo. Principalmente, el formalismo de operadores de densidad, el Principio de Máxima Entropía, y los modelos de grano grueso. Una vez establecida una base teórica, se introducirá la aplicación de grano grueso que se utilizará, se construirán la \textit{aplicación de asignación de máxima entropía}, y se definirá matemáticamente a la dinámica efectiva como la evolución seguida por el sistema efectivo. Finalmente, con dichas herramientas en mano, se desarrollará el estudio de las dinámicas efectivas generadas por diferentes tipos de dinámicas microscópicas. Se considerarán evoluciones unitarias locales, algunas compuertas de uso común en cómputo cuántico, canales de Pauli, la cadena de Ising, entre otros. De manera adicional, se incluirá un apéndice comparando los resultados obtenidos a través de la aplicación de asignación de máxima entropía con aquellos que pueden surgir del uso de otra aplicación de asignación: la \textit{aplicación de asignación promedio}.