\section*{Resumen}

En el presente trabajo se utiliza el Principio de Máxima Entropía para crear una aplicación de asignación que permita definir un estado microscópico dada una descripción \textit{gruesa} de un sistema conformado por un número arbitrario de qubits. En particular, la descripción gruesa corresponde a la obtenida de un aparato de medición al que se le asocian dos tipos de errores: por un lado es capaz de resolver únicamente una de las partículas, y por otro lado, existe una probabilidad no nula de que mida una partícula diferente a la de interés. A través de la asignación de Máxima Entropía, y asumiendo que se conoce la evolución microscópica, se estudiaron diferentes dinámicas efectivas. Esto es, el cambio observable en la descripción gruesa del sistema. Algunas de las dinámicas estudiadas resultaron ser no lineales y dependientes en el estado efectivo inicial, cosa que ofrece un mecanismo prototípico para el surgimiento de la dinámica no lineal a partir de la mecánica cuántica. En particular, la dinámica inducida por evoluciones unitarias locales experimenta una pérdida de la periodicidad de la dinámica subyacente, pero muestra convergencia a una dinámica unitaria cuando el número de partículas crece. Otros casos, como el de los canales de desfasamiento, el canal de despolarización, y un canal de estabilización, la dinámica efectiva resulta ser del mismo tipo que la dinámica subyacente y por lo mismo, un canal cuántico. Se encontró también que las dinámicas son lineales en el caso extremo en el que la probabilidad de error es nula, debido a que el único elemento no lineal de la dinámica, la aplicación de asignación, se hace lineal. Se estudió también, numéricamente, una cadena de Ising de hasta nueve partículas. Adicionalmente se compararon los resultados obtenidos de la aplicación de asignación de máxima entropía y otro tipo de aplicación, la aplicación de asignación promedio, que asigna a un estado efectivo el promedio de todos los estados microscópicos puros compatibles.
