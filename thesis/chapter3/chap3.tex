\chapter{Dinámica efectiva}\label{sec:chapter3}

En este capítulo se estudiarán las dinámicas efectivas generadas \acnote{no sé si es correcto decir que las dinámicas efectivas son "generadas" por las microscópicas} \ddnote{cambiemos por inducidas} por diferentes tipos de dinámicas microscópicas. Primero, motivados por el hecho de que el estado de máxima entropía es factorizable, se analizarán dinámicas generadas por hamiltonianos que no tengan partes de interacción. Luego se estudiarán compuertas de dos qubits bien conocidas en cómputo cuántico: la compuerta SWAP y la compuerta CNOT \ddnote{(que si presentan interacciones entre las dos partículas)}. Finalmente se profundizará en dinámicas más específicas, \ddnote*{como una cadena de partículas con interacción Ising}{como un modelo de Ising uno-dimensional}, así como evoluciones no unitarias, como el canal de despolarización, el canal de amortiguamiento de amplitud, entre otras.

\section{Dinámicas separables}

En la sección \ref{sec:Ch1PartialTrace} se habló de estados separables como aquellos estados que, descritos por un operador de densidad $\rho\in\densityspace{n}$, tienen la forma
\begin{equation*}
    \rho=\rho_{A}\otimes\rho_{B}
\end{equation*}
donde $\rho_{A}\in\densityspace{m}$, $\rho_{B}\in\densityspace{l}$ y $l+m=n$. Siguiendo esta línea de pensamiento, con \textit{dinámicas separables} nos referimos a dinámicas unitarias descritas por operadores $U\in\unitaryspace{n}$ que pueden reescribirse como
\begin{equation*}
    U=U_{A}\otimes U_{B}
\end{equation*}
donde, una vez más, $U_{A}\in\unitaryspace{m}$, $U_{B}\in\unitaryspace{l}$ y $l+m=n$. Los operadores separables están compuestos por operadores que actúan de forma independiente sobre diferentes subsistemas del sistema en custión. En el caso de un sistema compuesto por dos subsistemas de dos niveles, el operador separable está compuesto por dos unitarias que actúan sobre $\hilbert_{2}$. Como el estado de máxima entropía resulta ser separable, las dinámicas separables son una muy buena primera forma de aplicar el formalismo descrito en las secciones anteriores.

\subsection{Caso general}

Consideramos una unitaria $\mcU=U_{1}\otimes U_{2}$ que evoluciona en el tiempo como $\mcU_{t}=(U_{1}\otimes U_{2})^{t}=U_{1}^{t}\otimes U_{2}^{t}$. Retomando la ecuación (\ref{eq:MaxEntSeparable}), la evolución del estado de máxima entropía es simplemente
\begin{equation*}
    \varrho_{\max}(t)=U_{1}^{t}\rho_{A}(U_{1}^{t})^{\dag}\otimes U_{2}^{t}\rho_{B} (U_{2}^{t})^{\dag}.
\end{equation*}
De esto, el estado efectivo evolucionado obtenido del principio de máxima entropía, en términos de los multiplicadores de Lagrange es
\begin{equation*}
    \rho(t)=pU_{1}^{t}\rho_{A}(U_{1}^{t})^{\dag}+(1-p)U_{2}^{t}\rho_{B} (U_{2}^{t})^{\dag}
\end{equation*}

Por supuesto, esta expresión puede expandirse en términos de exponenciales o de funciones hiperbólicas del vector de Bloch del estado efecivo incial, $\vec{r}_{\rho}$. Si se hace esto para hallar la evolución de un observable $\pauli{i}\in\obspace{2}$ se encuentra que
\begin{equation*}
    \expval{\pauli{i}(t)}=\frac{p\tanh(p\lambda)}{2}\Tr[\pauli{i}U_{1}^{t}(\paulivec{r_{\rho}})(U_{1}^{t})^{\dag}]+\frac{p\tanh((1-p)\lambda)}{2}\Tr[\pauli{i}U_{2}^{t}(\paulivec{r_{\rho}})(U_{2}^{t})^{\dag}]
\end{equation*}
En la evolución de los observables (que, depués de todo, son las cantidades que permiten describir al sistema efectivo), se observan, de nueva cuenta, dos términos: el primero está asociado a la evolución de grano grueso sin error. Como el modelo toma en cuenta únicamente a la primera partícula, se espera observar únicamente la acción de la primera parte del operador de evolución separable. El primer término contiene un coeficiente de peso $p\tanh(p\lambda)$ inducido por la aplicación de grano grueso, y el elemento de valor esperado, que depende únicamente de la dirección del vector de Bloch del estado efectivo inicial y de la primera parte del operador de evolución. En contraste, el segundo término contiene la evolución generada por $U_{2}^{t}$, y depende de $(1-p)$, la probabilidad de error. Por esto, este es el término de ruido. Por la naturaleza separable y unitaria de la evolución, se verá que el ruido son oscilaciones periódicas, pero esto es más claro si se toman en cuenta ejemplos particulares.

\subsection{Dinámica simétrica}

Comenzamos con el caso en el que la dinámica separable simétrica, esto es, de una unitaria $mcU\in\text{U}(4)$ de la forma
\begin{equation*}
    \mcU_{t}=(U \otimes U)^{t}
\end{equation*}
donde $U\in\text{U}(2)$. Se realiza el mismo proceso: aplicamos la evolución al estado de máxima entropía compatible con un conjunto de observables tomográficamente completos en $\hilbert_{2}$ y propagamos al estado con la uitaria subytacente, para luego pasarlo por la aplicación de grano grueso y recuperar el estado efectivo evolucionado. El caso de la dinámica separable es quizá el caso más sencillo, pues la simetría de la unitaria permite factorizarla:
\begin{align*}
\CG{(U^{t}\otimes U^{t})\varrho_{max}(U^{t}\otimes U^{t})^{\dag}}&=p\frac{1}{Z_{1}}e^{\lambda p U^{t}\sigma_{z}(U^t)^{\dag}}+(1-p)\frac{1}{Z_{2}}e^{\lambda (1-p)U^{t}\sigma_{z}(U^t)^{\dag}}\\
&=p\frac{1}{Z_{1}}U^{t}e^{\lambda p \sigma_{z}}(U^t)^{\dag}+(1-p)\frac{1}{Z_{2}}U^{t}e^{\lambda (1-p)\sigma_{z}}(U^t)^{\dag}\\
&=U^{t}\qty(p\frac{1}{Z_{1}}e^{\lambda p \sigma_{z}}+(1-p)\frac{1}{Z_{2}}e^{\lambda (1-p)\sigma_{z}})(U^t)^{\dag}\\
\end{align*}
La dinámica efectiva tiene la forma:
\begin{equation}
    \rho\xrightarrow{U\otimes U}U\rho U^{\dagger}
\end{equation}
Así como demostramos previamente que si el estado efectivo es puro, entonces el único estado de máxima entropía compatible es justamente el producto tensorial del estado efectivo consigo mismo, podemos mostrar que si la evolución efectiva es unitaria, la dinámica subyacente es el producto tensorial de dicha dinámica consigo misma:

\subsection{Identidad de un lado}

Retomando a expresión (\ref{eq:SeparableDynamics}), y en virtud de (\ref{eq:PauliVectorExp}), vemos que el estado efectivo inicial $\rho$ puede verse como una combinación de dos operadores con vector de Bloch con dirección $\hat{r}_{\rho}$. El vector de Bloch de $\rho$ se ve modificado al ser una de sus dos componentes (paralelas) rotada. La rotación siendo $U_{1}$ \notaAd{Creo que dependo mucho de las parametrizaciones de Bloch para entender lo que está pasadno, ¿qué sucede en el espacio de operadores de densidad?}. En general:
\begin{equation}\label{eq:SeparableDynamicsUxI}
    \rho\xrightarrow{\mcU=U_{1}\otimes \Id} p\frac{1}{Z_{1}}U_{1}e^{\lambda p\hat{r}_{\rho}\cdot\vec{\sigma}}U_{1}^{\dag}+(1-p)\frac{1}{Z_{2}}e^{\lambda(1-p)\hat{r}_{\rho}\cdot\vec{\sigma}}
\end{equation}
En términos del vector de Bloch, denotando $r_{A}=p\tan(p\lambda)$, $r_{B}=(1-p)\tan((1-p)\lambda)$, y $O$ la rotación generada por $U_{1}$:
\begin{equation}
    r\hat{r}_{\rho}\xrightarrow{\mcU=U_{1}\otimes \Id}r_{A}O\hat{r}_{\rho}+r_{B}\hat{r}_{\rho}=O(r\hat{r}_{\rho}-r_{B}\hat{r}_{\rho})+r_{B}\hat{r}_{\rho}
\end{equation}\label{eq:SeparableDynamicsUxIBloch}
El resultado es una rotación alrededor de una línea que no pasa por el origen. Una rotación de esta naturaleza puede descomponerse en una rotación a través de un eje que pasa por el origen $R$ y una traslación $T$ como $T^{-1}\circ R\circ T$. Notar que una transformación así no tendría por qué mantener a los estados dentro de la esfera de Bloch, por lo que esta debe depender del estado mismo. En efecto traslación tiene una magnitud $r_{B}$ en la dirección opuesta a la del estado (depende del estado tanto en magnitud como en dirección). Así que, aunque esto podría parecer una transformación afín, no lo es, pues depende enteramente del estado.

De (\ref{eq:SeparableDynamicsUxIBloch}) también se ve que si $U=e^{it\hat{n}\cdot\vec{\sigma}}$ entonces se ve que cualquier estado con vector de Bloch $r\hat{n}$ será invariante bajo la transformación subyacente. Lo que es mejor, esto aplica para cualquiera de los casos $U_{1}=\Id$, $U_{2}=\Id$ o $U_{1}=U_{2}$.

\subsubsection{Cambio de fase $H=\sigma_{z}$}
Considérese el hamiltoniano $H=\sigma_{z}$. La rotación en la esfera debida a la unitaria generada por el hamiltoniano es una alrededor del eje $z$. La representación de esto, y de el resultado general (\ref{eq:SeparableDynamicsUxIBloch}) puede verse en la figura \ref{fig:ZRot}.


En el espacio de operadores de densidad, esto equivale a insertar una fase relativa en el primer subsistema fino. En efecto, ignorando fases globales,
\begin{equation}
    e^{it\sigma_{z}}=\begin{pmatrix}
        1&0\\0&e^{-i2t}
    \end{pmatrix}
\end{equation}
El estado grueso siente el cambio de fase relativa en su primera componente \notaAd{¿Qué son las trazas del MaxEnt?}.
\begin{equation}
    \rho\xrightarrow{\mcU=e^{it\sigma_{z}}\otimes \Id} p\frac{1}{Z_{1}}e^{it\sigma_{z}}e^{\lambda_{3}p\hat{r}_{\rho}\cdot\vec{\sigma}}e^{-it\sigma_{z}}+(1-p)\frac{1}{Z_{2}}e^{\lambda_{3}(1-p)\hat{r}_{\rho}\cdot\vec{\sigma}}
\end{equation}

\subsubsection{Tengo que ver cómo interpreto esta $H=a\sigma_{x}+b\sigma_{y}$}
La transformación es una rotación respecto al eje $(a,b,0)$. Al aplicarse sobre el primer subsistema, el resultado es una rotación de la primera componente del estado grueso. De nuevo, la condición de normalización entre dichas componentes asegura que el estado grueso se mantenga dentro de la esfera de Bloch. 

\subsection{Régimen de error pequeño y ejemplos particulares}

El caso $p\rightarrow 1$ puede interpretarse como aquel en el que el aparato de medición tiene una baja probabilidad de fallar (poco ruido). Las evoluciones separables $U=U_{1}\otimes U_{2}$ pueden verse como una evolución gruesa $U_{1}$ más una perturbación. La perturbación, al ser unitaria, es también una rotación, así que lo que se observa es una especie de hélice. El estado precesa alrededor de una traslación de la que sería su trayectoria no perturbada. Siempre que $U_{2}=\Id$ lo que se observa es una traslación en dirección del estado con magnitud $r_{B}$. Si, por el contrario, $U_{1}=\Id$, una vez más el estado grueso se ve desplazado, para luego comenzar a girar según la rotación inducida por la unitaria $U_{2}$. 

Algunos ejemplos pueden verse en las figuras siguientes.

\section{Compuertas de cómputo cuántico}

Las compuertas cuánticas son el análogo cuántico de las compuertas lógicas utilizadas en cómputo clásico. En un circuito cuántico, las compuertas permiten manipular la información (qubits). Una compuerta $U$ válida es un operador de evolución unitario que actúa sobre el espacio generado por $n$ qubits (esto es, actúa sobre $\hilbert_{2^{n}}$):
\begin{equation*}
  U:\hilbert_{2^{n}}\rightarrow\hilbert_{2^{n}}.
\end{equation*}
Compuertas de un qubit comunes icluyen a los operadores de Pauli, mientras que para dos qubits existen compuertas como el SWAP, el controlled-not (CNOT), y la compuerta de Hadamard. Como una primera aplicación del formalismo descrito en las secciones anteriores, en este trabajo se analizará la evolución efectiva bajo evoluciones subyacentes descritas por las compuertas SWAP y CNOT. 

\subsection{La compuerta cuántica SWAP}

La compuerta SWAP, $S$, actúa sobre dos qubits permutándolos. Su acción sobre la base de $\hilbert_{4}$ contruida mediante los eigenestados de $\pauli{3}\otimes\pauli{3}$ es
\begin{align*}
    \ket{0}\otimes\ket{0}\mapsto\ket{0}\otimes\ket{0}\\
    \ket{0}\otimes\ket{1}\mapsto\ket{1}\otimes\ket{0}\\
    \ket{1}\otimes\ket{0}\mapsto\ket{0}\otimes\ket{1}\\
    \ket{1}\otimes\ket{1}\mapsto\ket{1}\otimes\ket{0} \rlap{.}
\end{align*}
Si un sistema está descrito por un operador de densidad separable, $\varrho=\rho_{A}\otimes\rho_{B}$, entonces el efecto de la compuerta SWAP es 
\begin{equation*}
    S\varrho S^{\dag}=\rho_{B}\otimes\rho_{A}.
\end{equation*}
La compuerta puede representarse como una matriz de permutación
\begin{equation*}
    S=\begin{pmatrix}
        1&0&0&0\\
        0&0&1&0\\
        0&1&0&0\\
        0&0&0&1
    \end{pmatrix}.
\end{equation*}
Descrito de esta manera, el operador SWAP es un operador de evolución \textit{discreto}. Esto es, lleva un estado $\varrho$ a otro $\varrho'$ sin considerar ninguna dependencia temporal ni ningún estado intermedio. Sin embargo, como esta evolución es unitaria, puede generalizarse a cualquier tiempo de acuerdo con la ecuación (\ref{eq:UnitaryTimeDependence}). Para extenderlo, utilizamos la acción de $S$ sobre la base de $\pauli{3}\otimes\pauli{3}$ y reconocemos que el operador deja invariantes (hasta un factor) a los estados $\ket{00}$, $\ket{11}$, $\ket{+_{2}}=\frac{\ket{01}+\ket{10}}{\sqrt{2}}$ y $\ket{-_{2}}=\frac{\ket{01}-\ket{10}}{\sqrt{2}}$. Dados estos eigenestados (y eigenvalores), la descomposición espectral del operador es
\begin{equation*}
S=P(\dyad{00}+\dyad{11}+\dyad{+_{2}}-\dyad{-_{2}})P^{\dag}.
\end{equation*}
donde $P$ es la matriz formada por los eigenestados del operador, y cumple que $P^{\dag}P=\Id$. Potenciando se halla que
\begin{align*}
S^{t}&=P(\dyad{00}+\dyad{11}+\dyad{+_{2}}+(-)^{t}\dyad{-_{2}})P^{\dag}\\
&=P(\dyad{00}+\dyad{11}+\dyad{+_{2}}+e^{i \pi t}\dyad{-_{2}})P^{\dag}.
\end{align*}
La forma matricial del operador \textsc{SWAP} a un tiempo $t$ es
\begin{equation}\label{eq:SWAP(t)}
S^{t}=\begin{pmatrix}
 1 & 0 & 0 & 0 \\
 0 & \frac{1}{2}(1+e^{i \pi t}) & \frac{1}{2} (1-e^{i \pi t}) & 0 \\
 0 & \frac{1}{2}(1-e^{i \pi t}) & \frac{1}{2}(1+e^{i \pi t}) & 0 \\
 0 & 0 & 0 & 1
\end{pmatrix}=\begin{pmatrix}
  1 & 0 & 0 & 0 \\
  0 & e^{i\frac{t\pi}{2}}\cos{\frac{t\pi}{2}} & -ie^{i\frac{t\pi}{2}}\sin{\frac{t\pi}{2}} & 0 \\
  0 & -ie^{i\frac{t\pi}{2}}\sin{\frac{t\pi}{2}} & e^{i\frac{t\pi}{2}}\cos{\frac{t\pi}{2}}  & 0 \\
  0 & 0 & 0 & 1
 \end{pmatrix}
\end{equation}

\subsubsection{Evolución discreta}

Para estudiar la dinámica efectiva de una evolución subyacente descrita por el operador SWAP, primero analizaremos el caso en que no se ha introducido la dependencia temporal. Sea $\rho\in\densityspace{2}$ el estado efectivo y $\varrho_{\max}\in\densityspace{4}$ el estado de máxima entropía compatible con $\rho$ y la aplicación de grano grueso descrita en la sección \ref{sec:CH2CG} según $\mcA_{\mcC}^{\max}(\rho)=\varrho_{\max}$. Estudiaremos la asignación
\begin{equation}
  \rho\mapsto \mcC(S \varrho_{\max} S^{\dag}).
\end{equation}
Por comodidad, los cálculos se expresarán en términos de los multiplicadores de Lagrange. Como el estado de máxima entropía es separable, la acción del operador SWAP sobre este es
\begin{equation*}
  \frac{e^{p\sum_{i}\lambda_{i}\sigma_{i}}}{Z_{1}} \otimes \frac{e^{(1-p)\sum_{i}\lambda_{i}\sigma_{i}}}{Z_{2}}\mapsto\frac{e^{(1-p)\sum_{i}\lambda_{i}\sigma_{i}}}{Z_{2}}\otimes\frac{e^{p\sum_{i}\lambda_{i}\sigma_{i}}}{Z_{1}}.
\end{equation*}
El estado de la izquierda corresponde a $\varrho_{\max}(t=0)$, mientras que el de la derecha corresponde a $\varrho_{\max}(t=1)$. Con esto, basta con aplicar la aplicación de grano grueso a ambos estados para hallar a los estados efectivos inicial y final en términos de los multiplicadores de Lagrange:
\begin{equation}
\rho(0)=\frac{1}{2}[\Id+(\hat{r}_{\rho}\cdot\vec{\sigma})(p\tanh{-\lambda p}+(1-p)\tanh{-\lambda (1-p)})],
\end{equation}
\begin{equation}
\rho(t=1)=\frac{1}{2}[\Id+(\hat{r}_{\rho}\cdot\vec{\sigma})((1-p)\tanh{-\lambda p}+p\tanh{-\lambda (1-p)})].
\end{equation}
Vemos que ambos estados tienen la misma orientación (orientación significando la dirección del vector de Bloch) pero pureza distinta. Esto significa que el efecto del \textsc{SWAP} subyacente sobre la esfera de Bloch es comprimir al estado efectivo inicial con un coeficiente $\kappa_{1}$ definido según
\begin{equation}\label{eq:SWAPFactor}
  \kappa_{1}=\frac{r_{\rho(1)}}{r_{\rho(0)}}=\frac{(1-p)\tanh{\lambda p}+p\tanh{\lambda (1-p)}}{
    p\tanh{\lambda p}+(1-p)\tanh{\lambda (1-p)}}.
\end{equation}
Claro está, el factor de compresión depende del multiplicador de Lagrange, que a su vez es una función de la pureza del estado inicial. La figura \ref{fig:SWAPFactor2Drl} muestra dicha dependencia. Si la dependencia del factor de compresión en el estado efectivo inicial se denota por un superíndice, la dinámica efectiva puede escribirse como
\begin{equation}\label{eq:EffectiveSWAP1}
  \boxed{\frac{1}{2}(\Id+\vec{r}_{\rho}\cdot\vec{\sigma}) \mapsto \frac{1}{2}(\Id+\kappa_{1}^{\rho}\vec{r}_{\rho}\cdot\vec{\sigma})}
\end{equation}
\begin{figure}[h!]
  \centering
  \includegraphics[width=0.9\linewidth]{chapter3/figures_toy/ContractionFactorSWAP_2D_both.png}
  \caption{Factor de compresión $\kappa_{1}$ como función de $r_{\rho}$ (der.) y como función de $\lambda$ (izq.), para diferentes valores de $p$.}
  \label{fig:SWAPFactor2Drl}
\end{figure}

De las ecuaciones (\ref{eq:SWAPFactor}) y (\ref{eq:EffectiveSWAP1}) distinguimos lo siguiente:
\begin{itemize}
  \item Si $p=\frac{1}{2}$, entonces $\kappa_{1}^{\rho}=1$. Esto se debe a que la aplicación borrosa es invariante bajo el $\textsc{SWAP}$ si $p=0.5$. Así, todos los estado gruesos son puntos fijos bajo una evolución subyacente SWAP con aplicación de grano grueso con parámetro $p=\frac{1}{2}$.
  \item $\kappa_{1}^{\rho}$ no depende de la orientación del vector de Bloch, únicamente depende de la magnitud $r_{\rho(0)}$ y $p$.
  \item En los casos extremos, $p=1$ o $p=0$, la esfera colapsa al origen.
\end{itemize}


Como el factor de compresión depende de $\lambda$, la dinámica no es lineal. Las operaciones cuánticas de un qubit se traducen como aplicaciones afines en la esfera de Bloch. Si quisiéramos ver el proceso asociado al \textsc{SWAP} subyacente como una transformación de la forma
\begin{equation*}
  \vec{r}\rightarrow M\vec{r}+\vec{c}
\end{equation*}
en la que $\vec{c}=0$, y $M=OS$ con $O=\Id$ y $S=\kappa_{1}(\vec{r})\Id$, de tal forma que
\begin{equation*}
  \vec{r}\rightarrow \kappa_{1}(\vec{r})\vec{r}
\end{equation*}
nos daríamos cuenta que la transformación no es afín, y por esto, el proceso no puede ser descrito a través del formalismo de las operaciones cuánticas (no tiene representación en operadores de Kraus) \cite{Chuang}.

\subsubsection{Evolución continua}
Utilizando la forma dependiente del tiempo del operador $S$ dada por la ecuación (\ref{eq:SWAP(t)}), puede seguirse el mismo proceso para hallar una expresión del estado efectivo evolucionado como función del tiempo y en términos de los operadores de Lagrange:
\begin{align}
  \begin{split}
  \rho(t)=\frac{1}{2}\{\Id-(\hat{r_{\rho}}\cdot\vec{\sigma})[&((1-p)\cos^{2}{\frac{\pi t}{2}}+p\sin^{2}{\frac{\pi t}{2}})\tanh{p\lambda}\\
  &+(p\cos^{2}{\frac{\pi t}{2}}+(1-p)\sin^{2}{\frac{\pi t}{2}})\tanh{(1-p)\lambda}]\}.
  \end{split}
\end{align}

El estado efectivo inicial siendo el mismo, el estado final vuelve a tener la misma orientación, y entonces es posible calcular el factor de compresión como la razón entre las normas de los vectores de Bloch de los estados inicial y final:
\begin{equation}\label{eq:SWAPFactort}
  \kappa_{t}^{\rho}=\frac{((1-p)\cos^{2}{\frac{\pi t}{2}}+p\sin^{2}{\frac{\pi t}{2}})\tanh{\lambda p}+(p\cos^{2}{\frac{\pi t}{2}}+(1-p)\sin^{2}{\frac{\pi t}{2}})\tanh{\lambda (1-p)}}{
    p\tanh{\lambda p}+(1-p)\tanh{\lambda (1-p)}}
\end{equation}

Nuevamente, el factor de compresión (y por consiguiente, toda la evolución) depende de la pureza del estado efectivo incial, codificada en los multiplicadores de Lagrange. El efecto gradual de la evolución sobre la esfera de Bloch puede verse en la figura \ref{fig:SWAPFactorSequence}. La dinámica efectiva puede escribirse como
\begin{equation}\label{eq:EffectiveSWAPt}
  \boxed{\frac{1}{2}(\Id+\vec{r}_{\rho}\cdot\vec{\sigma}) \mapsto \frac{1}{2}(\Id+\kappa_{t}^{\rho}\vec{r}_{\rho}\cdot\vec{\sigma})}
\end{equation}

\begin{figure}[h!]
  \centering
  \begin{subfigure}{0.32\textwidth}
    \centering
    \includegraphics[width=0.9\linewidth]{chapter3/figures_toy/sphere_swapcontraction_t=0.0_z=0.9_p=0.9.png}
    \caption{$t=0$}
  \end{subfigure}%
  \begin{subfigure}{0.32\textwidth}
    \centering
    \includegraphics[width=0.9\linewidth]{chapter3/figures_toy/sphere_swapcontraction_t=0.5_z=0.9_p=0.9.png}
    \caption{$t=0.5$}
  \end{subfigure}
  \begin{subfigure}{0.32\textwidth}
    \centering
    \includegraphics[width=0.9\linewidth]{chapter3/figures_toy/sphere_swapcontraction_t=1.0_z=0.9_p=0.9.png}
    \caption{$t=1$}
  \end{subfigure}
  \caption{Efecto de la evolución subyacente si $r_{z}=0.9$, $p=0.9$. La dramática contracción se asocia a una pérdida casi total de información.}
  \label{fig:SWAPFactorSequence}
  \end{figure}

De las ecuaciones (\ref{eq:SWAPFactort}) y (\ref{eq:EffectiveSWAPt}) es posible concluir:
\begin{itemize}
  \item $\kappa_{t}^{\rho}$ es una función periódica del tiempo, y su periodo es de $T=2$ (observable en la figura \ref{fig:SWAPFactor2Dt}). Esto viene de que el operador SWAP, además de ser unitario, es hermitiano, i.e. $SS=\Id$.
  \item $\kappa_{t}^{\rho}$ es una función decreciente en $0\leq t\leq 1$.
  \item Se cumplen las observaciones hechas para el caso discreto: la esfera colapsa al origen si $p=1$ o $p=0$, y los puntos se mantienen fijos si $p=\frac{1}{2}$.
\end{itemize}

\begin{figure}[h!]
  \centering
  \includegraphics[width=0.6\linewidth]{chapter3/figures_toy/ContractionFactorSWAP_z=0.8_t=0_to_t=2.png}
  \caption{Factor de compresión $\kappa_{t}$ como función de $t$, para diferentes valores de $p$ y $r_{\rho(0)}=0.8$.}
  \label{fig:SWAPFactor2Dt}
\end{figure}

En términos del valor esperado del observable $\sigma_{3}$, la evolución del estado se da como
\begin{equation}
  \expval{\sigma_{3}(t)}=\kappa_{t}^{\rho}\expval{\pauli{3}(0)}
\end{equation}
que puede escribirse, también, como las probabilidades de que $\rho(t)$ se halle en el estado $\ket{0}$ o $\ket{1}$
 \begin{align}
  \bra{0}\rho(t)\ket{0}=\frac{1}{2}(1+\kappa_{t}^{\rho}\expval{\pauli{3}(0)}) && \bra{1}\rho(t)\ket{1}=\frac{1}{2}(1-\kappa_{t}^{\rho}\expval{\pauli{3}(0)})
 \end{align}
 donde la dependencia temporal está completamente contenida dentro del factor $\kappa_{t}^{\rho}$. 


\subsection{La compuerta cuántica controlled not}

La compuerta \textit{controlled not}, o CNOT, es el análogo cuántico de la compuerta lógica XOR. La compuerta XOR recibe como entrada dos bits, y arroja uno que puede ser $0$ si los bits de entrada tienen el mismo valor, o $1$ si tienen valores diferentes. Por otro lado, la compuerta cuántica CNOT actúa sobre un sistema de dos qubits, aplicando sobre el segundo qubit la compuerta $\sigma_{1}$ (NOT) si el primer qubit se halla en el estado $\ket{1}$, o dejándolo invariante si el primer qubit se halla en el estado $\ket{0}$. Esto es, cumple que \cite{Chuang}
\begin{align*}
    \ket{0}\otimes\ket{0}\mapsto\ket{0}\otimes\ket{0}\\
    \ket{0}\otimes\ket{1}\mapsto\ket{0}\otimes\ket{1}\\
    \ket{1}\otimes\ket{0}\mapsto\ket{1}\otimes\ket{1}\\
    \ket{1}\otimes\ket{1}\mapsto\ket{1}\otimes\ket{0} \rlap{.}
\end{align*}
En la base de los eigenestados de $\pauli{3}\otimes\pauli{3}$, la compuerta puede representarse como la matriz de permutación
\begin{equation*}
    \cnot=\begin{pmatrix}
        1&0&0&0\\
        0&1&0&0\\
        0&0&0&1\\
        0&0&1&0
    \end{pmatrix}.
\end{equation*}
Claro está, esta matriz corresponde a la evolución discreta. Para hallar la extensión a cualquier tiempo $t$, notamos que los eigenestados de este operador son $\ket{01}$, $\ket{00}$, $\ket{+}_{\cnot}$ y $\ket{-}_{\cnot}$ donde se definen
\begin{align*}
  \ket{+}_{\cnot}=\frac{\ket{10}+\ket{11}}{\sqrt{2}} & & \text{y} & & \ket{-}_{\cnot}=\frac{\ket{10}-\ket{11}}{\sqrt{2}}.
\end{align*}
Con esto es posible escribir la descomposición espectral del operador, y luego potenciarla 
\begin{align*}
\cnot=&P(\dyad{00}+\dyad{01}+\dyad{+}_{\cnot}-\dyad{-}_{\cnot})P^{\dag}\\
\Rightarrow \cnot^{t}=&P(\dyad{00}+\dyad{11}+\dyad{+}_{\cnot}+e^{i \pi t}\dyad{-}_{\cnot})P^{\dag}.
\end{align*}
La forma matricial del operador controlled not a un tiempo $t$ es análoga a la del operador SWAP:
\begin{equation}\label{eq:CNOT(t)}
\cnot^{t}=\begin{pmatrix}
  1 & 0 & 0 & 0 \\
  0 & 1 & 0 & 0 \\
  0 & 0 & e^{i\frac{t\pi}{2}}\cos{\frac{t\pi}{2}} & -ie^{i\frac{t\pi}{2}}\sin{\frac{t\pi}{2}}\\
  0 & 0 & -ie^{i\frac{t\pi}{2}}\sin{\frac{t\pi}{2}} & e^{i\frac{t\pi}{2}}\cos{\frac{t\pi}{2}}
 \end{pmatrix}
\end{equation}

El operador \textsc{CNOT} puede expandirse de la siguiente manera:
\begin{equation*}
        \cnot=\frac{1}{2}(\Id+\pauli{3}\otimes\Id+\Id\otimes\pauli{1}-\pauli{3}\otimes\pauli{1}).
\end{equation*}
Si se calcula el logaritmo de la compuerta es posible hallar el hamiltoniano generador de la unitaria,
\begin{equation*}
    H_{\cnot}=\frac{\pi}{4}\qty(\Id-\pauli{3}\otimes\Id-\Id\otimes\pauli{1}+\pauli{3}\otimes\pauli{1}),
\end{equation*}
que por ser una suma de operadores que conmutan entre sí, nos permite ver al controlled not como una aplicación consecutiva de tres unitarias diferentes
\begin{align*}
    \cnot&=e^{-i\frac{\pi}{4}\Id}e^{i\frac{\pi}{4}\pauli{3}\otimes\Id}e^{i\frac{\pi}{4}\Id\otimes\pauli{1}}e^{-i\frac{\pi}{4}\pauli{3}\otimes\pauli{1}}\\
    &=e^{-i\frac{\pi}{4}} (e^{i\frac{\pi}{4}\pauli{3}}\otimes \Id) (\Id \otimes e^{i\frac{\pi}{4}\pauli{1}}) e^{-i\frac{\pi}{4}\pauli{3}\otimes\pauli{1}}.
\end{align*}
De momento no le vi como sacarle jugo a esto, pero será útil para la extensión a tiempo arbitrario.


\subsubsection{CNOT completo efectivo}

Para estudiar la dinámica efectiva del operador $\cnot$ son particularmente útiles las expresiones (\ref{eq:rhoArhoB}). Como el estado de máxima entropía compatible con la aplicación de grano grueso puede escribirse como $\varrho_{\max}=\rho_{A}\otimes\rho_{B}$, entonces hallar el estado efectivo final es un problema de álgebra, en efecto,
\begin{align*}
    \rho(t=1)=\frac{1}{2}[&p(\rho(0)+\sigma_{3}\rho_{A}\sigma_{3}+\Tr{\sigma_{1}\rho_{B}}[\rho_{A}-\sigma_{3}\rho_{A}\sigma_{3}])\\
    &+(1-p)(\rho(0)+\sigma_{1}\rho_{B}\sigma_{1}+\Tr{\sigma_{3}\rho_{A}}[\rho_{B}-\sigma_{1}\rho_{B}\sigma_{1}])].
\end{align*}
La estructura del estado final es una consecuencia directa de la aplicación borrosa. Para entender el significado de cada uno de los términos, considérese el caso en que el aparato de medición no tiene un error asociado ($p=1$). En dicho caso, a través del principio de máxima entropía, se esperaría que el estado efectivo final fuera
\begin{equation*}
  \rho(t=1)=\rho(0)+\pauli{3}\rho(0)\pauli{3}=\frac{1}{2}(\Id+r_{3}\pauli{3})
\end{equation*}.
Este resultado viene del hecho que, en el caso $p=1$, el estado de máxima entropía es simplemente $\rho\otimes\frac{1}{2}$. Si se conociera la preparación microscópica del estado inicial, el resultado de aplicar la compuerta de controlled not seguido de trazar al segundo sistema sería
\begin{equation*}
  \rho(t=1)=
\end{equation*}
En términos del vector de Bloch, que es una forma rápida de obtener el cambio de los observables y de la esfera, la dinámica se ve como
\begin{equation*}
    \vec{r}_{\rho}=(pr_{A}+(1-p)r_{B})\hat{r}_{\rho}\mapsto\begin{pmatrix}
        r_{B}(pr_{A}(\hat{r}_{\rho,1})^2+(1-p)\hat{r}_{\rho,1})\\
        r_{B}r_{A}(p\hat{r}_{\rho,1}\hat{r}_{\rho,2}+(1-p)\hat{r}_{\rho,2}\hat{r}_{\rho,3})\\
        r_{A}(p\hat{r}_{\rho,3}+(1-p)r_{A}(\hat{r}_{\rho,3})^{2})
    \end{pmatrix}.
  \end{equation*}
  De momento tengo que ver esto en términos de matrices. El controlled not es una aplicación consecutiva de 3 unitarias, así que no debería ser demasiado complicado obtener la transformación de forma más elegante.

\subsubsection{CNOT efectivo a un tiempo arbitrario}

\section{Dinámicas especiales}

\subsection{Modelo de Ising}

\subsection{Canal de despolarización}

\subsection{Amortiguamiento de amplitud}
\newpage