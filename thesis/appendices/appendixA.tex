\chapter{Demostraciones de relaciones frecuentemente socorridas}

\subsubsection{Cuadrado de vector de pauli}
Se cumple que
\begin{align}
    (\paulivec{r})(\paulivec{r})=&\sum_{j}r_{j}\pauli{j}\sum_{k}r_{k}\pauli{k}\nonumber\\
    =&\sum_{j}r_{j}\sum_{k}r_{k}\pauli{j}\pauli{k}\nonumber\\
    =&\sum_{j}r_{j}\sum_{k}r_{k}(\Id\delta_{jk}+i\epsilon_{jkl}\pauli{l})\nonumber\\
    =&\sum_{j}r_{j}\sum_{k}r_{k}(\Id\delta_{jk})+i\sum_{j}r_{j}\sum_{k}r_{k}\epsilon_{jkl}\pauli{l}\nonumber\\
    =&\Id\nonumber
\end{align}
Donde en la última línea se ha utilizado la antisimetría del tensor de Lévi-Civita y $\hat{r}$ es un vector unitario. Se sigue que para todo entero positivo $p$
\begin{equation}\label{ap:PauliSquare}
    (\paulivec{n})^{2p}=\Id
\end{equation}

\subsubsection{Exponencial real de vector de Pauli}
Si se expande la serie de Taylor usando como argumento un vector de pauli $r\paulivec{r}$,
\begin{align}
    e^{r\paulivec{r}}=&\sum_{k=0}^{\infty}\frac{1}{k!}(r\paulivec{r})^k\nonumber\\
    =&\sum_{k}\frac{r^{2k}(\paulivec{r})^{2k}}{(2k)!}+\sum_{k}\frac{r^{2k+1}(\paulivec{r})^{2k+1}}{(2k+1)!},\nonumber
\end{align}
se puede usar (\ref{ap:PauliSquare}) para ver que
\begin{equation}
    e^{r\paulivec{r}}=\Id\sum_{k}\frac{r^{2k}}{(2k)!}+\paulivec{r}\sum_{k}\frac{r^{2k+1}}{(2k+1)!},\nonumber
\end{equation}
que, claro está, corresponde a
\begin{equation}\label{ap:PauliRealExp}
    e^{r\paulivec{r}}=\Id\cosh{r}+\paulivec{r}\sinh{r}
\end{equation}


\subsubsection{Exponencial compleja de un vector de Pauli}
Si se expande la serie de Taylor usando como argumento un vector de pauli $r\paulivec{r}$,
\begin{align}
    e^{\rmi r\paulivec{r}}=&\sum_{k=0}^{\infty}\frac{1}{k!}(r\paulivec{r})^k\nonumber\\
    =&\sum_{k}\frac{r^{2k}(\paulivec{r})^{2k}}{(2k)!}+\rmi \sum_{k}\frac{r^{2k+1}(\paulivec{r})^{2k+1}}{(2k+1)!},\nonumber
\end{align}
se puede usar (\ref{ap:PauliSquare}) para ver que
\begin{align}
    e^{r\paulivec{r}}=&\Id\sum_{k}(-1)^{2k}\frac{r^{2k}}{(2k)!}+\rmi \paulivec{r}\sum_{k}(-1)^{2k+1}\frac{r^{2k+1}}{(2k+1)!},\nonumber
\end{align}
que, claro está, corresponde a
\begin{equation}\label{ap:PauliCompExp}
    e^{\rmi r\paulivec{r}}=\Id\cos{r}+\rmi \paulivec{r}\sin{r}
\end{equation}
\subsubsection{Unitaria generada por un operador hermítico}
Toda unitaria de $2\times 2$ puede generarse a través de un operador hermítico $H$ como
\begin{equation}
    U=e^{-\rmi H}.\nonumber
\end{equation}
Pues bien, como el conjunto de las matrices de Pauli, junto a la identidad, forman una base del espacio de operadores (respecto al producto interno de Hilbert-Schmidt), $H$ puede expandirse como $H=r_{0}\Id+r_{x}\pauli{x}+r_{y}+\pauli{y}+r_{z}\pauli{z}$. Si se utiliza este para construir una unitaria, desarrollando la serie se encuentra que
\begin{align}
    e^{\rmi H}=&e^{\rmi(r_{0}\Id+r\paulivec{r})}\nonumber\\
    =&e^{\rmi r_{0}\Id}e^{\rmi r\paulivec{r}}\nonumber\\
    =&e^{\rmi r\paulivec{r}}\nonumber
\end{align}
Por (\ref{ap:PauliCompExp}), se sigue que 
\begin{equation}
    e^{-\rmi H}=\Id \cos{r}+\rmi(\paulivec{r})\sin{r}
\end{equation}


\subsubsection{Evolución de operador de densidad por operador unitario}
Podemos expandir la ecuación
\begin{equation}
    \rho(t)=(e^{-i\omega t \paulivec{r}})\rho(0)(e^{i\omega t \paulivec{r}})\nonumber
\end{equation}
haciendo uso de la ecuación (\ref{ap:PauliCompExp}) como
\begin{equation}
    \rho(t)=\rho(0)\cos^{2}(\omega t)+(\paulivec{r})\rho(0)(\paulivec{r})\sin^{2}(\omega t)+i\sin(\omega t)\cos(\omega t)[\rho(0),\paulivec{r}]
\end{equation}