\chapter{Conclusiones}\label{sec:conclusions}

\ddnote{Comienza explicando que hiciste. Tipo: se estudió la dinámica emergente al combinar el principio de MaxEnt con un modelo de grano grueso, tal que en el nivel fino se cumple la mecánica cuántica. En particular se estudiaron los sistemas tales, presentados en el capitulo tal.}


\ddnote*{ideas buenas, pero muy cantinfleado, el párrafo es una sola frase también :S. No hay necesidad de ser extremadamente breves, explica con calma y separando en más oraciones}{En esta tesis se estudió la dinámica efectiva que emerge de considerar un modelo de grano grueso y el principio de máxima entropía (o de mínima información) para lidiar con la no reversibilidad del modelo de grano grueso. En particular estudiamos el escenario en el que se relacionan sistemas de $n$ qubits, que evolucionan de acuerdo a la mecánica cuántica, con un sistema de un qubit.} \acnote{Iterado}

Se estudió la dinámica efectiva que emergió de considerar una aplicación de grano grueso resultante de concatenar dos tipos de errores. El primero es el inducido por un aparato que tiene una probabilidad no nula de medir una partícula diferente a la pretendida. El segundo proviene de la incapacidad del detector de resolver todos los grados de libertad del sistema. Esta falta de resolución provoca que el sistema de $n$ qubits se vea como un sistema de uno solo. Se construyó una aplicación de asignación basada en el Principio de Máxima Entropía para hallar al estado microscópico menos sesgado posible, tal que fuera compatible con un estado efectivo dado. Suponiendo que el sistema fino evoluciona de manera que cumple con las leyes de la mecánica cuántica, la dinámica efectiva estudiada es la composición de la aplicación de asignación, las evoluciones microscópicas, y la aplicación de grano grueso, como se ve en la ecuación \ref{eq:EffectiveDynamics}.


Como se discutió en el capítulo \ref{ch:3}, del modelo de grano grueso utilizado se reconocieron dos regímenes particularmente interesantes, el \textit{régimen imparcial}\ddnote{discutamos este nombre. Propongo algo como \textit{caso no sesgado}}\acnote{\checkmark}, asociado a una caja de $n$ partículas idénticas en la que cada una es igualmente probable de ser medida, y el régimen de partícula preferencial, en el que una partícula tiene una probabilidad mucho mayor de ser medida. En estos regímenes, la asignación de máxima entropía está dada por las ecuaciones (\ref{eq:BoltzmannAss}) y (\ref{eq:PreferentialAss}) respectivamente. \acnote{nuevo:} Además, el estado asignado a través del Principio de Máxima Entropía resultó ser un estado mixto y factorizable (esto es, sin correlaciones entre los subsistemas), como se aprecia en la ecuación (\ref{eq:MaxEntSeparable}). Esta propiedad es un resultado natural del principio de mínima información. En efecto, debido a que las correlaciones entre los subsistemas no son accesibles a través de la descripción gruesa, estas son minimizadas.

De acuerdo a lo hallado en el capítulo \ref{ch:4}, a través de la aplicación de grano grueso y la aplicación de asignación de maxima entropía se estudió la dinámica efectiva de sistemas de muchas partículas. A lo largo de este análisis, se hallaron comportamientos más afines a la física clásica que a la mecánica cuántica, como la no linealidad de las evoluciones. Casos como el de las dinámicas inducidas por las compuertas SWAP y CNOT resultaron ser versiones no lineales de canales cuánticos conocidos. En particular, la compuerta SWAP efectiva (\ref{eq:EffectiveSWAPt}) corresponde a un canal de despolarización no lineal, mientras que la compuerta CNOT efectiva es una combinación de un canal \textit{phase flip} y un canal \textit{bit flip} no lineales. Sin embargo, las no linealidades encontradas, a diferencia de las dinámicas clásicas no lineales, no son universales. Esto debido a que todas resultaron dependientes del estado efectivo inicial. 
%

\acnote{Párrafo iterado}
Otra diferencia viene del hecho de que las evoluciones deterministas \ddnote{$deterministas$}\acnote{\checkmark} no conllevan cambios en la entropía del sistema, mientras que las evoluciones efectivas estudiadas se traducían en contracciones de la esfera de Bloch, quizá mejor representadas por la compuerta SWAP efectiva (\ref{eq:EffectiveSWAPt}), caso en el que la no linealidad de la dinámica se halla enteramente en términos de la pureza del estado efectivo inicial. Dicho de otra forma, las dinámicas efectivas resultaron ser procesos irreversibles, manifestado en el aumento de la entropía del sistema. \ddnote{, manifestado en el aumento de la entropía del sistema.}\acnote{\checkmark}

Algunas de las dinámicas estudiadas, particularmente aquellas generadas por evoluciones subyacentes con fuertes simetrías, resultaron ser no solo lineales, sino canales cuánticos, como el caso de ciertos tipos de canales de Pauli, (\ref{eq:EffectiveDephasing}) y (\ref{eq:EffectiveDepolarizing}), y el canal de estabilización (\ref{eq:EffectiveStabilizing}), o aún más, evoluciones unitarias, como el caso de la dinámica local simétrica (\ref{eq:EffectiveSymmetricLocal}). 

Ninguna de las dinámicas estudiadas es no lineal en el caso en que las partículas no preferenciales tienen una probabilidad nula de ser detectadas, esto es, cuando el error es nulo, que equivale al caso en que la aplicación de medición borrosa sale del escenario. Aunque el único elemento no lineal en la composición que define a la dinámica efectiva es la aplicación de asignación, queda por investigar si es la aplicación de asignación o la aplicación de medición borrosa la causante de las no linealidades en las dinámicas efectivas.


\acnote{Párrafo nuevo.}
Finalmente, de la comparación entre las aplicaciones de asignación de máxima entropía y promedio realizada en el capítulo \ref{ch:5}, se encontró que estas coinciden en los casos en que no existe la posibilidad de que se detecte una partícula diferente a la pretendida. Además, se demostró que todas las aplicaciones de asignación deben coincidir cuando el estado efectivo inicial es puro y hay participación de todos los subsistemas en la aplicación de medición borrosa (esto es, todas las permutaciones tienen una probabilidad no nula de ocurrir). En este caso, cualquier aplicación de asignación debe asignar al estado efectivo un estado coherente de espín, dado por la ecuación \ref{eq:PureEffectiveState}.



\ddnote{Mañana hagamos un recuento de observaciones, para ver que mas ponemos aquí:
Agregar caso donde se ve que pasa en el límite termodinámica. Equivalencia de mapas de asignación cuando los estados son puros, y decir que los finos son coherentes de espín (de pronto mencionarlo donde pones el ejemplo también, estos es importante por que los estados coherentes son los ``clásicos'' desde el punto de vista de las fluctuaciones.). Luego mencionar que en general no son equivalentes las asignaciones, y que se comparó la nuestra con una que usa mínima información pero solo con estados puros, y los resultados no coinciden.
Mencionar SWAP en las conclusiones, es una no linealidad que solo se manifiesta en términos de la pureza. Luego mencionar que el SWAP y el CNOT dan origen a dinámicas bien conocidas, tales como desfasamiento, flips etc., pero no lineales.
}