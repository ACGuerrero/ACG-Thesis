\chapter{Conclusiones}

\ddnote{Comienza explicando que hiciste. Tipo: se estudió la dinámica emergente al combinar el principio de MaxEnt con un modelo de grano grueso, tal que en el nivel fino se cumple la mecánica cuántica. En particular se estudiaron los sistemas tales, presentados en el capitulo tal.}


Se estudió la dinámica efectiva que emergió de considerar un modelo de grano grueso que relaciona un estado efectivo de un qubit a un espacio microscópico de $n$ qubits, de utilizar el Principio de Máxima Entropía para hallar un estado fino compatible con dicho estado efectivo, y de asumir que el sistema microscópico evoluciona de acuerdo a las leyes de la mecánica cuántica.


Como se discutió en el capítulo \ref{sec:chapter2},del modelo de grano grueso utilizado se reconocieron dos regímenes particularmente interesantes, el régimen de Boltzmann, asociado a una caja de $n$ partículas idénticas en la que cada una es igualmente probable de ser medida, y el régimen de partícula preferencial, en el que una partícula tiene una probabilidad mucho mayor de ser medida. En estos regímenes, la asignación de máxima entropía está dada por las ecuaciones (\ref{eq:BoltzmannAss}) y (\ref{eq:PreferentialAss}) respectivamente.

De acuerdo a lo hallado en el capítulo \ref{sec:chapter3}, a través de la utilización de un modelo de grano grueso para estudiar la dinámica efectiva de sistemas de muchas partículas es posible hallar comportamientos más afines a la física clásica que a la mecánica cuántica, como la no linealidad de las evoluciones. Sin embargo, las no linealidades encontradas, a diferencia de las dinámicas clásicas no lineales, no son universales. Esto debido a que todas resultaron dependientes del estado efectivo inicial. Esto es, no hallamos no linealidades en las componentes de $\rho_{\ef}(t)$ sino dinámicas que varían según la elección del estado efectivo inicial, como se pudo observar en las ecuaciones (\ref{eq:EffectiveSWAPt}), (\ref{eq:EffectiveCNOT}), o (\ref{eq:EffectiveIsing2}). Otra diferencia viene del hecho que las evoluciones clásicas no conllevan cambios en la entropía del sistema, mientras que las evoluciones efectivas estudiadas se traducían en contracciones de la esfera de Bloch, quizá mejor representadas por la compuerta SWAP efectiva, (\ref{eq:EffectiveSWAPt}). Dicho de otra forma, las dinámicas efectivas resultaron ser procesos irreversibles con cambios en la entropía del sistema efectivo.

Alguna de las dinámicas estudiadas, particularmente aquellas generadas por evoluciones subyacentes con fuertes simetrías, resultaron ser no solo lineales, sino canales cuánticos, como el caso de ciertos tipos de canales de Pauli, (\ref{eq:EffectiveDephasing}) y (\ref{eq:EffectiveDepolarizing}), y el canal de estabilización (\ref{eq:EffectiveStabilizing}), o aún más, evoluciones unitarias, como el caso de la dinámica local simétrica (\ref{eq:EffectiveSymmetricLocal}). 

Ninguna de las dinámicas estudiadas es no lineal en el caso en que las partículas no preferenciales tienen una probabilidad nula de ser detectadas, esto es, cuando el error es nulo, que equivale al caso en que la aplicación de medición borrosa sale del escenario. Aunque el único elemento no lineal en la composición que define a la dinámica efectiva es la aplicación de asignación, queda por investigar si es la aplicación de asignación o la aplicación de medición borrosa la causante de las no linealidades en las dinámicas efectivas.