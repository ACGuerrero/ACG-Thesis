\chapter{El Principio de Máxima Entropía y los modelos de grano grueso}\label{sec:chapter1}

\epigraph{``Nadie sabe lo que es realmente la entropía, así que siempre tendrás la ventaja en cualquier debate.''}{--- \textup{John von Neumann}}


El desarrollo del presente trabajo descansa completamente sobre los hombros de tres conceptos: el operador de densidad, el principio de máxima entropía, y los modelos de grano grueso. En este capítulo se introducirán cada una de estas ideas. Primero, se discutirá el formalismo del operador de densidad, introducido mediante la necesidad de este a la hora de estudiar mezclas estadísticas de estados cuánticos. Luego se analizará el concepto de entropía en los contextos de las teorías de información tanto clásica como cuántica, a través del cual se derivará la expresión de la inferencia de máxima entropía de un sistema cuántico. Finalmente, se comentará sobre los modelos de grano grueso, como descripciones \textit{efectivas} de sistemas de dimensión alta, donde por \textit{efectivas} nos referimos a descripciones en las que no se tiene acceso a todos los grados de libertad del sistema.

\section{El operador de densidad}
\subsection{Derivación del operador de densidad}

Los vectores de estado no pueden describir a todos los sistemas estudiables en el contexto de la mecánica cuántica. Por esto, y porque será particularmente útil para nuestro trabajo, introducimos el concepto del operador de densidad (también llamado, en el caso discreto, que es el que nos incumbe, matriz de densidad) de forma similar a como lo hizo L.D. Landau en 1927 \cite{Landau}.

Considérese un sistema descrito por el vector de estado $\ket{\varphi}\in\hilbert_{n}$, con $\hilbert_{n}$ el espacio de Hilbert $\hilbert_{n}=\Complex^{n}$; y $A$ un observable (un operador hermítico). Se sabe que el valor de expectación del observable está dado por $\expval{A}=\bra{\varphi}A\ket{\varphi}$. Pues bien, esta expresión puede ser manipulada a través de una base ortogonal $\{\ket{e_{k}}\}$ del espacio $\hilbert_{n}$:
\begin{align*}
\expval{A}&=\qty(\sum_{i}\dyad{e_{i}})\bra{\varphi}A\ket{\varphi}\qty(\sum_{j}\dyad{e_{j}})\\
&=\bra{\varphi}\qty(\sum_{i}\dyad{e_{i}})A\qty(\sum_{j}\dyad{e_{j}})\ket{\varphi}\\
&=\sum_{i,j}\bra{e_{j}}\ket{\varphi}\bra{\varphi}\ket{e_{i}}\bra{e_{i}}A\ket{e_{j}}\rlap{.}\\
\end{align*}
Esta es una suma sobre los elementos de dos matrices: la del observable $A$ y la definida por $\dyad{\varphi}$. Por completez de la base $\{\ket{e_{i}}\}$,
\begin{align*}
\expval{A}&=\sum_{j}\bra{e_{j}}\ket{\varphi}\bra{\varphi}\qty(\sum_{i}\dyad{e_{i}})A\ket{e_{j}}\\
&=\sum_{j}\bra{e_{j}}\ket{\phi}\bra{\varphi}A\ket{e_{j}}\\
&=\Tr(\dyad{\varphi}A)\rlap{,}
\end{align*}
donde definimos al operador de densidad $\rho$ para un sistema descrito por $\ket{\varphi}$ como
\begin{equation}\label{eq:DensOpPure}
\rho=\dyad{\varphi},
\end{equation} 
y vemos que es posible hallar el valor esperado de un observable respecto a un estado a través del operador de densidad de este según
\begin{equation}\label{eq:ExpValFromDensOp}
\expval{A}=\Tr(A\rho).
\end{equation}

El nombre ``operador de densidad'' puede resultar más claro comparando la ecuación (\ref{eq:ExpValFromDensOp}) con el valor esperado en estadística. Si $X$ es una variable aleatoria cuya función de densidad de probabilidad es $\rho(x)$, entonces el valor esperado de una función $A$ de los valores de $X$ es
\begin{equation*}
E[A(x)]=\int A(x) \rho(x) dx.
\end{equation*}

En este sentido, la matriz de densidad ocupa un rol similar al de la función de densidad.

\subsection{Mezclas estadísticas}

Ahora que conocemos el operador de densidad para un sistema descrito por un vector de estado, supóngase que en lugar de estudiar un sistema que está completamente descrito por $\ket{\varphi}$, se trabaja con uno que está en el estado $\ket{\varphi_{i}}$ con probabilidad $p_{i}$, donde $\{\ket{\varphi_{i}}\}$ es un conjunto no necesariamente ortogonal de estados de $n$ niveles $\ket{\varphi_{i}}\in\hilbert_{n}$, y $\{p_{i}\}$ es un conjunto de números reales tales que $\sum_{i}p_{i}=1$. A este sistema se le llama ``mezlca estadística'', y no debe confundirse con una superposición de estados $\ket{\varphi_{i}}$ con coeficientes $\sqrt{p_{i}}$, ya que una superposición está bien caracterizada, y está completamente descrita por $\ket{\psi}=\sum_{i}\sqrt{p_{i}}\ket{\varphi_{i}}$, mientras que la mezcla no lo está: el elemento probabilístico está asociado a un grado de ignorancia sobre la preparación del sistema.

Ahora vuélvase a considerar un observable $A$. El valor esperado de dicho observable con respecto al sistema será, justamente
\begin{equation}
\expval{A}=\sum_{i}p_{i}\bra{\varphi_{i}}A\ket{\varphi_{i}}.
\end{equation}
Sea $\{\ket{e_{k}}\}$ una base ortogonal del espacio $\hilbert_{n}$.La expresión puede manipularse de forma similar a como se hizo antes:
\begin{align*}
\expval{A}&=\sum_{i}p_{i}\bra{\varphi_{i}}A\ket{\varphi_{i}}\\
&=\sum_{i,j,k}p_{i}\bra{e_{k}}\ket{\varphi_{i}}\bra{\varphi_{i}}\ket{e_{j}} \bra{e_{j}}A\ket{e_{k}}\\
&=\sum_{j,k}\bra{e_{k}}\qty(\sum_{i}p_{i}\dyad{\varphi_{i}})\ket{e_{j}} \bra{e_{j}}A\ket{e_{k}}\\
&=\sum_{k}\bra{e_{k}}\qty(\sum_{i}p_{i}\dyad{\varphi_{i}})A\ket{e_{k}}\\
&=\Tr[\qty(\sum_{i}p_{i}\dyad{\varphi_{i}})A]\rlap{,}
\end{align*}
Con lo que la mezcla queda descrita por el operador de densidad $\rho$ definido según
\begin{equation}\label{eq:DensOpMix}
\rho=\sum_{i}p_{i}\dyad{\varphi_{i}}.
\end{equation}
\subsection{Propiedades del operador de densidad}
\subsubsection{Validez}
De la definición del operador de densidad destilan algunas propiedades que permiten reconocer si un operador es un operador de densidad válido, o no \cite{Holevo}:
\begin{enumerate}
    \item $\Tr(\rho)=1$
    \item $\bra{\varphi}\rho\ket{\varphi}\geq 0$ $\forall$ $\ket{\varphi}\in\hilbert_{n}$
\end{enumerate}
Estas dos propiedades funcionan como una definición alternatica del operador de densidad. La primera propiedad se deriva de la normalización de los estados $\ket{\varphi_{i}}$ que definen a la matriz de densidad. La segunda puede interpretarse como la necesidad de que la probabilidad de que $\rho$ se halle en el estado $\dyad{\varphi}$ sea mayor o igual a $0$.
\subsubsection{Pureza}
La diferencia entre una mezcla estadística y una superposición puede no ser del todo clara. ¿Cómo son diferentes un sistema que tiene una probabilidad $p_{i}$ de hallarse en el estado $\ket{\varphi_{i}}$ y otro que se halla en una superposición de cada estado $\ket{\varphi_{i}}$ con coeficientes $\sqrt{p_{i}}$? 

Para responder, considérense dos sistemas de dos niveles. El primero puede hallarse en cualquiera de los siguientes estados
\begin{align*}
    \ket{0}=\begin{pmatrix}
        1\\
        0
    \end{pmatrix} && \text{y} && \ket{1}=\begin{pmatrix}
        0\\
        1
    \end{pmatrix}\rlap{,}
\end{align*}
con la misma probabilidad $p=\frac{1}{2}$. Entonces el operador de densidad que describe al sistema es 
\begin{equation*}
    \rho=\frac{1}{2}(\dyad{0}+\dyad{1})=\frac{1}{2}\Id_{2}.
\end{equation*}
Por otro lado, el segundo sistema se halla en una superposción de los mismos estados, con coeficientes $\sqrt{p}$. El operador de densidad que describe al segundo sistema es 
\begin{align*}
    \dyad{\psi} && \text{con} && \ket{\psi}=\frac{1}{\sqrt{2}}(\ket{0}+\ket{1})\rlap{.}
\end{align*}
Es claro, al menos matemáticamente, que los sistemas no se hallan en el mismo estado. Si nos propusiéramos calcular la probabilidad de cada uno de hallarse en el estado $\ket{0}$ encontraríamos que
\begin{align*}
    \bra{0}\rho\ket{0}=\frac{1}{2} && \text{y} &&\langle 0 \dyad{\psi} 0\rangle=\frac{1}{2}\rlap{.}
\end{align*}
y el resultado es el mismo si se hiciera con el estado $\ket{1}$. Parecería entonces que, experimentalmente, los sistemas se hallan en el mismo estado. Esto es falso. Si realizamos un cambio de base, de $\{\ket{1},\ket{2}\}$ a $\{\ket{+},\ket{-}\}$, donde
\begin{align*}
    \ket{+}=\frac{1}{\sqrt{2}}\begin{pmatrix}
        1\\
        1
    \end{pmatrix} && \text{y} && \ket{-}=\frac{1}{\sqrt{2}}\begin{pmatrix}
        1\\
        -1
    \end{pmatrix}\rlap{,}
\end{align*}
y calculamos la probabilidad de que cada sistema se halle en el estado $\ket{+}$ encontraremos
\begin{align*}
    \bra{+}\rho\ket{+}=\frac{1}{2} && \text{pero} &&\langle + \dyad{\psi} +\rangle=1\rlap{.}
\end{align*}
Este resultado puede interpretarse como que el sistema descrito por $\rho_{2}$ se halla en el estado $\ket{+}$, mientras que el sistema $\rho_{1}$ siempre tendrá una probabilidad $\frac{1}{2}$ de hallarse en cualquiera de los dos elementos de cualquier base ortogonal que escojamos. Esta es una propiedad del estado máximamente mezclado. La diferencia entre ambos sistemas es que ele elemento probabilístico asociado a las mediciones sobre $\dyad{\psi}$ es de naturaleza cuántica, y se debe a que el sistema se halla en una superposición de estados ortogonales, mietras que en el caso de $\rho$, el elemento probabilístico se debe a nuestra ignorancia sobre la preparación del estado \cite{Chuang}.

Vemos, pues, que hay una diferencia fundamental entre los sistemas que pueden ser descritos por un vector de estado (para los que es posible contruir una matriz de densidad), y aquellos que no. Si para $\rho$ un operador de densidad,
\begin{equation*}
    \rho=\sum_{i}p_{i}\dyad{\varphi_{i}},
\end{equation*}
se cumple que $\rho=\dyad{\varphi_{i}}$ $\forall i$, entonces decimos que $\rho$ es un estado puro, y está completamente caracterizado por el vector de estado $\ket{\varphi}$. En este sentido, los estados puros (aquellos que están descritos por un vector de estado, i.e. su operador de densidad es un proyector) son los puntos extremos del conjunto convexo de operadores de densidad. Estos estados cumplen que
\begin{itemize}
    \item $\rho=\dyad{\psi}$
    \item $\rho=\rho^{n}$
    \item $\Tr(\rho^{2})=1$
\end{itemize}
La pureza es una medida de qué tan puro es un estado, y, dado $\rho\in\mcS(\hilbert_{n})$, se define como \cite{Jaeger}
\begin{equation}
    \text{Pu}(\rho)=\Tr(\rho^{2}).
\end{equation}
Destilan de dicha definición las siguientes dos propiedades:
\begin{itemize}
    \item Un estado es puro si y sólo si $\text{Pu}(\rho)=1$.
    \item Para todo estado, $\frac{1}{n}\geq \text{Pu}(\rho)\geq 1$.
\end{itemize}
\subsubsection{Evolución del operador de densidad}
El postulado de la mecánica cuántica asociado al vector de estado puede reformularse para que funcione con operadores de densidad.\cite{Breuer}

\subsubsection{Parametrización del operador de densidad}
Cualquier matriz de densidad puede descomponerse en términos de una base del espacio de matrices hermitianas de $n\times n$. Una elección común de base para el espacio es el de los generadores $\{\varsigma_{k}\}$ del grupo $\text{SU}(n)$, junto a la matriz identidad $\Id_{n}$. Aunque no está dentro del alcance de este trabajo estudiar las propiedades y caracterizaciones de lestos generadores, su utilizacion permite parametrizar a las matrices de densidad de forma vectorial \cite{Bruning}. En efecto, sea $\{\varsigma_{k}\}$ un conjunto de generadores de $\text{SU}(n)$ y $\rho$ una matriz de densidad $\rho\in\mcS(\hilbert_{n})$. Entonces $\rho$ está completamente descrita por el vector generalizado de Bloch de dimensión $2n^{2}-1$, $\vec{\gamma}$ definido según
\begin{equation}
    \rho=\frac{1}{n}\Id_{n}+\frac{1}{2}\vec{\gamma}\cdot\vec{\varsigma}.
\end{equation}
Si $n=2$, los generadores corresponden a las matrices de Pauli $\sigma_{i}$. En tal caso, el conjunto de vectores de Bloch corresponde a la bola unitaria tridimensional, con los estados puros en la superficie y las mezclas en el interior. Para casos en los que la dimensión es una potencia de $k$, es posible obtener nuevos generadores a través de los productos tensoriales de las matrices de Pauli consigo mismas y con la matriz identidad correspondiente. El caso $n=4$, por ejemplo \cite{Chuang}:
\begin{equation}
    \rho=\frac{1}{4}\sum_{i,j}\gamma_{ij}\sigma_{i}\otimes \sigma_{j} \ \ i,j\in\{0,1,2,3\},
\end{equation}
donde $\sigma_{0}=\Id$ y $\gamma_{i.j}=\sigma_{i}\otimes \sigma_{j}\Tr(\rho)$.
\newpage
\section{Entropía}
\label{sec:ch2_entropy}

\subsection{Entropía de Shannon}
A finales de los años cuarenta, Claude Shannon se preguntaba sobre una medida de la \textit{incertidumbre}, o de la \textit{información} \footnote{En teoría de información clásica, los términos \textit{información}, \textit{incertidumbre} y \textit{sorpresa} se utilizan de manera intercambiable.} asociada a un proceso cuyo resultado estuviera descrito por una variable aleatoria $X$ con distribución de probabilidad $p(x_{j})$.

\acnote{Párrafo iterado: notas}

La cantidad de información provista por el resultado de un experimento depende de la probabilidad asociada a dicho suceso. Por ejemplo, al tirar un dado es mucho menos informativo saber que no cayó un $6$ que saber que cayó un $6$, ya que cada número tiene una probabilidad de $\frac{5}{6}$ de no caer, pero sólo $\frac{1}{6}$ de caer. Con la misma línea de razonamiento, conocer el resultado de un evento que ocurre con probabilidad $p=1$ no transmite ninguna información. Si a cada valor de $X$ se le puede asociar una cantidad de información, entonces debe poder calcularse la cantidad de información promedio: esta es la medida que buscaba Shannon. La forma de esta medida, denotada $H(p)$, vino de las propiedades que el matemático estadounidense afirmó que debía cumplir \cite{Shannon,Wilde}
\begin{enumerate}
    \item $S_{\text{S}}(p)$ debe ser continua en $p$.
    \item $S_{\text{S}}(p)$ debe ser una función creciente, monotónica de $n$ cuando $p_{j}=\frac{1}{n}$.
    \item Si $X$ e $Y$ son procesos independientes, $S_{\text{S}}(p_{X}(x_{j})p_{Y}(y_{l}))=S_{\text{S}}(p_{X}(x_{j}))+S_{\text{S}}(p_{Y}(y_{l}))$.
\end{enumerate}
Además, demostró que
\begin{equation}\label{eq:ShannonEntropy}
    S_{\text{S}}=-k\sum_{j}p(x_{j})\log{p(x_{j})},
\end{equation}
donde $k$ es una constante que depende de la naturaleza del sistema estudiado. Fue a través de discusiones con von Neumann que Shannon descubrió que su medida ya era ampliamente utilizada en física, y que llevaba el nombre de \textit{entropía} \cite{McIrvine}. En efecto, la entropía de Gibbs es
\begin{equation}\label{eq:GibbsEntropy}
    S_{\text{G}}=-k_{\text{B}}\sum_{j}p_{j}\log{p_{j}},
\end{equation}
donde $k_{B}$ es la constante de Boltzmann, y $p_{j}$ es la probabilidad de que el sistema se halle en la $j$-ésima configuración microscópica posible.

Como medida de incertidumbre, la entropía de Shannon (\ref{eq:ShannonEntropy}) es máxima para distribuciones equiprobables. Retomando la idea del dado bien balanceado, como no es posible tener ningún tipo de seguridad sobre el resultado de un tiro, la incertidumbre (la entropía) es máxima.

\acnote{Párrafo iterado: notas}

En teoría de información clásica, la entropía de Shannon se suele utilizar como la cantidad promedio de bits requerida para trasmitir un mensaje, tomando el logaritmo en base $2$ y $k=1$ en la ecuación (\ref{eq:ShannonEntropy}). Para ejemplificar la naturaleza de ``medida de información'' de la entropía de Shannon, supóngase que se desea transmitir un mensaje encriptado en el que únicamente se utilizan los caracteres A, B, C y D. Si el método de encriptación es tal que todos los caracteres tienen la misma probabilidad de aparecer, entonces una forma de transmitir el mensaje es asignándoles los valores $00$, $01$, $10$, y $11$ respectivamente\footnote{Esta forma de codificar los caracteres no es única, pero es la más sencilla. Otra sería asignarle a los caracteres A, B, C y D los valores $101$, $1001$, $10001$ y $100001$. Esta codificación, aunque produzca una cadena de bits que se traduzca de forma única a la cadena de caracteres, es altamente ineficiente. }. Calculando la entropía de Shannon se halla que, en promedio, se requieren dos bits para transmitir cada caracter del mensaje. 

Si, en cambio, las probabilidades de que aparezca cada uno de los caracteres son $p(A)=\frac{1}{2}$, $p(B)=\frac{1}{4}$, $p(C)=\frac{1}{8}$, $p(D)=\frac{1}{8}$. En este caso, una codificación posible, tal que no haya ambigüedad en la cadena de bits, es $A \rightarrow 0$, $B\rightarrow 10$, $C\rightarrow 110$, $D\rightarrow 111$. Nótese que ahora solo se requiere un bit para transmitir la letra más común. Pues bien, si se calcula la entropía de Shannon, se encuentra que cada letra requerirá $1.75$ bits para ser transmitida \cite{Cryptography}.

\subsection{Entropía de von Neumann}

La entropía de von Neumann, a pesar de haber sido obtenida veinte años antes, puede verse como la extensión cuántica de la entropía clásica de Shannon. Von Neumann introdujo el concepto del operador de densidad de forma paralela e independendiente a L. Landau, y definió la entropía $S$ asociada a un sistema descrito por un operador de densidad $\rho$ como \cite{vonNeumann}
\begin{equation}\label{eq:VonNeumannEntropy}
    S_{\text{N}}(\rho)=-\Tr(\rho\ln{\rho}).
\end{equation}
\acnote{El siguiente párrafo ha sido iterado, solo notas}

La entropía de von Neumann puede interpretarse de manera similar a la entropía de Shannon. Si se desea transmitir un qubit preparado como $\ket{\psi_{i}}$ con probabilidad $p_{i}$, entonces el operador de densidad que representa al estado enviado es justamente $\rho=\sum p_{i}\dyad{\psi_{i}}$. La cantidad de información recibida, o la incertidumbre sobre el qubit enviado, es justamente $S(\rho)$. Debe hacerse hincapié en el hecho que la entropía de un sistema cuántico es fundamentalmente diferente a la de un sistema clásico. El sistema cuántico presenta dos tipos de incertidumbres: la incertidumbre clásica, relacionada a nuestra falta de conocimiento relativa a un sistema, y la incertidumbre cuántica, una propiedad intrínseca a los sistemas ondulatorios, matemáticamente expresada a través del Principio de Incertidumbre de Heisenberg \cite{Wilde}.

\acnote{Lista iterada: notas y reescritura}

De la entropía de von Neumann de un sistema descrito por un operador de densidad $\rho\in\mcS(\hilbert_{n})$, nos interesan las siguientes propiedades \cite{Chuang}:
\begin{enumerate}
    \item La entropía puede escribirse en términos de los eigenvalores de $\rho$, $\eta_{j}$, como $S_{\text{N}}(\rho)=-\sum_{j}\eta_{j}\ln{\eta_{j}}$. Esto coincide con la entropía de Shannon si se envían los eigenestados de $\rho$ con probabilidades $\eta_{j}$.
    \item La entropía es no negativa, y es nula si y sólo sí $\rho$ es de la forma $\dyad{\psi}$ con $\ket{\psi}\in\hilbert_{n}$.
    \item La entropía es máxima cuando $\rho=\frac{1}{n}\Id_{n}$, y $S_{\text{N}}(\rho)=n$. Esto es de esperarse, de acuerdo con nuestra discusión previa, el estado máximamente mezclado es aquel del que somos máximamente ignorantes, y por lo mismo debe ser el que tiene la máxima entropía (recordando a la entropía como medida de incertidumbre).
    \item La entropía de un estado producto es igual a la suma de las entropías de cada factor, $S_{\text{N}}(\rho_{A}\otimes\rho_{B})=S_{\text{N}}(\rho_{A})+S_{\text{N}}(\rho_{B})$.
\end{enumerate}
Nótese que la última propiedad es análoga al caso clásico en el que se tienen dos variables aleatorias independientes.
\section{El principio de máxima entropía}\label{sec:CH1MaxEnt}

\subsection{El principio de máxima entropía clásico}

Supóngase que los resultados de un proceso corresponden a los valores $x_{i}$ de una variable aleatoria $X$. Sea, además $f$, una función sobre $X$, de la que conocemos el valor esperado
\begin{equation}\label{eq:JaynesRestrictions}
    \expval{f(x)}=\sum_{i}p(x_{i})f(x_{i}).
\end{equation}
Con esta información, nos interesa hallar la distribución $p(x_{i})$. Introducimos así al Principio de Máxima Entropía.

El principio de máxima entropía fue introducido por E. T. Jaynes en 1957. En su artículo, \textit{Information Theory and Statistical Mechanics}, Jaynes afirma que la distribución de probabilidad obtenida a través del principio de máxima entropía es la mejor estimación que se puede hacer a través de la información disponible, independientemente de si las predicciones coinciden, o no, con los resultados experimentales \cite{JaynesI}.

El problema de hallar una distribución de probabilidad adecuada es también un problema de contaminación de la información accesible. Esta contaminación proviene de suposiciones arbitrarias, y sin sustento físico, que pueden hacerse sobre el sistema. El objetivo es, entonces, hallar la estimación de $p$ menos sesgada posible.

Jaynes relaciona la teoría de información clásica con la mecánica estadística no por la simple coincidencia en la forma de las entropías de Shannon y de Gibbs, sino a través de una reinterpretación de la mecánica estadística como una forma de inferencia estadística. En este contexto, viendo la entropía física como una medida de la incertidumbre asociada a una distribución de probabilidad, una distribución $p$ que no maximice la entropía, es una distribución que introduce información arbitraria no incluída en las hipótesis iniciales.

A través del método de multiplicadores de Lagrange, Jaynes demuestra que la distribución de probabilidad $p$ que maximiza la entropía de Shannon (\ref{eq:ShannonEntropy}), sujeta a las restricciones (\ref{eq:JaynesRestrictions}) es 
\begin{equation}
    p(x_{i})=e^{-\lambda-\mu f(x_{i})}
\end{equation}

Como ejemplo, supongamos que se tira un dado de seis caras $40$ veces, y el promedio de los tirajes es $\expval{X}=3.2$. Si tratáramos con un dado perfectamente equilibrado, y dispusiéramos del tiempo para hacer una infinidad de tirajes, hallaríamos que $\expval{X}=3.5$. Podríamos asumir, entonces, que el dado está bien equilibrado (una susposición de ergodicidad), y que el valor de expectación hallado experimentalmente difiere por simple falta de tirajes, pero esto equivale a hacer una suposición sobre algo que no sabemos a ciencia cierta.

\notaAd{Aqui voy a sacar la distribución de máxima entropía. No es complicado.}

\subsection{Extensión a la mecánica cuántica}

En su segundo artículo, Jaynes

\newpage
\section{Modelos de grano grueso}\label{sec:Ch1CG}

\subsection{Descripciones gruesas en física}

Una descripción de \textit{grano grueso} es aquella que no toma en cuenta todas los detalles de un sistema o fenómeno. Nuestra interacción del día a día con el mundo que nos rodea es fundamentalmente gruesa: al bañarnos, no nos preocupa la energía cinética individial de cada una de las $10^{23}$ moléculas de agua presente en cada una de las gotas que caen sobre nosotros, sino de qué tan caliente, o frío parece el chorro que sale de la llave. Una descripción de grano grueso puede omitir dichos detalles microscópicos por voluntad del observador (puede que no le sea útil toda la información del sistema, o que la cantidad de información sea demasiado grande como para manjearla) o por simple ignorancia de la información omitida.

La termodinámica es un área de la física que trata casi exclusivamente con modelos de grano grueso. Las cantidades termodinámicas: temperatura, presión, volumen, no son sino el resultado de una descripción gruesa de sistemas extremadamente complejos, pues promedian las interacciones y propiedades de $10^{23}$ partículas. La descripción de todo el sistema se reduce a un puñado de coordenadas gruesas.

Cuando se habla de modelos de grano grueso, no se suele hacer referencia a las descripciones efectivas inducidas por la ignorancia. En realidad, cuando se habla de modelos de grano grueso se hace referencia a un modelo impuesto por el observador sobre el sistema. Los modelos de grano grueso que buscan simplificar un problema deshechando información poco útil son comunes en física química. [REFERENCIAS?] El tipo de modelo de grano grueso en el que se centra estre trabajo no es el impuesto por el científico, sino el que proviene de su incapacidad de acceder a toda la información del sistema.

La descripción termodinámica de un sistema de $10^{23}$ partículas corresponde justamente a un modelo de grano grueso inducido por una ignorancia sobre los grados de libertad del sistema. Aún así, auque el observador no cuente con acceso a dicha información, puede deducir que su descripción es meramente efectiva. n efecto, la entropía de un sistema termodinámico es una cantidad que relaciona las coordenadas gruesas con la realidad microscópica.

\subsection{Grano grueso en mecánica cuántica}

En el contexto de la mecánica cuántica, un modelo de grano grueso se obtiene trazando sobre un subsistema del sistema de interés. Al subsistema deshechado se le puede llamar \textit{entorno}, y aunque la separación separación sistema - entrono no es siempre posible \cite{Macro-To-Micro}, nos limitamos a los casos en los que los grados de libertad ignorados pueden trazarse a través de la operación de traza parcial usual.

Un ejemplo sencillo de un modelo de grano grueso es el de un sistema de dos partículas, del cual únicamente nos importa una. En dicho caso, el modelo puede consistir en estudiar únicamente al operador de densidad reducido correspondiente a la partícula de nuestro interés. Es importante notar que el subsistema ignorado no es necesariamente una parte que puede ser separada del sistema, como en el caso de las dos partículas, sino que puede representar un conjunto de información intrínseca al sistema, pero que se ha decidido ignorar. Por ejemplo, puede que se tome en cuenta el momento angular orbital de una partícula, pero no su espín.

Matemáticamente, el modelo de grano grueso corresponde a separar un espacio $\hilbert^{C}$ en dos espacios $\hilbert^{A}$ y $\hilbert^{B}$...


\subsection{Canales cuánticos}



\newpage