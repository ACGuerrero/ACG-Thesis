\section{El operador de densidad}

\subsection{Mezclas estadísticas}

\acnote{El siguiente párrafo ha sido iterado, solo notas}

En el contexto de la mecánica cuántica nos enfrentamos a dos tipos de probabilidades. La primera, la probabilidad cuántica, está codificada dentro de los vectores de estado que se utilizan para describir el estado en el que se puede hallar un sistema. Sin embargo, los vectores de estado no contemplan el segundo tipo de probabilidad: la asociada a la ignorancia. Esta no es una probabilidad cuántica (aquella que se define como el valor absoluto al cuadrado de una amplitud de probabilidad), sino una clásica. Por esto, y porque será particularmente útil para nuestro trabajo, introducimos el concepto del operador de densidad (también llamado, en el caso discreto, que es el que nos incumbe, matriz de densidad).

\acnote{El siguiente párrafo ha sido iterado, solo notas}

Supóngase que en lugar de estudiar un sistema que está completamente descrito por $\ket{\varphi}\in\hilbert_{n}$, con $\hilbert_{n}$ el espacio de Hilbert de dimensión $n$, \ie{} $\hilbert_{n}=\Complex^{n}$, se trabaja con uno que está en el estado $\ket{\varphi_{j}}$ con probabilidad $p_{j}$, donde $\{\ket{\varphi_{j}}\}_{j=1}^{m}$ es un conjunto no necesariamente ortogonal de $m$ estados de $n$ niveles $\ket{\varphi_{j}}\in\hilbert_{n}$, y $\{p_{j}\}_{j=1}^{m}$ es un conjunto de números reales no negativos tales que $\sum_{j=1}^{m} p_{j}=1$.

De este sistema se dice que se halla en un estado de \textit{mezcla estadística}.Esto no debe confundirse con que el sistema se halle en una superposición coherente de estados $\ket{\varphi_{j}}$ con coeficientes $\sqrt{p_{j}}$, ya que dicha superposición está bien determinada, y está completamente descrita por $\ket{\psi}=\sum_{j} e^{\rmi \theta_j} \sqrt{p_{j}}\ket{\varphi_{j}}$, mientras que la mezcla no lo está, pues parte del elemento probabilístico está asociado a un grado de ignorancia sobre la preparación del sistema. La mezcla estadística, en este sentido, toma en cuenta no sólo la probabilidad intrínseca a cada estado cuántico, sino una probabilidad clásica, $p_{j}$. 

Consideremos ahora un observable descrito por un operador hermítico $A$, se sabe que el valor de expectación del observable, con respecto a un estado $\ket{\varphi_{j}}$, está dado por $\expval{A}_{j}=\bra{\varphi_{j}}A\ket{\varphi_{j}}$. El valor esperado de dicho observable con respecto a la mezcla estadística será, justamente, el promedio de los valores esperados respecto a los estados cuánticos en la mezcla:
\begin{equation}
\expval{A}=\sum_{j=1}^{m}p_{j}\bra{\varphi_{j}}A\ket{\varphi_{j}}. \nonumber
\end{equation}
Pues bien, esta expresión puede ser manipulada a través de una base ortogonal $\{\ket{e_{k}}\}_{k=1}^{n}$ del espacio $\hilbert_{n}$:
\begin{align}
\expval{A}&=\sum_{j=1}^{m}p_{j}\bra{\varphi_{j}}A\ket{\varphi_{j}}\nonumber\\ 
&=\sum_{j=1}^{m}\sum_{k,l=1}^{n}p_{j}\bra{e_{l}}\ket{\varphi_{j}}\bra{\varphi_{j}}\ket{e_{k}} \bra{e_{k}}A\ket{e_{l}}. \nonumber
\end{align}
Esta es una suma sobre los elementos de la matriz del observador $A$ y los de las matrices definidas por $\dyad{\varphi_{j}}$. Agrupando la suma sobre $j$ y tomando en cuenta completez de la base $\{\ket{e_{k}}\}_{k=1}^{n}$,
\begin{align}
\expval{A}&=\sum_{k,l=1}^{n}\bra{e_{l}}\qty(\sum_{j=1}^{m}p_{j}\dyad{\varphi_{j}})\ket{e_{k}} \bra{e_{k}}A\ket{e_{l}}\nonumber\\
&=\sum_{l=1}^{n}\bra{e_{l}}\qty(\sum_{j=1}^{m}p_{j}\dyad{\varphi_{j}})A\ket{e_{l}}\nonumber\\
&=\Tr[\qty(\sum_{j=1}^{m}p_{j}\dyad{\varphi_{j}})A]\rlap{,}\nonumber
\end{align}
Con lo que la mezcla queda descrita por el operador de densidad $\rho$, definido según
\begin{equation}\label{eq:DensOpMix}
\rho=\sum_{j=1}^{m}p_{j}\dyad{\varphi_{j}}.
\end{equation}
\acnote{El siguiente párrafo ha sido iterado, solo notas}
Entonces podemos observar que es posible hallar el valor esperado de un observable respecto a un estado  utilizando el operador de densidad, cuya expresión está dada por
\begin{equation}\label{eq:ExpValFromDensOp}
\expval{A}=\Tr(A\rho).
\end{equation}

\acnote{Párrafo eliminado: ver iffalse}

\iffalse
El nombre ``operador de densidad'' puede resultar más claro comparando la ecuación (\ref{eq:ExpValFromDensOp}) con el valor esperado en estadística. Si $X$ es una variable aleatoria con dominio discreto $X=x_1,x_2,\dots$ y cuya distribución probabilidad es $p(x_k)$, entonces el valor esperado de una función $A$ de los valores de $X$ es
\begin{equation}
E[A]=\sum_{k} A(x_k) p(x_k).\nonumber
\end{equation}
\acnote{El siguiente párrafo no ha sido iterado}

\ddnote*{Este texto está raro. Ciertamente hay una analogía, pero no es claro que va por aquí}{Reconociendo que la operación traza no es sino la suma sobre los elementos diagonales de la matriz, es posible ver que la matriz de densidad ocupa un rol similar al de la función de densidad.}
\fi

Ahora, es necesario ser capaz de distinguir si un operador cualquiera corresponde a un operador de la forma (\ref{eq:DensOpMix}). De esta definición destilan dos propiedades que permiten reconocer si un operador arbitrario es un operador de densidad válido, o no \cite{Holevo}:
\begin{enumerate}
    \item $\Tr(\rho)=1$
    \item $\rho\geq 0$,
\end{enumerate}
donde $\rho\geq 0$ se define como $\bra{\varphi}\rho\ket{\varphi}\geq 0$ $\forall$ $\ket{\varphi}\in\hilbert_{n}$.
La primera propiedad se deriva de la normalización de los estados $\ket{\varphi_{j}}$ que definen al sistema. La segunda establece que $\rho$ es una matriz positiva semidefinida y puede interpretarse como la necesidad de que la probabilidad de que el sistema descrito por $\rho$ se halle en el estado $\ket{\varphi}$ sea mayor o igual a $0$. Un operador es un operador de densidad si cumple con estas propiedades. Por esto, estos dos puntos funcionan como una definición alternativa al operador de densidad.

\acnote{En el siguiente párrafo hay tres oraciones idénticas que me toca reescribir.}

A partir de este momento se asume que todos los espacios de Hilbert con los que se trabaja son complejos y de dimension finita. Esto es, son todos del tipo $\hilbert_{n}=\Complex^{n}$. Al conjunto de todos los operadores lineales acotados que actúan sobre un espacio $\hilbert_{n}$ se le denotará como $\mcB(\hilbert_{n})$. Luego, al conjunto de operadores lineales Hermitianos se le denotará mediante $\mcL(\hilbert_{n})$. Finalmente, al conjunto de operadores de densidad se le denotará mediante $\mcS(\hilbert_{n})$. Como nos concentramos en espacios de dimension finita, los operadores tienen representación matricial. Los términos \textit{matriz de densidad} y \textit{operador de densidad} se consideran intercambiables.



\subsection{Pureza}

La diferencia entre una mezcla estadística y una superposición puede no ser del todo clara. ¿Cómo son diferentes un sistema que tiene una probabilidad $p_{j}$ de hallarse en el estado $\ket{\varphi_{j}}$ y otro que se halla en una superposición de cada estado $\ket{\varphi_{j}}$ con coeficientes $\sqrt{p_{j}}$? 

Para responder, considérense dos sistemas de dos niveles. El primero puede hallarse en cualquiera de los siguientes estados
\begin{align}
    \ket{0}=\begin{pmatrix}
        1\\
        0
    \end{pmatrix} && \text{y} && \ket{1}=\begin{pmatrix}
        0\\
        1
    \end{pmatrix}\rlap{,}\nonumber
\end{align}
con la misma probabilidad $p=\frac{1}{2}$. Entonces el operador de densidad que describe al sistema es 
\begin{equation}
    \rho=\frac{1}{2}(\dyad{0}+\dyad{1})=\frac{1}{2}\Id_{2}.\nonumber
\end{equation}
Por otro lado, el segundo sistema se halla en una superposición de los mismos estados, con coeficientes $\sqrt{p}$. El operador de densidad que describe al segundo sistema es 
\begin{align}
    \dyad{\psi} && \text{con} && \ket{\psi}=\frac{1}{\sqrt{2}}(\ket{0}+\ket{1})\rlap{.}\nonumber
\end{align}
Es claro que $\ket{\psi}$ y $\rho$ no describen al mismo objeto, pues $\rho\neq\dyad{\psi}$. Si nos propusiéramos calcular la probabilidad de cada uno de hallarse en el estado $\ket{0}$ encontraríamos que
\begin{align}
    \bra{0}\rho\ket{0}=\frac{1}{2} && \text{y} &&\langle 0 \dyad{\psi} 0\rangle=\frac{1}{2}\rlap{,}\nonumber
\end{align}
y el resultado es el mismo si se hiciera con el estado $\ket{1}$. Parecería entonces que los sistemas se hallan en el mismo estado. Esto es falso. Si realizamos un cambio de base, de $\{\ket{0},\ket{1}\}$ a $\{\ket{+},\ket{-}\}$, donde
\begin{align}
    \ket{+}=\frac{1}{\sqrt{2}}\begin{pmatrix}
        1\\
        1
    \end{pmatrix} && \text{y} && \ket{-}=\frac{1}{\sqrt{2}}\begin{pmatrix}
        1\\
        -1
    \end{pmatrix}\rlap{,}\nonumber
\end{align}
y calculamos la probabilidad de que cada sistema se halle en el estado $\ket{+}$ encontraremos
\begin{align}
    \bra{+}\rho\ket{+}=\frac{1}{2} && \text{pero} &&\langle + \dyad{\psi} +\rangle=1\rlap{.}\nonumber
\end{align}
El segundo resultado es de esperarse, pues $\dyad{\psi}$ se halla en el estado $\ket{+}$. Por otro lado, el sistema $\rho$ siempre tendrá una probabilidad $\frac{1}{2}$ de hallarse en cualquiera de los dos elementos de cualquier base ortogonal que escojamos. La diferencia entre ambos sistemas es que el elemento probabilístico asociado a las mediciones sobre $\dyad{\psi}$ es de naturaleza cuántica, y viene de que el sistema se halla en una superposición de estados ortogonales, mientras que en el caso de $\rho$, el elemento probabilístico se debe a nuestra ignorancia sobre la preparación del estado \cite{Chuang}. El hecho de que hallemos que $\rho$ siempre tenga una probabilidad $\frac{1}{2}$ de hallarse en alguno de los dos elementos de cualquier base ortogonal es una propiedad del estado máximamente mezclado, que puede verse como un estado de cuya preparación somos máximamente ignorantes.

\acnote{La siguiente sección ha sido iterada: reescritura. (viejo en iffalse - iffalse eliminado)}

\acnote{Sección iterada revisada: notas}

Observamos entonces que hay una diferencia fundamental entre los sistemas que pueden ser descritos por un vector de estado y aquellos que no. Considérese el caso en que, dada la expresión \ref{eq:DensOpMix}, el estado del sistema es $\ket{\varphi_{1}}$ con  probabilidad $p_{1}=1$, i.e. $\rho=\dyad{\varphi_{1}}$. En tal caso decimos que $\rho$ es un estado puro: los estados puros no pueden expresarse como una combinación convexa de otros estados. Claramente, $\rho^{2}=\rho$. Esto hace de $\rho$ un proyector de rango $1$, de lo que se sigue que $\Tr(\rho^{2})=1$.

Como en general se cumple que $\Tr(\rho^{2})=\sum_i p^2_i\leq 1$, definimos a la pureza como una medida de que tan puro es un estado como \cite{Jaeger}
\begin{equation}
    \text{Pu}(\rho)=\Tr(\rho^{2}).\nonumber
\end{equation}
De esta definición es posible afirmar que
\begin{itemize}
    \item Un estado es puro si y sólo si $\text{Pu}(\rho)=1$.
    \item Para todo estado $\rho$, $\frac{1}{n}\leq \text{Pu}(\rho)\leq 1$.
\end{itemize}

\subsection{Sistemas multipartitos}\label{sec:Ch1PartialTrace}
\acnote{El siguiente párrafo ha sido iterado: notas}
Hasta ahora hemos hablado de sistemas descritos por operadores de densidad en $\densityspace{n}$, pero, ¿qué sucede si el sistema que estudiamos está conformado por dos subsistemas, cada uno descrito a través de sus respectivos espacios de Hilbert? Sean, pues, $A$ y $B$ dos sistemas con espacios de Hilbert de dimensión finita $\hilbert^{A}$ y $\hilbert^{B}$, y sea $C$ un sistema compuesto por $A$ y $B$. Entonces el producto tensorial de los espacios $\hilbert^{A}$ y $\hilbert^{B}$ es otro espacio de Hilbert, uno asociado al sistema $C$:
 \begin{equation}
     \hilbert^{C}=\hilbert^{A}\otimes\hilbert^{B}.\nonumber
 \end{equation}
 La dimensión del espacio de Hilbert del sistema multipartito cumple
\begin{equation}
    \text{dim}(\hilbert^{C})=\text{dim}(\hilbert^{A})\text{dim}(\hilbert^{B}).\nonumber
\end{equation}
Si $A$ y $B$ representaran dos partículas diferentes, entonces $C$ representa a las partículas como conjunto, como sistema de dos partículas. Si cada una de las partículas puede ser descrita mediante un vector de estado, el estado del sistema es simplemente el producto tensorial de dichos vectores de estado:
\begin{equation}
    \ket{\psi^{A}}\otimes\ket{\psi^{B}}\in\hilbert^{C}\,\; \; \forall\ket{\psi^{A}}\in\hilbert^{A},\ket{\psi^{B}}\in\hilbert^{B}.\nonumber
\end{equation}

\acnote{Párrafo iterado: notas}

Si un estado puede escribirse como un producto tensorial de estados pertenecientes a los subsistemas del sistema multipartito, entonces se dice que es un estado \textit{producto} o \textit{separable}. Nótese que, en general, los estados del sistema compuesto no son estados separables. En realidad, dadas $\{\ket{\varphi_{i}^{A}\}}$ y $\{\ket{\varphi_{j}^{B}\}}$ bases ortonormales de los espacios $\hilbert^{A}$ y $\hilbert^{B}$ respectivamente, podemos escribir a todo estado puro $\ket{\psi^{C}}$ del sistema multipartito como
\begin{equation}
    \ket{\psi^{AB}}=\sum_{j,k}\alpha_{j,k}\ket{\varphi_{j}^{A}}\otimes\ket{\varphi_{k}^{B}}.\nonumber
\end{equation}
El significado físico de que un sistema se halle en un estado producto es que el sistema se halla en un estado en el que no hay correlaciones entre sus subsistemas (de esto que puedan separarse). Un estado que no puede separarse tiene cierto grado de entrelazamiento, y por esto deja de tener sentido hablar de vectores de estado individuales a cada partícula. Ahora, sean $G^{A}$ y $G^{B}$ dos operadores que actúan en $\hilbert^{A}$ y $\hilbert^{B}$ respectivamente, correspondientes a observables de cada subsistema. Entonces se cumple:
\begin{equation}
    G^{A}\ket{\psi^{A}}\otimes G^{B}\ket{\psi^{B}}=(G^{A}\otimes G^{B})\ket{\psi^{A}}\otimes\ket{\psi^{B}}.\nonumber
\end{equation}
\acnote{El siguiente párrafo ha sido iterado: notas}

\acnote{Párrafo iterado: notas}

¿Qué sucede si únicamente nos es relevante el estado de una partícula? Es en este caso en el que surge el concepto de la matriz de densidad reducida. Si $\rho^{C}$ es la matriz de densidad del sistema compuesto por $A$ y $B$, entonces la matriz de densidad reducida del sistema $A$ es
\begin{equation}
    \rho^{A}=\Tr_{B}(\rho^{C}),\nonumber
\end{equation}
donde $\Tr_{B}$ representa la operación de traza parcial con respecto al subsistema $B$. Si la traza de $\rho^{C}$ es 
\begin{equation}
    \Tr(\rho^{C})=\sum_{j}\bra{\varphi^{C}_{j}}\rho^{C}\ket{\varphi^{C}_{j}},\nonumber
\end{equation}
para toda base ortonormal $\{\ket{\varphi^{C}_{j}}\}$ de $\hilbert^{C}$, entonces, para toda base ortonormal $\{\ket{\varphi^{B}_{j}}\}$ de $\hilbert^{B}$  la traza parcial respecto a $B$ es \cite{Hardy}
\begin{equation}
    \Tr_{B}(\rho^{C})=\sum_{j}(\Id^{A}\otimes \bra{\varphi^{B}_{j}})\rho^{C}(\Id^{A}\otimes \ket{\varphi^{B}_{j}}).\nonumber
\end{equation}

\acnote{Párrafo iterado: añadido valor esperado}

Puede verse que el resultado de la operación es trazar sobre los elementos del sistema que no es de interés. La matriz reducida del sistema $A$, o traza parcial con respecto al sistema $B$, actúa como matriz de densidad de $A$, ya que contiene toda la descripción estadística de dicho subsistema. Ahora, sea $G^{A}$ una observable del subsistema $A$. Esto quiere decir que $G^{A}$ actúa sobre $\rho^{C}$ como $(G^{A}\otimes \Id_{B})$. Obsérvese que
\begin{equation}
    \expval{G^{A}}=\Tr\qty((G^A\otimes \Id_B)\rho^C)=\Tr \qty( G^A \rho^A ).\nonumber
\end{equation}

\subsection{Evolución y parametrización}

\acnote{Esta subsección ha sido iterada: reescritura.}
\acnote{Reescritura iterada: notas}

\subsubsection{Evolución de sistemas cerrados}

La evolución de un sistema cuántico cerrado descrito por un vector de estado está dada por la ecuación de Schrödinger,
\begin{equation}
    \rmi\hbar\frac{d}{dt}\ket{\psi(t)}=H\ket{\psi(t)},\nonumber
\end{equation}
cuya solución formal está dada en términos de un operador unitario $U(t,t_{0})$ según
\begin{align}
    \ket{\psi(t)}=U(t,t_{0})\ket{\psi(t_{0})} && \text{con} && U(t,t_{0})=e^{-\rmi H(t-t_{0})/\hbar}\rlap{,}\nonumber
\end{align}
siempre y cuando el hamiltoniano $H$ no dependa explícitamente del tiempo. Pues bien, los postulados de la mecánica cuántica pueden adaptarse al formalismo de operadores de densidad. De la ecuación de Schrödinger se sigue que la evolución de un sistema descrito por un operador de densidad $\rho$ está descrita por ecuación de Liouville-von Neumann,
\begin{equation}
    \rmi\hbar\frac{d}{d t} \rho(t)=[H,\rho(t)].\nonumber
\end{equation}
De la misma forma que antes, la solución queda expresada en términos de un operador unitario,
\begin{equation}
    \rho(t)=U(t,t_{0})\rho(t_{0})U^{\dagger}(t,t_{0}).\nonumber
\end{equation}

\subsubsection{Evolución de sistemas abiertos}

\acnote{Reescritura iterada: notas}

Considérese, en cambio, que el sistema de interés está acoplado a un entorno $E$ cuyo estado es $\rho_{E}$ y que, inicialmente, el conjunto de estos dos conforma un sistema cerrado descrito por el operador de densidad $\rho(0)=\rho_{S}(0)\otimes\rho_{E}$. Esta condición inicial está justificada experimentalmente si se considera que el sistema $S$ se prepara al hacer una medición proyectiva, de tal manera que el estado total queda separable. Dado que el total es cerrado, este evoluciona siguiendo la ecuación de Liouville-von Neumann,
\begin{align}
    \rmi\hbar\frac{d}{d t} \rho(t)=[H,\rho(t)] && \text{con} && \rho(0)=\rho_{S}(0)\otimes\rho_{E}\rlap{.}\nonumber
\end{align}
Sin embargo, para hallar la ecuación de la dinámica del sistema de interés es necesario trazar al entorno de ambos lados de esta ecuación, de forma que se halla
\begin{align}
    \rmi\hbar\frac{d}{d t} \rho_{S}(t)=\Tr_{E}([H,\rho(t)]) && \text{con} && \rho(0)=\rho_{S}(0)\otimes\rho_{E}\rlap{,}\nonumber
\end{align}
cuya solución formal está dada en términos de un superoperador parametrizado por $t$, $\mcE_{t}$,
\begin{equation}
    \rho_{S}(t)=\mcE_{t}(\rho_{S}(0)).\nonumber
\end{equation}
En esta ecuación, $\mcE_{t}$ está definido como
 \begin{equation}
    \mcE_{t}(\rho_{S}(0))=\Tr_{E}\qty[U(t,0)\left(\rho_{S}(0)\otimes\rho_{E}\right)U^{\dag}(t,0)],\nonumber
 \end{equation}
 y cumple que es un canal cuántico \cite{Ziman} (también llamado operación cuántica \cite{Chuang} o aplicación dinámica \cite{Breuer}) $\forall t$, y que $\mcE_0=\text{id}_S$. El formalismo de los canales cuánticos queda fuera del alcance de este trabajo, y sin entrar en más detalle, basta con señalar que los canales cuánticos son aplicaciones lineales completamente positivas que preservan la traza, \ie{} todo canal cuántico $\mcE:\mcB(\hilbert_{A})\rightarrow \mcB(\hilbert_{B})$ cumple que \cite{Watrous}
 \begin{enumerate}
    \item $\mcE\otimes \text{id}_{k}: \mcB(\hilbert_A \otimes \hilbert_k)\to \mcB(\hilbert_B \otimes \hilbert_k)$ es positivo para todo entero positivo $k$.
    \item $\Tr\qty(\mcE(\Delta))=\Tr\qty(\Delta)$.
 \end{enumerate}
Los canales cuánticos están relacionados directamente al esquema de sistemas abiertos a través del teorema de Stinespring \cite{Watrous}, de acuerdo con el cual dado un canal cuántico
\begin{equation}
    \mcE:\mcB(\hilbert)\rightarrow \mcB(\hilbert),\nonumber
\end{equation}    
existen un espacio de Hilbert $\tilde \hilbert$ y un operador unitario $U$ que actúa sobre $\hilbert\otimes \tilde \hilbert$ tales que
\begin{equation}
    \mcE(\rho)=\Tr_{C}\qty(U  \left(\rho \otimes \dyad{e}{e}\right) U^{\dag}),\nonumber
\end{equation}
donde $\rho\in\mcS(\hilbert)$ y $\ket{e}\in \tilde \hilbert$. \ddnote{Hice varios cambios aquí, cuando el canal opera entre diferentes espacios de Hilbert, el teorema es ligeramente mas complicado. Mejor simplifiqué a cuando se usa el mismo espacio.}\acnote{\checkmark}

\subsubsection{Parametrización del operador de densidad}
\acnote{Sección iterada: reescritura}

\acnote{Reescritura iterada: notas, reescritura}

Es común escoger alguna base hermítica para poder parametrizar a las matrices de densidad. El beneficio de hacer esto es que, por ser $\mcS(\hilbert_{n})$ un subconjunto de $\mcL(\hilbert_{n})$, dicha parametrización será real, y aún más: los parámetros serán promedios de observables. Dicho de otro modo, se puede escoger una base de tal forma que sea un conjunto de observables a través del cual sea posible reconstruir el estado de un sistema. Una elección común de base son las matrices generalizadas de Gell-Mann, junto a la matriz identidad.

Sea $\{\varsigma_{k}\}_{k}$ el conjunto de $n^{2}-1$ matrices generalizadas de Gell-Mann que generan a $\text{SU}(n)$ y $\rho$ una matriz de densidad $\rho\in\mcS(\hilbert_{n})$. Entonces $\rho$ puede expandirse en la base formada por dichas matrices y la identidad a través del producto punto de Hilbert-Schmidt como
\begin{equation}
    \rho=\frac{1}{n}\qty(\Id_{n}\Tr(\rho)+\sum_{k=1}^{n^{2}-1}\Tr(\rho\varsigma_{k})\varsigma_{k}).\nonumber
\end{equation}
Lo que significa que está completamente descrita por el vector generalizado de Bloch de dimensión $n^{2}-1$, $\vec{\gamma}$ definido a través de
\begin{equation}
    \gamma_{k}=\Tr(\rho\varsigma_{k}),\nonumber
\end{equation}
de tal manera que la matriz de densidad $\rho$ puede escribirse como
\begin{equation}\label{eq:BlochParametrization}
    \rho=\frac{1}{n}\qty(\Id_{n}+\vec{\gamma}\cdot\vec{\varsigma}).
\end{equation}
Si $n=2$, los generadores corresponden a las matrices de Pauli $\sigma_{j}$. En tal caso, el conjunto de vectores de Bloch corresponde a la bola unitaria tridimensional, con los estados puros en la superficie y los estados mixtos en el interior. Para casos en los que la dimensión es una potencia de $2$, es posible obtener nuevos generadores a través de los productos tensoriales de las matrices de Pauli consigo mismas y con la matriz identidad. El caso $n=4$, por ejemplo \cite{Chuang}:
\begin{equation}\label{eq:BlochParametrization4}
    \rho=\frac{1}{4}\sum_{j,k=0}^{3}\gamma_{jk}\sigma_{i}\otimes \sigma_{k},\nonumber
\end{equation}
donde $\sigma_{0}=\Id$ y 
\begin{equation}\label{eq:BlochComponents}
        \gamma_{jk}=\Tr(\sigma_{j}\otimes \sigma_{k}\rho).
\end{equation}
Obsérvese que, como se mencionó previamente, los parámetros $\gamma_{jk}$ son promedios de las observables $\sigma_{j} \otimes \sigma_{k}$. Combinar las ecuaciones (\ref{eq:BlochComponents}) y (\ref{eq:BlochParametrization}) permite reconstruir el estado del sistema. A esto se le llama \textit{tomografía de estados cuánticos} \cite{Chuang}.
