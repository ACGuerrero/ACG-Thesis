\chapter{Introducción}

\acnote{Párrafo de introducción al tema: iterado}

\acnote{Segunda iteración: notas}

Muchas áreas de la física tratan casi exclusivamente con descripciones efectivas de los sistemas que estudian. Por ejemplo, la mecánica estadística y la mecánica clásica se encargan de los efectos observables de una realidad microscópica. En mecánica clásica, la interacción entre dos superficies rugosas y la disipación de energía en forma de calor se piensa como una fuerza que se opone al movimiento. En mecánica estadística, la energía cinética de miles de millones de partículas dentro de un recipiente es proporcional a la temperatura el gas. A este tipo de descripciones las llamamos modelos de grano grueso (\textit{coarse-grained models}), mientras que a descripciones detalladas del comportamiento microscópico de un sistema se llaman modelos de grano fino (\textit{fine-grained models}). Aunque \textit{grano grueso} no sea un término que sea comúnmente leído en los libros de texto de física, ha estado en el centro del progreso de la física como ciencia, en particular en la física estadística. \footnote{Aunque históricamente el reduccionismo metodológico ha ido de la mano con el reduccionismo de la teoría, no hay razón para pensar que esto deba ser así. El problema de la medida es un ejemplo de un problema que trata con un fenómeno que emerge a grandes escalas. Sin embargo, una discusión más detallada de este tema queda fuera del alcance de esta tesis.}. En efecto, la estructura gruesa atómica (\textit{atomic gross structure}), en la que únicamente se estudia la interacción de Coulomb entre electrones puntuales y un núcleo también puntual, puede verse como una descripción efectiva de la estructura fina del átomo, en la que sí se toma en cuenta la interacción espín-órbita, así como una corrección relativista a la energía cinética y el término de Darwin \cite{Bransden}. Esta, a su vez, puede verse como una descripción efectiva de la estructura hiperfina del átomo, que sí considera al espín nuclear y las interacciones que este tiene con el resto del sistema.

\acnote{Párrafo de motivación: iterado}

Aunque la mecánica cuántica describe con alta precisión los fenómenos del mundo microscópico, los objetos macroscópicos son mejor descritos por la física clásica. De acuerdo con el principio de correspondencia de Bohr, la mecánica cuántica debería poder realizar las mismas predicciones que la física clásica en el límite de números cuánticos grandes. La transición cuántica-clásica es objeto de estudio hasta el día de hoy. Los modelos de colapso objetivo y el darwinismo cuántico son ejemplos de teorías a través de las que se busca explicar la emergencia de los fenómenos de la física clásica desde el mundo cuántico \cite{Zurek, Bassi}. En este contexto, un modelo de grano grueso puede utilizarse para describir un número arbitrariamente grande de partículas utilizando un número pequeño de variables. De esta manera, los modelos de grano grueso podrían permitir echar una mirada a la transición entre el mundo cuántico y el mundo clásico.

\acnote{Enunciación de la problemática: iterado, reescritura}

\acnote{Segunda iteración: reescritura}

Con esto en mente, nos preguntamos sobre las características de la dinámica de un sistema cuántico descrito a través un modelo de grano grueso. Esto es, investigamos las propiedades que emerjan de las reglas de la mecánica cuántica cuando se tiene una descripción gruesa del sistema analizado. En particular, el modelo de grano grueso utilizado concatena la posibilidad de medir una partícula diferente a la pretendida y una falta de resolución en el aparato de detección. Se supone que el observador conoce tanto el número de partículas del sistema microscópico como la evolución experimentada por este, que se propaga siguiendo las leyes de la mecánica cuántica. Es aquí donde surge uno de los primeros problemas, pues si solo se tiene acceso a la información \textit{gruesa}, ¿cómo saber qué estado microscópico es el que se propaga? En efecto, el estado observado puede ser compatible con un gran número de estados microscópicos. Una manera de enfrentarse a este problema es utilizar el \textit{principio de mínima información}, o \textit{principio de máxima entropía}. Esto es, toda la información no contenida dentro del estado efectivo inicial, respecto al estado fino inicial, se considera mínima. De esta manera se abordará el problema de la emergencia de dinámicas macroscópicas, usando este principio \cite{JaynesI,Brillouin} en combinación con el modelo de grano grueso. Dicho de otra forma, se utilizará, a la par de dicho modelo, un enfoque de física estadística. %Este párrafo a sido modificado para tomar en cuenta los comentarios de Miguel B. Ahora se menciona de forma explícita el uso de la física estadística. Para complementar estas ediciones, tengo que tener bien estudiados los postulados de la mecánica estadística.


Para abordar el problema, en el capítulo \ref{ch:2} se introducen los conceptos que serán fundamentales a lo largo del trabajo. Principalmente, el formalismo de operadores de densidad, el Principio de Máxima Entropía, y los modelos de grano grueso. Una vez establecida la base teórica, el capítulo \ref{ch:3} presenta la aplicación (mapeo o \textit{mapping} en inglés) de grano grueso que se utilizará, así como la \textit{aplicación de asignación de máxima entropía}, y se define matemáticamente a la dinámica efectiva como la evolución seguida por el sistema efectivo. Con dichas herramientas en mano, en el capítulo \ref{ch:4} se desarrolla el estudio de las dinámicas efectivas inducidas por diferentes tipos de dinámicas microscópicas. Se consideran evoluciones unitarias locales, algunas compuertas de uso común en cómputo cuántico, canales de Pauli, la cadena de Ising, entre otros. Finalmente, en el capítulo \ref{ch:5} se comparan los resultados obtenidos a través de la aplicación de asignación de máxima entropía con aquellos que pueden surgir del uso de otra aplicación de asignación: la \textit{aplicación de asignación promedio}.