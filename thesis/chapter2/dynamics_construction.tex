\section{Construcción de la dinámica}\label{sec:ch2dycon}

Ahora que hemos establecido que usaremos como modelo de grano grueso uno que incluye tanto problemas de resolución como errores de permutación, y que hemos contruído nuestra aplicación de asignación a través del Principio de Máxima Entropía, podemos preguntarnos sobre la evolución del sistema efectivo, la ``dinámica gruesa'', denotada como $\Gamma_t$. La dinámica efectiva es una aplicación dinámica que corresponde a la evolución observada por un experimentalista. Dado un estado efectivo inicial $\rho_{\ef}(0)$,
\begin{gather}
\Gamma_{t}:\mcS(\hilbert_2)\rightarrow \mcS(\hilbert_2)\nonumber\\
\rho_{\ef}(0) \mapsto \Gamma_{t}(\rho_{\ef}(0))\rlap{.}\nonumber
\end{gather}
Debido que asumimos que el estado que se propaga debido a la evolución subyacente es justamente un estado compatible con $\rho_{\ef}$, seleccionado a través de una aplicación de asignación, a la dinámica gruesa la definimos como la composición
\begin{equation}\label{eq:EffectiveDynamics}
\Gamma_t:=\mcC \circ \mcV_t \circ \mcA_{\mcC}^{\max}.
\end{equation}

\acnote{Párrafo iterado: reescritura}

Donde $\mcV_{t}$ es la evolución seguida por el sistema microscópico. Esta puede ser unitaria, o un canal cuántico. El siguiente diagrama ilustra la ecuación anterior,
\[\begin{tikzcd}[arrows={<-|}]
    \rho_{\ef}(0)  & \rho_{\ef}(t) \arrow{l}{\Gamma_{t}} \arrow{d}{\mcC}\\
\varrho_{\max}(0) \arrow{u}{\mcA_{\mcC}^{\max}} & \varrho_{\max}(t). \arrow{l}{\mcV_{t}}
\end{tikzcd}
\]

\acnote{Párrafo iterado: reescritura}

Debe notarse que, debido a la no invertibilidad de la aplicación de grano grueso, en general
\begin{equation}
    (\mcU_{t}\circ\mcA_{\mcC}^{\max})(\rho_{\ef}) \neq (\mcA_{\mcC}^{\max}\circ\mcC \circ \mcU_t \circ \mcA_{\mcC}^{\max})(\rho_{\ef}),\nonumber
\end{equation}
que también puede escribirse como
\begin{equation}
    \varrho_{\max}(t)\neq\mcA_{\mcC}^{\max}(\rho_{\ef}(t))\nonumber
\end{equation}
Después de todo, la maximización de la entropía se restringe de acuerdo a las observaciones experimentales, así que estados de máxima entropía que cumplan un conjunto particular de restricciones no tienen por qué satisfacer un conjunto diferente de restricciones.



En el siguiente capítulo se analizarán dinámicas efectivas generadas por diferentes dinámicas subyacentes. Si se asume que el sistema conformado por las partículas es cerrado, entonces la evolución $\mcV_{t}$ será unitaria, generada por un hamiltoniano $H$. Algunos ejemplos de dinámicas subyacentes no unitarias son los canales de ruido usuales, como el canal de \textit{despolarización}, el canal de \textit{amortiguamiento de amplitud}, o el canal de \textit{amortiguamiento de fase}. \ddnote{puse italicas en algunas partes}\acnote{enterado}


\acnote{Párrafo iterado: notas}

A diferencia de los mapas dinámicos usualmente estudiados en teoría de sistemas cuánticos abiertos, la dinámica efectiva $\Gamma_{t}$ no tiene por qué ser lineal (sí debe, por supuesto, mandar estados cuánticos a estados cuánticos), debido a que uno de los elementos de la composición que la originan no siempre es lineal: la aplicación de asignación. El estudio de las particularidades de algunas de estas dinámicas efectivas es el foco de este trabajo.


