\section{Preámbulo}

\subsection{El operador de densidad}

\begin{frame}{Mezclas estadísticas}
    \begin{columns}
        \begin{column}{0.5\textwidth}
            En el contexto de la mecánica cuántica nos enfrentamos a dos tipos de probabilidades. La primera, la probabilidad cuántica, está codificada dentro de los vectores de estado que se utilizan para describir al estado en el que se pueda hallar un sistema. Los vectores de estado, sin embargo, no contemplan el segundo tipo de probabilidad: la asociada a la ignorancia.
        \end{column}
        \pause
        \begin{column}{0.5\textwidth}
            Supóngase que se estudia un sistema del que se sabe se halla en el estado $\ket{\varphi_{j}}$ con probabilidad $p_{j}$, donde $\{\ket{\varphi_{j}}\}_{j=1}^{m}$ es un conjunto no necesariamente ortogonal de $m$ estados $\ket{\varphi_{j}}$ pertenecientes al espacio de Hilbert de dimensión $n$, $\hilbert_{n}$, y $\{p_{j}\}_{j=1}^{m}$ es un conjunto de números reales no negativos tales que $\sum_{j=1}^{m} p_{j}=1$. De este sistema se dice que se halla en un estado de \textit{mezcla estadística}.
        \end{column}
    \end{columns}
\end{frame}

\begin{frame}{Parametrización}
    \begin{columns}
        \begin{column}{0.5\textwidth}
            Es común escoger alguna base hermítica para poder parametrizar a las matrices de densidad.

            Sea $\{\varsigma_{k}\}_{k}$ el conjunto de $n^{2}-1$ matrices generalizadas de Gell-Mann que generan a $\text{SU}(n)$ y $\rho$ una matriz de densidad $\rho\in\mcS(\hilbert_{n})$.
        \end{column}
        \pause
        \begin{column}{0.5\textwidth}
            Entonces $\rho$ puede expandirse en la base formada por dichas matrices y la identidad a través del producto punto de Hilbert-Schmidt como
            \begin{equation}
                \rho=\frac{1}{n}\qty(\Id_{n}\Tr(\rho)+\sum_{k=1}^{n^{2}-1}\Tr(\rho\varsigma_{k})\varsigma_{k}).\nonumber
            \end{equation}
        \end{column}
    \end{columns}
\end{frame}

\subsection{Evolución}
\begin{frame}{Sistemas cerrados}
    \begin{columns}
        \begin{column}{0.5\textwidth}
            La evolución de un sistema cuántico cerrado descrito por un vector de estado está dada por la ecuación de Schrödinger,
            \begin{equation}
                \rmi\hbar\frac{d}{dt}\ket{\psi(t)}=H\ket{\psi(t)}.\nonumber
            \end{equation}
        \end{column}
        \pause
        \begin{column}{0.5\textwidth}
            La evolución de un sistema descrito por un operador de densidad $\rho$ está descrita por ecuación de Liouville-von Neumann,
            \begin{equation}
                \rmi\hbar\frac{d}{d t} \rho(t)=[H,\rho(t)].\nonumber
            \end{equation}
        \end{column}
    \end{columns}
\end{frame}

\begin{frame}{Sistemas abiertos}
    \begin{columns}
        \begin{column}{0.5\textwidth}
            para hallar la ecuación de la dinámica del sistema de interés es necesario trazar al entorno de ambos lados de esta ecuación, de forma que se halla
            \begin{align}
                \rmi\hbar\frac{d}{d t} \rho_{S}(t)=\Tr_{E}([H,\rho(t)])\rlap{,}\nonumber
            \end{align}
            con $\rho(0)=\rho_{S}(0)\otimes\rho_{E}$, y cuya solución formal está dada en términos de un superoperador parametrizado por $t$, $\mcE_{t}$,
                \begin{equation}
                    \rho_{S}(t)=\mcE_{t}(\rho_{S}(0)).\nonumber
                \end{equation}
        \end{column}
        \pause
        \begin{column}{0.5\textwidth}
            En esta ecuación, $\mcE_{t}$ está definido como
            \begin{equation}
                \mcE_{t}(\rho_{S}(0))=\Tr_{E}\qty[U(t,0)\left(\rho_{S}(0)\otimes\rho_{E}\right)U^{\dag}(t,0)],\nonumber
            \end{equation}
            y cumple que es un canal cuántico
        \end{column}
    \end{columns}
\end{frame}

\begin{frame}{Canal de desfasamiento}
    \begin{columns}
        \begin{column}{0.5\textwidth}
            El \textit{canal de desfasamiento} tiene operadores de Kraus $\{\sqrt{p}\Id,\sqrt{(1-p)}\pauli{3}\}$. Su efecto sobre un estado $\rho$ es
            \begin{equation}
                \rho\mapsto p\rho+(1-p)\pauli{3}\rho\pauli{3}.\nonumber
            \end{equation}
        \end{column}
        \begin{column}{0.5\textwidth}
            FIGURA
        \end{column}
    \end{columns}
\end{frame}

\begin{frame}{Canal de bitflip}
    \begin{columns}
        \begin{column}{0.5\textwidth}
            El \textit{canal de bitflip} es análogo al canal de desfasamiento. Este canal tiene por operadores de Kraus $\{\sqrt{p}\Id,\sqrt{(1-p)}\pauli{1}\}$, y no será difícil ver que su efecto es el de reducir la magnitud de las componentes de $\pauli{2}$ y $\pauli{3}$ por un factor de $2p-1$.
        \end{column}
        \begin{column}{0.5\textwidth}
            FIGURA
        \end{column}
    \end{columns}
\end{frame}

\begin{frame}{Canal de despolarización}
    \begin{columns}
        \begin{column}{0.5\textwidth}
            Finalmente, el \textit{canal de despolarización} se define mediante
            \begin{equation}
                \rho\mapsto p\frac{1}{2}\Id+(1-p)\rho\nonumber.
            \end{equation}
Esto equivale a perder toda la información acerca del estado con una probabilidad $p$.
        \end{column}
        \begin{column}{0.5\textwidth}
            FIGURA
        \end{column}
    \end{columns}
\end{frame}

\subsection{Entropía}
\begin{frame}{Entropía clásica}
    \lipsum[1]
\end{frame}

\begin{frame}{Entropía cuántica}
    \lipsum[1]
\end{frame}

\subsection{El principio de Máxima Entropía}

\begin{frame}{El PME clásico}
    \lipsum[1]
\end{frame}

\begin{frame}{El PME cuántico}
    \lipsum[1]
\end{frame}

\subsection{Modelos de grano grueso}

\begin{frame}{Modelos de grano grueso}
    \lipsum[1]
\end{frame}