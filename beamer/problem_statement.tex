\section{Introducción}


\begin{frame}{Qubits}
    \begin{center}
        ``\textbf{Dinámica} efectiva de un sistema de N \textbf{qubits}''
    \end{center}
    \pause
    \begin{columns}
        \begin{column}{0.5\textwidth}
            \begin{center}
                Un qubit es un sistema cuántico de dos estados.
                \pause
                Todo qubit se puede escribir como
                \begin{equation}
                    \ket{\psi}=\alpha\ket{0}+\beta\ket{1}.\nonumber
                \end{equation}
                \pause
                Por convención usamos los eigenestados del operador $\pauli{3}$.
                \begin{align}
                    \ket{0} & & \text{y} & & \ket{1}. \nonumber
                \end{align} 
            \end{center}
        \end{column}
        \pause
        \begin{column}{0.5\textwidth}
            La evolución de un sistema cuántico descrito por $\ket{\psi}$ está dada por
            \begin{equation}
                \rmi\hbar\frac{d}{dt}\ket{\psi(t)}=H\ket{\psi(t)}.\nonumber
            \end{equation}
            \pause
            Cuya solución es una evolución unitaria $U(t,t_{0})=e^{-\rmi H(t-t_{0})/\hbar}$
            \begin{align}
                \ket{\psi(t)}=U(t,t_{0})\ket{\psi(t_{0})} \rlap{.}\nonumber
            \end{align}
        \end{column}
    \end{columns}
\end{frame}

\begin{frame}{Parametrización de un qubit}
    \begin{columns}
        \begin{column}{0.5\textwidth}
            Un qubit $\ket{\psi}\in\hilbert_{2}$ arbitrario
            \begin{equation}
                \ket{\psi}=\alpha\ket{0}+\beta\ket{1},\nonumber
            \end{equation}
            \pause
            se puede reescribir como:
            \begin{equation}
                \ket{\psi}=\cos(\frac{\theta}{2})\ket{0}+e^{\rmi\varphi}\sin(\frac{\theta}{2})\ket{1}.\nonumber
            \end{equation}
        \end{column}
        \begin{column}{0.5\textwidth}
            \centering
            \BlochSphere
        \end{column}
    \end{columns}
\end{frame}



\begin{frame}{Efectiva}
    \begin{center}
        ``Dinámica \textbf{efectiva} de un sistema de N qubits''
    \end{center}
    \pause
    \begin{columns}
        \begin{column}{0.5\textwidth}
            \begin{center}
                Dinámica \textbf{efectiva}\\
                \pause
                $\downarrow$\\
                Dinámica {\tiny(que emerge de una descripción)} \textbf{efectiva}
            \end{center}
            \pause
            Aquí, 
            \begin{center}
                \textit{efectiva}$\iff$\textit{gruesa}.
            \end{center}
        \end{column}
        \pause
        \begin{column}{0.5\textwidth}
            La descripción gruesa de un sistema puede ser resultado de
            \begin{itemize}
                \item Aparato de medición imperfecto
                \item Descarte de grados de libertad del sistema
            \end{itemize}
        \end{column}
    \end{columns}
\end{frame}

\begin{frame}{Modelos de grano grueso}
    \begin{center}
        Una descripción de grano grueso es aquella que no toma en cuenta todos los detalles de un sistema o fenómeno.
    \end{center}
    \pause
    \begin{columns}
        \begin{column}{0.5\textwidth}
            \begin{center}
                Variables termodinámicas\\
                $\downarrow$\\
                Descripción efectiva\\
                \pause
                \begin{equation}
                    T=\frac{1}{k_\text{B}}\frac{2}{3}\expval{E_{\text{cin}}}.\nonumber
                \end{equation}
            \end{center}
        \end{column}
        \begin{column}{0.5\textwidth}
            \begin{figure}
                \includegraphics[width=1\textwidth]{figures/CGT.png}
            \end{figure}
        \end{column}
    \end{columns}
\end{frame}

\subsection{El operador de densidad}

\begin{frame}{Mezclas estadísticas}
    \begin{columns}
        \begin{column}{0.5\textwidth}
            Los vectores de estado contienen probabilidad cuántica:
            \begin{equation}
                \ket{\psi}=\frac{1}{\sqrt{2}}(\ket{0}+\ket{1}),\nonumber
            \end{equation}
            \pause
            pero no el segundo tipo de probabilidad: la asociada a la ignorancia.
        \end{column}
        \pause
        \begin{column}{0.5\textwidth}
            Un sistema del que se sabe se halla en el estado $\ket{\varphi_{j}}$ con probabilidad $p_{j}$.\\
            \vspace{0.2cm}
            \pause
            De este sistema se halla en un estado de \textit{mezcla estadística}.\\
            \vspace{0.2cm}
            \pause
            Es descrito por el operador de densidad
            \begin{equation}
                \rho=\sum_{j}p_{j}\dyad{\varphi_{j}}.\nonumber
            \end{equation}
        \end{column}
    \end{columns}
\end{frame}

\begin{frame}{Parametrización}
    \begin{columns}
        \begin{column}{0.5\textwidth}
            Una base hermítica permite parametrizar a una matriz de densidad a través del producto punto de Hilbert-Schmidt
            \begin{equation}
                \rho=\frac{1}{2}\qty(\Id_{2}\Tr(\rho)+\sum_{k=1}^{3}\Tr(\rho\pauli{k})\pauli{k}).\nonumber
            \end{equation}
            \pause
            El vector de Bloch es
            \begin{equation}
                \vec{r}_{\rho}=\begin{pmatrix}
                    \Tr(\rho\pauli{x})\\
                    \Tr(\rho\pauli{y})\\
                    \Tr(\rho\pauli{z})
                \end{pmatrix}\nonumber
            \end{equation}
        \end{column}
        \pause
        \begin{column}{0.5\textwidth}
            \centering
            \BlochSphereDensity
        \end{column}
    \end{columns}
\end{frame}


\begin{frame}{Sistemas multipartitos}
    \begin{center}
        ``Dinámica efectiva de un sistema de \textbf{N} qubits''
    \end{center}
    \pause
    \begin{columns}
        \begin{column}{0.5\textwidth}
            \begin{equation}
                \hilbert_{\text{bi}}=\hilbert_{2}\otimes\hilbert_{2}.\nonumber
            \end{equation} \pause
            La dimensión cumple
           \begin{equation}
               \text{dim}(\hilbert_{\text{bi}})=\text{dim}(\hilbert_{2})\text{dim}(\hilbert_{2}).\nonumber
           \end{equation}
           \pause
           ¿Qué sucede si únicamente nos es relevante el estado de una partícula?
        \end{column}
        \pause
        \begin{column}{0.5\textwidth}
             Si $\rho_{\text{bi}}$ describe dos qubits A y B, \pause
            \begin{equation}
                \rho^{A}=\Tr_{B}(\rho_{\text{bi}}),\nonumber
            \end{equation}
            donde $\Tr_{B}$ es la traza parcial respecto al qubit $B$.\\
            \pause
            \begin{center}
                ¿Y la evolución?
            \end{center}
        \end{column}
    \end{columns}
\end{frame}

\begin{frame}{Evolución abierta}
    Supóngase que el sistema de interés está acoplado a un \textit{entorno}: $\rho(0)=\rho_{S}(0)\otimes\rho_{E}$
    \vspace*{0.3cm}
    \begin{columns}
        \begin{column}{0.5\textwidth}
            El sistema completo evoluciona de forma unitaria.\begin{equation}
                \rmi\hbar\frac{d}{d t} \rho(t)=[H,\rho(t)].\nonumber
            \end{equation}\pause
            Para hallar la dinámica del sistema de interés hay que trazar:
            \begin{align}
                \rmi\hbar\frac{d}{d t} \rho_{S}(t)=\Tr_{E}([H,\rho(t)])\rlap{,}\nonumber
            \end{align}
        \end{column}
        \pause
        \begin{column}{0.5\textwidth}
            \begin{equation}
                \rho_{S}(t)=\mcE_{t}(\rho_{S}(0)).\nonumber
            \end{equation}\pause
            \begin{equation}
                \mcE_{t}(\rho_{S}(0))=\Tr_{E}\qty[U(t,0)\left(\rho_{S}(0)\otimes\rho_{E}\right)U^{\dag}(t,0)],\nonumber
            \end{equation}\pause
            Es un canal cuántico y\pause
            \begin{equation}
                \mcE(\rho)=\sum_{k}A_{k}\rho A^{\dagger}_{k}\nonumber
            \end{equation}\pause
            es su representación en suma de operadores de Kraus.
        \end{column}
    \end{columns}
\end{frame}

\begin{frame}{Canal de desfasamiento}
    El \textit{canal de desfasamiento} tiene operadores de Kraus $\{\sqrt{p}\Id,\sqrt{(1-p)}\pauli{3}\}$. Su efecto sobre un estado $\rho$ es
    \begin{equation}
        \rho\mapsto p\rho+(1-p)\pauli{3}\rho\pauli{3}.\nonumber
    \end{equation}
    \begin{figure}
        \centering
        \begin{subfigure}{0.45\textwidth}
            \centering
            \includegraphics[width=0.5\textwidth]{figures/whole_sphere.png}
        \end{subfigure}
        \begin{subfigure}{0.45\textwidth}
            \centering
            \includegraphics[width=0.5\textwidth]{figures/dephased.png}
        \end{subfigure}
    \end{figure}
\end{frame}

\begin{frame}{Canal de bitflip}
    El \textit{canal de bitflip} tiene operadores de Kraus $\{\sqrt{p}\Id,\sqrt{(1-p)}\pauli{1}\}$. Su efecto es el de reducir la magnitud de las componentes de $\pauli{2}$ y $\pauli{3}$ por un factor de $2p-1$.
    \begin{figure}
        \centering
        \begin{subfigure}{0.45\textwidth}
            \centering
            \includegraphics[width=0.5\textwidth]{figures/whole_sphere.png}
        \end{subfigure}
        \begin{subfigure}{0.45\textwidth}
            \centering
            \includegraphics[width=0.5\textwidth]{figures/bitflip.png}
        \end{subfigure}
    \end{figure}
\end{frame}

\begin{frame}{Canal de despolarización}
    Finalmente, el \textit{canal de despolarización} se define mediante
    \begin{equation}
        \rho\mapsto p\frac{1}{2}\Id+(1-p)\rho\nonumber.
    \end{equation}
    \begin{figure}
        \centering
        \begin{subfigure}{0.45\textwidth}
            \centering
            \includegraphics[width=0.5\textwidth]{figures/whole_sphere.png}
        \end{subfigure}
        \begin{subfigure}{0.45\textwidth}
            \centering
            \includegraphics[width=0.5\textwidth]{figures/depol.png}
        \end{subfigure}
    \end{figure}
\end{frame}

\begin{frame}{El problema}
    \begin{center}
        ``Dinámica efectiva de un sistema de N qubits''
    \end{center}
    \begin{columns}
        \begin{column}{0.5\textwidth}
            \begin{displaymath}
                \xymatrix{
                  {\rho_{\ef}(0)} \ar[rr]^{\text{?}}
                  && {\rho_{\ef}(t)}\\
                  {\varrho_{\text{?}}(0)} \ar[rr]^{\mcE_{t}} \ar[u]^{\mcC}
                  && {\varrho_{\text{?}}(t)} \ar[u]^{\mcC}
                }
              \end{displaymath}
        \end{column}
        \pause
        \begin{column}{0.5\textwidth}
            \begin{displaymath}
                \xymatrix{
                  {\rho_{\ef}(0)} \ar[rr]^{\Gamma} \ar[d]^{\mcA}
                  && {\rho_{\ef}(t)}\\
                  {\varrho(0)} \ar[rr]^{\mcE_{t}}
                  && {\varrho(t)} \ar[u]^{\mcC}
                }
              \end{displaymath}
              \pause
        \end{column}
    \end{columns}
    \begin{center}
        ``Utilizando el Principio de Máxima Entropía''
    \end{center}
\end{frame}

\subsection{Entropía e información}

\begin{frame}{Información clásica}
    \begin{columns}
        \begin{column}{0.5\textwidth}
            \begin{center}
                A cada evento se le puede asociar una cantidad de información\pause
            \end{center}
            \begin{itemize}
                \item ``No cayó 6''$\rightarrow$poca información\pause
                \item ``Cayó 6''$\rightarrow$más información\pause
            \end{itemize}
            Entonces es decreciente con la probabilidad, monótona y Un evento $p=1$ no transmite información.
        \end{column}
        \pause
        \begin{column}{0.5\textwidth}
            Si , entonces debe poder calcularse la cantidad de información promedio:\\
            \pause
            Shannon.
        \end{column}
    \end{columns}
\end{frame}

\begin{frame}{Entropía de Shannon}
    \begin{columns}
        \begin{column}{0.5\textwidth}
            Claude Shannon demostró 
            \begin{equation}
                S_{\text{S}}=-k\sum_{j}p(x_{j})\log{p(x_{j})}.\nonumber
            \end{equation}
            Fue a través de discusiones con von Neumann que Shannon descubrió que su medida ya era ampliamente utilizada en física, y que llevaba el nombre de \textit{entropía}.   
            La entropía de Shannon se utiliza como la cantidad promedio de bits requerida para trasmitir un mensaje.
        \end{column}
        \begin{column}{0.5\textwidth}
            Ejemplo en apéndice?
        \end{column}
    \end{columns}
\end{frame}



\subsection{El principio de Máxima Entropía}

\begin{frame}{El PME clásico}
    \begin{columns}
        \begin{column}{0.5\textwidth}
            Supóngase que
            $\expval{\text{\Pisymbol{dice3d}{102}}}=3.5$\\
            ¿$p(x)$?\\
            $p(\epsdice{5})=\frac{1}{2}$ y $p(\epsdice{2})=\frac{1}{2}$\\
            Dado bien balanceado
            $\downarrow$
            maximiza la entropía de Shannon
        \end{column}
        \begin{column}{0.5\textwidth}
            El principio de máxima entropía fue introducido por E. T. Jaynes en 1957.
            Jaynes afirma que la distribución de probabilidad que maximice la entropía es la estimación menos sesgada que se puede hacer.
        \end{column}
    \end{columns}
\end{frame}

\begin{frame}{El PME clásico}
    Sean $x_{j}$ los valores de $X$, $f_{k}$, funciones de valor esperado conocido. La información:
    \begin{equation}
    \expval{f_{l}(x)}=\sum_{j}p(x_{j})f_{l}(x_{j}).\nonumber
    \end{equation}
    Maximizamos la entropía usando multiplicadores de Lagrange:
    \begin{equation}
        \mcL=-S_{\text{S}}(p)+\sum_{l}\lambda_{l}\qty(\sum_{j}p(x_{j})f_{l}(x_{j})-\expval{f_{l}(x)})+\mu\qty(\sum_{j}p(x_{j})-1).\nonumber
    \end{equation}
    Se deriva, se iguala a cero, se halla un sistema de ecuaciones.
\end{frame}

\begin{frame}{El PME clásico}
    Las soluciones:
    \begin{equation}
        p(x_{j})=\exp[-(1+\frac{\mu}{k})-\frac{1}{k}\sum_{l}\lambda_{l}f_{l}(x_{j})].\nonumber
    \end{equation}
    La función de partición,
    \begin{equation}
        \frac{1}{e^{-(1+\mu)}}=\sum_{j}\exp[-\frac{1}{k}\sum_{l}\lambda_{l}f_{l}(x_{j})] \Rightarrow Z=\sum_{j}\exp[-\frac{1}{k}\sum_{l}\lambda_{l}f_{l}(x_{j})],\nonumber
    \end{equation}
    La distribución de probabilidad que maximiza la entropía: 
    \begin{equation}
        p(x_{j})=\frac{1}{Z}\exp[-\frac{1}{k}\sum_{l}\lambda_{l}f_{l}(x_{j})].\nonumber
    \end{equation}
\end{frame}

\begin{frame}{El PME cuántico}
    Usamos la entropía de von Neuman $S=-Tr(\rho\ln(\rho))$, que para $\rho$ $H$ tales que $[\rho,H]=0$ es
    \begin{equation}
        S_{\text{N}}=-\sum_{k}\eta_{k}\log{\eta_{k}}.\nonumber
    \end{equation}
    Para maximizarla:
    \begin{equation}
        \mcL=\sum_{k}\eta_{k}\log{\eta_{k}}+\lambda_{0}\qty(\sum_{k}\eta_{k}-1)+\lambda_{1}\qty(\sum_{k}\eta_{k}E_{k}-\expval{H}).\nonumber
    \end{equation}
    Las soluciones:
    \begin{equation}
        \eta_{k}=-e^{(\lambda_{0}+1+\lambda_{1}E_{k})}=\frac{1}{Z}e^{-\lambda E_{k}}.\nonumber
    \end{equation}
\end{frame}

\begin{frame}{El PME cuántico}
    Utilizando notación de Dirac, $\rho=\sum_{k}\eta_{k}\dyad{e_{k}}$, podemos hallar el resultado independiente de la elección de la base:
    \begin{equation}
        \rho=\frac{1}{Z}e^{-\lambda H}.\nonumber
    \end{equation}
    Este resultado es fácilmente generalizable al caso en que contemos con un  conjunto $\{A_{j}\}_{j}$ de $N$ observables tales que $[A_{j},\rho]=0$, de los que conozcamos sus valores esperados. En general, la expresión del estado de máxima entropía es
    \begin{equation}\label{eq:GeneralMaxEnt}
        \rho=\frac{1}{Z}e^{-\sum_{k}\lambda_{k} A_{k}}.\nonumber
    \end{equation}
\end{frame}


\begin{frame}{El problema}
    \begin{center}
        ``Dinámica efectiva de un sistema de N qubits\\
        Utilizando el Principio de Máxima Entropía''
        \begin{displaymath}
            \xymatrix{
              {\rho_{\ef}(0)} \ar[rr]^{\Gamma_{t}} \ar[d]^{\mcA_{\mcC}^{\max}}
              && {\rho_{\ef}(t)}\\
              {\varrho_{\max}(0)} \ar[rr]^{\mcE_{t}}
              && {\varrho_{\max}(t)} \ar[u]^{\mcC}
            }
          \end{displaymath}
    \end{center}
\end{frame}