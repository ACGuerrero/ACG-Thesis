\section{Construcción del modelo y la asignación}

%###########################
%########### CG ############
\subsection{El modelo de grano grueso}
\begin{frame}{Medición borrosa y falta de resolución}
    \begin{center}
        El modelo de grano grueso considera dos tipos de error\mycite{FuzzyMeasurements}\pause:
    \end{center}
    \vspace{0.3cm}
    \begin{columns}
        \begin{column}{0.5\textwidth}
            \begin{block}{Permutación}
            \centering
            La posibilidad de medir una partícula diferente a la pretendida\pause:
            \begin{align}
                \mcF:&\mcS(\hilbert_2 \otimes \hilbert_2)\to \mcS(\hilbert_2 \otimes \hilbert_2)\pause\nonumber\\
                &\varrho \mapsto p\varrho+(1-p)S\varrho S.\nonumber
            \end{align}
        \end{block}
        \end{column}
        \pause
        \begin{column}{0.5\textwidth}
            \begin{block}{Resolución}
            \centering
            Falta de resolución en el aparato de detección\pause:
            \begin{gather}
                \mcC:\mcS(\hilbert_2 \otimes \hilbert_2)\to \mcS(\hilbert_2)\pause\nonumber\\
                \varrho_{AB} \mapsto \Tr_{B}(\Fuzzy{\varrho_{AB}})\pause=p\rho_A+(1-p)\rho_B\rlap{,}\nonumber
            \end{gather}
        \end{block}
        \end{column}
    \end{columns}
\end{frame}
\begin{frame}{Extensión a N qubits}
    \begin{columns}
        \begin{column}{0.5\textwidth}
            \begin{block}{Permutación}
            Para $N$ qubits la aplicación borrosa\pause:
            \begin{equation}
                \begin{gathered}
                \mcF:\mcS\qty( \hilbert_{2}^{\otimes n})\to \mcS\qty( \hilbert_{2}^{\otimes n})\pause\nonumber\\
                \varrho \mapsto p_{1}\varrho+\sum_{j=2}^{n}p_{j}(S_{1,j})\varrho(S_{1,j}),\nonumber
            \end{gathered}\nonumber
        \end{equation}
    \end{block}
        \end{column}
        \pause
        \begin{column}{0.5\textwidth}
            \begin{block}{Resolución}
        La aplicación de grano grueso\pause:
        \begin{equation}
            \begin{gathered}
                \mcC:\mcS( \hilbert_{2}^{\otimes n})\to \mcS(\hilbert_{2})\pause\\
                \varrho\mapsto\Tr_{\overline{1}}(\Fuzzy{\varrho})\pause = \sum_{k}p_{k}\rho_{k}.
            \end{gathered}\nonumber
        \end{equation}
    \end{block}
        \end{column}
    \end{columns}
\end{frame}
%###########################


%###########################
%######### MaxEnt ##########
\subsection{La aplicación de asignación}
\begin{frame}{El estado de máxima entropía}
    \begin{columns}
        \begin{column}{0.5\textwidth}
            \begin{displaymath}
                \xymatrix{
                  {\rho_{\ef}(0)} \ar[rr]^{\Gamma_{t}} \ar[d]^{\mcA_{\mcC}^{\max}}
                  && {\rho_{\ef}(t)}\\
                  {\varrho_{\max}(0)} \ar[rr]^{\mcE_{t}}
                  && {\varrho_{\max}(t)} \ar[u]^{\mcC}
                }
              \end{displaymath}\pause
              El estado que maximiza la entropía es:
              \begin{equation}
                  \varrho_{\max}=\frac{1}{Z}\exp(\sum_{i}\lambda_{i}{\color{red}G_{i}})\nonumber
              \end{equation}\pause
              ¿Qué valores de expectación \textbf{conocemos}?
        \end{column}
        \pause
        \begin{column}{0.5\textwidth}
            Los de $\rho_{\ef}$\pause:
            \begin{align}
                \expval{\pauli{i}}_{\rho_{\ef}}&=\Tr(\pauli{i}\rho_{\ef})\pause\nonumber\\
                &=\Tr[\sigma_{i}\CG{\varrho}]\pause\nonumber\\
                &=\Tr[\sigma_{i}\otimes\Id_{2^{n-1}}\pause\qty(p_{1}\varrho+\sum_{k=2}^{n}p_{k}S_{1,k}\varrho S_{1,k})]\pause\nonumber\\
                &=\Tr[{\color{red}\qty(\sum_{k=1}^{n}p_{k}(\Id_{2^{k-1}}\otimes\sigma_{i}\otimes\Id_{2^{n-k}}))}\varrho]\pause\nonumber\\
                &=\expval{{\color{red}G_{i}}}_{\varrho}.\nonumber
            \end{align}
        \end{column}
    \end{columns}
\end{frame}
\begin{frame}{Definición de la aplicación de Máxima Entropía}
    \begin{columns}
        \begin{column}{0.5\textwidth}
            Tenemos
            \begin{equation}
                \varrho_{\max}=\varrho_{\max}(\lambda_{1},\lambda_{2},\lambda_{3})\nonumber
            \end{equation}\pause
            La relación:
            \begin{equation}
                \expval{\pauli{k}}=\frac{\lambda_{k}}{\lambda}\sum_{j=1}^{n}p_{j}\tanh(p_{j}\lambda)\nonumber
            \end{equation}\pause
            donde $\lambda=\sqrt{\lambda_{1}^{2}+\lambda_{2}^{2}+\lambda_{3}^{2}}$
        \end{column}
        \pause
        \begin{column}{0.5\textwidth}
            \begin{block}{La aplicación}
            Definimos la aplicación de asignación de máxima entropía:\pause
            \tcbox{$
                \begin{gathered}
                    \mcA_{\mcC}^{\max}:\densityspace{2}\rightarrow\mcS(\hilbert_{2}^{\otimes n})\pause\nonumber\\
                    \rho_{\ef} \mapsto \Motimes_{j=1}^{n}\frac{1}{Z_{j}}\text{exp}\qty(p_{j}\sum_{k=1}^{3}\lambda_{k}\sigma_{k}).
                \end{gathered}$}
                A notar \textbf{factorizabilidad}: $\varrho_{\max}=\Motimes_{j=1}^{n}\rho_{k}$
            \end{block}
        \end{column}
    \end{columns}
\end{frame}
%###########################


%###########################
%######### Gammat ##########
\begin{frame}{Uniendo todo}
    \begin{displaymath}
        \xymatrix{
          {\rho(0)} \ar[rr]^{\Gamma_{t}} \ar[d]^{\mcA^{max}_{\mcC}}
          && {\rho(t)}\\
          {\varrho_{max}(0)} \ar[rr]^{\mcE_{t}}
          && {\varrho_{max}(t)} \ar[u]^{\mcC}
        }
      \end{displaymath}\pause
      Donde:
        \begin{equation}
            \mcA_{\mcC}^{\max}[\rho]=\Motimes_{j=1}^{n}\frac{1}{Z_{j}}\text{exp}\qty(p_{j}\sum_{k=1}^{3}\lambda_{k}\sigma_{k}) \qquad \pause \CG{\varrho}=\Tr_{\overline{1}}\qty(p_{1}\varrho+\sum_{j=2}^{n}p_{j}(S_{1,j})\varrho(S_{1,j}))\nonumber
        \end{equation}\pause
        \begin{center}
            \tcbox{$\Gamma_t[\rho]:=(\mcC \circ \mcE_t \circ \mcA^{max}_{\mcC})[\rho].$}
        \end{center}
\end{frame}
%###########################
