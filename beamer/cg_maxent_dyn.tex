\section{El modelo, la asignación, y la dinámica}
\iffalse
\subsection{El modelo de grano grueso}

\begin{frame}{Motivación}
    \begin{columns}
        \begin{column}
            El modelo de grano grueso considera dos tipos de error:
            \begin{itemize}
                \item posibilidad de medir una partícula diferente a la pretendida
                \item falta de resolución en el aparato de detección
            \end{itemize} 
        \end{column}
        \begin{column}
            Un prototipo sencillo de error consiste en el inducido por un aparato que no distingue diferentes conjuntos de partículas entre sí. El caso más simple corresponde a la permutación de dos partículas.
            \begin{align}
                \mcF:&\mcS(\hilbert_2 \otimes \hilbert_2)\to \mcS(\hilbert_2 \otimes \hilbert_2)\nonumber\\
                &\varrho \mapsto p\varrho+(1-p)S\varrho S,\nonumber
            \end{align}
        \end{column}
    \end{columns}
\end{frame}

\begin{frame}{Motivación}
    \begin{columns}
        \begin{column}
            Al error se le añade la falta de resolución: solo se resuelve una partícula. Matemáticamente, la composición del error y de la falta de resolución puede escribirse como
            \begin{gather}
                \mcC:\mcS(\hilbert_2 \otimes \hilbert_2)\to \mcS(\hilbert_2)\nonumber\\
                \varrho_{AB} \mapsto p\rho_A+(1-p)\rho_B\rlap{,}\nonumber
            \end{gather}
        \end{column}
        \begin{column}
            Considerando un sistema de $n$ subsistemas de dos niveles, la aplicación borrosa se define entonces como
\begin{gather}
    \mcF:\mcS\qty( \hilbert_{2}^{\otimes n})\to \mcS\qty( \hilbert_{2}^{\otimes n})\nonumber\\
    \varrho \mapsto p_{1}\varrho+\sum_{j=2}^{n}p_{j}(S_{1,j})\varrho(S_{1,j}),\nonumber
\end{gather}
donde $S_{1,j}$ es el operador que permuta la primera y la $j$-ésima partícula. De esta manera, donde $\Tr_{\overline{i}}$ denota la traza parcial sobre todos menos el $i$-ésimo qubit, la aplicación de grano grueso que resuelve un qubit donde hay $n$ qubits es
\begin{equation}
    \begin{gathered}
        \mcC:\mcS( \hilbert_{2}^{\otimes n})\to \mcS(\hilbert_{2})\\
        \varrho\mapsto\Tr_{\overline{1}}(\Fuzzy{\varrho}).
    \end{gathered}
\end{equation}
        \end{column}
    \end{columns}
\end{frame}
\fi

\subsection{La aplicación de asignación}
\begin{frame}{Repaso: PME}
    \lipsum[1]
\end{frame}

\begin{frame}{Construcción del estado asignado}
    \lipsum[1]
\end{frame}


\subsection{La dinámica efectiva}
\begin{frame}{Construcción de la dinámica}
    \lipsum[1]
\end{frame}
