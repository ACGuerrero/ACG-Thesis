\section{Conclusiones}

\begin{frame}{Conclusiones}
    \begin{columns}
        \begin{column}{0.5\textwidth}
            \begin{itemize}
                \item Se obtuvieron resultados analíticos.\pause
                \item Se observó la emergencia de no linealidad.\pause
                \item Las dinámicas ser descritas por el formalismo de canales cuánticos.\pause
                \item Las dinámicas no son universales.\pause
                \item La no linealidad depende de la asignación.\pause
            \end{itemize}
        \end{column}
        \begin{column}{0.5\textwidth}
            \begin{itemize}
                \item La asignación depende de la aplicación de grano grueso.\pause
                \item Ninguna dinámica estudiada es no lineal en el caso de no-error.\pause
                \item Queda por estudiar la relación entre la no-linealidad, la asignación, y la aplicación borrosa.\pause
                \item El estado de máxima entropía puede modificarse.modificarse.\pause
            \end{itemize}
        \end{column}
    \end{columns}
\end{frame}
