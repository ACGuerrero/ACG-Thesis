\section{Conclusiones}

\begin{frame}{Conclusiones}
    \begin{itemize}
        \item Se estudió la dinámica emergente de un modelo de grano grueso a través del principio de máxima entropía.
        \item Se hallaron comportamientos más afines a la física clásica que a la mecánica cuántica, como la no linealidad de las evoluciones.
        \item Las dinámicas inducidas por las compuertas SWAP y CNOT resultaron ser versiones no lineales de canales cuánticos conocidos
        \item Sin embargo, las no linealidades encontradas, a diferencia de las dinámicas clásicas no lineales, no son universales.
    \end{itemize}
\end{frame}

\begin{frame}{Conclusiones}
    \begin{itemize}
        \item A diferencia de las evoluciones deterministas, las evoluciones efectivas estudiadas se traducían en contracciones de la esfera de Bloch.
        \item Las dinámicas efectivas resultaron ser procesos irreversibles.
        \item Algunas de las dinámicas estudiadas, particularmente aquellas generadas por evoluciones subyacentes con fuertes simetrías, resultaron ser no solo lineales, sino canales cuánticos.
    \end{itemize}
\end{frame}

\begin{frame}
    \begin{itemize}
        \item Se encontró que las dos asignaciones estudiadas coinciden en los casos en que no existe la posibilidad de que se detecte una partícula diferente a la pretendida. 
        \item Se demostró que todas las aplicaciones de asignación deben coincidir cuando el estado efectivo inicial es puro y hay participación de todos los subsistemas en la aplicación de medición borrosa.
    \end{itemize}
\end{frame}

\begin{frame}{Conclusiones}
    \begin{itemize}
        \item Ninguna de las dinámicas estudiadas es no lineal en el caso en que las partículas no preferenciales tienen una probabilidad nula de ser detectadas, esto es, cuando el error es nulo, que equivale al caso en que la aplicación de medición borrosa sale del escenario.
        \item Aunque el único elemento no lineal en la composición que define a la dinámica efectiva es la aplicación de asignación, queda por investigar si es la aplicación de asignación o la aplicación de medición borrosa la causante de las no linealidades en las dinámicas efectivas.
    \end{itemize}
\end{frame}