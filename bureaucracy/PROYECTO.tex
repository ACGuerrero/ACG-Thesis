\documentclass[onecolumn,11pt]{article}
\usepackage[spanish]{babel}
\usepackage[utf8]{inputenc}
\usepackage[a4paper, total={7in, 9in}]{geometry}
\usepackage{amsfonts}
\usepackage{physics}
\usepackage{xcolor}
\newcommand{\mcU}{\mathcal{U}}
\newcommand{\mcO}{\mathcal{O}}
\newcommand{\mcI}{\mathcal{I}}
\newcommand{\mcL}{\mathcal{L}}
\newcommand{\mcB}{\mathcal{B}}
\newcommand{\mcH}{\mathcal{H}}
\newcommand{\mcF}{\mathcal{F}}
\newcommand{\mcC}{\mathcal{C}}
\newcommand{\mcT}{\mathcal{T}}
\newcommand{\mcE}{\ensuremath{\mathcal{E}} }
\newcommand{\mcG}{\ensuremath{\mathcal{G}} }
\newcommand{\mcM}{\mathcal{M}}
\newcommand{\mcN}{\mathcal{N}}
\newcommand{\nnn}{\mathcal{N}}
\newcommand{\choi}{\ensuremath{\mcD} }
\newcommand{\mmm}{\mathcal{M}}
\newcommand{\sss}{\mathcal{S}}
\newcommand{\mcD}{\mathcal{D}}
\newcommand{\mcA}{\mathcal{A}}
\newcommand{\mcP}{\mathcal{P}}
\usepackage{bm}
\usepackage{hyperref}
\title{Proyecto de Tesis} \date{}
\newcommand{\Cc}{\mathcal{C}} %Para escribir la C caligráfica que denota al mapeo CG
\newcommand{\Hh}{\mathbb{C}} %Para escribir la H caligráfica que denota esp de Hilbert
\newcommand{\ident}{\mathbb{1}}% Para escribir el op. indentidad con notación chida
\newcommand{\CG}[1]{\Cc\left[#1\right]}
\newcommand*{\B}[1]{\ifmmode\bm{#1}\else\textbf{#1}\fi}
\newcommand{\nota}[1]{{\color{red} [#1]}}
\begin{document}
\maketitle
\thispagestyle{empty}
Se presenta el proyecto de Tesis de Licenciatura del alumno de la carrera de Física en la Facultad de Ciencias de la UNAM, Adán Castillo Guerrero, bajo el tutelaje del Dr. David Dávalos González. El título tentativo es ``Título chido''.

\section{Marco teórico}


Dentro del marco de la \textit{teoría de información cuántica}, especialmente en la conexión con la parte experimental, es natural suponer que no siempre es posible tener disponible toda la información sobre el \textit{estado} del sistema de interés. Esto ya sea por insuficiencia en la resolución de los aparatos de medición o por el inevitable error inherente a las herramientas de medición. Un prototipo sencillo de error, pero El aparato a disposición, por ejemplo, puede confundir diferentes conjuntos de partículas entre sí. El caso más sencillo corresponde a la permutación de dos partículas. Es decir, si un sistema está compuesto por dos partículas cuyos operadores de densidad son $\rho_{1}$ y $\rho_{2}$ \nota{Esto es parte de la nota que viene después, si bien estos son los estados reducidos para cada partícula, tener conocimiento de ellos no es equivalente a conocer el estado en conjunto. Es una de las cosas que diferencia a la cuántica de la clásica.}, entonces el sistema está descrito por el operador $\rho_{1}\otimes\rho_{2}$ \nota{Aguas, no todos los estados de dos partículas tienen esta forma, dime si tienes problemas y lo hablamos}, y la permutación equivaldría a detectar, en cambio, $\rho_{2}\otimes\rho_{1}$ \cite{FuzzyMeasurements}. Esta \textit{confusión} a la hora de la medición es una aplicación borrosa. Considérese, pues, que se aplica esta a un sistema de dos parículas descrito por un operador de densidad $\rho$:
\begin{equation}\label{eq:Fuzzy2}
\mcF[\rho]=p\rho+(1-p)S\rho S^{\dag}
\end{equation}


Al modelaje de un grupo como una unidad, o como otro grupo, pero con un número inferior de grados de libertad, se le llama \textit{coarse graining}.


Tengo que desarrollar mejor el marco.

\begin{equation}\label{eq:CG}
    \Cc\left[\rho\right]=\Tr_{B}\left(p\rho+(1-p)S\rho S^{\dag}\right)
\end{equation}

donde $\rho\in \Hh^{2}_{A}\otimes \Hh^{2}_{B}$ es el operador de densidad de un sistema de cuatro niveles, y $\Tr_{B}$ denota la traza parcial respecto a $\Hh^{2}_{B}$. Al estado $\rho$ se le llama estado ``microscópico'' y a $\CG{\rho}$, estado ``macroscópico''. En este documento, a los estados macroscópicos se les escribirá en negritas. De esta forma:
\begin{equation}
    \B{\rho}=\CG{\rho}
\end{equation}

\vspace{0.5cm}
\section{Resumen del proyecto}

El trabajo tiene como objetivo el estudio y posterior descripción de la dinámica bajo un mapeo particular de coarse graining, dado por la ecuación (\ref{eq:CG}). Esto es, dada una operación $M$ \nota{Aqui mas bien deberiamos hablar directamente de unitarias. Sabes Que otras operaciones hay?, si no, no importa, hablar de las unitarias es suficiente}:
\begin{equation}
    M:\Hh^{4}\rightarrow\Hh^{4}
\end{equation}
aplicada sobre el sistema descrito por $\rho$, nos interesa saber el mapeo efectivo resultante sobre $\B{\rho}$. Tomando en cuenta que si:
\begin{align}
    \rho'=&M\rho M^{\dag}\\
    \Rightarrow \B{\rho}'=&\CG{M\rho M^{\dag}}
\end{align}
Para un observador externo, que solo tiene acceso a la descripción \textit{gruesa}, la dinámica se ve como:
\begin{equation}\label{eq:obs}
    \B{\rho}\rightarrow\mathcal{A}\left[\B{\rho}\right]=\B{\rho}'
\end{equation}
con $\mathcal{A}$ un mapeo:
\begin{equation}
    \mathcal{A}:\Hh^{2}\rightarrow\Hh^{2}
\end{equation}
Una idea sería determinar la forma de $\mathcal{A}$ mediante tomografía de procesos cuántica. 

\nota{Aquí podemos hablar del meollo de la dinámica. Ya hablando con Carlos, parte del interés es que no es lineal la dinámica gruesa. Esto es interesante, ya que la cuántica siempre es lineal, pero el universo clásico no lo es siempre. Parte de la idea de Fernando del nuevo paper de Fernando, donde habla del Assignement map, es eso, la emergencia de no linealidad. Aquí puedes hablar un poco de la ecuación: $\rho(t)=\mcC \circ \mcU_t \circ \mcA [\rho(0)]$} Claro, un ejercicio muy interesante sería el de estudiar el problema \textit{inverso}, esto es, que a través de la observación dada por la ecuación (\ref{eq:obs}), se obtenga un mapeo $\mathfrak{A}$ tal que:
\begin{equation}
    \rho'=\mathfrak{A}\left[\rho\right]
\end{equation}
La cuestión es que me parece que esto requiere haber resuelto primero el problema del mapeo de asignación inverso para el coarse graining.


\begin{center}
    \rule{200pt}{0.4pt}\\
    Adán Castillo Guerrero \\
    Estudiante de Física \\
    Facultad de Ciencias, UNAM \\
    
\end{center}

\vspace{1.0cm}

\begin{center}
    \rule{200pt}{0.4pt}\\
    Dr. David Dávallos Gonzáles \\
    Investigador postdoctoral \\
    RCQI, Instituto de Física, Academia Eslovaca de Ciencias (SAV), Eslovaquia\\
    
\end{center}
\bibliographystyle{plain}
\bibliography{bibliography}

\end{document}